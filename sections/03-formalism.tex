\section{Formulation}

%\subsection{\label{sec:rwa}The rotating-wave approximation}

In the systems under consideration here, $\omega_0$ lies in the optical frequency band ($\sim \SI{1500}{\milli\eV\per\hbar}$).
As such, na\"ively integrating \cref{eq:liouville} to resolve the Rabi dynamics that occur on the order of \SI{1}{\pico\second} becomes computationally infeasible.
By introducing $\tilde{\rho} = \hat{U} \hat{\rho} \hat{U}^\dagger$ where $\hat{U} = \mathrm{diag}(1, e^{i \omega_L t})$, we may instead write \cref{eq:liouville} as
\begin{equation}
  \pdv{\tilde{\rho}}{t} = \frac{-i}{\hbar} \commutator{\hat{U} \hat{\mathcal{H}} \hat{U}^\dagger - i \hbar \hat{V}}{\tilde{\rho}} - \hat{\mathcal{D}}\qty[\tilde{\rho}], \quad \hat{V} \equiv \hat{U} \pdv{\hat{U}^\dagger}{t}
  \label{eq:rotating liouville}
\end{equation}
which will contain only terms proportional to $e^{i (\omega_0 \pm \omega_L) t}$ assuming a monochromatic incident field.
Ignoring the high-frequency quantities under the assumption that such terms will integrate to approximately zero in solving \cref{eq:rotating liouville} over any appreciable timescale produces the familiar rotating-wave approximation\cite{}---as the system no longer contains any optical frequencies, numerical differential equation solvers require orders of magnitude fewer timesteps to solve \cref{eq:rotating liouville}.


%Models of isolated \qds{} or \qd{} ensembles in which the separation between dots far exceeds a typical interaction length (perhaps through a finite-difference time-domain grid) often use a rotating-wave approximation to great effect in eliminating high-frequency terms in the solution to \cref{eq:liouville} with only a small change in the resulting dynamics.
%Assuming the incident radiation takes the form $\vb{E}(t) = \tilde{\vb{E}}(t) \cos(\omega_L t)$
%where $\tilde{\vb{E}}(t)$ represents a slowly-varying envelope function, we motivating $\hat{U} \hat{\mathcal{H}} \hat{U}^\dagger - i \hbar \hat{V} \equiv \tilde{\mathcal{H}}$ as a ``rotating-frame Hamiltonian.''
%By neglecting the $e^{i \qty(\omega_0 + \omega_L) t}$ terms under the assumption that they oscillate quickly and will therefore integrate to zero over any appreciable timescales, we produce a governing equation much more amenable to numerical integration as it contains only smooth (low-frequency) quantities.

Due to the quantum mechanical transitions at play in producing secondary radiation, we may assume similarly monochromatic radiated fields.
As such, a similar transformation applies to the source distribution in \cref{eq:integral operator}.
Writing $\vb{P}(\vb{r}, t) = \tilde{\vb{P}}(\vb{r}, t)e^{i \omega_L t}$, $\vb{E}_\text{rad}(\vb{r}, t) = \tilde{\vb{E}}_\text{rad}(\vb{r}, t)e^{i \omega_L t}$, and ignoring high-frequency terms, the radiated field envelope becomes
\begin{widetext}
\begin{equation}
  \begin{gathered}
    \tilde{\mathfrak{F}}\{ \tilde{\vb{P}}(\vb{r}, t) \} = \frac{-1}{4\pi \varepsilon} \int
    \qty(\tensor{\mathrm{I}} -  \bar{\vb{r}}\bar{\vb{r}}^\dagger) \cdot \frac{\qty(\partial_t^2 \tilde{\vb{P}}(\vb{r}', t_R) + 2 i \omega_L \partial_t \tilde{\vb{P}}(\vb{r}', t_R) - \omega_L^2 \tilde{\vb{P}}(\vb{r}', t_R)) e^{-i \omega_L \abs{\vb{r} - \vb{r}'}/c}}{c^2 \abs{\vb{r}-\vb{r}'}} + \\
    \qty(\tensor{\mathrm{I}} - 3\bar{\vb{r}}\bar{\vb{r}}^\dagger) \cdot \frac{\qty(\partial_t \tilde{\vb{P}}(\vb{r}', t_R) + i \omega_L \tilde{\vb{P}}(\vb{r}', t_R))e^{-i \omega_L \abs{\vb{r} - \vb{r}'}/c}}{c \abs{\vb{r}-\vb{r}'}^2} +
    \qty(\tensor{\mathrm{I}} - 3\bar{\vb{r}}\bar{\vb{r}}^\dagger) \cdot \frac{                \tilde{\vb{P}}(\vb{r}', t_R) e^{-i \omega_L \abs{\vb{r} - \vb{r}'}/c}}{\abs{\vb{r}-\vb{r}'}^3}
    \, \dd[3]{\vb{r'}}.
  \end{gathered}
  \label{eq:radiated envelope}
\end{equation}
\end{widetext}
where $\bar{\vb{r}} \equiv \qty(\vb{r} - \vb{r}')/\abs{\vb{r} - \vb{r}'}$ and $t_R \equiv t - \abs{\vb{r} - \vb{r}'}/c$.

%Critically, \cref{eq:radiated envelope} maintains the phase relationship between every pair of \qds{} for arbitrarily large timesteps via the $e^{-i \omega_L \abs{\vb{r} - \vb{r}'}/c}$ factors that appear.
%Without this transformation, the evolution of \cref{eq:liouville} would require $c \, \Delta t$ at least as small as the minimum separation between \qds{}---roughly three orders of magnitude smaller than the $\Delta t$ required to resolve the maximum frequency in \cref{eq:rotating liouville}.


% Look at http://za2uf4ps7f.scholar.serialssolutions.com/?sid=google&auinit=B&aulast=Shanker&atitle=Fast+analysis+of+transient+electromagnetic+scattering+phenomena+using+the+multilevel+plane+wave+time+domain+algorithm&id=doi:10.1109/TAP.2003.809054&title=I.R.E.+transactions+on+antennas+and+propagation&volume=51&issue=3&date=2003&spage=628&issn=0018-926X for time basis notation
