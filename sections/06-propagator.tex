\section{Derivation of \cref{eq:integral operator}}

For time-harmonic (monochromatic) electric fields, we require
\begin{equation}
  \vb{E}_\omega = - \frac{1}{c} \qty(i \omega \vb{A}_\omega) - \grad \phi_\omega
  \label{eq:potential e-field}
\end{equation}
where $G_\omega(\vb{r} - \vb{r}') = e^{i k \abs{\vb{r} - \vb{r}'}}/\abs{\vb{r} - \vb{r}'}$ and 
\begin{subequations}
  \begin{align}
      \phi_\omega(\vb{r}) = \int G(\vb{r} - \vb{r}') \rho_\omega(\vb{r}') \dd[3]{\vb{r}'} \label{eq:scalar potential} \\
    \vb{A}_\omega(\vb{r}) = \frac{1}{c} \int G(\vb{r} - \vb{r}') \vb{J}_\omega(\vb{r}') \dd[3]{\vb{r}'}. \label{eq:vector potential}
  \end{align}
\end{subequations}
Choosing
\begin{equation}
  \div{\vb{A}_\omega} + \frac{1}{c} \qty(i \omega \phi_\omega) = 0 
\end{equation}
and inserting \cref{eq:vector potential}, we may rewrite \cref{eq:potential e-field} as
\begin{equation}
  \vb{E}_\omega = -i \omega \int \tensor{\mathrm{G}}_d \cdot \vb{J}_\omega(\vb{r}') \dd[3]{\vb{r}'}
\end{equation}
where
\begin{equation}
  \tensor{\mathrm{G}}_d \equiv \qty[ \frac{\tensor{\mathrm{I}}}{c^2} + \frac{\grad\grad}{k^2} ] G_\omega(\vb{r} - \vb{r}').
  \label{eq:gdyad}
\end{equation}
Noting that $\grad{\bar{\vb{r}}} \equiv (\tensor{\mathrm{I}} - \outerprod{\bar{\vb{r}}}{\bar{\vb{r}}})/r$, we may expand \cref{eq:gdyad} as
\begin{equation}
  \begin{aligned}
    \tensor{\mathrm{G}}_d \equiv &  \qty[ \qty(\frac{1}{c^2} - \frac{i}{k r} - \frac{1}{(kr)^2})\tensor{\mathrm{I}} ] G_\omega(\vb{r}) \\
    &- \qty[ \qty(1 - \frac{3i}{kr} - \frac{3}{(kr)^2}) \outerprod{\bar{\vb{r}}}{\bar{\vb{r}}} ] G_\omega(\vb{r}).
  \end{aligned}
\end{equation}
Finally, making the substitution $k \to \omega/c$ will give powers of $i \omega$ that produce the $0^\text{th}$, $1^\text{st}$, and $2^\text{nd}$ derivatives in \cref{eq:integral operator} after applying an inverse Fourier transform in time.
