\section{\label{section:introduction}Introduction}

\subsection{Why study quantum dots?}
Semiconductor structures containing a large number of \qds{} offer ideal environments for exploring light-matter interactions.
Often, these structures exhibit new phenomena due to geometrical randomness and non-linearities in the underlying system dynamics.
Optical excitations (excitons) produce characteristic Rabi oscillations~\cite{Stievater2001,Kamada2001,Htoon2002} in \qds{} analogous to those observed in atomic systems.
As \qds{} have stronger dipolar transitions than atoms, these light-induced oscillations generate considerable secondary fields that act to couple the system together.
We therefore expect---at least in some regions of the sample---these secondary fields to produce modified collective behavior in the exciton dynamics.
\textcolor{red}{Can we say something more concrete than what we expect?}
Researchers in atomic and solid-state optics have thoroughly investigated light-matter interactions in ensembles of two level systems, producing several variants of the Maxwell-Bloch equations~\cite{Gross1982} to describe characteristic features such as ringing in pulse propagation~\cite{Burnham1969,MacGillivray1976} or emission fluctuations~\cite{Haake1979}.
The interplay between long-range radiative and short-range dipolar interactions in these formulations introduces a van der Waals dephasing mechanism~\cite{Coffey1978} \textcolor{red}{to do what?}.
The direct investigation of such competition within the Maxwell-Bloch equations is hampered by the fact that those are defined using spatial averages in a continuum formulations, often limited to geometries of reduced dimensionality, in which the short-range dynamics is not described.
However, novel algorithmic approaches, as well as current computing capabilities, enable a microscopic formulation of Maxwell-Bloch dynamics in large systems, without resorting to spatial averages.

In this paper, we develop a computational framework to discover signatures of collective effects in strongly-driven \qds{} within a microscopic formalism, in which the effect of the radiation is studied at the level of individual \qds{}.
We describe three dimensional systems that extend over dimensions comparable to the light wavelength containing up to $10^4$ \qds{}.
We use a full-wave approach without resorting to a continuum treatment of macroscopic fields~\cite{} or mesh-based strategies (such as finite-difference time-domain method)~\cite{Vanneste2001,Fratalocchi2008}, techniques undermined by their computational overhead and numerical sensitivities~\cite{Baczewski2013}.
Our integral equation-based approach together with a technique to evolve the  state of the \qd{}, facilitates fast and accurate field calculations by allowing direct communication of field quantities between \qds{}.
This approach has been successfully used in other electromagnetic \cite{otherpapersfromshanker} and acoustic \cite{Glosser2016} systems characterized by localized centers and nonlinear wave propagation.
Specifically, an time domain integral equation based full wave model used to propagate fields between \qds{} affords three salient advantages:
\begin{inparaenum}[(i)]
  \item directly propagating fields between points removes the need for a computational grids to propagate fields, thereby improving accuracy by eliminating numerical dispersion,
  \item the propagator explicitly includes the Sommerfeld radiation condition \cite{Stratton2007}, and
  \item an integral equation methodology models both near and far fields to very high accuracy (unlike differential equation methods that rely on an underlying grid).
\end{inparaenum}
These  integral equations are paired with a  highly-tuned predictor-corrector scheme to provide an accurate account of the propagation and phase effects between \qds{}, even for very close particles.
The entire system steps through time to provide a self-consistent transient solution. 

By explicitly describing each \qd{} in the system, we will numerically demonstrate  that the collective Rabi oscillation can induce significant coupling in sufficiently close \qds{}.
This laser-induced inter-dot coupling manifests itself in different forms:
\begin{inparaenum}[(i)]
  \item For isolated \qd{} pairs, the polarization generated by the optical pulse is able to dynamically screen the \qds{} from Rabi rotation.
    We interpret this as the consequence of a time-dependent energy shift that brings the pair temporarily out of resonance with the external driving field.
  \item For multiplets of \qds{}, in addition to the screening, we observe oscillations in the free induction decay that are due to many-dot coupling.
  \item For optical pulses of integer $\pi$ area, for which no polarization is expected in the system after the pulse, we observe that patterns of residual localized polarization remain in the system.
\end{inparaenum}
% This is due to the two effects described above, which effectively reduce the area pulse in some areas of the sample.
% We characterize this radiation localization effect using the inverse participation ratio~\cite{Schwartz2007} of the dot polarization.
% This effect could be used, for instance, to selectively identify multiplets of dots that are dynamically coupled during Rabi oscillations.
\textcolor{red}{I do not like this -- presents data and explanation before material is presented}

The structure of this paper as follows: \cref{section:formulation} contains the formulation of our semi-classical approach describing the \qds{} and how they dynamically couple due light-induced effects.
Here we focus on effects that arise from the non-equilibrium nature of the oscillations and consider only the homogeneous case of identical \qds{} that do not couple in equilibrium.
\Cref{section:computational approach} contains the details of the our efficient computational method that allows for 3D simulations with over $10^4$ \qds{} in reasonable time.
We share our efficient computational method, which could be used in other solid state or atomic systems, in \cite{githubpage}.
Examples of our numerical experiments showing the effects (i)-(iii) described above are shown in \cref{section:results}.
We draw conclusions and discuss some future directions in \cref{section:conclusion}.
