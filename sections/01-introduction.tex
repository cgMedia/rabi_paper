Semiconductor structures containing a large number of quantum dots offer ideal environments for exploring light-matter interactions in regimes where non-equilibrium, non-linearity, and randomness lead to new phenomena.
Optical excitations (excitons) in give rise to characteristic Rabi oscillations in \qds{} that mimic two-level atoms.
Time-resolved~\cite{stievater,shih} and spectral~\cite{kamada} methods have revealed Rabi oscillations in many such systems.
In these experiments, the excitation typically involves a large collection of dots while an optical mask isolates the signal from a single dot.
Compared to atomic systems, quantum dots have stronger dipolar transitions and light-induced dipolar oscillations generate strong secondary fields that  depend on the random spatial configurations.
We therefore expect that, at least in some regions of the sample, these secondary fields lead to modified collective behavior in the exciton dynamics.
Signatures of collective effects due to long-range electromagnetic coupling have arisen in the photoluminescence emissions of ensembles of \qds{}~\cite{forchel}.

Here, we develop a microscopic computational framework to discover signatures of collective effects in strongly-driven quantum dots.
Current solution strategies for Maxwell-Bloch systems either resort to a continuum treatment of macroscopic fields~\cite{} or finite-difference or other mesh-based strategies~\cite{Vanneste2001, Fratalocchi2008} for full-wave models of the simulation domain---techniques undermined by their computational overhead and numerical sensitivities~\cite{Baczewski2013}.
We turn our attention to an integral approach that facilitates fast and accurate field calculations by allowing direct communication of field quantities between \qds{} without the need for a computationally-burdensome radiation grid.
Moreover, a highly-tuned predictor-corrector scheme facilitates an accurate account of propagation and phase effects between \qds{} even for very close particles. 

In analogy to the super- and sub-radiant regimes in the linear case, we find that in some instances the Rabi oscillation of a set of strongly coupled dots features additional oscillations due to an exchange of energy between the dot, and in other cases the dynamical creation of dark multiplets that screen completely the dots from Rabi rotation. ....Anything else we can mention?
