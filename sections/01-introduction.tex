\section{\label{section:introduction}Introduction}

%\subsection{Why study quantum dots?}
Semiconductor structures containing a large number of \qds{} offer ideal environments for exploring   collective effects induced by light-matter interaction.
Often, these structures exhibit new phenomena due to geometrical randomness and non-linearities in the underlying system dynamics.
Optical excitations (excitons) undergo characteristic Rabi oscillations~\cite{Stievater2001,Kamada2001,Htoon2002} in \qds{} analogous to those observed in atomic systems.
As \qds{} have stronger dipolar transitions than atoms, these light-induced oscillations generate  secondary fields that are sufficiently strong so as to couple the system more strongly than equivalent atomic species. We can therefore expect---at least in some regions of the sample---these local secondary fields to produce modified collective behavior in the exciton dynamics.
Both theoretical/computational~\cite{Slepyan2002,Slepyan2004} and experimental investigations~\cite{Asakura2013} into phenomena induced by these secondary fields have shown promise as avenues toward novel \textcolor{red}{solid-state} devices~\cite{Noginov2005,Bertolotti2010}.

%\subsection{What tools exist?}
In the realm of theoretical/computational investigation, researchers in atomic and solid-state optics have developed numerous variations of the Maxwell-Bloch equations~\cite{Gross1982} to describe features such as ringing in pulse propagation~\cite{Burnham1969,MacGillivray1976} or emission fluctuations~\cite{Haake1979}.
Early solution strategies for these equations fell to continuum models~\cite{Rehler1971,MacGillivray1976} that recover effects arising from far-field interactions but cannot adequately describe near-field regimes.
More recently, mesh-based PDE solvers~\cite{Vanneste2001,Fratalocchi2008,Bachelard2015} added a large degree of fidelity to these models, though the finite size of the mesh means they still have trouble resolving extremely short-range effects without unduly increasing the computational cost.
Additionally, the nature of these meshes make them prohibitively expensive to extend into three and higher dimensions for optically-large systems.
\textcolor{red}{In this work we develop a computational framework to discover signatures of collective effects in strongly-driven \qds{} within a microscopic formalism.}
By constructing the Maxwell-Bloch equations with an integral kernel and backwards-looking temporal basis functions to describe radiation, we recover near- and far-electric fields with full fidelity across the simulation while allowing for dynamics at the level of individual \qds{}. 
Our implementation---based on successful models of other electromagnetic \cite{otherpapersfromshanker} and acoustic \cite{Glosser2016} systems---readily tracks $10^4$ particles distributed over an optically large region in three dimensions.

As we explicitly track the evolution of each \qd{} in the system, we will numerically demonstrate that the collective Rabi oscillation can induce significant coupling in sufficiently close \qds{}.
This laser-induced inter-dot coupling manifests itself in different forms:
\begin{inparaenum}[(i)]
  \item For isolated \qd{} pairs, the polarization generated by the optical pulse is able to dynamically screen the \qds{} from Rabi rotation.
    We interpret this as the consequence of a time-dependent energy shift that brings the pair temporarily out of resonance with the external driving field.
  \item For multiplets of \qds{}, in addition to the screening, we observe oscillations in the free induction decay that are due to many-dot coupling.
  \item For optical pulses of integer $\pi$ area, for which no polarization is expected in the system after the pulse, we observe that patterns of residual localized polarization remain in the system.
\end{inparaenum}
These effects could, for instance, help identify multiplets of dots that dynamically couple during Rabi oscillations.

We structure the remainder of this paper as follows: \cref{section:problem statement} motivates the physical model of an ensemble of two-level systems that interact through a classical electric field.
\Cref{section:computational approach} present the details of our algorithm in the context of a global rotating-wave approximation and we offer an implementation of this algorithm at~\cite{githubpage}.
\Cref{section:results} contains the results of our investigation where we observe polarization features not present in noninteracting systems at both sub- and super-wavelength scales.
Finally, \cref{section:conclusion} contains concluding remarks where we speculate on the mechanisms underpinning the observed polarization features as well as comment on our future work in this area.
