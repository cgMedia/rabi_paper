\section{Numerical Results}
Here we detail the results of several investigations into coupled \qd{} behavior with the model presented thusfar.
Our algorithm reliably handles tens of thousands of \qds{} and can simulate ten picoseconds of system dynamics in two days on a single processor.
We perform simulations of systems of \qds{} randomly distributed throughout a cubic simulation volume at various densities.
\Cref{table:parameters} gives approximate values of the system parameters.

\begin{table}
  \begin{ruledtabular}
    \begin{tabular}{lll}
      Quantity                 & Symbol         & Value                        \\ \hline
      Transition frequency     & $\omega_0$     & $\SI{1500}{\milli\eV}/\hbar$ \\
      Transition dipole moment & $\abs{\vb{d}}$ & \SI{10}{\elementarycharge\bohr} \\
      Relaxation times         & $T_{1}, T_{2}$ & \SIlist{10;20}{\pico\second} \\
      Laser frequency          & $\omega_L$     & $\SI{1500}{\milli\eV}/\hbar$ \\ \hline
      Speed of light           & $c$            & \SI{299.792458}{\micro\meter\per\pico\second} \\
      Reduced Planck constant  & $\hbar$        & \SI{0.65821193}{\milli\eV \pico\second} \\
      Vacuum permeability      & $\mu_0$        & \SI{2.0133545e-4}{\milli\eV \pico\second\squared \per \elementarycharge \per \micro\meter}
    \end{tabular}
  \end{ruledtabular}
  \caption{\label{table:parameters}Rough simulation parameters.}
\end{table}

\begin{figure}
  \usetikzlibrary{pgfplots.groupplots}

\begin{filecontents}{ipr_1024.dat}
0.000000000000000000e+00 0.000000000000000000e+00
2.500000000000000139e-02 3.381020394210842994e-10
5.000000000000000278e-02 1.840404700598902155e-09
7.499999999999999722e-02 8.033211570338677499e-09
1.000000000000000056e-01 2.158837546033763599e-08
1.250000000000000000e-01 4.766930931281512795e-08
1.499999999999999944e-01 9.100822592755797302e-08
1.749999999999999889e-01 1.532972601608461466e-07
2.000000000000000111e-01 2.369038891636334604e-07
2.250000000000000056e-01 3.422420040461067459e-07
2.500000000000000000e-01 4.681486278444030313e-07
2.750000000000000222e-01 6.117639690802426783e-07
2.999999999999999889e-01 7.692248697121172041e-07
3.250000000000000111e-01 9.360082609832526962e-07
3.499999999999999778e-01 1.107389336343087352e-06
3.750000000000000000e-01 1.278841105706958123e-06
4.000000000000000222e-01 1.446443866589749994e-06
4.249999999999999889e-01 1.607074230902281672e-06
4.500000000000000111e-01 1.758516512770412835e-06
4.749999999999999778e-01 1.899458037815876868e-06
5.000000000000000000e-01 2.029400021924402178e-06
5.250000000000000222e-01 2.148513335702715091e-06
5.500000000000000444e-01 2.257463509529235995e-06
5.749999999999999556e-01 2.357237091006256224e-06
5.999999999999999778e-01 2.448984700587075736e-06
6.250000000000000000e-01 2.533892668434706286e-06
6.500000000000000222e-01 2.613087782425260047e-06
6.750000000000000444e-01 2.687574294546078366e-06
6.999999999999999556e-01 2.758199257206928739e-06
7.249999999999999778e-01 2.825640380475780145e-06
7.500000000000000000e-01 2.890410412151062786e-06
7.750000000000000222e-01 2.952872560571615651e-06
8.000000000000000444e-01 3.013262491573676788e-06
8.249999999999999556e-01 3.071713496050759717e-06
8.499999999999999778e-01 3.128282391075395616e-06
8.750000000000000000e-01 3.182974453097651285e-06
9.000000000000000222e-01 3.235766212925875879e-06
9.250000000000000444e-01 3.286625301406704766e-06
9.499999999999999556e-01 3.335526797607415994e-06
9.749999999999999778e-01 3.382465748892702490e-06
1.000000000000000000e+00 3.427465741071434452e-06
1.024999999999999911e+00 3.470583607187958941e-06
1.050000000000000044e+00 3.511910568371667471e-06
1.074999999999999956e+00 3.551570282031525291e-06
1.100000000000000089e+00 3.589714412759225162e-06
1.125000000000000000e+00 3.626516425651871156e-06
1.149999999999999911e+00 3.662164324548305636e-06
1.175000000000000044e+00 3.696853022110277096e-06
1.199999999999999956e+00 3.730776944928715019e-06
1.225000000000000089e+00 3.764123359608620087e-06
1.250000000000000000e+00 3.797066771237455826e-06
1.274999999999999911e+00 3.829764608610482582e-06
1.300000000000000044e+00 3.862354283488923443e-06
1.324999999999999956e+00 3.894951602124952991e-06
1.350000000000000089e+00 3.927650421283162413e-06
1.375000000000000000e+00 3.960523379467704302e-06
1.399999999999999911e+00 3.993623496010775820e-06
1.425000000000000044e+00 4.026986413490241806e-06
1.449999999999999956e+00 4.060633059392408650e-06
1.475000000000000089e+00 4.094572516820035868e-06
1.500000000000000000e+00 4.128804918072026195e-06
1.524999999999999911e+00 4.163324205460391253e-06
1.550000000000000044e+00 4.198120637795288435e-06
1.574999999999999956e+00 4.233182956060645846e-06
1.600000000000000089e+00 4.268500155741672755e-06
1.625000000000000000e+00 4.304062844353098716e-06
1.649999999999999911e+00 4.339864189782357065e-06
1.675000000000000044e+00 4.375900487067329453e-06
1.699999999999999956e+00 4.412171387983962120e-06
1.725000000000000089e+00 4.448679848966397704e-06
1.750000000000000000e+00 4.485431858933632346e-06
1.774999999999999911e+00 4.522436009863706544e-06
1.800000000000000044e+00 4.559702970323995311e-06
1.824999999999999956e+00 4.597244916378692009e-06
1.850000000000000089e+00 4.635074966307262817e-06
1.875000000000000000e+00 4.673206656210550369e-06
1.899999999999999911e+00 4.711653483786305829e-06
1.925000000000000044e+00 4.750428537845814009e-06
1.949999999999999956e+00 4.789544222175189450e-06
1.975000000000000089e+00 4.829012074641881799e-06
2.000000000000000000e+00 4.868842675949305171e-06
2.024999999999999911e+00 4.909045637700856432e-06
2.049999999999999822e+00 4.949629656082667889e-06
2.075000000000000178e+00 4.990602615678350548e-06
2.100000000000000089e+00 5.031971727404880787e-06
2.125000000000000000e+00 5.073743685204100407e-06
2.149999999999999911e+00 5.115924827499502073e-06
2.174999999999999822e+00 5.158521291627601521e-06
2.200000000000000178e+00 5.201539151873570827e-06
2.225000000000000089e+00 5.244984534391898198e-06
2.250000000000000000e+00 5.288863704942846702e-06
2.274999999999999911e+00 5.333183127787400474e-06
2.299999999999999822e+00 5.377949496222951542e-06
2.325000000000000178e+00 5.423169736981257524e-06
2.350000000000000089e+00 5.468850992031626043e-06
2.375000000000000000e+00 5.515000582160395348e-06
2.399999999999999911e+00 5.561625957198031503e-06
2.424999999999999822e+00 5.608734637735455358e-06
2.450000000000000178e+00 5.656334152962828876e-06
2.475000000000000089e+00 5.704431978694173705e-06
2.500000000000000000e+00 5.753035478895046975e-06
2.524999999999999911e+00 5.802151853255077887e-06
2.549999999999999822e+00 5.851788092389384466e-06
2.575000000000000178e+00 5.901950941449496044e-06
2.600000000000000089e+00 5.952646872081981955e-06
2.625000000000000000e+00 6.003882062003817327e-06
2.649999999999999911e+00 6.055662380899514904e-06
2.674999999999999822e+00 6.107993380893590035e-06
2.700000000000000178e+00 6.160880289657636941e-06
2.725000000000000089e+00 6.214328003987298205e-06
2.750000000000000000e+00 6.268341081792612804e-06
2.774999999999999911e+00 6.322923730495101299e-06
2.799999999999999822e+00 6.378079790043738803e-06
2.825000000000000178e+00 6.433812709043208386e-06
2.850000000000000089e+00 6.490125512807016703e-06
2.875000000000000000e+00 6.547020762395468554e-06
2.899999999999999911e+00 6.604500504117649662e-06
2.924999999999999822e+00 6.662566209181697979e-06
2.950000000000000178e+00 6.721218703464957450e-06
2.975000000000000089e+00 6.780458087564042446e-06
3.000000000000000000e+00 6.840283647454193511e-06
3.024999999999999911e+00 6.900693756160813930e-06
3.049999999999999822e+00 6.961685766915248763e-06
3.075000000000000178e+00 7.023255898226969315e-06
3.100000000000000089e+00 7.085399111335767899e-06
3.125000000000000000e+00 7.148108980325855455e-06
3.149999999999999911e+00 7.211377555236663656e-06
3.174999999999999822e+00 7.275195218258388719e-06
3.200000000000000178e+00 7.339550533118528730e-06
3.225000000000000089e+00 7.404430087596097082e-06
3.250000000000000000e+00 7.469818329078076885e-06
3.274999999999999911e+00 7.535697392971850428e-06
3.299999999999999822e+00 7.602046923803167100e-06
3.325000000000000178e+00 7.668843888807312312e-06
3.350000000000000089e+00 7.736062383853666014e-06
3.375000000000000000e+00 7.803673431631442947e-06
3.399999999999999911e+00 7.871644772088808376e-06
3.424999999999999822e+00 7.939940645274338668e-06
3.450000000000000178e+00 8.008521566842553147e-06
3.475000000000000089e+00 8.077344096680833668e-06
3.500000000000000000e+00 8.146360601298495430e-06
3.524999999999999911e+00 8.215519010817831994e-06
3.549999999999999822e+00 8.284762571616208562e-06
3.575000000000000178e+00 8.354029595936942817e-06
3.600000000000000089e+00 8.423253209975797504e-06
3.625000000000000000e+00 8.492361102189517619e-06
3.649999999999999911e+00 8.561275273927596019e-06
3.674999999999999822e+00 8.629911794571501135e-06
3.700000000000000178e+00 8.698180563810557070e-06
3.725000000000000089e+00 8.765985083825913668e-06
3.750000000000000000e+00 8.833222244512264551e-06
3.774999999999999911e+00 8.899782125118072332e-06
3.799999999999999822e+00 8.965547815996009404e-06
3.825000000000000178e+00 9.030395264488172771e-06
3.850000000000000089e+00 9.094193149239730364e-06
3.875000000000000000e+00 9.156802787619114370e-06
3.899999999999999911e+00 9.218078081236211543e-06
3.924999999999999822e+00 9.277865504890006262e-06
3.950000000000000178e+00 9.336004144650314474e-06
3.975000000000000089e+00 9.392325791127793463e-06
4.000000000000000000e+00 9.446655094339862294e-06
4.025000000000000355e+00 9.498809786956169052e-06
4.049999999999999822e+00 9.548600983021659702e-06
4.075000000000000178e+00 9.595833559687683145e-06
4.099999999999999645e+00 9.640306629723561019e-06
4.125000000000000000e+00 9.681814113000420177e-06
4.150000000000000355e+00 9.720145415417292059e-06
4.174999999999999822e+00 9.755086224117219193e-06
4.200000000000000178e+00 9.786419428103919696e-06
4.224999999999999645e+00 9.813926173748597701e-06
4.250000000000000000e+00 9.837387065026901786e-06
4.275000000000000355e+00 9.856583518654113071e-06
4.299999999999999822e+00 9.871299284769548266e-06
4.325000000000000178e+00 9.881322144262650112e-06
4.349999999999999645e+00 9.886445794415237368e-06
4.375000000000000000e+00 9.886471935267523992e-06
4.400000000000000355e+00 9.881212569889203864e-06
4.424999999999999822e+00 9.870492532880296987e-06
4.450000000000000178e+00 9.854152262633386963e-06
4.474999999999999645e+00 9.832050834546190998e-06
4.500000000000000000e+00 9.804069274282900537e-06
4.525000000000000355e+00 9.770114172616113993e-06
4.549999999999999822e+00 9.730121626254535638e-06
4.575000000000000178e+00 9.684061532556215157e-06
4.599999999999999645e+00 9.631942270291703190e-06
4.625000000000000000e+00 9.573815803548763395e-06
4.650000000000000355e+00 9.509783251869389012e-06
4.674999999999999822e+00 9.440000976700373830e-06
4.700000000000000178e+00 9.364687242437207473e-06
4.724999999999999645e+00 9.284129519924327889e-06
4.750000000000000000e+00 9.198692511452020992e-06
4.775000000000000355e+00 9.108826989143907829e-06
4.799999999999999822e+00 9.015079553561676159e-06
4.825000000000000178e+00 8.918103436446811125e-06
4.849999999999999645e+00 8.818670491233561448e-06
4.875000000000000000e+00 8.717684537443539175e-06
4.900000000000000355e+00 8.616196250843619758e-06
4.924999999999999822e+00 8.515419820661262670e-06
4.950000000000000178e+00 8.416751628676204938e-06
4.974999999999999645e+00 8.321791243200236249e-06
5.000000000000000000e+00 8.232365064525862253e-06
5.025000000000000355e+00 8.150553007950367412e-06
5.049999999999999822e+00 8.078718666893478298e-06
5.075000000000000178e+00 8.019543462894863570e-06
5.099999999999999645e+00 7.976065362302192652e-06
5.125000000000000000e+00 7.951722822751921757e-06
5.150000000000000355e+00 7.950404727257625107e-06
5.174999999999999822e+00 7.976507171554569897e-06
5.200000000000000178e+00 8.034998093165270430e-06
5.224999999999999645e+00 8.131490870351941458e-06
5.250000000000000000e+00 8.272328178115465752e-06
5.275000000000000355e+00 8.464677569155207651e-06
5.299999999999999822e+00 8.716640453140583542e-06
5.325000000000000178e+00 9.037376380791918573e-06
5.349999999999999645e+00 9.437244803676622662e-06
5.375000000000000000e+00 9.927966779887335141e-06
5.400000000000000355e+00 1.052280943392939344e-05
5.424999999999999822e+00 1.123679636040163937e-05
5.450000000000000178e+00 1.208694758961497556e-05
5.474999999999999645e+00 1.309255321366316858e-05
5.500000000000000000e+00 1.427548530759208123e-05
5.525000000000000355e+00 1.566055337624155420e-05
5.549999999999999822e+00 1.727590921580114507e-05
5.575000000000000178e+00 1.915350780230177900e-05
5.599999999999999645e+00 2.132963160694626749e-05
5.625000000000000000e+00 2.384548658838220314e-05
5.650000000000000355e+00 2.674787901824461980e-05
5.674999999999999822e+00 3.008998324868891912e-05
5.700000000000000178e+00 3.393221151155628897e-05
5.724999999999999645e+00 3.834319782406564979e-05
5.750000000000000000e+00 4.340090902933669247e-05
5.775000000000000355e+00 4.919389687402567733e-05
5.799999999999999822e+00 5.582270575438903412e-05
5.825000000000000178e+00 6.340145126215225059e-05
5.849999999999999645e+00 7.205958482459757196e-05
5.875000000000000000e+00 8.194385942184999501e-05
5.900000000000000355e+00 9.322051041211988284e-05
5.924999999999999822e+00 1.060776636958014364e-04
5.950000000000000178e+00 1.207279805608178162e-04
5.974999999999999645e+00 1.374115442882690129e-04
6.000000000000000000e+00 1.563989876386554043e-04
6.025000000000000355e+00 1.779948523254017752e-04
6.049999999999999822e+00 2.025411611370681909e-04
6.075000000000000178e+00 2.304211701012414369e-04
6.099999999999999645e+00 2.620632516353107750e-04
6.125000000000000000e+00 2.979448396904708744e-04
6.150000000000000355e+00 3.385963442823760813e-04
6.174999999999999822e+00 3.846049154956003146e-04
6.200000000000000178e+00 4.366179063064233937e-04
6.224999999999999645e+00 4.953458499823907950e-04
6.250000000000000000e+00 5.615647324527816599e-04
6.275000000000000355e+00 6.361173045056280236e-04
6.299999999999999822e+00 7.199131452141581660e-04
6.325000000000000178e+00 8.139271595745932193e-04
6.349999999999999645e+00 9.191961735978745190e-04
6.375000000000000000e+00 1.036813283305771565e-03
6.400000000000000355e+00 1.167919624946724697e-03
6.424999999999999822e+00 1.313693267071955805e-03
6.450000000000000178e+00 1.475334985406596836e-03
6.474999999999999645e+00 1.654050772240552762e-03
6.500000000000000000e+00 1.851031055266060552e-03
6.525000000000000355e+00 2.067426755934826700e-03
6.549999999999999822e+00 2.304322500867509935e-03
6.575000000000000178e+00 2.562707504401924748e-03
6.599999999999999645e+00 2.843444854851292711e-03
6.625000000000000000e+00 3.147240146588531177e-03
6.650000000000000355e+00 3.474610586962521494e-03
6.674999999999999822e+00 3.825855851846530348e-03
6.700000000000000178e+00 4.201032047327631774e-03
6.724999999999999645e+00 4.599930141212649551e-03
6.750000000000000000e+00 5.022060145114380647e-03
6.775000000000000355e+00 5.466642151613310303e-03
6.799999999999999822e+00 5.932605065565188344e-03
6.825000000000000178e+00 6.418593527470541418e-03
6.849999999999999645e+00 6.922983131876544148e-03
6.875000000000000000e+00 7.443903623827027340e-03
6.900000000000000355e+00 7.979269343535421236e-03
6.924999999999999822e+00 8.526815816875399009e-03
6.950000000000000178e+00 9.084141085998956658e-03
6.974999999999999645e+00 9.648750163179328765e-03
7.000000000000000000e+00 1.021810088567150047e-02
7.025000000000000355e+00 1.078964945319233620e-02
7.049999999999999822e+00 1.136089403635549139e-02
7.075000000000000178e+00 1.192941503868675049e-02
7.099999999999999645e+00 1.249291085555314859e-02
7.125000000000000000e+00 1.304922827528647543e-02
7.150000000000000355e+00 1.359638698591449611e-02
7.174999999999999822e+00 1.413259796133174359e-02
7.200000000000000178e+00 1.465627578424140778e-02
7.224999999999999645e+00 1.516604520518402902e-02
7.250000000000000000e+00 1.566074242850051465e-02
7.275000000000000355e+00 1.613941175387613447e-02
7.299999999999999822e+00 1.660129828684622430e-02
7.325000000000000178e+00 1.704583746841571365e-02
7.349999999999999645e+00 1.747264216924202321e-02
7.375000000000000000e+00 1.788148805664039470e-02
7.400000000000000355e+00 1.827229788108904879e-02
7.424999999999999822e+00 1.864512525175903573e-02
7.450000000000000178e+00 1.900013838488753132e-02
7.474999999999999645e+00 1.933760422100810200e-02
7.500000000000000000e+00 1.965787322160726658e-02
7.525000000000000355e+00 1.996136507653397188e-02
7.549999999999999822e+00 2.024855548246192252e-02
7.575000000000000178e+00 2.051996409118309397e-02
7.599999999999999645e+00 2.077614367488024236e-02
7.625000000000000000e+00 2.101767051388565843e-02
7.650000000000000355e+00 2.124513597940922219e-02
7.674999999999999822e+00 2.145913925930280700e-02
7.700000000000000178e+00 2.166028115738602086e-02
7.724999999999999645e+00 2.184915888530385847e-02
7.750000000000000000e+00 2.202636175938275584e-02
7.775000000000000355e+00 2.219246771225638279e-02
7.799999999999999822e+00 2.234804052942293562e-02
7.825000000000000178e+00 2.249362772363637519e-02
7.849999999999999645e+00 2.262975896416962590e-02
7.875000000000000000e+00 2.275694498343874481e-02
7.900000000000000355e+00 2.287567688949646363e-02
7.924999999999999822e+00 2.298642581915191427e-02
7.950000000000000178e+00 2.308964287295068446e-02
7.974999999999999645e+00 2.318575927949062532e-02
8.000000000000000000e+00 2.327518674258610698e-02
8.025000000000000355e+00 2.335831793049822011e-02
8.050000000000000711e+00 2.343552707180802594e-02
8.074999999999999289e+00 2.350717062728551257e-02
8.099999999999999645e+00 2.357358801180233637e-02
8.125000000000000000e+00 2.363510234420298714e-02
8.150000000000000355e+00 2.369202120692290997e-02
8.175000000000000711e+00 2.374463740026628092e-02
8.199999999999999289e+00 2.379322967935249347e-02
8.224999999999999645e+00 2.383806346426767600e-02
8.250000000000000000e+00 2.387939151631241971e-02
8.275000000000000355e+00 2.391745457522923349e-02
8.300000000000000711e+00 2.395248195404663799e-02
8.324999999999999289e+00 2.398469208968434171e-02
8.349999999999999645e+00 2.401429304877187795e-02
8.375000000000000000e+00 2.404148298913459658e-02
8.400000000000000355e+00 2.406645057831360152e-02
8.425000000000000711e+00 2.408937537116508446e-02
8.449999999999999289e+00 2.411042814917547725e-02
8.474999999999999645e+00 2.412977122440155067e-02
8.500000000000000000e+00 2.414755871135420076e-02
8.525000000000000355e+00 2.416393677020737055e-02
8.550000000000000711e+00 2.417904382481648670e-02
8.574999999999999289e+00 2.419301075904059517e-02
8.599999999999999645e+00 2.420596109470529028e-02
8.625000000000000000e+00 2.421801115455015155e-02
8.650000000000000355e+00 2.422927021321213958e-02
8.675000000000000711e+00 2.423984063912237863e-02
8.699999999999999289e+00 2.424981803007635398e-02
8.724999999999999645e+00 2.425929134483719232e-02
8.750000000000000000e+00 2.426834303308740212e-02
8.775000000000000355e+00 2.427704916560495876e-02
8.800000000000000711e+00 2.428547956652226530e-02
8.824999999999999289e+00 2.429369794912247457e-02
8.849999999999999645e+00 2.430176205655940058e-02
8.875000000000000000e+00 2.430972380858890253e-02
8.900000000000000355e+00 2.431762945537082124e-02
8.925000000000000711e+00 2.432551973901950562e-02
8.949999999999999289e+00 2.433343006372560083e-02
8.974999999999999645e+00 2.434139067488635261e-02
9.000000000000000000e+00 2.434942684772879382e-02
9.025000000000000355e+00 2.435755908573753944e-02
9.050000000000000711e+00 2.436580332907367988e-02
9.074999999999999289e+00 2.437417117315380383e-02
9.099999999999999645e+00 2.438267009726740805e-02
9.125000000000000000e+00 2.439130370318125240e-02
9.150000000000000355e+00 2.440007196340597193e-02
9.175000000000000711e+00 2.440897147865973490e-02
9.199999999999999289e+00 2.441799574394451500e-02
9.224999999999999645e+00 2.442713542241795391e-02
9.250000000000000000e+00 2.443637862613231432e-02
9.275000000000000355e+00 2.444571120240290538e-02
9.300000000000000711e+00 2.445511702451431263e-02
9.324999999999999289e+00 2.446457828521868619e-02
9.349999999999999645e+00 2.447407579138710654e-02
9.375000000000000000e+00 2.448358925799895211e-02
9.400000000000000355e+00 2.449309759957526630e-02
9.425000000000000711e+00 2.450257921713422729e-02
9.449999999999999289e+00 2.451201227865591400e-02
9.474999999999999645e+00 2.452137499113439881e-02
9.500000000000000000e+00 2.453064586222603732e-02
9.525000000000000355e+00 2.453980394978012847e-02
9.550000000000000711e+00 2.454882909737131358e-02
9.574999999999999289e+00 2.455770215437253629e-02
9.599999999999999645e+00 2.456640517910199314e-02
9.625000000000000000e+00 2.457492162376247180e-02
9.650000000000000355e+00 2.458323650015078674e-02
9.675000000000000711e+00 2.459133652526020869e-02
9.699999999999999289e+00 2.459921024611001067e-02
9.724999999999999645e+00 2.460684814332428361e-02
9.750000000000000000e+00 2.461424271324548149e-02
9.775000000000000355e+00 2.462138852843897760e-02
9.800000000000000711e+00 2.462828227678258430e-02
9.824999999999999289e+00 2.463492277938441818e-02
9.849999999999999645e+00 2.464131098786670471e-02
9.875000000000000000e+00 2.464744996159249149e-02
9.900000000000000355e+00 2.465334482568259586e-02
9.925000000000000711e+00 2.465900271071825092e-02
9.949999999999999289e+00 2.466443267521956431e-02
9.974999999999999645e+00 2.466964561207560247e-02
1.000000000000000000e+01 2.467465414018863573e-02
1.002500000000000036e+01 2.467947248274215966e-02
1.005000000000000071e+01 2.468411633349863396e-02
1.007499999999999929e+01 2.468860271271210854e-02
1.009999999999999964e+01 2.469294981413359263e-02
1.012500000000000000e+01 2.469717684480047387e-02
1.015000000000000036e+01 2.470130385918641036e-02
1.017500000000000071e+01 2.470535158939832915e-02
1.019999999999999929e+01 2.470934127307072500e-02
1.022499999999999964e+01 2.471329448055642475e-02
1.025000000000000000e+01 2.471723294308322480e-02
1.027500000000000036e+01 2.472117838337234072e-02
1.030000000000000071e+01 2.472515235025389210e-02
1.032499999999999929e+01 2.472917605869618535e-02
1.034999999999999964e+01 2.473327023647389766e-02
1.037500000000000000e+01 2.473745497874509439e-02
1.040000000000000036e+01 2.474174961150046176e-02
1.042500000000000071e+01 2.474617256477201765e-02
1.044999999999999929e+01 2.475074125628968363e-02
1.047499999999999964e+01 2.475547198613571934e-02
1.050000000000000000e+01 2.476037984270585204e-02
1.052500000000000036e+01 2.476547862019073606e-02
1.055000000000000071e+01 2.477078074754766879e-02
1.057499999999999929e+01 2.477629722889332364e-02
1.059999999999999964e+01 2.478203759505925169e-02
1.062500000000000000e+01 2.478800986598559547e-02
1.065000000000000036e+01 2.479422052356109601e-02
1.067500000000000071e+01 2.480067449452973485e-02
1.069999999999999929e+01 2.480737514300777533e-02
1.072499999999999964e+01 2.481432427224377485e-02
1.075000000000000000e+01 2.482152213524577850e-02
1.077500000000000036e+01 2.482896745396353730e-02
1.080000000000000071e+01 2.483665744678233839e-02
1.082499999999999929e+01 2.484458786411447226e-02
1.084999999999999964e+01 2.485275303196188279e-02
1.087500000000000000e+01 2.486114590328867766e-02
1.090000000000000036e+01 2.486975811713504000e-02
1.092500000000000071e+01 2.487858006536037767e-02
1.094999999999999929e+01 2.488760096686109050e-02
1.097499999999999964e+01 2.489680894915025436e-02
1.100000000000000000e+01 2.490619113695287032e-02
1.102500000000000036e+01 2.491573374764543755e-02
1.105000000000000071e+01 2.492542219294880540e-02
1.107499999999999929e+01 2.493524118649827898e-02
1.109999999999999964e+01 2.494517485661254530e-02
1.112500000000000000e+01 2.495520686345326089e-02
1.115000000000000036e+01 2.496532051984334810e-02
1.117500000000000071e+01 2.497549891472660918e-02
1.119999999999999929e+01 2.498572503823870417e-02
1.122499999999999964e+01 2.499598190731505687e-02
1.125000000000000000e+01 2.500625269065101172e-02
1.127500000000000036e+01 2.501652083184742470e-02
1.130000000000000071e+01 2.502677016948353164e-02
1.132499999999999929e+01 2.503698505300597926e-02
1.134999999999999964e+01 2.504715045315005287e-02
1.137500000000000000e+01 2.505725206591677065e-02
1.140000000000000036e+01 2.506727640894396147e-02
1.142500000000000071e+01 2.507721090942096470e-02
1.144999999999999929e+01 2.508704398268099198e-02
1.147499999999999964e+01 2.509676510078125491e-02
1.150000000000000000e+01 2.510636485050359717e-02
1.152500000000000036e+01 2.511583498033856734e-02
1.155000000000000071e+01 2.512516843624037696e-02
1.157499999999999929e+01 2.513435938588595361e-02
1.159999999999999964e+01 2.514340323155275086e-02
1.162500000000000000e+01 2.515229661163820296e-02
1.165000000000000036e+01 2.516103739109072993e-02
1.167500000000000071e+01 2.516962464101188748e-02
1.169999999999999929e+01 2.517805860788972783e-02
1.172499999999999964e+01 2.518634067283866823e-02
1.175000000000000000e+01 2.519447330148856870e-02
1.177500000000000036e+01 2.520245998501215168e-02
1.180000000000000071e+01 2.521030517293302090e-02
1.182499999999999929e+01 2.521801419850316917e-02
1.184999999999999964e+01 2.522559319723724838e-02
1.187500000000000000e+01 2.523304901951420667e-02
1.190000000000000036e+01 2.524038913805745632e-02
1.192500000000000071e+01 2.524762155115689924e-02
1.194999999999999929e+01 2.525475468269584856e-02
1.197499999999999964e+01 2.526179727987743970e-02
1.200000000000000000e+01 2.526875830979527504e-02
1.202500000000000036e+01 2.527564685592801133e-02
1.205000000000000071e+01 2.528247201558605009e-02
1.207499999999999929e+01 2.528924279955865537e-02
1.209999999999999964e+01 2.529596803490593085e-02
1.212500000000000000e+01 2.530265627200050868e-02
1.215000000000000036e+01 2.530931569684598606e-02
1.217500000000000071e+01 2.531595404941242369e-02
1.219999999999999929e+01 2.532257854895953939e-02
1.222499999999999964e+01 2.532919582686381579e-02
1.225000000000000000e+01 2.533581186755491654e-02
1.227500000000000036e+01 2.534243195797192955e-02
1.230000000000000071e+01 2.534906064576996493e-02
1.232499999999999929e+01 2.535570170637884910e-02
1.234999999999999964e+01 2.536235811901364767e-02
1.237500000000000000e+01 2.536903205135009748e-02
1.240000000000000036e+01 2.537572485274491169e-02
1.242500000000000071e+01 2.538243705560210006e-02
1.244999999999999929e+01 2.538916838443394317e-02
1.247499999999999964e+01 2.539591777215091661e-02
1.250000000000000000e+01 2.540268338298936820e-02
1.252500000000000036e+01 2.540946264146253658e-02
1.255000000000000071e+01 2.541625226670089777e-02
1.257499999999999929e+01 2.542304831156277922e-02
1.259999999999999964e+01 2.542984620582820765e-02
1.262500000000000000e+01 2.543664080283363632e-02
1.265000000000000036e+01 2.544342642897039555e-02
1.267500000000000071e+01 2.545019693545445133e-02
1.269999999999999929e+01 2.545694575183840569e-02
1.272499999999999964e+01 2.546366594074126952e-02
1.275000000000000000e+01 2.547035025341910439e-02
1.277500000000000036e+01 2.547699118575018698e-02
1.280000000000000071e+01 2.548358103425330326e-02
1.282499999999999929e+01 2.549011195195140592e-02
1.284999999999999964e+01 2.549657600366793744e-02
1.287500000000000000e+01 2.550296522056546070e-02
1.290000000000000036e+01 2.550927165369671484e-02
1.292500000000000071e+01 2.551548742632427785e-02
1.294999999999999929e+01 2.552160478467189325e-02
1.297499999999999964e+01 2.552761614698942014e-02
1.300000000000000000e+01 2.553351415052228582e-02
1.302500000000000036e+01 2.553929169614397102e-02
1.305000000000000071e+01 2.554494199032975726e-02
1.307499999999999929e+01 2.555045858412600610e-02
1.309999999999999964e+01 2.555583540880638355e-02
1.312500000000000000e+01 2.556106680789614616e-02
1.315000000000000036e+01 2.556614756520968201e-02
1.317500000000000071e+01 2.557107292868943810e-02
1.319999999999999929e+01 2.557583862974626907e-02
1.322499999999999964e+01 2.558044089793514581e-02
1.325000000000000000e+01 2.558487647081236424e-02
1.327500000000000036e+01 2.558914259881852477e-02
1.330000000000000071e+01 2.559323704524328266e-02
1.332499999999999929e+01 2.559715808118518537e-02
1.334999999999999964e+01 2.560090447561477744e-02
1.337500000000000000e+01 2.560447548065418813e-02
1.340000000000000036e+01 2.560787081218911243e-02
1.342500000000000071e+01 2.561109062608598469e-02
1.344999999999999929e+01 2.561413549013693419e-02
1.347499999999999964e+01 2.561700635217769792e-02
1.350000000000000000e+01 2.561970450451757711e-02
1.352500000000000036e+01 2.562223154522999882e-02
1.355000000000000071e+01 2.562458933652179971e-02
1.357499999999999929e+01 2.562677996075189008e-02
1.359999999999999964e+01 2.562880567453170583e-02
1.362500000000000000e+01 2.563066886136151565e-02
1.365000000000000036e+01 2.563237198340370671e-02
1.367500000000000071e+01 2.563391753294342809e-02
1.369999999999999929e+01 2.563530798402048269e-02
1.372499999999999964e+01 2.563654574491817245e-02
1.375000000000000000e+01 2.563763311196257805e-02
1.377500000000000036e+01 2.563857222525135413e-02
1.380000000000000071e+01 2.563936502681777099e-02
1.382499999999999929e+01 2.564001322165262650e-02
1.384999999999999964e+01 2.564051824213956290e-02
1.387500000000000000e+01 2.564088121619629429e-02
1.390000000000000036e+01 2.564110293956113965e-02
1.392500000000000071e+01 2.564118385248075754e-02
1.394999999999999929e+01 2.564112402108531183e-02
1.397499999999999964e+01 2.564092312366464746e-02
1.400000000000000000e+01 2.564058044203090456e-02
1.402500000000000036e+01 2.564009485810581079e-02
1.405000000000000071e+01 2.563946485582041523e-02
1.407499999999999929e+01 2.563868852835179610e-02
1.409999999999999964e+01 2.563776359081820777e-02
1.412500000000000000e+01 2.563668739824342957e-02
1.415000000000000036e+01 2.563545696886827124e-02
1.417500000000000071e+01 2.563406901249468128e-02
1.419999999999999929e+01 2.563251996384217565e-02
1.422499999999999964e+01 2.563080602046789030e-02
1.425000000000000000e+01 2.562892318510313905e-02
1.427500000000000036e+01 2.562686731178786745e-02
1.430000000000000071e+01 2.562463415557043184e-02
1.432499999999999929e+01 2.562221942498248486e-02
1.434999999999999964e+01 2.561961883693138903e-02
1.437500000000000000e+01 2.561682817318404470e-02
1.440000000000000036e+01 2.561384333785453674e-02
1.442500000000000071e+01 2.561066041511892225e-02
1.444999999999999929e+01 2.560727572642170977e-02
1.447499999999999964e+01 2.560368588643687632e-02
1.450000000000000000e+01 2.559988785692790883e-02
1.452500000000000036e+01 2.559587899788214396e-02
1.455000000000000071e+01 2.559165711501748541e-02
1.457499999999999929e+01 2.558722050305097667e-02
1.459999999999999964e+01 2.558256798398898801e-02
1.462500000000000000e+01 2.557769893979320100e-02
1.465000000000000036e+01 2.557261333886802143e-02
1.467500000000000071e+01 2.556731175578568632e-02
1.469999999999999929e+01 2.556179538391588346e-02
1.472499999999999964e+01 2.555606604045275454e-02
1.475000000000000000e+01 2.555012616375228890e-02
1.477500000000000036e+01 2.554397880263729881e-02
1.480000000000000071e+01 2.553762759782023453e-02
1.482499999999999929e+01 2.553107675535602750e-02
1.484999999999999964e+01 2.552433101235654020e-02
1.487500000000000000e+01 2.551739559530100468e-02
1.490000000000000036e+01 2.551027617120206828e-02
1.492500000000000071e+01 2.550297879224280861e-02
1.494999999999999929e+01 2.549550983429011222e-02
1.497499999999999964e+01 2.548787593003901814e-02
1.500000000000000000e+01 2.548008389737500476e-02
1.502500000000000036e+01 2.547214066370457816e-02
1.505000000000000071e+01 2.546405318699678302e-02
1.507499999999999929e+01 2.545582837434254286e-02
1.509999999999999964e+01 2.544747299886166575e-02
1.512500000000000000e+01 2.543899361573617388e-02
1.515000000000000036e+01 2.543039647829670180e-02
1.517500000000000071e+01 2.542168745508036412e-02
1.519999999999999929e+01 2.541287194869107716e-02
1.522499999999999964e+01 2.540395481754285922e-02
1.525000000000000000e+01 2.539494030144200457e-02
1.527500000000000036e+01 2.538583195200610002e-02
1.530000000000000071e+01 2.537663256907012055e-02
1.532499999999999929e+01 2.536734414403225488e-02
1.534999999999999964e+01 2.535796781136905043e-02
1.537500000000000000e+01 2.534850380922959445e-02
1.540000000000000036e+01 2.533895145018574477e-02
1.542500000000000071e+01 2.532930910310297185e-02
1.544999999999999929e+01 2.531957418694251796e-02
1.547499999999999964e+01 2.530974317724619255e-02
1.550000000000000000e+01 2.529981162582123008e-02
1.552500000000000036e+01 2.528977419416100422e-02
1.555000000000000071e+01 2.527962470065264755e-02
1.557499999999999929e+01 2.526935618173262177e-02
1.559999999999999964e+01 2.525896096663312468e-02
1.562500000000000000e+01 2.524843076541989639e-02
1.565000000000000036e+01 2.523775676954137515e-02
1.567500000000000071e+01 2.522692976405497070e-02
1.569999999999999929e+01 2.521594025050414439e-02
1.572499999999999964e+01 2.520477857894754328e-02
1.575000000000000000e+01 2.519343508793324762e-02
1.577500000000000036e+01 2.518190025058647183e-02
1.580000000000000071e+01 2.517016482515260084e-02
1.582499999999999929e+01 2.515822000812416304e-02
1.584999999999999964e+01 2.514605758798585042e-02
1.587500000000000000e+01 2.513367009765067953e-02
1.590000000000000036e+01 2.512105096361877399e-02
1.592500000000000071e+01 2.510819464981373442e-02
1.594999999999999929e+01 2.509509679431631834e-02
1.597499999999999964e+01 2.508175433707870300e-02
1.600000000000000000e+01 2.506816563697127770e-02
1.602499999999999858e+01 2.505433057647769857e-02
1.605000000000000071e+01 2.504025065269869024e-02
1.607499999999999929e+01 2.502592905332713669e-02
1.610000000000000142e+01 2.501137071639169401e-02
1.612500000000000000e+01 2.499658237298347727e-02
1.614999999999999858e+01 2.498157257198853690e-02
1.617500000000000071e+01 2.496635168631412133e-02
1.619999999999999929e+01 2.495093190010143347e-02
1.622500000000000142e+01 2.493532717664130594e-02
1.625000000000000000e+01 2.491955320682024549e-02
1.627499999999999858e+01 2.490362733811047474e-02
1.630000000000000071e+01 2.488756848425561358e-02
1.632499999999999929e+01 2.487139701600367056e-02
1.635000000000000142e+01 2.485513463327341036e-02
1.637500000000000000e+01 2.483880421959209356e-02
1.639999999999999858e+01 2.482242967942733039e-02
1.642500000000000071e+01 2.480603575963855056e-02
1.644999999999999929e+01 2.478964785623342590e-02
1.647500000000000142e+01 2.477329180786800103e-02
1.650000000000000000e+01 2.475699367783683325e-02
1.652499999999999858e+01 2.474077952642638953e-02
1.655000000000000071e+01 2.472467517574773732e-02
1.657499999999999929e+01 2.470870596939924932e-02
1.660000000000000142e+01 2.469289652936590071e-02
1.662500000000000000e+01 2.467727051288726853e-02
1.664999999999999858e+01 2.466185037196333418e-02
1.667500000000000071e+01 2.464665711832528250e-02
1.669999999999999929e+01 2.463171009674172685e-02
1.672500000000000142e+01 2.461702676951170801e-02
1.675000000000000000e+01 2.460262251486798266e-02
1.677499999999999858e+01 2.458851044203127362e-02
1.680000000000000071e+01 2.457470122543059521e-02
1.682499999999999929e+01 2.456120296051732960e-02
1.685000000000000142e+01 2.454802104324819989e-02
1.687500000000000000e+01 2.453515807523561423e-02
1.689999999999999858e+01 2.452261379623168688e-02
1.692500000000000071e+01 2.451038504526661943e-02
1.694999999999999929e+01 2.449846575153775963e-02
1.697500000000000142e+01 2.448684695586179211e-02
1.700000000000000000e+01 2.447551686298238965e-02
1.702499999999999858e+01 2.446446092493074348e-02
1.705000000000000071e+01 2.445366195514580426e-02
1.707499999999999929e+01 2.444310027268626825e-02
1.710000000000000142e+01 2.443275387563073608e-02
1.712500000000000000e+01 2.442259864240227049e-02
1.714999999999999858e+01 2.441260855928679627e-02
1.717500000000000071e+01 2.440275597227934204e-02
1.719999999999999929e+01 2.439301186090611639e-02
1.722500000000000142e+01 2.438334613156557806e-02
1.725000000000000000e+01 2.437372792741783478e-02
1.727499999999999858e+01 2.436412595178059914e-02
1.730000000000000071e+01 2.435450880171605853e-02
1.732499999999999929e+01 2.434484530818289971e-02
1.735000000000000142e+01 2.433510487907184286e-02
1.737500000000000000e+01 2.432525784125102314e-02
1.739999999999999858e+01 2.431527577756011776e-02
1.742500000000000071e+01 2.430513185475358454e-02
1.744999999999999929e+01 2.429480113823983059e-02
1.747500000000000142e+01 2.428426088953667544e-02
1.750000000000000000e+01 2.427349084246692781e-02
1.752499999999999858e+01 2.426247345416483042e-02
1.755000000000000071e+01 2.425119412714908662e-02
1.757499999999999929e+01 2.423964139917589300e-02
1.760000000000000142e+01 2.422780709753543857e-02
1.762500000000000000e+01 2.421568645519888568e-02
1.764999999999999858e+01 2.420327818644092688e-02
1.767500000000000071e+01 2.419058452013450600e-02
1.769999999999999929e+01 2.417761118946584048e-02
1.772500000000000142e+01 2.416436737737721885e-02
1.775000000000000000e+01 2.415086561767812445e-02
1.777499999999999858e+01 2.413712165240174123e-02
1.780000000000000071e+01 2.412315424660449806e-02
1.782499999999999929e+01 2.410898496239952116e-02
1.785000000000000142e+01 2.409463789471201489e-02
1.787500000000000000e+01 2.408013937170767893e-02
1.789999999999999858e+01 2.406551762341831158e-02
1.792500000000000071e+01 2.405080242265772786e-02
1.794999999999999929e+01 2.403602470243930422e-02
1.797500000000000142e+01 2.402121615490528453e-02
1.800000000000000000e+01 2.400640881664227802e-02
1.802499999999999858e+01 2.399163464568006277e-02
1.805000000000000071e+01 2.397692509563347646e-02
1.807499999999999929e+01 2.396231069242221040e-02
1.810000000000000142e+01 2.394782061903942896e-02
1.812500000000000000e+01 2.393348231384339816e-02
1.814999999999999858e+01 2.391932108763113879e-02
1.817500000000000071e+01 2.390535976461270751e-02
1.819999999999999929e+01 2.389161835219049540e-02
1.822500000000000142e+01 2.387811374404294820e-02
1.825000000000000000e+01 2.386485946075728518e-02
1.827499999999999858e+01 2.385186543177578861e-02
1.830000000000000071e+01 2.383913782193312556e-02
1.832499999999999929e+01 2.382667890548388276e-02
1.835000000000000142e+01 2.381448698973636161e-02
1.837500000000000000e+01 2.380255639006091314e-02
1.839999999999999858e+01 2.379087745722099564e-02
1.842500000000000071e+01 2.377943665742328697e-02
1.844999999999999929e+01 2.376821670479272244e-02
1.847500000000000142e+01 2.375719674526696745e-02
1.850000000000000000e+01 2.374635259024555564e-02
1.852499999999999858e+01 2.373565699757284111e-02
1.855000000000000071e+01 2.372507999686753258e-02
1.857499999999999929e+01 2.371458925547504162e-02
1.860000000000000142e+01 2.370415048068027705e-02
1.862500000000000000e+01 2.369372785324462549e-02
1.864999999999999858e+01 2.368328448692282734e-02
1.867500000000000071e+01 2.367278290781998223e-02
1.869999999999999929e+01 2.366218554745123590e-02
1.872500000000000142e+01 2.365145524270968400e-02
1.875000000000000000e+01 2.364055573582544631e-02
1.877499999999999858e+01 2.362945216736721255e-02
1.880000000000000071e+01 2.361811155509140059e-02
1.882499999999999929e+01 2.360650325167382851e-02
1.885000000000000142e+01 2.359459937451414274e-02
1.887500000000000000e+01 2.358237520105152876e-02
1.889999999999999858e+01 2.356980952356732897e-02
1.892500000000000071e+01 2.355688495774187555e-02
1.894999999999999929e+01 2.354358820012327630e-02
1.897500000000000142e+01 2.352991023023408984e-02
1.900000000000000000e+01 2.351584645373339805e-02
1.902499999999999858e+01 2.350139678412760738e-02
1.905000000000000071e+01 2.348656566121706118e-02
1.907499999999999929e+01 2.347136200550383606e-02
1.910000000000000142e+01 2.345579910868292317e-02
1.912500000000000000e+01 2.343989446122367662e-02
1.914999999999999858e+01 2.342366951915423462e-02
1.917500000000000071e+01 2.340714941282797876e-02
1.919999999999999929e+01 2.339036260138587647e-02
1.922500000000000142e+01 2.337334047755671232e-02
1.925000000000000000e+01 2.335611692787279217e-02
1.927499999999999858e+01 2.333872785438678016e-02
1.930000000000000071e+01 2.332121066420169211e-02
1.932499999999999929e+01 2.330360373382064920e-02
1.935000000000000142e+01 2.328594585579809792e-02
1.937500000000000000e+01 2.326827567506363145e-02
1.939999999999999858e+01 2.325063112297134221e-02
1.942500000000000071e+01 2.323304885696992131e-02
1.944999999999999929e+01 2.321556371364920032e-02
1.947500000000000142e+01 2.319820818305346963e-02
1.950000000000000000e+01 2.318101191180585094e-02
1.952499999999999858e+01 2.316400124218363082e-02
1.955000000000000071e+01 2.314719879392043364e-02
1.957499999999999929e+01 2.313062309505353048e-02
1.960000000000000142e+01 2.311428826731866740e-02
1.962500000000000000e+01 2.309820377113978862e-02
1.964999999999999858e+01 2.308237421423173924e-02
1.967500000000000071e+01 2.306679922730119070e-02
1.969999999999999929e+01 2.305147340911736439e-02
1.972500000000000142e+01 2.303638634259424708e-02
1.975000000000000000e+01 2.302152268236964255e-02
1.977499999999999858e+01 2.300686231356007666e-02
1.980000000000000071e+01 2.299238058020425246e-02
1.982499999999999929e+01 2.297804858112996104e-02
1.985000000000000142e+01 2.296383352992013155e-02
1.987500000000000000e+01 2.294969917472871462e-02
1.989999999999999858e+01 2.293560627286255621e-02
1.992500000000000071e+01 2.292151311425300736e-02
1.994999999999999929e+01 2.290737608696842217e-02
1.997500000000000142e+01 2.289315027761365529e-02
2.000000000000000000e+01 2.287879009847454750e-02
2.002499999999999858e+01 2.286424993306641201e-02
2.005000000000000071e+01 2.284948479119426265e-02
2.007499999999999929e+01 2.283445096448017547e-02
2.010000000000000142e+01 2.281910667312276111e-02
2.012500000000000000e+01 2.280341269475618174e-02
2.014999999999999858e+01 2.278733296617036289e-02
2.017500000000000071e+01 2.277083514932410832e-02
2.019999999999999929e+01 2.275389115311817270e-02
2.022500000000000142e+01 2.273647760313972366e-02
2.025000000000000000e+01 2.271857625223700286e-02
2.027499999999999858e+01 2.270017432541866775e-02
2.030000000000000071e+01 2.268126479373010901e-02
2.032499999999999929e+01 2.266184657231357785e-02
2.035000000000000142e+01 2.264192463932347341e-02
2.037500000000000000e+01 2.262151007311670112e-02
2.039999999999999858e+01 2.260062000647261593e-02
2.042500000000000071e+01 2.257927749768282380e-02
2.044999999999999929e+01 2.255751131947066404e-02
2.047500000000000142e+01 2.253535566794533790e-02
2.050000000000000000e+01 2.251284979496600766e-02
2.052499999999999858e+01 2.249003756807598994e-02
2.055000000000000071e+01 2.246696696368762070e-02
2.057499999999999929e+01 2.244368949974861333e-02
2.060000000000000142e+01 2.242025961517540508e-02
2.062500000000000000e+01 2.239673400414726387e-02
2.064999999999999858e+01 2.237317091391849422e-02
2.067500000000000071e+01 2.234962941548411519e-02
2.069999999999999929e+01 2.232616865672302706e-02
2.072500000000000142e+01 2.230284710810670826e-02
2.075000000000000000e+01 2.227972181102593624e-02
2.077499999999999858e+01 2.225684763899403798e-02
2.080000000000000071e+01 2.223427658168239049e-02
2.082499999999999929e+01 2.221205706171088673e-02
2.085000000000000142e+01 2.219023329345858747e-02
2.087500000000000000e+01 2.216884469296192400e-02
2.089999999999999858e+01 2.214792534702555993e-02
2.092500000000000071e+01 2.212750354924274826e-02
2.094999999999999929e+01 2.210760140950319627e-02
2.097500000000000142e+01 2.208823454295068267e-02
2.100000000000000000e+01 2.206941184304254955e-02
2.102499999999999858e+01 2.205113534262081315e-02
2.105000000000000071e+01 2.203340016544910740e-02
2.107499999999999929e+01 2.201619456994983667e-02
2.110000000000000142e+01 2.199950008506522045e-02
2.112500000000000000e+01 2.198329173772724243e-02
2.114999999999999858e+01 2.196753836944744659e-02
2.117500000000000071e+01 2.195220303883231236e-02
2.119999999999999929e+01 2.193724350544693988e-02
2.122500000000000142e+01 2.192261278929747564e-02
2.125000000000000000e+01 2.190825979908333707e-02
2.127499999999999858e+01 2.189413002143424103e-02
2.130000000000000071e+01 2.188016626225186209e-02
2.132499999999999929e+01 2.186630943055513712e-02
2.135000000000000142e+01 2.185249935438556348e-02
2.137500000000000000e+01 2.183867561786162481e-02
2.139999999999999858e+01 2.182477840789845761e-02
2.142500000000000071e+01 2.181074935882424889e-02
2.144999999999999929e+01 2.179653238317886751e-02
2.147500000000000142e+01 2.178207447668526961e-02
2.150000000000000000e+01 2.176732648599268363e-02
2.152499999999999858e+01 2.175224382781133781e-02
2.155000000000000071e+01 2.173678714896345315e-02
2.157499999999999929e+01 2.172092291733949271e-02
2.160000000000000142e+01 2.170462393474955134e-02
2.162500000000000000e+01 2.168786976374726880e-02
2.164999999999999858e+01 2.167064706153249212e-02
2.167500000000000071e+01 2.165294981545038203e-02
2.169999999999999929e+01 2.163477947585781130e-02
2.172500000000000142e+01 2.161614498372590662e-02
2.175000000000000000e+01 2.159706269169553439e-02
2.177499999999999858e+01 2.157755617895291272e-02
2.180000000000000071e+01 2.155765596183179325e-02
2.182499999999999929e+01 2.153739910347492875e-02
2.185000000000000142e+01 2.151682872760311491e-02
2.187500000000000000e+01 2.149599344252938263e-02
2.189999999999999858e+01 2.147494668328861836e-02
2.192500000000000071e+01 2.145374598067308416e-02
2.194999999999999929e+01 2.143245216729457864e-02
2.197500000000000142e+01 2.141112853156915824e-02
2.200000000000000000e+01 2.138983993157111616e-02
2.202499999999999858e+01 2.136865188109419306e-02
2.205000000000000071e+01 2.134762962100392181e-02
2.207499999999999929e+01 2.132683718900302630e-02
2.210000000000000142e+01 2.130633650118445022e-02
2.212500000000000000e+01 2.128618645871123080e-02
2.214999999999999858e+01 2.126644209257894103e-02
2.217500000000000071e+01 2.124715375900511002e-02
2.219999999999999929e+01 2.122836639753961085e-02
2.222500000000000142e+01 2.121011886296325644e-02
2.225000000000000000e+01 2.119244334126172552e-02
2.227499999999999858e+01 2.117536485886062358e-02
2.230000000000000071e+01 2.115890089295032958e-02
2.232499999999999929e+01 2.114306108956446120e-02
2.235000000000000142e+01 2.112784709437342048e-02
2.237500000000000000e+01 2.111325249972407711e-02
2.239999999999999858e+01 2.109926290974986976e-02
2.242500000000000071e+01 2.108585612360755826e-02
2.244999999999999929e+01 2.107300243521025018e-02
2.247500000000000142e+01 2.106066504596580333e-02
2.250000000000000000e+01 2.104880058531029274e-02
2.252499999999999858e+01 2.103735973212355823e-02
2.255000000000000071e+01 2.102628792842837005e-02
2.257499999999999929e+01 2.101552617514033591e-02
2.260000000000000142e+01 2.100501189846332509e-02
2.262500000000000000e+01 2.099467987398138252e-02
2.264999999999999858e+01 2.098446319459393999e-02
2.267500000000000071e+01 2.097429426753096710e-02
2.269999999999999929e+01 2.096410582501345710e-02
2.272500000000000142e+01 2.095383193279127046e-02
2.275000000000000000e+01 2.094340898060500708e-02
2.277499999999999858e+01 2.093277663873843730e-02
2.280000000000000071e+01 2.092187876528997861e-02
2.282499999999999929e+01 2.091066424931548387e-02
2.285000000000000142e+01 2.089908777608738555e-02
2.287500000000000000e+01 2.088711050166980332e-02
2.289999999999999858e+01 2.087470062557387693e-02
2.292500000000000071e+01 2.086183385170763821e-02
2.294999999999999929e+01 2.084849372962703506e-02
2.297500000000000142e+01 2.083467187018802139e-02
2.300000000000000000e+01 2.082036803137151532e-02
2.302499999999999858e+01 2.080559007258781304e-02
2.305000000000000071e+01 2.079035377763387149e-02
2.307499999999999929e+01 2.077468254884656459e-02
2.310000000000000142e+01 2.075860697697759841e-02
2.312500000000000000e+01 2.074216429359974218e-02
2.314999999999999858e+01 2.072539771473946163e-02
2.317500000000000071e+01 2.070835568623421649e-02
2.319999999999999929e+01 2.069109104308302152e-02
2.322500000000000142e+01 2.067366009661498333e-02
2.325000000000000000e+01 2.065612166438152500e-02
2.327499999999999858e+01 2.063853605881227349e-02
2.330000000000000071e+01 2.062096405155275716e-02
2.332499999999999929e+01 2.060346583064787823e-02
2.335000000000000142e+01 2.058609996816404206e-02
2.337500000000000000e+01 2.056892241566195267e-02
2.339999999999999858e+01 2.055198554461639196e-02
2.342500000000000071e+01 2.053533724821108436e-02
2.344999999999999929e+01 2.051902011998832723e-02
2.347500000000000142e+01 2.050307072384632487e-02
2.350000000000000000e+01 2.048751896804376291e-02
2.352499999999999858e+01 2.047238759474812389e-02
2.355000000000000071e+01 2.045769179430529175e-02
2.357499999999999929e+01 2.044343895166361377e-02
2.360000000000000142e+01 2.042962853010405963e-02
2.362500000000000000e+01 2.041625209502563251e-02
2.364999999999999858e+01 2.040329347825997966e-02
2.367500000000000071e+01 2.039072908089340982e-02
2.369999999999999929e+01 2.037852831001374376e-02
2.372500000000000142e+01 2.036665414263814575e-02
2.375000000000000000e+01 2.035506380739678448e-02
2.377499999999999858e+01 2.034370957280153339e-02
2.380000000000000071e+01 2.033253962841053558e-02
2.382499999999999929e+01 2.032149904380915750e-02
2.385000000000000142e+01 2.031053078857593397e-02
2.387500000000000000e+01 2.029957679503791024e-02
2.389999999999999858e+01 2.028857904502530743e-02
2.392500000000000071e+01 2.027748066080501191e-02
2.394999999999999929e+01 2.026622698040366280e-02
2.397500000000000142e+01 2.025476659759088000e-02
2.400000000000000000e+01 2.024305234726433630e-02
2.402499999999999858e+01 2.023104221777032030e-02
2.405000000000000071e+01 2.021870017303304540e-02
2.407499999999999929e+01 2.020599686879654550e-02
2.410000000000000142e+01 2.019291024920302074e-02
2.412500000000000000e+01 2.017942601192583216e-02
2.414999999999999858e+01 2.016553793258914901e-02
2.417500000000000071e+01 2.015124804156993854e-02
2.419999999999999929e+01 2.013656664913942088e-02
2.422500000000000142e+01 2.012151221744230140e-02
2.425000000000000000e+01 2.010611108087716203e-02
2.427499999999999858e+01 2.009039701905514230e-02
2.430000000000000071e+01 2.007441068933893402e-02
2.432499999999999929e+01 2.005819892855312259e-02
2.435000000000000142e+01 2.004181393593808605e-02
2.437500000000000000e+01 2.002531235169104085e-02
2.439999999999999858e+01 2.000875424725564317e-02
2.442500000000000071e+01 1.999220204564628248e-02
2.444999999999999929e+01 1.997571939082627701e-02
2.447500000000000142e+01 1.995936998691041908e-02
2.450000000000000000e+01 1.994321642804471745e-02
2.452499999999999858e+01 1.992731904047381417e-02
2.455000000000000071e+01 1.991173475808797472e-02
2.457499999999999929e+01 1.989651605215149097e-02
2.460000000000000142e+01 1.988170993536471365e-02
2.462500000000000000e+01 1.986735705886695513e-02
2.464999999999999858e+01 1.985349091948999598e-02
2.467500000000000071e+01 1.984013719257659084e-02
2.469999999999999929e+01 1.982731320359261765e-02
2.472500000000000142e+01 1.981502754935046684e-02
2.475000000000000000e+01 1.980327987704099887e-02
2.477499999999999858e+01 1.979206082646448039e-02
2.480000000000000071e+01 1.978135213812437510e-02
2.482499999999999929e+01 1.977112692661018439e-02
2.485000000000000142e+01 1.976135011593543878e-02
2.487500000000000000e+01 1.975197903032168203e-02
2.489999999999999858e+01 1.974296413128453706e-02
2.492500000000000071e+01 1.973424988884899672e-02
2.494999999999999929e+01 1.972577577225861911e-02
2.497500000000000142e+01 1.971747734335167063e-02
2.500000000000000000e+01 1.970928743333348609e-02
2.502499999999999858e+01 1.970113738244061788e-02
2.505000000000000071e+01 1.969295832010492520e-02
2.507499999999999929e+01 1.968468246267257235e-02
2.510000000000000142e+01 1.967624440487363360e-02
2.512500000000000000e+01 1.966758238119536636e-02
2.514999999999999858e+01 1.965863947377237408e-02
2.517500000000000071e+01 1.964936474389239116e-02
2.519999999999999929e+01 1.963971426561221187e-02
2.522500000000000142e+01 1.962965204146700088e-02
2.525000000000000000e+01 1.961915078233951101e-02
2.527499999999999858e+01 1.960819253574922968e-02
2.530000000000000071e+01 1.959676914966560976e-02
2.532499999999999929e+01 1.958488256155179949e-02
2.535000000000000142e+01 1.957254490586138318e-02
2.537500000000000000e+01 1.955977843605268690e-02
2.539999999999999858e+01 1.954661526084770090e-02
2.542500000000000071e+01 1.953309689763962562e-02
2.544999999999999929e+01 1.951927364933209522e-02
2.547500000000000142e+01 1.950520381426889757e-02
2.550000000000000000e+01 1.949095274168244352e-02
2.552499999999999858e+01 1.947659174827594611e-02
2.555000000000000071e+01 1.946219691393915804e-02
2.557499999999999929e+01 1.944784777694938752e-02
2.560000000000000142e+01 1.943362595091667824e-02
2.562500000000000000e+01 1.941961368727235585e-02
2.564999999999999858e+01 1.940589240818554884e-02
2.567500000000000071e+01 1.939254123554307388e-02
2.569999999999999929e+01 1.937963554171734729e-02
2.572500000000000142e+01 1.936724554774432380e-02
2.575000000000000000e+01 1.935543499380764851e-02
2.577499999999999858e+01 1.934425990575595888e-02
2.580000000000000071e+01 1.933376747997276462e-02
2.582499999999999929e+01 1.932399510678740537e-02
2.585000000000000142e+01 1.931496955060692244e-02
2.587500000000000000e+01 1.930670630205597732e-02
2.589999999999999858e+01 1.929920911477005230e-02
2.592500000000000071e+01 1.929246973617964461e-02
2.594999999999999929e+01 1.928646783856594743e-02
2.597500000000000142e+01 1.928117115304122178e-02
2.600000000000000000e+01 1.927653580577468745e-02
2.602499999999999858e+01 1.927250685229034421e-02
2.605000000000000071e+01 1.926901900216464181e-02
2.607499999999999929e+01 1.926599752325478324e-02
2.610000000000000142e+01 1.926335931126196485e-02
2.612500000000000000e+01 1.926101410767278727e-02
2.614999999999999858e+01 1.925886584626549172e-02
2.617500000000000071e+01 1.925681410618850301e-02
2.619999999999999929e+01 1.925475564745937965e-02
2.622500000000000142e+01 1.925258600324939243e-02
2.625000000000000000e+01 1.925020110214515495e-02
2.627499999999999858e+01 1.924749889277115392e-02
2.630000000000000071e+01 1.924438094306760508e-02
2.632499999999999929e+01 1.924075398673671597e-02
2.635000000000000142e+01 1.923653139003508178e-02
2.637500000000000000e+01 1.923163451344061040e-02
2.639999999999999858e+01 1.922599394432780717e-02
2.642500000000000071e+01 1.921955057898968103e-02
2.644999999999999929e+01 1.921225653470776465e-02
2.647500000000000142e+01 1.920407587567500227e-02
2.650000000000000000e+01 1.919498513963563760e-02
2.652499999999999858e+01 1.918497365549908656e-02
2.655000000000000071e+01 1.917404364591841076e-02
2.657499999999999929e+01 1.916221011250960823e-02
2.660000000000000142e+01 1.914950050527297609e-02
2.662500000000000000e+01 1.913595418138085841e-02
2.664999999999999858e+01 1.912162166243615985e-02
2.667500000000000071e+01 1.910656370278985411e-02
2.669999999999999929e+01 1.909085018474728987e-02
2.672500000000000142e+01 1.907455885971279935e-02
2.675000000000000000e+01 1.905777395708863065e-02
2.677499999999999858e+01 1.904058468497215120e-02
2.680000000000000071e+01 1.902308364882689812e-02
2.682499999999999929e+01 1.900536521567904730e-02
2.685000000000000142e+01 1.898752385249308922e-02
2.687500000000000000e+01 1.896965246780621048e-02
2.689999999999999858e+01 1.895184078577986583e-02
2.692500000000000071e+01 1.893417378124535172e-02
2.694999999999999929e+01 1.891673020339987993e-02
2.697500000000000142e+01 1.889958121397577670e-02
2.700000000000000000e+01 1.888278916424735235e-02
2.702499999999999858e+01 1.886640653219117877e-02
2.705000000000000071e+01 1.885047503881654568e-02
2.707499999999999929e+01 1.883502495914512326e-02
2.710000000000000142e+01 1.882007464018823836e-02
2.712500000000000000e+01 1.880563023416163826e-02
2.714999999999999858e+01 1.879168565191543031e-02
2.717500000000000071e+01 1.877822273694989413e-02
2.719999999999999929e+01 1.876521165699640906e-02
2.722500000000000142e+01 1.875261150583615311e-02
2.725000000000000000e+01 1.874037110439537301e-02
2.727499999999999858e+01 1.872842998649045312e-02
2.730000000000000071e+01 1.871671955115226244e-02
2.732499999999999929e+01 1.870516436050979892e-02
2.735000000000000142e+01 1.869368355934969764e-02
2.737500000000000000e+01 1.868219239038144963e-02
2.739999999999999858e+01 1.867060377718994177e-02
2.742500000000000071e+01 1.865882994579163429e-02
2.744999999999999929e+01 1.864678405468913128e-02
2.747500000000000142e+01 1.863438180332909269e-02
2.750000000000000000e+01 1.862154298899713900e-02
2.752499999999999858e+01 1.860819298323948545e-02
2.755000000000000071e+01 1.859426410027468016e-02
2.757499999999999929e+01 1.857969683185509147e-02
2.760000000000000142e+01 1.856444092558349676e-02
2.762500000000000000e+01 1.854845628652869796e-02
2.764999999999999858e+01 1.853171368538896607e-02
2.767500000000000071e+01 1.851419526008400013e-02
2.769999999999999929e+01 1.849589480168840547e-02
2.772500000000000142e+01 1.847681781969792020e-02
2.775000000000000000e+01 1.845698138570162766e-02
2.777499999999999858e+01 1.843641375929820289e-02
2.780000000000000071e+01 1.841515380365427168e-02
2.782499999999999929e+01 1.839325020282586362e-02
2.785000000000000142e+01 1.837076049643138892e-02
2.787500000000000000e+01 1.834774995080633297e-02
2.789999999999999858e+01 1.832429028949790137e-02
2.792500000000000071e+01 1.830045830784939737e-02
2.794999999999999929e+01 1.827633439967840320e-02
2.797500000000000142e+01 1.825200102483361120e-02
2.800000000000000000e+01 1.822754114848977244e-02
2.802499999999999858e+01 1.820303668284567120e-02
2.805000000000000071e+01 1.817856696224762897e-02
2.807499999999999929e+01 1.815420728209327442e-02
2.810000000000000142e+01 1.813002753035696316e-02
2.812500000000000000e+01 1.810609093888842364e-02
2.814999999999999858e+01 1.808245297926933115e-02
2.817500000000000071e+01 1.805916042484389389e-02
2.819999999999999929e+01 1.803625059747274509e-02
2.822500000000000142e+01 1.801375081350104282e-02
2.825000000000000000e+01 1.799167803966893187e-02
2.827499999999999858e+01 1.797003876505489164e-02
2.830000000000000071e+01 1.794882909101285992e-02
2.832499999999999929e+01 1.792803503639005952e-02
2.835000000000000142e+01 1.790763305088229063e-02
2.837500000000000000e+01 1.788759072520578267e-02
2.839999999999999858e+01 1.786786768248407528e-02
2.842500000000000071e+01 1.784841663171957485e-02
2.844999999999999929e+01 1.782918456050234038e-02
2.847500000000000142e+01 1.781011404169829601e-02
2.850000000000000000e+01 1.779114462596337667e-02
2.852499999999999858e+01 1.777221429045032150e-02
2.855000000000000071e+01 1.775326091285853863e-02
2.857499999999999929e+01 1.773422373931008572e-02
2.860000000000000142e+01 1.771504481486143104e-02
2.862500000000000000e+01 1.769567034604431488e-02
2.864999999999999858e+01 1.767605196651271526e-02
2.867500000000000071e+01 1.765614787851985112e-02
2.869999999999999929e+01 1.763592384620487827e-02
2.872500000000000142e+01 1.761535401911580243e-02
2.875000000000000000e+01 1.759442156857876985e-02
2.877499999999999858e+01 1.757311912309894239e-02
2.880000000000000071e+01 1.755144899361631150e-02
2.882499999999999929e+01 1.752942318365200564e-02
2.885000000000000142e+01 1.750706318417775617e-02
2.887500000000000000e+01 1.748439955756969644e-02
2.889999999999999858e+01 1.746147131965258142e-02
2.892500000000000071e+01 1.743832513310018817e-02
2.894999999999999929e+01 1.741501432962929211e-02
2.897500000000000142e+01 1.739159778213669391e-02
2.900000000000000000e+01 1.736813865124155198e-02
2.902499999999999858e+01 1.734470303333788405e-02
2.905000000000000071e+01 1.732135853977760653e-02
2.907499999999999929e+01 1.729817283797027705e-02
2.910000000000000142e+01 1.727521218649606671e-02
2.912500000000000000e+01 1.725253999631523941e-02
2.914999999999999858e+01 1.723021544982290473e-02
2.917500000000000071e+01 1.720829220846645594e-02
2.919999999999999929e+01 1.718681723766818034e-02
2.922500000000000142e+01 1.716582977553489028e-02
2.925000000000000000e+01 1.714536046893699178e-02
2.927499999999999858e+01 1.712543069684640748e-02
2.930000000000000071e+01 1.710605209705664750e-02
2.932499999999999929e+01 1.708722630808168913e-02
2.935000000000000142e+01 1.706894493334025487e-02
2.937500000000000000e+01 1.705118973014092820e-02
2.939999999999999858e+01 1.703393302090839417e-02
2.942500000000000071e+01 1.701713831945410316e-02
2.944999999999999929e+01 1.700076116030382542e-02
2.947500000000000142e+01 1.698475011435191320e-02
2.950000000000000000e+01 1.696904797040872312e-02
2.952499999999999858e+01 1.695359305789603810e-02
2.955000000000000071e+01 1.693832068299079216e-02
2.957499999999999929e+01 1.692316464765643291e-02
2.960000000000000142e+01 1.690805881902540317e-02
2.962500000000000000e+01 1.689293871502522623e-02
2.964999999999999858e+01 1.687774307166677729e-02
2.967500000000000071e+01 1.686241535733456684e-02
2.969999999999999929e+01 1.684690520037122538e-02
2.972500000000000142e+01 1.683116969766240353e-02
2.975000000000000000e+01 1.681517457436286847e-02
2.977499999999999858e+01 1.679889516774201244e-02
2.980000000000000071e+01 1.678231721165615586e-02
2.982499999999999929e+01 1.676543740228178536e-02
2.985000000000000142e+01 1.674826373021413503e-02
2.987500000000000000e+01 1.673081556896207495e-02
2.989999999999999858e+01 1.671312351480868166e-02
2.992500000000000071e+01 1.669522897848622900e-02
2.994999999999999929e+01 1.667718353413670876e-02
2.997500000000000142e+01 1.665904803642641319e-02
3.000000000000000000e+01 1.664089152141648431e-02
3.002499999999999858e+01 1.662278991168268660e-02
3.005000000000000071e+01 1.660482455033891053e-02
3.007499999999999929e+01 1.658708059235972782e-02
3.010000000000000142e+01 1.656964528486033067e-02
3.012500000000000000e+01 1.655260617063188805e-02
3.014999999999999858e+01 1.653604925079939919e-02
3.017500000000000071e+01 1.652005714400732461e-02
3.019999999999999929e+01 1.650470727961851625e-02
3.022500000000000142e+01 1.649007016208313067e-02
3.025000000000000000e+01 1.647620774254592718e-02
3.027499999999999858e+01 1.646317193186173575e-02
3.030000000000000071e+01 1.645100328647694510e-02
3.032499999999999929e+01 1.643972989549384936e-02
3.035000000000000142e+01 1.642936649347629882e-02
3.037500000000000000e+01 1.641991381891728594e-02
3.039999999999999858e+01 1.641135823396912621e-02
3.042500000000000071e+01 1.640367161541545285e-02
3.044999999999999929e+01 1.639681152204181327e-02
3.047500000000000142e+01 1.639072163767004223e-02
3.050000000000000000e+01 1.638533248396706679e-02
3.052499999999999858e+01 1.638056239146714124e-02
3.055000000000000071e+01 1.637631871236586503e-02
3.057499999999999929e+01 1.637249925346484156e-02
3.060000000000000142e+01 1.636899390321358561e-02
3.062500000000000000e+01 1.636568642297432305e-02
3.064999999999999858e+01 1.636245636888343513e-02
3.067500000000000071e+01 1.635918110813404727e-02
3.069999999999999929e+01 1.635573789134407707e-02
3.072500000000000142e+01 1.635200594119261275e-02
3.075000000000000000e+01 1.634786851711437020e-02
3.077499999999999858e+01 1.634321491609974294e-02
3.080000000000000071e+01 1.633794237045902917e-02
3.082499999999999929e+01 1.633195780553397397e-02
3.085000000000000142e+01 1.632517942265424016e-02
3.087500000000000000e+01 1.631753807621639379e-02
3.089999999999999858e+01 1.630897841737109777e-02
3.092500000000000071e+01 1.629945978166217821e-02
3.094999999999999929e+01 1.628895680259198966e-02
3.097500000000000142e+01 1.627745973883951147e-02
3.100000000000000000e+01 1.626497450802657369e-02
3.102499999999999858e+01 1.625152242623402812e-02
3.105000000000000071e+01 1.623713965777346876e-02
3.107499999999999929e+01 1.622187638587991101e-02
3.110000000000000142e+01 1.620579572010813832e-02
3.112500000000000000e+01 1.618897236193675773e-02
3.114999999999999858e+01 1.617149105417303376e-02
3.117500000000000071e+01 1.615344484476254061e-02
3.119999999999999929e+01 1.613493319861421249e-02
3.122500000000000142e+01 1.611605999448270654e-02
3.125000000000000000e+01 1.609693144589564515e-02
3.127499999999999858e+01 1.607765398673319593e-02
3.130000000000000071e+01 1.605833216257865348e-02
3.132499999999999929e+01 1.603906656908016259e-02
3.135000000000000142e+01 1.601995187703641327e-02
3.137500000000000000e+01 1.600107498289817312e-02
3.139999999999999858e+01 1.598251332008220754e-02
3.142500000000000071e+01 1.596433336393861604e-02
3.144999999999999929e+01 1.594658935884249104e-02
3.147500000000000142e+01 1.592932229183807030e-02
3.150000000000000000e+01 1.591255913213026132e-02
3.152499999999999858e+01 1.589631235056663747e-02
3.155000000000000071e+01 1.588057972759983338e-02
3.157499999999999929e+01 1.586534445279576960e-02
3.160000000000000142e+01 1.585057551272044118e-02
3.162500000000000000e+01 1.583622835882740262e-02
3.164999999999999858e+01 1.582224584105942911e-02
3.167500000000000071e+01 1.580855938800539151e-02
3.169999999999999929e+01 1.579509040912769474e-02
3.172500000000000142e+01 1.578175189064305595e-02
3.175000000000000000e+01 1.576845015260348523e-02
3.177499999999999858e+01 1.575508673159214251e-02
3.180000000000000071e+01 1.574156035098579418e-02
3.182499999999999929e+01 1.572776893922388924e-02
3.185000000000000142e+01 1.571361165529413378e-02
3.187500000000000000e+01 1.569899088099314943e-02
3.189999999999999858e+01 1.568381413996587648e-02
3.192500000000000071e+01 1.566799590516795501e-02
3.194999999999999929e+01 1.565145925893163606e-02
3.197500000000000142e+01 1.563413737256842337e-02
3.200000000000000000e+01 1.561597477661797601e-02
3.202499999999999858e+01 1.559692839677793985e-02
3.204999999999999716e+01 1.557696833580023837e-02
3.207500000000000284e+01 1.555607838682051755e-02
3.210000000000000142e+01 1.553425626901244749e-02
3.212500000000000000e+01 1.551151358263710284e-02
3.214999999999999858e+01 1.548787548592050686e-02
3.217499999999999716e+01 1.546338010251911352e-02
3.220000000000000284e+01 1.543807767353162427e-02
3.222500000000000142e+01 1.541202947356310957e-02
3.225000000000000000e+01 1.538530651515179065e-02
3.227499999999999858e+01 1.535798807030387680e-02
3.229999999999999716e+01 1.533016004154457573e-02
3.232500000000000284e+01 1.530191321811096711e-02
3.235000000000000142e+01 1.527334145492780307e-02
3.237500000000000000e+01 1.524453981375795825e-02
3.239999999999999858e+01 1.521560270629709194e-02
3.242499999999999716e+01 1.518662207892045729e-02
3.245000000000000284e+01 1.515768567749789866e-02
3.247500000000000142e+01 1.512887542888571857e-02
3.250000000000000000e+01 1.510026597294101755e-02
3.252499999999999858e+01 1.507192337541434293e-02
3.254999999999999716e+01 1.504390404792114999e-02
3.257500000000000284e+01 1.501625389674826963e-02
3.260000000000000142e+01 1.498900771682280461e-02
3.262500000000000000e+01 1.496218884180989389e-02
3.264999999999999858e+01 1.493580905584581189e-02
3.267499999999999716e+01 1.490986876611256895e-02
3.270000000000000284e+01 1.488435743015344842e-02
3.272500000000000142e+01 1.485925422601903217e-02
3.275000000000000000e+01 1.483452894783751641e-02
3.277499999999999858e+01 1.481014310481653176e-02
3.279999999999999716e+01 1.478605119684579805e-02
3.282500000000000284e+01 1.476220213631927736e-02
3.285000000000000142e+01 1.473854078246982988e-02
3.287500000000000000e+01 1.471500955223375290e-02
3.289999999999999858e+01 1.469155007001820071e-02
3.292499999999999716e+01 1.466810481805998212e-02
3.295000000000000284e+01 1.464461874919364276e-02
3.297500000000000142e+01 1.462104082490502631e-02
3.300000000000000000e+01 1.459732544336647675e-02
3.302499999999999858e+01 1.457343372476903667e-02
3.304999999999999716e+01 1.454933462485691331e-02
3.307500000000000284e+01 1.452500585139059573e-02
3.310000000000000142e+01 1.450043456301790094e-02
3.312500000000000000e+01 1.447561783522945296e-02
3.314999999999999858e+01 1.445056288324483330e-02
3.317499999999999716e+01 1.442528703768404183e-02
3.320000000000000284e+01 1.439981747424109136e-02
3.322500000000000142e+01 1.437419070468643097e-02
3.325000000000000000e+01 1.434845184166790889e-02
3.327499999999999858e+01 1.432265365530774673e-02
3.329999999999999716e+01 1.429685544419893525e-02
3.332500000000000284e+01 1.427112174781426483e-02
3.335000000000000142e+01 1.424552093088234973e-02
3.337500000000000000e+01 1.422012367328475725e-02
3.339999999999999858e+01 1.419500140113485152e-02
3.342499999999999716e+01 1.417022469607094257e-02
3.345000000000000284e+01 1.414586172023038838e-02
3.347500000000000142e+01 1.412197669391445157e-02
3.350000000000000000e+01 1.409862846184859950e-02
3.352499999999999858e+01 1.407586918161689293e-02
3.354999999999999716e+01 1.405374316528451052e-02
3.357500000000000284e+01 1.403228590116749330e-02
3.360000000000000142e+01 1.401152327910907755e-02
3.362500000000000000e+01 1.399147103717847396e-02
3.364999999999999858e+01 1.397213444310547309e-02
3.367499999999999716e+01 1.395350821787212300e-02
3.370000000000000284e+01 1.393557670341563984e-02
3.372500000000000142e+01 1.391831427057740357e-02
3.375000000000000000e+01 1.390168595794694269e-02
3.377499999999999858e+01 1.388564832649606673e-02
3.379999999999999716e+01 1.387015051024204429e-02
3.382500000000000284e+01 1.385513543822103691e-02
3.385000000000000142e+01 1.384054119914095507e-02
3.387500000000000000e+01 1.382630251658486902e-02
3.389999999999999858e+01 1.381235230010930196e-02
3.392499999999999716e+01 1.379862323556411453e-02
3.395000000000000284e+01 1.378504937694810181e-02
3.397500000000000142e+01 1.377156770210836816e-02
3.400000000000000000e+01 1.375811959514257506e-02
3.402499999999999858e+01 1.374465221997992100e-02
3.404999999999999716e+01 1.373111975215152158e-02
3.407500000000000284e+01 1.371748443875381827e-02
3.410000000000000142e+01 1.370371746064637415e-02
3.412500000000000000e+01 1.368979957533954066e-02
3.414999999999999858e+01 1.367572152419447781e-02
3.417499999999999716e+01 1.366148419273237671e-02
3.420000000000000284e+01 1.364709851885745166e-02
3.422500000000000142e+01 1.363258514939450113e-02
3.425000000000000000e+01 1.361797385139566254e-02
3.427499999999999858e+01 1.360330269032856282e-02
3.429999999999999716e+01 1.358861699275814433e-02
3.432500000000000284e+01 1.357396811626671203e-02
3.435000000000000142e+01 1.355941205399582850e-02
3.437500000000000000e+01 1.354500790511396942e-02
3.439999999999999858e+01 1.353081624594147485e-02
3.442499999999999716e+01 1.351689743880910878e-02
3.445000000000000284e+01 1.350330991763998290e-02
3.447500000000000142e+01 1.349010848977814751e-02
3.450000000000000000e+01 1.347734269376041523e-02
3.452499999999999858e+01 1.346505525144923825e-02
3.454999999999999716e+01 1.345328065137269338e-02
3.457500000000000284e+01 1.344204389707654747e-02
3.460000000000000142e+01 1.343135945106661670e-02
3.462500000000000000e+01 1.342123040055618818e-02
3.464999999999999858e+01 1.341164786646494023e-02
3.467499999999999716e+01 1.340259067186735584e-02
3.470000000000000284e+01 1.339402528024114664e-02
3.472500000000000142e+01 1.338590600808981478e-02
3.475000000000000000e+01 1.337817551020609695e-02
3.477499999999999858e+01 1.337076553001715143e-02
3.479999999999999716e+01 1.336359790121196059e-02
3.482500000000000284e+01 1.335658578146319171e-02
3.485000000000000142e+01 1.334963509354596643e-02
3.487500000000000000e+01 1.334264614464900425e-02
3.489999999999999858e+01 1.333551539022105256e-02
3.492499999999999716e+01 1.332813730570834437e-02
3.495000000000000284e+01 1.332040632654185076e-02
3.497500000000000142e+01 1.331221881525473591e-02
3.500000000000000000e+01 1.330347501368430753e-02
3.502499999999999858e+01 1.329408093824076091e-02
3.504999999999999716e+01 1.328395017738307100e-02
3.507500000000000284e+01 1.327300555219607971e-02
3.510000000000000142e+01 1.326118060391786799e-02
3.512500000000000000e+01 1.324842087564765195e-02
3.514999999999999858e+01 1.323468496009511455e-02
3.517499999999999716e+01 1.321994528976808310e-02
3.520000000000000284e+01 1.320418865197184669e-02
3.522500000000000142e+01 1.318741641632425673e-02
3.525000000000000000e+01 1.316964446931305661e-02
3.527499999999999858e+01 1.315090285595601517e-02
3.529999999999999716e+01 1.313123513566611528e-02
3.532500000000000284e+01 1.311069746504370942e-02
3.535000000000000142e+01 1.308935742650518992e-02
3.537500000000000000e+01 1.306729262709943086e-02
3.539999999999999858e+01 1.304458909663805052e-02
3.542499999999999716e+01 1.302133951898968439e-02
3.545000000000000284e+01 1.299764133349133773e-02
3.547500000000000142e+01 1.297359474674627741e-02
3.550000000000000000e+01 1.294930069657059500e-02
3.552499999999999858e+01 1.292485881116311672e-02
3.554999999999999716e+01 1.290036540647465993e-02
3.557500000000000284e+01 1.287591156402657748e-02
3.560000000000000142e+01 1.285158132960547740e-02
3.562500000000000000e+01 1.282745007045375113e-02
3.564999999999999858e+01 1.280358302551523782e-02
3.567499999999999716e+01 1.278003407848464154e-02
3.570000000000000284e+01 1.275684477914539386e-02
3.572500000000000142e+01 1.273404363256964468e-02
3.575000000000000000e+01 1.271164567022586497e-02
3.577499999999999858e+01 1.268965231076232998e-02
3.579999999999999716e+01 1.266805151197032166e-02
3.582500000000000284e+01 1.264681820914599630e-02
3.585000000000000142e+01 1.262591502882345375e-02
3.587500000000000000e+01 1.260529326095576053e-02
3.589999999999999858e+01 1.258489406707002413e-02
3.592499999999999716e+01 1.256464989684219838e-02
3.595000000000000284e+01 1.254448608117215806e-02
3.597500000000000142e+01 1.252432256612519626e-02
3.600000000000000000e+01 1.250407574913746898e-02
3.602499999999999858e+01 1.248366037695441098e-02
3.604999999999999716e+01 1.246299146353039428e-02
3.607500000000000284e+01 1.244198618598852009e-02
3.610000000000000142e+01 1.242056571736488771e-02
3.612500000000000000e+01 1.239865695657643707e-02
3.614999999999999858e+01 1.237619411848721492e-02
3.617499999999999716e+01 1.235312015040694079e-02
3.620000000000000284e+01 1.232938794513679243e-02
3.622500000000000142e+01 1.230496132567374806e-02
3.625000000000000000e+01 1.227981578181268005e-02
3.627499999999999858e+01 1.225393894444916885e-02
3.629999999999999716e+01 1.222733078944886718e-02
3.632500000000000284e+01 1.220000356910072928e-02
3.635000000000000142e+01 1.217198147509777895e-02
3.637500000000000000e+01 1.214330004315368645e-02
3.639999999999999858e+01 1.211400531508024028e-02
3.642499999999999716e+01 1.208415277938033536e-02
3.645000000000000284e+01 1.205380611641370531e-02
3.647500000000000142e+01 1.202303577831745005e-02
3.650000000000000000e+01 1.199191743738403995e-02
3.652499999999999858e+01 1.196053033946322709e-02
3.654999999999999716e+01 1.192895560060384247e-02
3.657500000000000284e+01 1.189727448651443573e-02
3.660000000000000142e+01 1.186556671416090156e-02
3.662500000000000000e+01 1.183390881432760423e-02
3.664999999999999858e+01 1.180237259218523682e-02
3.667499999999999716e+01 1.177102372039435553e-02
3.670000000000000284e+01 1.173992049607600584e-02
3.672500000000000142e+01 1.170911278893562585e-02
3.675000000000000000e+01 1.167864120328531569e-02
3.677499999999999858e+01 1.164853647161173331e-02
3.679999999999999716e+01 1.161881909184534108e-02
3.682500000000000284e+01 1.158949921478112308e-02
3.685000000000000142e+01 1.156057678220509186e-02
3.687500000000000000e+01 1.153204191042317764e-02
3.689999999999999858e+01 1.150387550818506022e-02
3.692499999999999716e+01 1.147605011248805512e-02
3.695000000000000284e+01 1.144853092074341223e-02
3.697500000000000142e+01 1.142127699321237623e-02
3.700000000000000000e+01 1.139424259567486768e-02
3.702499999999999858e+01 1.136737864912659192e-02
3.704999999999999716e+01 1.134063425083489327e-02
3.707500000000000284e+01 1.131395822952913598e-02
3.710000000000000142e+01 1.128730069679509569e-02
3.712500000000000000e+01 1.126061455693937174e-02
3.714999999999999858e+01 1.123385693875405232e-02
3.717499999999999716e+01 1.120699051445262634e-02
3.720000000000000284e+01 1.117998467404015747e-02
3.722500000000000142e+01 1.115281652668928883e-02
3.725000000000000000e+01 1.112547170516173071e-02
3.727499999999999858e+01 1.109794495394861996e-02
3.729999999999999716e+01 1.107024048704903507e-02
3.732500000000000284e+01 1.104237210698786123e-02
3.735000000000000142e+01 1.101436308233337026e-02
3.737500000000000000e+01 1.098624578679524913e-02
3.739999999999999858e+01 1.095806110884401698e-02
3.742499999999999716e+01 1.092985764612583975e-02
3.745000000000000284e+01 1.090169070420141280e-02
3.747500000000000142e+01 1.087362112385547100e-02
3.750000000000000000e+01 1.084571396518769354e-02
3.752499999999999858e+01 1.081803708023991277e-02
3.754999999999999716e+01 1.079065960842017467e-02
3.757500000000000284e+01 1.076365043092002323e-02
3.760000000000000142e+01 1.073707662116483302e-02
3.762500000000000000e+01 1.071100192844245090e-02
3.764999999999999858e+01 1.068548533101638942e-02
3.767499999999999716e+01 1.066057969332060047e-02
3.770000000000000284e+01 1.063633055930645645e-02
3.772500000000000142e+01 1.061277511075009103e-02
3.775000000000000000e+01 1.058994131535251695e-02
3.777499999999999858e+01 1.056784728491794167e-02
3.779999999999999716e+01 1.054650085889405862e-02
3.782500000000000284e+01 1.052589942316668921e-02
3.785000000000000142e+01 1.050602996842858565e-02
3.787500000000000000e+01 1.048686938672653036e-02
3.789999999999999858e+01 1.046838499919346480e-02
3.792499999999999716e+01 1.045053530245064824e-02
3.795000000000000284e+01 1.043327091604467904e-02
3.797500000000000142e+01 1.041653570853325224e-02
3.800000000000000000e+01 1.040026807561847573e-02
3.802499999999999858e+01 1.038440234016796242e-02
3.804999999999999716e+01 1.036887024110840286e-02
3.807500000000000284e+01 1.035360247607101242e-02
3.810000000000000142e+01 1.033853026143606406e-02
3.812500000000000000e+01 1.032358687302590942e-02
3.814999999999999858e+01 1.030870913112245067e-02
3.817499999999999716e+01 1.029383879482492770e-02
3.820000000000000284e+01 1.027892383289777910e-02
3.822500000000000142e+01 1.026391954109554146e-02
3.825000000000000000e+01 1.024878947964688418e-02
3.827499999999999858e+01 1.023350620868773188e-02
3.829999999999999716e+01 1.021805180424626043e-02
3.832500000000000284e+01 1.020241814247407550e-02
3.835000000000000142e+01 1.018660694526495394e-02
3.837500000000000000e+01 1.017062958598657918e-02
3.839999999999999858e+01 1.015450665966601690e-02
3.842499999999999716e+01 1.013826732748119089e-02
3.845000000000000284e+01 1.012194845071113375e-02
3.847500000000000142e+01 1.010559353422471086e-02
3.850000000000000000e+01 1.008925150404217132e-02
3.852499999999999858e+01 1.007297534741705310e-02
3.854999999999999716e+01 1.005682064709088885e-02
3.857500000000000284e+01 1.004084404389391470e-02
3.860000000000000142e+01 1.002510166355010182e-02
3.862500000000000000e+01 1.000964754439805961e-02
3.864999999999999858e+01 9.994532102788037178e-03
3.867499999999999716e+01 9.979800672049725374e-03
3.870000000000000284e+01 9.965492149278052866e-03
3.872500000000000142e+01 9.951637781731142648e-03
3.875000000000000000e+01 9.938260121433598396e-03
3.877499999999999858e+01 9.925372172752178895e-03
3.879999999999999716e+01 9.912976753287018258e-03
3.882500000000000284e+01 9.901066083556842035e-03
3.885000000000000142e+01 9.889621615699369442e-03
3.887500000000000000e+01 9.878614105972530202e-03
3.889999999999999858e+01 9.868003930242735464e-03
3.892499999999999716e+01 9.857741636118100628e-03
3.895000000000000284e+01 9.847768719943760657e-03
3.897500000000000142e+01 9.838018611772825564e-03
3.900000000000000000e+01 9.828417846656558113e-03
3.902499999999999858e+01 9.818887396370566678e-03
3.904999999999999716e+01 9.809344131998166960e-03
3.907500000000000284e+01 9.799702384869636573e-03
3.910000000000000142e+01 9.789875571073738611e-03
3.912500000000000000e+01 9.779777843357385578e-03
3.914999999999999858e+01 9.769325733619273547e-03
3.917499999999999716e+01 9.758439749432459495e-03
3.920000000000000284e+01 9.747045889122975953e-03
3.922500000000000142e+01 9.735077041793493979e-03
3.925000000000000000e+01 9.722474241372921697e-03
3.927499999999999858e+01 9.709187747075916086e-03
3.929999999999999716e+01 9.695177926695844328e-03
3.932500000000000284e+01 9.680415923608069739e-03
3.935000000000000142e+01 9.664884093342254387e-03
3.937500000000000000e+01 9.648576200820898827e-03
3.939999999999999858e+01 9.631497374796847352e-03
3.942499999999999716e+01 9.613663821516616609e-03
3.945000000000000284e+01 9.595102305132840320e-03
3.947500000000000142e+01 9.575849407568596053e-03
3.950000000000000000e+01 9.555950585560728933e-03
3.952499999999999858e+01 9.535459047085101045e-03
3.954999999999999716e+01 9.514434473391951072e-03
3.957500000000000284e+01 9.492941616293024826e-03
3.960000000000000142e+01 9.471048802989186569e-03
3.962500000000000000e+01 9.448826382726090850e-03
3.964999999999999858e+01 9.426345150691180924e-03
3.967499999999999716e+01 9.403674784907325654e-03
3.970000000000000284e+01 9.380882331386741074e-03
3.972500000000000142e+01 9.358030771499279213e-03
3.975000000000000000e+01 9.335177703442585490e-03
3.977499999999999858e+01 9.312374166860870606e-03
3.979999999999999716e+01 9.289663636234377769e-03
3.982500000000000284e+01 9.267081204550550039e-03
3.985000000000000142e+01 9.244652974285647723e-03
3.987500000000000000e+01 9.222395667803386793e-03
3.989999999999999858e+01 9.200316464082715015e-03
3.992499999999999716e+01 9.178413063447419043e-03
3.995000000000000284e+01 9.156673976582863778e-03
3.997500000000000142e+01 9.135079028987273461e-03
\end{filecontents}

\begin{tikzpicture}
  \begin{groupplot}[
      group style={
          group name=my plots,
          group size=1 by 2,
          xlabels at=edge bottom,
          xticklabels at=edge bottom,
          vertical sep=2pt,
          every axis yticklabel/.style={/pgf/number format/fixed}
      },
      width=\columnwidth,
      height=0.5\columnwidth,
      xlabel={Time (\si{\pico\second})},
      %xmin=0, xmax=40,
      %ytick align=outside,
      %xtick align=outside
  ]
  \nextgroupplot[ylabel = {Population}, tick label style={/pgf/number format/fixed}]
  \addplot graphics[xmin=0, xmax=40, ymin=0, ymax=1]{figures/rendered_1024.png};
  \nextgroupplot[ylabel = {IPR}, yticklabel style={/pgf/number format/fixed}] 
  \addplot[smooth, thick] table {ipr_1024.dat};
  \end{groupplot}

\end{tikzpicture}

  \caption{\label{fig:population dynamics}Population dynamics of 512 interacting \qds{}.
    The majority of the \qds{} follow a trajectory prescribed by the incident pulse; several pairs of dots, however, become strongly correlated due to the small distance between particles.
  }
\end{figure}

\begin{figure}
  \usetikzlibrary{pgfplots.groupplots}

\begin{tikzpicture}
  %\begin{axis}[
      %width=\columnwidth,
      %xlabel={Time (\si{\pico\second})},
      %xmin=-2.5, xmax=32.5,
      %ymin = -0.05, ymax = 1.05,
      %%ytick align=outside,
      %%xtick align=outside
  %]
    %\addplot graphics[xmin=0, xmax=30, ymin=0.064480053660309997, ymax=1]{figures/full_neighborhood.png};

  %\end{axis}
  \begin{groupplot}[
      group style={
          group name=my plots,
          group size=1 by 2,
          xlabels at=edge bottom,
          xticklabels at=edge bottom,
          vertical sep=2pt,
      },
      width=\columnwidth,
      xlabel={Time (\si{\pico\second})},
      %xmin=0, xmax=40,
      %ytick align=outside,
      %xtick align=outside
  ]
  \nextgroupplot[ylabel={Population (full system)}]
  \addplot graphics[xmin=0, xmax=30, ymin=0.064480053660309997, ymax=1]{figures/full_neighborhood.png};

  \nextgroupplot[ylabel={Population (reduced system)}] 
  \addplot graphics[xmin=0, xmax=30, ymin=0.069909905907700007, ymax=1]{figures/empty_neighborhood.png};
  \end{groupplot}

\end{tikzpicture}

\end{figure}

\begin{figure}
  \usepgfplotslibrary{fillbetween}
\usetikzlibrary{patterns}

\begin{filecontents}{low_density_stats.dat}
Time, lbound, ubound, p1, p2, p3, p4, p5
0.       ,1.                 ,1.                 ,1.                 ,1.                 ,1.                ,1.                 ,1.                
0.078125 ,0.9999999999999    ,0.9999999999998946 ,0.9999999999999    ,0.9999999999999    ,0.9999999999999   ,0.9999999999999    ,0.9999999999999   
0.15625  ,0.9999999999996766 ,0.9999999999997039 ,0.9999999999997039 ,0.9999999999997039 ,0.9999999999997039,0.9999999999997039 ,0.9999999999997039
0.234375 ,0.9999999999989145 ,0.9999999999989125 ,0.9999999999989145 ,0.9999999999989145 ,0.9999999999989125,0.9999999999989145 ,0.9999999999989125
0.3125   ,0.9999999999970063 ,0.9999999999970499 ,0.9999999999969937 ,0.9999999999970001 ,0.9999999999970938,0.9999999999970001 ,0.9999999999970938
0.390625 ,0.99999999999245   ,0.9999999999927423 ,0.999999999992752  ,0.99999999999265   ,0.99999999999305  ,0.9999999999925521 ,0.99999999999305  
0.46875  ,0.9999999999825251 ,0.9999999999831289 ,0.999999999983229  ,0.9999999999832015 ,0.9999999999843   ,0.9999999999827249 ,0.999999999984382 
0.546875 ,0.9999999999612627 ,0.9999999999627056 ,0.9999999999632002 ,0.9999999999631002 ,0.9999999999661752,0.9999999999617992 ,0.9999999999662054
0.625    ,0.9999999999168    ,0.99999999992      ,0.9999999999217    ,0.9999999999214    ,0.9999999999293   ,0.999999999918     ,0.9999999999294   
0.703125 ,0.9999999998256843 ,0.999999999832929  ,0.9999999998375467 ,0.9999999998368531 ,0.9999999998555663,0.9999999998285041 ,0.9999999998558663
0.78125  ,0.9999999996423422 ,0.9999999996581422 ,0.9999999996697118 ,0.9999999996681102 ,0.9999999997106547,0.9999999996487445 ,0.9999999997112508
0.859375 ,0.9999999992791107 ,0.9999999993125065 ,0.999999999339728  ,0.9999999993364441 ,0.9999999994292796,0.9999999992928211 ,0.9999999994305672
0.9375   ,0.999999998569725  ,0.9999999986390062 ,0.9999999987001875 ,0.9999999986931375 ,0.9999999988897376,0.9999999985987187 ,0.9999999988923937
1.015625 ,0.9999999972029845 ,0.9999999973439773 ,0.9999999974765635 ,0.9999999974620872 ,0.9999999978672954,0.999999997262927  ,0.9999999978725185
1.09375  ,0.9999999946040414 ,0.9999999948860445 ,0.9999999951645633 ,0.9999999951355742 ,0.999999995952236 ,0.9999999947255962 ,0.9999999959624032
1.171875 ,0.9999999897246853 ,0.9999999902794305 ,0.9999999908502152 ,0.9999999907926893 ,0.9999999924063944,0.9999999899667134 ,0.9999999924261191
1.25     ,0.9999999806772    ,0.9999999817522    ,0.9999999828966    ,0.9999999827847    ,0.9999999859178   ,0.9999999811514    ,0.9999999859555   
1.328125 ,0.9999999641058444 ,0.9999999661596316 ,0.9999999684107173 ,0.9999999681956588 ,0.9999999741833879,0.9999999650209497 ,0.9999999742544897
1.40625  ,0.9999999341203234 ,0.9999999379919086 ,0.9999999423434203 ,0.999999941936007  ,0.9999999532134328,0.9999999358618914 ,0.9999999533456156
1.484375 ,0.9999998805127724 ,0.9999998877176602 ,0.9999998959969445 ,0.999999895235224  ,0.9999999161864744,0.9999998837832101 ,0.9999999164292046
1.5625   ,0.9999997858163875 ,0.9999997990588125 ,0.99999981458      ,0.9999998131732561 ,0.9999998515939688,0.9999997918798688 ,0.9999998520345937
1.640625 ,0.9999996205227129 ,0.9999996445691579 ,0.9999996732637487 ,0.9999996706974601 ,0.9999997402786032,0.9999996316262906 ,0.9999997410687443
1.71875  ,0.9999993354131078 ,0.9999993785644281 ,0.9999994309118304 ,0.9999994262858226 ,0.999999550778725 ,0.9999993555024804 ,0.9999995521787812
1.796875 ,0.9999988494369534 ,0.999998925977659  ,0.9999990202588946 ,0.9999990120170591 ,0.9999992321243149,0.99999888535668   ,0.9999992345760081
1.875    ,0.9999980308254    ,0.9999981650469    ,0.999998332755     ,0.9999983182406    ,0.9999987028665   ,0.9999980943075    ,0.9999987071089   
1.953125 ,0.9999966680912614 ,0.9999969008178453 ,0.9999971955368236 ,0.9999971702685125 ,0.9999978346397712,0.9999967790055864 ,0.9999978418943024
2.03125  ,0.999994426148486  ,0.9999948251885695 ,0.9999953369727758 ,0.999995293482036  ,0.9999964279399992,0.9999946177490157 ,0.9999964401989296
2.109375 ,0.9999907809010966 ,0.9999914575729711 ,0.99999233592475   ,0.999992261915205  ,0.9999941770377875,0.9999911081837475 ,0.999994197508629 
2.1875   ,0.9999849231996499 ,0.9999860581307249 ,0.9999875482287812 ,0.9999874236997938 ,0.9999906200385063,0.9999854760468125 ,0.999990653818025 
2.265625 ,0.9999756199722132 ,0.9999775028612659 ,0.9999800019172024 ,0.999979794730624  ,0.9999850690845915,0.9999765435555782 ,0.9999851241645275
2.34375  ,0.9999610165564773 ,0.9999641066693219 ,0.999968250414368  ,0.9999679095543047 ,0.9999765146326617,0.9999625425946477 ,0.9999766033784648
2.421875 ,0.9999383597919819 ,0.9999433768154811 ,0.9999501704220667 ,0.9999496158906328 ,0.999963496744935 ,0.9999408538091639 ,0.9999636380325826
2.5      ,0.9999036164126    ,0.9999116751277    ,0.9999226886779    ,0.9999217965613    ,0.9999439356015   ,0.9999076482459    ,0.9999441578559   
2.578125 ,0.9998509559287367 ,0.9998637632172755 ,0.9998814194949128 ,0.9998800002057994 ,0.9999149132101003,0.9998574035177563 ,0.9999152586500555
2.65625  ,0.9997720619329703 ,0.9997922011488258 ,0.9998201934104148 ,0.9998179604529672 ,0.9998723988821985,0.999782262040454  ,0.9998729293454305
2.734375 ,0.9996552313587024 ,0.9996865672792986 ,0.9997304570512395 ,0.9997269828478313 ,0.9998109129012488,0.9996711954468688 ,0.9998117176878228
2.8125   ,0.9994842184356062 ,0.9995324660277938 ,0.9996005260670563 ,0.9995951804461313 ,0.9997231262587625,0.9995089375550875 ,0.9997243324854812
2.890625 ,0.9992367803661268 ,0.9993102924433263 ,0.9994146777213414 ,0.9994065436156955 ,0.9995993999024834,0.9992746496217709 ,0.9996011858835544
2.96875  ,0.9988828865807867 ,0.9989937288633953 ,0.9991520783172664 ,0.9991398381285891 ,0.9994272748815274,0.9989402873965453 ,0.9994298870327328
3.046875 ,0.9983825644733001 ,0.998547960999227  ,0.9987855538430577 ,0.998767338902892  ,0.999190935172475 ,0.9984686511718573 ,0.9991947088364997
3.125    ,0.9976833742329    ,0.9979276202989    ,0.9982802309919    ,0.9982534257575    ,0.9988706777735   ,0.9978111197006    ,0.9988760621717   
3.203125 ,0.9967175353372095 ,0.9970744872736295 ,0.9975920999346912 ,0.9975530922530047 ,0.9984424390343205,0.9969050976892144 ,0.9984500262217978
3.28125  ,0.9953987694584641 ,0.9959150276202571 ,0.996666579574732  ,0.9966104488341602 ,0.9978774410358398,0.9956712459817781 ,0.9978879982489891
3.359375 ,0.993618980804713  ,0.9943578798974713 ,0.9954371993460109 ,0.9953573360308129 ,0.9971420354165149,0.9940106145988868 ,0.9971565395432571
3.4375   ,0.9912449640562062 ,0.9922914678207688 ,0.9938245454621187 ,0.9937121989406187 ,0.9961978315914313,0.9918018595785375 ,0.9962175036541   
3.515625 ,0.9881154122026002 ,0.9895832537542455 ,0.9917356507524706 ,0.9915794075223294 ,0.9950021993872545,0.9888987942821125 ,0.995028535939921 
3.59375  ,0.9840330502335469 ,0.9860774374394672 ,0.9890640296693398 ,0.9888492320821516 ,0.9935092294113211,0.9851285988180914 ,0.9935440266756203
3.671875 ,0.9787684104187114 ,0.9815922022295206 ,0.9856905670694373 ,0.9853986928637383 ,0.9916712154872249,0.9802910797627862 ,0.9917165809829128
3.75     ,0.9720671475764    ,0.9759240651306    ,0.9814854545973    ,0.981093490141     ,0.9894406908299   ,0.9741594270168    ,0.9894990363475   
3.828125 ,0.9636412417677562 ,0.9688495115798114 ,0.9763113246628029 ,0.9757911788057045 ,0.9867730032590429,0.9664829396424272 ,0.9868470124472211
3.90625  ,0.9531782830854844 ,0.9601298573257531 ,0.9700276559084242 ,0.969345675318825  ,0.9836293575181367,0.9569921728563906 ,0.9837219213149399
3.984375 ,0.940350226836327  ,0.949518443833311  ,0.9624964154839586 ,0.9616130741384014 ,0.9799801894266061,0.9454068768705962 ,0.98009430324919  
4.0625   ,0.9248259809316125 ,0.9367702515743875 ,0.9535887681843438 ,0.9524586106098876 ,0.97580867518195  ,0.931446939039875  ,0.9759472933459937
4.140625 ,0.9062879653314677 ,0.9216537952707667 ,0.9431925334936083 ,0.9417644506184915 ,0.9711141281132818,0.9148462973822669 ,0.9712799734755783
4.21875  ,0.8844523825700101 ,0.9039648930312609 ,0.9312199270412773 ,0.9294378326836258 ,0.9659150037942922,0.8953694669815132 ,0.9661103340706507
4.296875 ,0.8590924201980394 ,0.8835416035653595 ,0.9176150069673171 ,0.9154189608348738 ,0.9602512303053388,0.8728299321343997 ,0.9604775667913488
4.375    ,0.8300630074565    ,0.8602793376968    ,0.9023601800649    ,0.8996879701553    ,0.9541856062409   ,0.8471092426488    ,0.954443435519    
4.453125 ,0.797325145606242  ,0.8341449204843179 ,0.8854811277603589 ,0.8822702839408716 ,0.9478040651546377,0.8181752719767977 ,0.9480925294732134
4.53125  ,0.7609673305668829 ,0.8051882565480345 ,0.867049596642357  ,0.8632397627559274 ,0.9412146848819641,0.7860978195251908 ,0.9415312818588633
4.609375 ,0.7212213095005034 ,0.7735502715516245 ,0.8471836577248939 ,0.8427192079612615 ,0.9345454135103701,0.7510596411414816 ,0.934885730233649 
4.6875   ,0.6784694917851125 ,0.7394659964531938 ,0.82604525953935   ,0.8208780127347    ,0.92794058035555  ,0.7133611413503063 ,0.9282980904025937
4.765625 ,0.6332418414808125 ,0.7032620171742071 ,0.8038351503348283 ,0.797927018798053  ,0.9215563463187204,0.6734173812542057 ,0.9219223003192204
4.84375  ,0.5862010330170928 ,0.6653480043942827 ,0.7807854938508507 ,0.7741109047024156 ,0.9155553157632312,0.6317467401821804 ,0.9159187565203296
4.921875 ,0.538115968158292  ,0.6262026077899774 ,0.757150717242194  ,0.7496986645210709 ,0.9101005751399077,0.5889514480089559 ,0.9104485066577732
5.       ,0.4898252400173    ,0.5863545679869    ,0.7331972809006    ,0.7249729029761    ,0.9053494407646   ,0.5456911552103    ,0.9056671764503   
5.078125 ,0.44219354330636235,0.5463603955121606 ,0.7091931341258738 ,0.7002187578403213 ,0.9014471928507287,0.5026515860651977 ,0.9017189012269529
5.15625  ,0.39606508879327046,0.5067803158115719 ,0.6853976116061203 ,0.6757132547739158 ,0.8985210507141719,0.4605109680063548 ,0.8987305066699179
5.234375 ,0.35221856968858267,0.468154339101631  ,0.6620524421363999 ,0.6517158118065453 ,0.896674614381025 ,0.41990723128475876,0.8968061496002859
5.3125   ,0.3113280358256625 ,0.43098026432631875,0.6393743975897314 ,0.6284604559323188 ,0.89598296787035  ,0.38140886752626246,0.896022594953625 
5.390625 ,0.27393319416960527,0.3956951795750097 ,0.6175499290259521 ,0.6061501168200264 ,0.8964886185594099,0.345491842720054  ,0.8964252789769228
5.46875  ,0.2404213408663085 ,0.36266161768042804,0.5967319410816742 ,0.5849531483534757 ,0.8981984391310984,0.31252417182835535,0.898025295748743 
5.546875 ,0.21102159395354841,0.33215902149818805,0.5770386668365752 ,0.565002022066357  ,0.9010817834869621,0.28275881647021744,0.9007974456378228
5.625    ,0.1858106261409    ,0.3043806422394    ,0.5585544438436    ,0.5463939611987    ,0.9050699621794   ,0.2563346293822    ,0.9046794988543   
5.703125 ,0.16472791822276858,0.27943549870748224,0.5413320689128489 ,0.5291931533260952 ,0.9100572722788027,0.23328427319008108,0.9095728436492537
5.78125  ,0.14759781793144613,0.2573546251950235 ,0.5253963332217032 ,0.5134341033904251 ,0.9159037660833914,0.21354748903087037,0.9153446946336875
5.859375 ,0.13415543392527152,0.23810056935337312,0.510748311181313  ,0.4991256685019188 ,0.9224398941346678,0.1969878238979159 ,0.9218320153566097
5.9375   ,0.12407355741493126,0.22157897775599375,0.49736998965388124,0.48625534339858123,0.929473052545925 ,0.1834109248416625 ,0.9288472416768312
6.015625 ,0.11698827884848356,0.20765112571917943,0.4852288724764208 ,0.474793432907348  ,0.9367958997933525,0.17258272799007054,0.9361857706203071
6.09375  ,0.1125216051042453 ,0.19614637758586093,0.4742822657190273 ,0.46469683917499766,0.944196090989293 ,0.1642462275342656 ,0.9436350005414844
6.171875 ,0.11030005869975097,0.18687377000618566,0.46448103134755026,0.4559122932882453 ,0.9514668391549044,0.1581359289162365 ,0.9509844895029012
6.25     ,0.1099688542827    ,0.1796321547372    ,0.4557726809548    ,0.4483789602564    ,0.9584175001163   ,0.1539895036076    ,0.9580365729666   
6.328125 ,0.1112017380133709 ,0.17421858094172218,0.4481037582534835 ,0.4420304317399074 ,0.9648832457866809,0.15155652026275185,0.9646165956086266
6.40625  ,0.11370691988851328,0.17043481410291486,0.4414215241530125 ,0.4367961859259047 ,0.9707328896146742,0.15060440321852733,0.9705818178316382
6.484375 ,0.11722972983031267,0.16809206067620872,0.4356750080126682 ,0.43260263481771233,0.975874081024693 ,0.15092195446866408,0.9758280953807934
6.5625   ,0.121552710518675  ,0.16701408811551874,0.4308155223749812 ,0.42937389574011875,0.9802553839087125,0.15232087705301248,0.9802936204468126
6.640625 ,0.12649385004151126,0.16703899958208604,0.4267967567764817 ,0.42703241924466434,0.9838651444049776,0.15463576892735253,0.9839593345232877
6.71875  ,0.13180978585417188,0.16801994865697187,0.4235745710494484 ,0.4254995844053281 ,0.9867274610521164,0.1577230361436922 ,0.9868460273668945
6.796875 ,0.1368780131693621 ,0.16982507221793966,0.421106602505772  ,0.4246963408501117 ,0.988895914006053 ,0.16145912074147775,0.9890085476895826
6.875    ,0.1422785092828    ,0.1723368893859    ,0.4193517873265    ,0.4245439409772    ,0.9904459241644   ,0.1657383669683    ,0.9905278893642   
6.953125 ,0.14792819647124347,0.17545137229778826,0.4182698774556434 ,0.42496477112151365,0.9914666712007429,0.17047077333946464,0.9915021243226685
7.03125  ,0.15376102510100467,0.17907684796776094,0.4178210129420086 ,0.42588326138748045,0.9920534089530983,0.17557980533789375,0.992037200116261 
7.109375 ,0.1597249852486237 ,0.18313284569146102,0.4179653884012985 ,0.4272268330389255 ,0.9923008181560509,0.1810003799397031 ,0.9922385165659727
7.1875   ,0.16577945997360624,0.1875489659493    ,0.41866303296544377,0.42892683099213746,0.9922977849912374,0.18667708212655   ,0.9922039804285062
7.265625 ,0.17189294936866595,0.19229451640393272,0.4198737070047285 ,0.4309193865137794 ,0.9921237428312617,0.19256263451916888,0.9920189620318984
7.34375  ,0.1780411614590508 ,0.19729025855536328,0.4215569068073485 ,0.4331461602853594 ,0.9918465058117696,0.198616614442943  ,0.9917532970893821
7.421875 ,0.18416214779896534,0.20248470496492743,0.4236719604097726 ,0.43555492652728905,0.9915213789526988,0.20480439606172238,0.9914602308341399
7.5      ,0.1901331409475    ,0.2078378972888    ,0.4261781936326    ,0.4380999722317    ,0.9911912566695   ,0.2110962864516    ,0.9911770164347   
7.578125 ,0.19607878823998018,0.21331550438418964,0.42903514444649815,0.4407422999496698 ,0.9908874103685527,0.21746682194394695,0.9909267635188014
7.65625  ,0.20203731004541325,0.21888818032252108,0.432202805295975  ,0.44344963602296955,0.9906307001389812,0.22389419244274453,0.9907210837702415
7.734375 ,0.20799740720795026,0.22453098356229353,0.43564187605218335,0.4461962573295664 ,0.9904330054767396,0.23035976516493747,0.9905630891668722
7.8125   ,0.21394991787624376,0.23022284920088126,0.43931401410473125,0.4489626583354    ,0.9902987360736251,0.2368476842659625 ,0.9904503508004062
7.890625 ,0.21988742915547912,0.23594610808663216,0.4431820720267711 ,0.4517350856920482 ,0.9902263422448175,0.24334452786704788,0.9903775097410021
7.96875  ,0.2258039639363719 ,0.24168604842453753,0.44721031680497736,0.454504970454868  ,0.9902097856393484,0.24983900875541018,0.9903383319111813
8.046875 ,0.2316947266001067 ,0.24743051697817509,0.45136462754678763,0.45726828849907414,0.9902399516047358,0.2563217090715103 ,0.9903271046512756
8.125    ,0.2375558949216    ,0.2531695578812    ,0.4556126706628    ,0.4600248782044    ,0.9903059870935   ,0.262784842427     ,0.9903393723524   
8.203125 ,0.24338444851371596,0.25910597262184604,0.45992405287754795,0.46277774185454856,0.9903965378826947,0.2692220392697212 ,0.9903720926738201
8.28125  ,0.2491780265059227 ,0.265045134167543  ,0.4642704530884875 ,0.4655323536803641 ,0.9905008442182632,0.27562815290367815,0.9904233561592125
8.359375 ,0.2549348089312525 ,0.2709449128319876 ,0.46862573419509046,0.46829599356210805,0.9906096424240007,0.2819990845511249 ,0.9904918459179316
8.4375   ,0.2605837799151375 ,0.27680140701073125,0.4729660358580625 ,0.47107712154826875,0.9907158177653125,0.2883316264165187 ,0.9905762193395251
8.515625 ,0.26596298415945857,0.2826115879583829 ,0.4772698487152713 ,0.47388480455135046,0.9908147641355589,0.2946233219125029 ,0.9906745727337896
8.59375  ,0.271311695191857  ,0.28837315095951793,0.48151807012661796,0.47672820319374765,0.9909044288690445,0.3008723422414148 ,0.9907841078908493
8.671875 ,0.2766290118364014 ,0.294084388021792  ,0.4856940411534851 ,0.4796161237739408 ,0.9909850530576904,0.3070773784464419 ,0.9909010646309178
8.75     ,0.2819143404577    ,0.2997440799078    ,0.4897835641456    ,0.4825566377651    ,0.9910586535059   ,0.3132375479105    ,0.9910209246521   
8.828125 ,0.2871673221533725 ,0.3053514053356017 ,0.49377490020374726,0.48555676914976253,0.9911283254660618,0.3193523141697719 ,0.9911388384561317
8.90625  ,0.2923877739185602 ,0.31090586522899455,0.49765874581053754,0.48862224818277344,0.9911974690506788,0.3254214188112227 ,0.9912501866646438
8.984375 ,0.29757564166902994,0.31640721999389104,0.5014281880371292 ,0.49175732887327334,0.9912690519067175,0.331444824171192  ,0.9913511650017766
9.0625   ,0.3027309632116438 ,0.321855437939075  ,0.5050786380734    ,0.494964666448825  ,0.9913450139146938,0.33742266555144373,0.9914392809429687
9.140625 ,0.30785383945980543,0.3272506531227288 ,0.5086077431470336 ,0.4982452503649033 ,0.991425896741987 ,0.3433552116971161 ,0.9915136684832524
9.21875  ,0.3129444124033484 ,0.33259313108524213,0.512015277313064  ,0.5015983879370781 ,0.9915107453673461,0.34924283235209763,0.991575161580875 
9.296875 ,0.31800284854201616,0.33788324111982293,0.5153030120612836 ,0.5050217333517634 ,0.9915972856723405,0.35508597179464624,0.991626110259366 
9.375    ,0.3230293266852    ,0.3431214339125    ,0.5184745680715    ,0.5085113566911    ,0.9916823388358   ,0.3608851273669    ,0.9916699683601   
9.453125 ,0.328024029195293  ,0.3483082235595546 ,0.5215352498806413 ,0.5120618475539424 ,0.9917623964531644,0.3666408321217679 ,0.9917107210303562
9.53125  ,0.3329871359097149 ,0.353444173130793  ,0.5244918655809985 ,0.5156664479089976 ,0.9918342560783782,0.3723536408300282 ,0.9917522466824539
9.609375 ,0.33791882011511537,0.35852988308870865,0.5273525339473634 ,0.5193172089655937 ,0.9918956090463183,0.37802411869936536,0.9917977182654139
9.6875   ,0.34281924606851877,0.3635659820003625 ,0.5301264816766125 ,0.5230051669807625 ,0.9919454821818563,0.38365283226365   ,0.991849140880875 
9.765625 ,0.34768856766028594,0.36855311908473004,0.5328238335743172 ,0.5267205331644714 ,0.9919844605542941,0.3892403419935816 ,0.9919070987455174
9.84375  ,0.35252692790101325,0.3734919582280586 ,0.5354553986480922 ,0.5304528930682609 ,0.9920146554595328,0.3947871962653859 ,0.9919707485916289
9.921875 ,0.3573344589846518 ,0.37838317317407294,0.538032455131922  ,0.5341914111041869 ,0.9920394241601718,0.4002939263960207 ,0.9920380549880999
10.      ,0.362111282738     ,0.3832274436555    ,0.5405665374377    ,0.5379250361294    ,0.9920628885126   ,0.4057610425161    ,0.9921062228835   
10.078125,0.36685751131299715,0.3880254522811636 ,0.5430692279800113 ,0.5416427043221002 ,0.9920893319557536,0.4111890301031819 ,0.9921722504957401
10.15625 ,0.37157324801643987,0.392777882031086  ,0.5455519566869798 ,0.5453335358965985 ,0.9921225731498124,0.4165783470402297 ,0.9922335068211884
10.234375,0.3762585882004435 ,0.39748541424307154,0.5480258108475594 ,0.5489870225254019 ,0.9921654170533544,0.42192942110016246,0.9922882353660757
10.3125  ,0.38091362016104374,0.4021487269933812 ,0.5505013577272875 ,0.5525932026848313 ,0.9922192701779438,0.42724264778212495,0.9923358994575625
10.390625,0.3855384260081182 ,0.4067684937971664 ,0.5529884821349631 ,0.5561428224866406 ,0.9922839787886216,0.4325183884484671 ,0.9923773121087502
10.46875 ,0.3901330824864703 ,0.41134538256383435,0.5554962408511609 ,0.5596274799121663 ,0.9923579116967454,0.43775696872708586,0.9924145301230234
10.546875,0.39469766173525994,0.41588005475607925,0.5580327355143144 ,0.563039750737643  ,0.9924382692210171,0.4429586771556318 ,0.9924505317551741
10.625   ,0.3992322319824    ,0.4203731647084    ,0.5606050052616    ,0.5663732947835    ,0.9925215635671   ,0.4481237640516    ,0.9924887331273   
10.703125,0.4037368581738812 ,0.42482535906928764,0.5632189400869609 ,0.569622941485622  ,0.9926041890797419,0.453252440600252  ,0.9925324249470944
10.78125 ,0.4082116025437969 ,0.42923727633641334,0.5658792155508391 ,0.5727847541392532 ,0.9926829878910695,0.45834487815352193,0.9925842235844641
10.859375,0.4126565251309756 ,0.433609546459345  ,0.5685892491761452 ,0.5758560724777502 ,0.992755719244901 ,0.46340120773546145,0.9926456274116648
10.9375  ,0.41707168425055624,0.4379427904913    ,0.5713511785206999 ,0.57883553360155   ,0.9928213585668625,0.46842151975033747,0.9927167511684812
11.015625,0.421457136927898  ,0.44223762027078745,0.5741658606410546 ,0.5817230715545981 ,0.9928801819693183,0.4734058638878528 ,0.9927962812009642
11.09375 ,0.42581293930232417,0.4464946381249656 ,0.5770328923714226 ,0.5845198961339594 ,0.9929336282643985,0.4783542492184375 ,0.9928816578005046
11.171875,0.4301391470089324 ,0.45071443658412585,0.5799506505561152 ,0.5872284518055612 ,0.9929839676452265,0.48326664446915213,0.9929694536196865
11.25    ,0.4344358155412    ,0.4548975981046    ,0.582916351148     ,0.5898523578186    ,0.9930338377627   ,0.488142978469     ,0.9930558854867   
11.328125,0.4387030006013341 ,0.45904469479754995,0.5859261258350449 ,0.5923963308618949 ,0.9930857289355334,0.49298314074957417,0.9931373759196923
11.40625 ,0.44294075843929614,0.46315628816727034,0.5889751146474624 ,0.594866091808921  ,0.9931415072737719,0.49778698228312657,0.9932110736765594
11.484375,0.44714914618191137,0.4672329288615839 ,0.5920575728228695 ,0.5972682582667176 ,0.9932020566742317,0.5025543163413837 ,0.9932752506355588
11.5625  ,0.4513282221533687 ,0.47127515644195   ,0.5951669900189562 ,0.599610224834125  ,0.9932670994147688,0.5072849194541312 ,0.9933295137646062
11.640625,0.4554780461834173 ,0.4752834991804065 ,0.5982962198365503 ,0.6019000330925527 ,0.9933352241925656,0.5119785324473417 ,0.993374802083918 
11.71875 ,0.4595986799039242 ,0.4792584738917195 ,0.6014376174949195 ,0.6041462334708398 ,0.9934041151703453,0.5166348615398773 ,0.9934131741434086
11.796875,0.4636901870282282 ,0.4832005858104404 ,0.6045831834096005 ,0.6063577412191634 ,0.9934709418181005,0.5212535794798349 ,0.9934474256693558
11.875   ,0.4677526336118    ,0.4873142647384    ,0.6077247103709    ,0.6085436887706    ,0.9935328427253   ,0.5258343267026    ,0.9934806040213   
11.953125,0.4717860882898117 ,0.4914381612318979 ,0.6108539319885845 ,0.6107132768056358 ,0.9935874213759821,0.5303767124943786 ,0.9935155016715592
12.03125 ,0.47579062248822973,0.49552872207311877,0.613962670077479  ,0.6128756263216468 ,0.9936331705613352,0.5348803161518391 ,0.9935542125956531
12.109375,0.4797663106053974 ,0.4995862267036934 ,0.6170429786918843 ,0.6150396339770419 ,0.9936697546333872,0.5393446881270525 ,0.9935978232772711
12.1875  ,0.4837132301614    ,0.5036109520237124 ,0.6200872825907312 ,0.6172138329037375 ,0.9936981029832936,0.5437693511583312 ,0.9936462861474562
12.265625,0.487631461913092  ,0.5076031723374983 ,0.6230885080172104 ,0.6194062610857479 ,0.9937202997178689,0.548153801386007  ,0.9936984919306333
12.34375 ,0.491521089934632  ,0.5115631593220304 ,0.626040203800396  ,0.6216243392703922 ,0.9937392882295875,0.552497509462493  ,0.9937525238317946
12.421875,0.4953822016617745 ,0.5154911820137695 ,0.6289366509593837 ,0.6238747602160737 ,0.993758439584911 ,0.5567999216672984 ,0.9938060463821995
12.5     ,0.4992148879023    ,0.5193875068128    ,0.63177295916      ,0.6261633909056    ,0.9937810554106   ,0.5610604610461    ,0.9938567600436   
12.578125,0.5030192428129104 ,0.523252397501074  ,0.6345451485872593 ,0.6284951891474467 ,0.9938098858280712,0.565278528592005  ,0.9939028427681201
12.65625 ,0.5067953638466828 ,0.5270861152708227 ,0.6372502160291134 ,0.63087413575665   ,0.9938467394481149,0.5694535044974836 ,0.9939433031390649
12.734375,0.510543351673126  ,0.5308889187605795 ,0.6398861841844257 ,0.6333031832899425 ,0.9938922463548509,0.573584749500605  ,0.9939781856325667
12.8125  ,0.5142633100756688 ,0.5346610640949062 ,0.642452133488325  ,0.6357842220362687 ,0.9939458091297625,0.57767160635795   ,0.9940085941807875
12.890625,0.5179553458312016 ,0.5384028049241574 ,0.6449482159937681 ,0.6383180637195657 ,0.9940057458027141,0.581713401470186  ,0.994036531108976 
12.96875 ,0.5216195685762696 ,0.5421143924625359 ,0.6473756511135117 ,0.6409044431090015 ,0.9940695973141414,0.5857094466916922 ,0.9940645795042382
13.046875,0.5252560906652103 ,0.545796075519494  ,0.6497367033177361 ,0.6435420374440524 ,0.9941345458002935,0.589659041349566  ,0.994095483010478 
13.125   ,0.5288650270257    ,0.5494481005248    ,0.652034642134     ,0.6462285033372    ,0.9941978730508   ,0.5935614744955    ,0.9941316936679   
13.203125,0.5324464950161933 ,0.5530707115429788 ,0.6542736850805702 ,0.6489605305310305 ,0.994257383413453 ,0.5974160274116874 ,0.9941749631490792
13.28125 ,0.5360006142900102 ,0.5566641502787024 ,0.6564589244343867 ,0.6517339116262845 ,0.9943117229475179,0.601221976384036  ,0.9942260449014305
13.359375,0.5395275066706949 ,0.5602286560696161 ,0.6585962389749408 ,0.654543626655606  ,0.9943605452226171,0.6049785957541558 ,0.9942845558259444
13.4375  ,0.5430272960413812 ,0.5637644658695625 ,0.6606921921171938 ,0.6573839411139313 ,0.9944045006672875,0.6086851612519687 ,0.9943490193705875
13.515625,0.5465001082516546 ,0.5672718142205406 ,0.6627539180587917 ,0.6602485158423922 ,0.9944450561784105,0.6123409536079112 ,0.9944170819198667
13.59375 ,0.5499460710440344 ,0.570750933216486  ,0.6647889977868344 ,0.6631305269509    ,0.9944841798904992,0.6159452624347891 ,0.994485866041918 
13.671875,0.5533653140005566 ,0.5742020524585959 ,0.666805326979398  ,0.6660227937707708 ,0.994523947897643 ,0.6194973903659973 ,0.9945524021953247
13.75    ,0.5567579685116    ,0.5776253990066    ,0.668810977992     ,0.6689179126804    ,0.9945661417648   ,0.6229966574298    ,0.9946140684541   
13.828125,0.5601241677641825 ,0.5810211973256116 ,0.6708140582539955 ,0.6718083945111836 ,0.9946119060418644,0.6264424056336362 ,0.9946689675000832
13.90625 ,0.563464046751739  ,0.5843896692335695 ,0.6728225674883852 ,0.6746868031517063 ,0.9946615237657243,0.6298340037309548 ,0.9947161814923429
13.984375,0.5667777423006846 ,0.5877310338501591 ,0.6748442562269857 ,0.6775458929120138 ,0.9947143472015124,0.6331708521364295 ,0.9947558663462516
14.0625  ,0.5700653931139875 ,0.591045507549825  ,0.6768864881086437 ,0.6803787421937563 ,0.9947688943455562,0.636452387956075  ,0.9947891739216125
14.140625,0.573327139827289  ,0.5943333039224656 ,0.6789561084145038 ,0.6831788810408519 ,0.9948230936153837,0.6396780900937522 ,0.9948180189957047
14.21875 ,0.5765631250758719 ,0.5975946337419616 ,0.6810593212383655 ,0.6859404102077391 ,0.9948746343277601,0.6428474843981211 ,0.9948447330215476
14.296875,0.5797734935676983 ,0.6008297049454566 ,0.6832015775580979 ,0.688658109503506  ,0.9949213631404877,0.6459601488117392 ,0.9948716644161124
14.375   ,0.5829583921611    ,0.6040387226244    ,0.6853874763419    ,0.691327533309     ,0.9949616594179   ,0.6490157184854    ,0.9949007926528   
14.453125,0.5861179699419936 ,0.6072218890280491 ,0.6876206806146443 ,0.6939450913668684 ,0.99499472652806  ,0.6520138908228763 ,0.9949334196893814
14.53125 ,0.5892523782989812 ,0.6103794035793414 ,0.6899038501776062 ,0.6965081131746297 ,0.9950207505898571,0.6549544304220632 ,0.9949699880234523
14.609375,0.5923617709921316 ,0.6135114629038454 ,0.6922385924242922 ,0.6990148945608758 ,0.9950409005725908,0.6578371738802314 ,0.9950100525287263
14.6875  ,0.5954463042144125 ,0.616618260870025  ,0.6946254323738813 ,0.7014647253445813 ,0.9950571701424374,0.66066203443595   ,0.9950524069176375
14.765625,0.5985061366424073 ,0.619699988640862  ,0.6970638027297081 ,0.7038578972879161 ,0.9950720877242032,0.6634290064188901 ,0.9950953397462304
14.84375 ,0.6015414294755281 ,0.6227568347343656 ,0.6995520544231586 ,0.706195691901093  ,0.9950883426400828,0.6661381694843672 ,0.9951369735554828
14.921875,0.6045523464624649 ,0.6257889850917553 ,0.7020874877112365 ,0.7084803480456403 ,0.9951083883998857,0.6687896926100734 ,0.9951756275500255
15.      ,0.6075390539139    ,0.6287966231513    ,0.704666403536     ,0.7107150096439    ,0.9951340871753   ,0.6713838378352    ,0.9952101412055   
15.078125,0.6105017207024379 ,0.631779929926099  ,0.7072841744405487 ,0.7129036542122341 ,0.9951664518760499,0.6739209637221826 ,0.9952401037072309
15.15625 ,0.6134405182489062 ,0.6347390840819578 ,0.7099353339404516 ,0.7150510033316898 ,0.9952055254485266,0.6764015285266758 ,0.9952659507540039
15.234375,0.6163556204968721 ,0.6376742620162815 ,0.7126136828743276 ,0.7171624165431066 ,0.9952504140077507,0.6788260930560318 ,0.9952889130679511
15.3125  ,0.6192472038761313 ,0.6405856379330875 ,0.7153124108537937 ,0.719243770556125  ,0.9952994650012374,0.6811953232043    ,0.9953108261109562
15.390625,0.6221154472571383 ,0.6434733839140777 ,0.7180242305937176 ,0.7213013260011212 ,0.995350558128838 ,0.6835099921446736 ,0.9953338337185632
15.46875 ,0.6249605318975179 ,0.6463376699845305 ,0.7207415225712804 ,0.7233415842830335 ,0.9954014589578242,0.6857709821653992 ,0.9953600358552812
15.546875,0.6277826413832558 ,0.6491786641716392 ,0.7234564871699501 ,0.7253711373898754 ,0.9954501759765457,0.6879792861301144 ,0.9953911396053745
15.625   ,0.6305819615666    ,0.6519965325558    ,0.7261613012309    ,0.7273965137415    ,0.9954952625779   ,0.6901360085443    ,0.9954281714865   
15.703125,0.6333586805016149 ,0.6547914393144041 ,0.7288482757457857 ,0.7294240233534198 ,0.9955360158664023,0.6922423662055353 ,0.9954712985113774
15.78125 ,0.6361129883810086 ,0.6575635467575071 ,0.7315100113091305 ,0.731459605705789  ,0.9955725423715797,0.6942996884152047 ,0.9955197870071094
15.859375,0.6388450774748748 ,0.6603130153565446 ,0.734139547891153  ,0.7335086837665727 ,0.9956056835385251,0.6963094167222392 ,0.9955721052290665
15.9375  ,0.6415551420727438 ,0.6630400037667687 ,0.7367305055268125 ,0.7355760275845687 ,0.9956368174626999,0.6982731041715062 ,0.9956261520659125
16.015625,0.6442433784314053 ,0.6657446688445936 ,0.7392772126139554 ,0.737665630768728  ,0.9956675737275044,0.7001924140230155 ,0.9956795736593828
16.09375 ,0.6469099847283563 ,0.6684271656606907 ,0.741774818695661  ,0.7397806029922297 ,0.9956995121072125,0.7020691179048806 ,0.9957301158271383
16.171875,0.649555161021294  ,0.6710876475112931 ,0.7442193888794598 ,0.7419230813796475 ,0.9957338210645945,0.7039050933646029 ,0.9957759550381481
16.25    ,0.6521791092156    ,0.6737262659286    ,0.7466079773718    ,0.7440941633099    ,0.9957710876864   ,0.7057023207768    ,0.9958159550602   
16.328125,0.6547820330371129 ,0.6763431706923537 ,0.7489386780380816 ,0.7462938627364493 ,0.995811177814914 ,0.7074628795677711 ,0.9958498094227561
16.40625 ,0.6573641380120616 ,0.6789385098439328 ,0.7512106503859609 ,0.7485210916324273 ,0.9958532459184547,0.7091889437189328 ,0.9958780492331282
16.484375,0.6599256314513219 ,0.6815124297051588 ,0.7534241198989847 ,0.7507736676410356 ,0.9958958720330804,0.7108827765095798 ,0.9959019183163137
16.5625  ,0.662466722439375  ,0.6840650749020125 ,0.7555803522731375 ,0.7530483483762063 ,0.9959373015989125,0.7125467244687125 ,0.9959231394051562
16.640625,0.6649876218247016 ,0.6865965883952989 ,0.7576816017227301 ,0.7553408922012459 ,0.9959757468649115,0.7141832105060709 ,0.9959436125183541
16.71875 ,0.667488542211368  ,0.6891071115191352 ,0.7597310341866399 ,0.7576461446403032 ,0.9960096984979656,0.7157947262051532 ,0.9959650969743367
16.796875,0.6699696979488616 ,0.6915967840267491 ,0.7617326269446593 ,0.7599581488912068 ,0.9960381947224889,0.7173838232673698 ,0.9959889300588604
16.875   ,0.6724313051185    ,0.694065744145     ,0.7636910467914    ,0.7622702782569    ,0.996061002819    ,0.7189531041088    ,0.9960158281055   
16.953125,0.6748735815136084 ,0.6965141286367211 ,0.7656115095592244 ,0.7645753876617463 ,0.9960786827749406,0.7205052116251615 ,0.9960458010028939
17.03125 ,0.6772967466118694 ,0.6989420728702727 ,0.7674996243559656 ,0.7668659808310961 ,0.9960925227053751,0.7220428181526961 ,0.9960781914958304
17.109375,0.679701021537347  ,0.701349710896879  ,0.7693612263894586 ,0.7691343891923779 ,0.9961043569628466,0.7235686136693439 ,0.9961118294810309
17.1875  ,0.682086629010575  ,0.7037371755324688 ,0.7712022026872563 ,0.7713729581122251 ,0.9961162971260438,0.7250852932946562 ,0.9961452723170625
17.265625,0.6844537932838078 ,0.7061045984453418 ,0.773028315330771  ,0.7735742357554969 ,0.9961304200474851,0.7265955441638849 ,0.9961770882071719
17.34375 ,0.6868027400609211 ,0.7084521102464625 ,0.7748450270338015 ,0.7757311596378478 ,0.9961484636785055,0.7281020317644922 ,0.9962061331582429
17.421875,0.6891336964007101 ,0.7107798405817375 ,0.776657333958334  ,0.7778372358628921 ,0.9961715794450942,0.7296073858388238 ,0.996231773890386 
17.5     ,0.6914468906008    ,0.7130879182249    ,0.7784696106028    ,0.7798867060937    ,0.9962001800086   ,0.7311141859681    ,0.9962540188904   
17.578125,0.6937425520640613 ,0.7153764711695582 ,0.7802854713772476 ,0.7818746975151739 ,0.99623390505235  ,0.7326249469638095 ,0.9962735358336846
17.65625 ,0.6960209111454577 ,0.7176456267185125 ,0.7821076531248875 ,0.7837973514026132 ,0.9962717080716984,0.7341421041991508 ,0.9962915530608414
17.734375,0.6982821989792518 ,0.7198955115705028 ,0.7839379223656895 ,0.7856519263879502 ,0.9963120473473833,0.7356679990232186 ,0.9963096623755419
17.8125  ,0.7005266472891063 ,0.7221262519016562 ,0.7857770103965375 ,0.7874368731644563 ,0.9963531475639374,0.7372048643975688 ,0.9963295568990187
17.890625,0.7027544881789043 ,0.724337973442675  ,0.7876245786540832 ,0.7891518780968169 ,0.996393287692789 ,0.7387548109004393 ,0.9963527483218896
17.96875 ,0.7049659539079985 ,0.7265308015500375 ,0.7894792159146126 ,0.7907978740483931 ,0.9964310674189547,0.7403198132394477 ,0.9963803109904196
18.046875,0.7071612766510068 ,0.7287048612718592 ,0.7913384679777816 ,0.7923770176698697 ,0.9964656090230675,0.7419016974078562 ,0.9964126954010861
18.125   ,0.7093406882442    ,0.7308602774071    ,0.7931988995454    ,0.7938926333426    ,0.9964966633119   ,0.7435021286142    ,0.9964496418048   
18.203125,0.7115044199197825 ,0.7329971745601407 ,0.7950561870161    ,0.7953491249736308 ,0.9965246048671724,0.745122600103277  ,0.9964902077640159
18.28125 ,0.7136527020320882 ,0.7351156771902047 ,0.7969052399508758 ,0.7967518578214507 ,0.996550320704243 ,0.7467644229769641 ,0.9965329045627165
18.359375,0.7157857637751667 ,0.7372159096555075 ,0.7987403480609148 ,0.7981070134467596 ,0.9965750142035873,0.7484287171102671 ,0.9965759195075666
18.4375  ,0.7179038328958249 ,0.7392979962546562 ,0.8005553497017    ,0.799421421774125  ,0.996599959998925 ,0.750116403242275  ,0.9966173872994187
18.515625,0.7200071354038492 ,0.7413620612652736 ,0.8023438171348204 ,0.8007023749831439 ,0.9966262530779726,0.751828196310803  ,0.9966556660966938
18.59375 ,0.7220958952814767 ,0.7434082289800211 ,0.8040992532057618 ,0.8019574285800063 ,0.9966545955591485,0.7535646000832898 ,0.9966895737385304
18.671875,0.7241703341937066 ,0.7454366237423505 ,0.8058152936375623 ,0.8031941954662954 ,0.9966851574828914,0.7553259031207624 ,0.9967185468297692
18.75    ,0.7262306712019    ,0.747447369982     ,0.807485908866     ,0.8044201391061    ,0.9967175348501   ,0.757112176104     ,0.9967426986383   
18.828125,0.7282771224814576 ,0.7494405922508812 ,0.8091055992586428 ,0.8056423719924104 ,0.9967508112859484,0.7589232705337601 ,0.9967627687883382
18.90625 ,0.7303099010455468 ,0.7514164152607922 ,0.8106695776748867 ,0.8068674655097594 ,0.9967837120235234,0.7607588188082414 ,0.9967799756239828
18.984375,0.7323292164752069 ,0.7533749639222129 ,0.8121739336176699 ,0.8081012770008613 ,0.9968148233456795,0.7626182356745047 ,0.9967957978891292
19.0625  ,0.7343352746566    ,0.7553163633855312 ,0.8136157737416    ,0.80934879934595   ,0.9968428397661813,0.764500721038075  ,0.99681172345975  
19.140625,0.7363282775262632 ,0.7572407390840086 ,0.8149933341405697 ,0.8106140377035326 ,0.996866796850191 ,0.766405264113712  ,0.9968290074648085
19.21875 ,0.7383084228241906 ,0.7591482167787532 ,0.8163060606590946 ,0.8118999172469219 ,0.9968862502909109,0.7683306488936649 ,0.9968484796292344
19.296875,0.7402759038532539 ,0.761038922605213  ,0.8175546544483303 ,0.8132082247570244 ,0.9969013711523854,0.770275460903983  ,0.996870431564346 
19.375   ,0.7422309092474    ,0.7629129831204    ,0.8187410810172    ,0.8145395859022    ,0.9969129413907   ,0.7722380952212    ,0.9968946006936   
19.453125,0.7441736227446378 ,0.7647705253504227 ,0.8198685421711123 ,0.8158934788961331 ,0.9969222504693668,0.7742167657137748 ,0.9969202509130252
19.53125 ,0.746104222966932  ,0.7666116768378195 ,0.8209414113861977 ,0.8172682840698023 ,0.99693091025415  ,0.7762095154719617 ,0.9969463338051711
19.609375,0.7480228832042265 ,0.7684365656871239 ,0.8219651342926749 ,0.8186613677665172 ,0.9969406186884732,0.7782142283876332 ,0.9969717009430338
19.6875  ,0.7499297712033937 ,0.7702453206077687 ,0.8229460970796625 ,0.8200691978365875 ,0.9969529108567562,0.7802286418374375 ,0.9969953297085813
19.765625,0.7518250489594664 ,0.7720380709543658 ,0.8238914666333097 ,0.8214874870035748 ,0.9969689376421822,0.7822503604208293 ,0.9970165233960062
19.84375 ,0.7537088725108859 ,0.7738149467624461 ,0.8248090071394625 ,0.8229113594625688 ,0.9969893071375305,0.7842768706964844 ,0.9970350513308804
19.921875,0.7555813917363687 ,0.7755760787790322 ,0.8257068786582082 ,0.8243355352950462 ,0.99701401317506  ,0.786305556852979  ,0.997051205390715 
20.      ,0.7574427501544    ,0.777321598488     ,0.8265934237592    ,0.825754526707     ,0.9970424607203   ,0.7883337172411    ,0.997065763778    
20.078125,0.7592930847248841 ,0.7790516381291691 ,0.8274769487259229 ,0.8271628396826464 ,0.9970735818718925,0.7903585816867407 ,0.997079868715139 
20.15625 ,0.7611325256539562 ,0.7807663307108992 ,0.8283655060266406 ,0.8285551744597976 ,0.997106021519332 ,0.792377329491     ,0.9970948392164336
20.234375,0.7629611962023446 ,0.7824658100160685 ,0.8292666847421007 ,0.829926618243359  ,0.9971383607490866,0.7943871080175902 ,0.9971119508558675
20.3125  ,0.764779212498     ,0.78415021060185   ,0.8301874154058125 ,0.8312728238149437 ,0.9971693405865688,0.7963850517526687 ,0.9971322197415438
20.390625,0.7665866833557794 ,0.7858196677931774 ,0.8311337952763155 ,0.8325901681301038 ,0.9971980494718418,0.7983683017180861 ,0.9971562269122173
20.46875 ,0.768383710104554  ,0.7874743176698062 ,0.8321109394359437 ,0.8338758856291915 ,0.9972240447156898,0.800334025105829  ,0.9971840123448664
20.546875,0.7701703864251445 ,0.7891142970492422 ,0.8331228622856245 ,0.8351281718059356 ,0.9972473898543381,0.802279434996038  ,0.9972150559460254
20.625   ,0.7719467981996    ,0.7907397434643    ,0.8341723930661    ,0.8363462535224    ,0.9972686042676   ,0.8042018100152    ,0.997248348339    
20.703125,0.7737130233767792 ,0.7923507951372473 ,0.8352611279456336 ,0.8375304236455732 ,0.9972885362118781,0.806098513784516  ,0.99728253939281  
20.78125 ,0.7754691318545015 ,0.7939475909510765 ,0.8363894200421914 ,0.8386820387487351 ,0.9973081830590148,0.8079670140091344 ,0.9973161397617992
20.859375,0.7772151853828411 ,0.795530270418927  ,0.8375564075548473 ,0.8398034798107992 ,0.997328490950011 ,0.8098049010540834 ,0.9973477422888243
20.9375  ,0.7789512374898563 ,0.7970989736523313 ,0.8387600789218125 ,0.8408980770978188 ,0.997350168940475 ,0.8116099058565875 ,0.997376227299225 
21.015625,0.7806773334330936 ,0.7986538413288461 ,0.839997372739354  ,0.8419700015841863 ,0.9973735496198128,0.8133799170285605 ,0.9974009189900325
21.09375 ,0.7823935101792195 ,0.8001950146603336 ,0.8412643090386868 ,0.8430241264016665 ,0.9973985197652594,0.8151129970025602 ,0.9974216686518875
21.171875,0.7840997964137676 ,0.801722635361402  ,0.8425561474707179 ,0.8440658628421001 ,0.9974245323087773,0.8168073970876433 ,0.9974388528927302
21.25    ,0.7857962125826    ,0.8032368456202    ,0.8438675670667    ,0.8451009763055    ,0.9974506969004   ,0.8184615713028    ,0.997453289188    
21.328125,0.7874827709673206 ,0.8047377880689796 ,0.845192861501661  ,0.8461353883235391 ,0.9974759329875711,0.8200741888692716 ,0.997466084629897 
21.40625 ,0.7891594757950929 ,0.8062256057573516 ,0.8465261432541374 ,0.8471749713000358 ,0.9974991588562414,0.8216441452497757 ,0.9974784444098266
21.484375,0.7908263233835867 ,0.8077004421259625 ,0.8478615497339699 ,0.848225342929537  ,0.9975194841994929,0.8231705716332416 ,0.9974914726313671
21.5625  ,0.7924833023218812 ,0.809162440981375  ,0.8491934443536313 ,0.8492916673324812 ,0.9975363734223438,0.8246528427737125 ,0.9975059985881125
21.640625,0.7941303936861446 ,0.8106117464706641 ,0.85051660566055   ,0.8503784697956779 ,0.9975497520453991,0.8260905831059979 ,0.997522456641322 
21.71875 ,0.7957675712904836 ,0.8120485030568414 ,0.851826398019111  ,0.8514894716250906 ,0.997560038293407 ,0.8274836710632851 ,0.9975408382491937
21.796875,0.7973948019705509 ,0.8134728554917051 ,0.8531189179421821 ,0.8526274509935096 ,0.9975680945757058,0.8288322415430139 ,0.9975607221822012
21.875   ,0.7990120459004    ,0.8148849487883    ,0.8543911109733    ,0.8537941348578    ,0.997575106913    ,0.8301366864673    ,0.9975813756836   
21.953125,0.8006192569388983 ,0.8162849281898114 ,0.8556408550269055 ,0.8549901260022732 ,0.9975824122329743,0.831397653401714  ,0.9976019075058425
22.03125 ,0.8022163830045672 ,0.8176729391355343 ,0.8568670072587461 ,0.8562148680891735 ,0.9975913018624836,0.8326160422063227 ,0.9976214453454133
22.109375,0.8038033664769132 ,0.8190491272225839 ,0.8580694127941869 ,0.8574666503359902 ,0.9976028331528656,0.8337929996989937 ,0.9976393065542121
22.1875  ,0.8053801446208125 ,0.8204136381629125 ,0.8592488750288312 ,0.8587426520366187 ,0.9976176794469875,0.8349299123312875 ,0.9976551326610062
22.265625,0.8069466500331282 ,0.8217666177352583 ,0.8604070885841976 ,0.8600390257667088 ,0.997636041902858 ,0.836028396884825  ,0.9976689648283229
22.34375 ,0.8085028111066578 ,0.8231082117313437 ,0.8615465373816765 ,0.8613510167213516 ,0.9976576362127688,0.8370902892133555 ,0.9976812477165343
22.421875,0.8100485525116279 ,0.8244385658962223 ,0.8626703616184639 ,0.8626731142978269 ,0.9976817547918637,0.8381176310721843 ,0.9976927615433416
22.5     ,0.8115837956897    ,0.8257578258637    ,0.8637821986072    ,0.8639992308428    ,0.9977073926681   ,0.8391126550949    ,0.9977044943052   
22.578125,0.8131084593604051 ,0.8270661370852822 ,0.8648860034916022 ,0.8653229014228563 ,0.997733415113008 ,0.840077767994853  ,0.9977174761538855
22.65625 ,0.8146224600363093 ,0.8283636447549844 ,0.8659858566713469 ,0.8666374976343952 ,0.997758738683186 ,0.8410155320945336 ,0.9977326041189648
22.734375,0.8161257125451652 ,0.8296504937287655 ,0.8670857653600913 ,0.8679364478657844 ,0.9977824957774419,0.8419286453006138 ,0.9977504867752776
22.8125  ,0.8176181305577312 ,0.8309268284407937 ,0.8681894670228375 ,0.8692134560887188 ,0.9978041562396063,0.8428199196727876 ,0.9977713348839312
22.890625,0.8190996271187619 ,0.8321927928162776 ,0.8693002424632451 ,0.8704627112128286 ,0.9978235872877205,0.8436922587492394 ,0.9977949161596853
22.96875 ,0.8205701151800758 ,0.8334485301822782 ,0.8704207460705501 ,0.8716790792907094 ,0.9978410437686891,0.8445486338167774 ,0.9978205815007594
23.046875,0.8220295081339197 ,0.8346941831775276 ,0.8715528601575585 ,0.8728582714246886 ,0.9978570925844896,0.8453920593332144 ,0.9978473581024997
23.125   ,0.8234777203455    ,0.835929893661     ,0.8726975794908    ,0.8739969810524    ,0.9978724861047   ,0.8462255677233    ,0.9978740938694   
23.203125,0.8249146676827833 ,0.837155802622483  ,0.8738549310003666 ,0.8750929853967461 ,0.9978880076645273,0.8470521837888807 ,0.9978996292543882
23.28125 ,0.826340268042832  ,0.8383720500933876 ,0.8750239323337086 ,0.876145207189307  ,0.9979043164950282,0.847874898981029  ,0.9979229684335923
23.359375,0.8277544418733981 ,0.8395787750599694 ,0.876202591449443  ,0.8771537342474593 ,0.9979218190048362,0.8486966457911379 ,0.9979434222200141
23.4375  ,0.82915711268785   ,0.8407761153794749 ,0.8773879478124874 ,0.8781197961310687 ,0.9979405883665126,0.84952027252025   ,0.9979607001702062
23.515625,0.8305482075721022 ,0.8419642076989208 ,0.8785761541210746 ,0.8790456987442686 ,0.9979603457723681,0.8503485186862608 ,0.9979749380450134
23.59375 ,0.8319276576826087 ,0.8431431873776977 ,0.8797625958493579 ,0.8799347194123992 ,0.9979805060066914,0.8511839913216672 ,0.9979866576375562
23.671875,0.8332953987332942 ,0.844313188412749  ,0.8809420443250876 ,0.880790966549261  ,0.9980002789579241,0.8520291424069835 ,0.9979966671519314
23.75    ,0.8346513714692    ,0.8454743433677    ,0.8821088376725    ,0.8816192094393    ,0.9980188092442   ,0.8528862476721    ,0.9980059199203   
23.828125,0.8359955221261904 ,0.8466267833043578 ,0.8832570827393237 ,0.8824246848947818 ,0.9980353297895632,0.8537573869818964 ,0.998015355736146 
23.90625 ,0.8373278028735671 ,0.8477706377171382 ,0.8843808702115757 ,0.8832128884961039 ,0.998049303010507 ,0.854644426502814  ,0.9980257514065024
23.984375,0.8386481722374461 ,0.8489060344691189 ,0.8854744944927632 ,0.8839893587669337 ,0.998060525544672 ,0.8555490028302373 ,0.9980376049869373
24.0625  ,0.8399565955041438 ,0.85003309972905   ,0.8865326696492812 ,0.8847594629484813 ,0.9980691787535813,0.8564725092300687 ,0.998051071978825 
24.140625,0.8412530450990156 ,0.8511519579095911 ,0.8875507328036635 ,0.885528192978126  ,0.9980758164318632,0.8574160841271075 ,0.9980659626582539
24.21875 ,0.8425375009409797 ,0.8522627316047672 ,0.8885248267878985 ,0.8862999798748923 ,0.9980812916966805,0.8583806019519875 ,0.9980817992316773
24.296875,0.8438099507680354 ,0.8533655415266331 ,0.8894520546622819 ,0.8870785339648136 ,0.998086635120984 ,0.8593666664291796 ,0.9980979214541253
24.375   ,0.8450703904334    ,0.8544605064407    ,0.8903305997859    ,0.8878667173226    ,0.9980929041412   ,0.8603746063716    ,0.9981136213877   
24.453125,0.8463188241696963 ,0.8555477430988656 ,0.8911598064970513 ,0.888666453450826  ,0.9981010283147904,0.861404474021119  ,0.9981282834028548
24.53125 ,0.8475552648183922 ,0.8566273661700367 ,0.8919402180472852 ,0.8894786776554195 ,0.9981116754106313,0.8624560459515765 ,0.998141505064939 
24.609375,0.8487797340236813 ,0.8576994881681875 ,0.8926735701326399 ,0.8903033298894112 ,0.9981251595917752,0.863528826533913  ,0.9981531781866068
24.6875  ,0.8499922623893812 ,0.8587642193775    ,0.8933627401982313 ,0.8911393900208687 ,0.9981414057519875,0.8646220539351499 ,0.9981635164240937
24.765625,0.8511928895960233 ,0.8598216677749126 ,0.8940116544646681 ,0.8919849537223529 ,0.9981599746711806,0.8657347086059853 ,0.9981730250675486
24.84375 ,0.852381664479082  ,0.8608719389507125 ,0.8946251563485937 ,0.892837345464125  ,0.9981801436310758,0.8668655241906094 ,0.9981824185639391
24.921875,0.85355864506567   ,0.861915136026738  ,0.8952088415303742 ,0.893693263531902  ,0.9982010281671725,0.8680130007679273 ,0.998192500140579 
25.      ,0.8547238985703    ,0.8629513595742    ,0.8957688662472    ,0.8945489506637    ,0.9982217241989   ,0.8691754203155    ,0.9982040241856   
25.078125,0.8558775013493106 ,0.8639807075318177 ,0.8963117364803518 ,0.8954003828307001 ,0.9982414468937304,0.8703508642642638 ,0.9982175647967954
25.15625 ,0.8570195388130836 ,0.8650032751233828 ,0.8968440864468445 ,0.8962434679467656 ,0.9982596437150805,0.8715372329888469 ,0.9982334126915031
25.234375,0.8581501052974736 ,0.8660191547792621 ,0.8973724552036461 ,0.8970742459056686 ,0.9982760639118282,0.8727322670626355 ,0.9982515177596814
25.3125  ,0.8592693038943999 ,0.8670284360598375 ,0.8979030702048437 ,0.8978890813185625 ,0.99829077438125  ,0.8739335700787688 ,0.9982714867773438
25.390625,0.8603772462418763 ,0.8680312055843645 ,0.898441646307068  ,0.8986848406757485 ,0.9983041210287504,0.8751386328256804 ,0.9982926365516765
25.46875 ,0.8614740522752953 ,0.8690275469647547 ,0.8989932080345016 ,0.8994590463460868 ,0.9983166439098109,0.8763448585834415 ,0.9983140935710398
25.546875,0.8625598499409106 ,0.8700175407468287 ,0.8995619418877222 ,0.9002100008550439 ,0.9983289620334873,0.8775495892937502 ,0.9983349236335187
25.625   ,0.8636347748718    ,0.871001264359     ,0.9001510842036    ,0.9009368761636    ,0.9983416484871   ,0.8787501323435    ,0.9983542701737   
25.703125,0.8646989700297787 ,0.8719787920704198 ,0.9007628485510044 ,0.9016397641873807 ,0.9983551177572181,0.879943787689268  ,0.9983714788360847
25.78125 ,0.8657525853129734 ,0.8729501949575266 ,0.9013983949738695 ,0.9023196864672383 ,0.9983695446256946,0.8811278750475453 ,0.9983861884222249
25.859375,0.8667957771320465 ,0.8739155408816837 ,0.9020578416562943 ,0.90297856262936   ,0.9983848282920539,0.8822997608688047 ,0.9983983741537472
25.9375  ,0.8678287079561    ,0.8748748944768    ,0.9027403177915125 ,0.9036191390595    ,0.9984006074227687,0.8834568848158437 ,0.9984083372626062
26.015625,0.8688515458318464 ,0.8758283171471796 ,0.9034440547488415 ,0.9042448808853957 ,0.9984163230358876,0.8845967854726896 ,0.99841664383197  
26.09375 ,0.869864463875786  ,0.8767758670760867 ,0.904166511058811  ,0.9048598319279297 ,0.9984313179782055,0.885717125014629  ,0.9984240240928384
26.171875,0.8708676397439172 ,0.8777175992440818 ,0.9049045253490527 ,0.9054684486425051 ,0.9984449556066298,0.886815712587023  ,0.9984312496571867
26.25    ,0.8718612550807    ,0.8786957212975    ,0.9056544902589    ,0.906075415163     ,0.9984567371612   ,0.8878905261498    ,0.9984390094162   
26.328125,0.8728454949484445 ,0.8796817885327566 ,0.9064125395293509 ,0.9066854473696764 ,0.998466397677966 ,0.888939732565619  ,0.9984478045848688
26.40625 ,0.8738205472421656 ,0.8806582548654625 ,0.9071747399812445 ,0.9073030943580305 ,0.9984739640569953,0.8899617057284367 ,0.9984578797027812
26.484375,0.8747866020906567 ,0.8816249801458812 ,0.9079372799632016 ,0.9079325458026297 ,0.998479765364347 ,0.8909550425484144 ,0.9984692000176715
26.5625  ,0.8757438512472813 ,0.8825818385839    ,0.9086966460725187 ,0.9085774534539375 ,0.9984843935235376,0.891918576635     ,0.9984814776529063
26.640625,0.876692487473997  ,0.8835287197585321 ,0.9094497805299956 ,0.9092407744111112 ,0.998488620835485 ,0.8928513895416108 ,0.9984942407066075
26.71875 ,0.8776327039204063 ,0.8844655295278961 ,0.9101942124826625 ,0.9099246428941938 ,0.9984932878938952,0.8937528194582188 ,0.9985069323072469
26.796875,0.8785646935025841 ,0.8853921908395411 ,0.9109281577042642 ,0.9106302760184178 ,0.998499180245259 ,0.8946224672671944 ,0.9985190218402903
26.875   ,0.8794886482832    ,0.8863086444448    ,0.9116505825748    ,0.9113579176408    ,0.9985069138217   ,0.8954601998961    ,0.9985301088389   
26.953125,0.8804047588573122 ,0.8872148495165516 ,0.9123612298292257 ,0.912106822715905  ,0.998516847512385 ,0.8962661509318376 ,0.9985400016216724
27.03125 ,0.8813132137465266 ,0.8881107841741696 ,0.9130606052805835 ,0.912875282861039  ,0.9985290365530414,0.897040718483675  ,0.9985487573556538
27.109375,0.8822141988052772 ,0.8889964459126709 ,0.9137499264524227 ,0.9136606920803829 ,0.9985432335442   ,0.8977845603053873 ,0.998556676992974 
27.1875  ,0.8831078966414313 ,0.8898718519357063 ,0.9144310357889626 ,0.91445964984      ,0.9985589360091688,0.898498586218725  ,0.9985642563118062
27.265625,0.8839944860552313 ,0.8907370393900631 ,0.9151062826900609 ,0.9152680970907597 ,0.9985754718188471,0.8991839479025893 ,0.9985721017566795
27.34375 ,0.8848741414997617 ,0.8915920654961782 ,0.9157783800353766 ,0.9160814793938906 ,0.9985921077935438,0.8998420261385883 ,0.9985808256783453
27.421875,0.8857470325647584 ,0.8924370075708449 ,0.9164502420355898 ,0.9168949301142126 ,0.9986081633033859,0.9004744156360953 ,0.9985909389421446
27.5     ,0.8866133234873    ,0.8932719629333    ,0.9171248111018    ,0.9177034657727    ,0.9986231102891   ,0.9010829075813    ,0.9986027591784   
27.578125,0.8874731726920423 ,0.8940970486918358 ,0.9178048819581146 ,0.918502185099147  ,0.9986366438001169,0.9016694700844172 ,0.9986163501822057
27.65625 ,0.8883267323626086 ,0.8949124014002774 ,0.9184929313676844 ,0.9192864631662554 ,0.9986487124260492,0.9022362267285023 ,0.9986315026252156
27.734375,0.8891741480473397 ,0.895718176581186  ,0.9191909616248075 ,0.9200521321883532 ,0.9986595049247203,0.9027854334418232 ,0.9986477592683523
27.8125  ,0.8900155583001126 ,0.8965145481093938 ,0.9199003653587687 ,0.920795641168775  ,0.9986693967700062,0.9033194539448313 ,0.9986644804181812
27.890625,0.8908510943592759 ,0.8973017074528643 ,0.9206218182491207 ,0.9215141875150914 ,0.9986788669778598,0.9038407340416504 ,0.9986809387371085
27.96875 ,0.8916808798646032 ,0.8980798627699328 ,0.9213552049949203 ,0.9222058149977125 ,0.9986884003444008,0.9043517750405422 ,0.9986964277658812
28.046875,0.8925050306150172 ,0.8988492378637433 ,0.922099582353724  ,0.9228694739561029 ,0.9986983923632741,0.9048551066065303 ,0.9987103664282901
28.125   ,0.8933236543669    ,0.8996100709985    ,0.9228531813751    ,0.9235050413498    ,0.9987090732728   ,0.9053532593555    ,0.9987223826709   
28.203125,0.8941368506745426 ,0.9003626135839264 ,0.9236134491289226 ,0.9241133000912486 ,0.9987204641044997,0.9058487375044365 ,0.9987323630544919
28.28125 ,0.8949447107723242 ,0.9011071287388938 ,0.9243771283838781 ,0.924695878953364  ,0.9987323719032258,0.9063439918945219 ,0.9987404609084195
28.359375,0.8957473175004914 ,0.9018573710993464 ,0.9251403719224675 ,0.9252551561321818 ,0.998744424471426 ,0.9068413936983892 ,0.9987470625805511
28.4375  ,0.8965447452728812 ,0.9026419665181062 ,0.9258988865204812 ,0.9257941312301312 ,0.9987561382020188,0.9073432091141812 ,0.9987527182130626
28.515625,0.897337060087988  ,0.9034212095160548 ,0.926648100217576  ,0.9263162718483847 ,0.9987670069736306,0.9078515753346866 ,0.9987580491699355
28.59375 ,0.8981243195824585 ,0.9041950893519204 ,0.9273833453765281 ,0.926825342135832  ,0.9987765965863517,0.908368478067336  ,0.9987636478359391
28.671875,0.8989065731272955 ,0.9049635946169873 ,0.9281000492539149 ,0.9273252214509838 ,0.9987846283996258,0.908895730853119  ,0.9987699864146639
28.75    ,0.8996838619647    ,0.9057267133476    ,0.9287939234179    ,0.9278197217056    ,0.9987910377796   ,0.9094349564164    ,0.9987773494833   
28.828125,0.9004562193870669 ,0.9064844331368034 ,0.9294611433712987 ,0.9283124119744596 ,0.9987959973155093,0.9099875702479817 ,0.9987858007637361
28.90625 ,0.9012236709558444 ,0.9072367412455171 ,0.930098510173907  ,0.9288064585525313 ,0.9987999007630555,0.9105547665958984 ,0.9987951886120929
28.984375,0.9019862347592158 ,0.9079836247101173 ,0.9307035866662279 ,0.9293044878550087 ,0.9988033102798954,0.9111375070135743 ,0.9988051881374048
29.0625  ,0.9027439217082875 ,0.908725070447775  ,0.9312748020908624 ,0.9298084784072312 ,0.9988068756340188,0.9117365115814312 ,0.9988153717392563
29.140625,0.9034967358697668 ,0.9094610653574895 ,0.9318115203611471 ,0.9303196867310792 ,0.9988112386582083,0.9123522528896056 ,0.9988252952371226
29.21875 ,0.9042446748332094 ,0.9101915964170352 ,0.9323140689223001 ,0.9308386102759265 ,0.9988169385584281,0.9129849528406054 ,0.9988345843835672
29.296875,0.9049877301125073 ,0.9109166507747851 ,0.9327837270002359 ,0.931364988713881  ,0.9988243333925277,0.9136345822973713 ,0.9988430067838489
29.375   ,0.9057258875777    ,0.9116362158364    ,0.9332226739009    ,0.9318978430709    ,0.9988335502119   ,0.9143008635744    ,0.9988505169911   
29.453125,0.9064591279166536 ,0.9123502793462435 ,0.9336338998930699 ,0.9324355503227478 ,0.9988444714987349,0.9149832757391282 ,0.9988572673220953
29.53125 ,0.9071874271228508 ,0.9130588294631188 ,0.9340210839229953 ,0.9329759493805743 ,0.9988567594816687,0.9156810626571757 ,0.9988635828929031
29.609375,0.9079107570076307 ,0.9137618548307004 ,0.9343884439148314 ,0.933516472907563  ,0.9988699136991527,0.9163932436930027 ,0.9988699054973921
29.6875  ,0.908629085733375  ,0.9144593446436999 ,0.9347405666537    ,0.9340542981772625 ,0.9988833518212562,0.9171186269378813 ,0.9988767162125438
29.765625,0.9093423783644667 ,0.9151512887082138 ,0.9350822251241907 ,0.9345865093243544 ,0.9988965001421032,0.9178558248172732 ,0.9988844501329367
29.84375 ,0.9100505974337781 ,0.915837677498918  ,0.9354181916868562 ,0.9351102628352    ,0.9989088788560281,0.9186032718965891 ,0.998893417855896 
29.921875,0.9107537035197426 ,0.916518502212841  ,0.9357530555800808 ,0.9356229480370152 ,0.9989201684331629,0.9193592446794607 ,0.9989037470754785
30.      ,0.9114516558314    ,0.9171937548195    ,0.9360910529303    ,0.9361223346506    ,0.9989302468526   ,0.9201218831666    ,0.9989153541556   
30.078125,0.9121444127975578 ,0.9178634281105916 ,0.9364359167651768 ,0.9366067001677133 ,0.9989391925344822,0.920889213922701  ,0.998927950437306 
30.15625 ,0.912831932655682  ,0.9185275157466055 ,0.9367907534812335 ,0.937074930860018  ,0.9989472536478546,0.9216591743753969 ,0.9989410821894937
30.234375,0.9135141740374625 ,0.9191860123042686 ,0.9371579508940363 ,0.9375265915485742 ,0.9989547900810337,0.9224296380570268 ,0.9989541975130747
30.3125  ,0.9141910965462438 ,0.9198389133237813 ,0.937539121417825  ,0.9379619608425812 ,0.9989621987999687,0.9231984404833312 ,0.9989667290892562
30.390625,0.9148626613231452 ,0.9204862153559579 ,0.9379350822172738 ,0.9383820302527679 ,0.9989698358771982,0.9239634053566852 ,0.9989781791526027
30.46875 ,0.9155288315975952 ,0.9211279160119016 ,0.9383458723913632 ,0.9387884673587883 ,0.9989779487396617,0.9247223707749578 ,0.9989881928565071
30.546875,0.9161895732190853 ,0.9217640140129775 ,0.9387708054739341 ,0.9391835449737196 ,0.9989866301636927,0.925473215128046  ,0.9989966082863889
30.625   ,0.9168448551657    ,0.9223945092431    ,0.9392085538896    ,0.9395700398764    ,0.9989958016115   ,0.9262138823665    ,0.9990034753843   
30.703125,0.917494650026943  ,0.9230194028031398 ,0.9396572605142487 ,0.9399511061574715 ,0.9990052283516331,0.9269424063382058 ,0.9990090412814698
30.78125 ,0.9181389344568203 ,0.9236386970667352 ,0.9401146712636773 ,0.940330129423875  ,0.9990145633191296,0.9276569338993274 ,0.9990137051018867
30.859375,0.9187776895951528 ,0.9242523957374678 ,0.9405782817201566 ,0.9407105690032225 ,0.9990234117869529,0.9283557465239516 ,0.9990179502602088
30.9375  ,0.9194109014532813 ,0.92486050390825   ,0.9410454902250938 ,0.941095795840075  ,0.9990314054404938,0.9290372801581563 ,0.9990222658323812
31.015625,0.9200385612631281 ,0.9254630281205385 ,0.9415137496738937 ,0.9414889339429615 ,0.9990382729531726,0.9297001430859183 ,0.9990270701308284
31.09375 ,0.9206606657859141 ,0.9260599764244477 ,0.9419807104270657 ,0.941892713039268  ,0.9990438948587086,0.9303431316015351 ,0.9990326489702829
31.171875,0.9212772175798336 ,0.9266513584386334 ,0.9424443472992204 ,0.9423093395124672 ,0.9990483332765923,0.9309652433126635 ,0.9990391183887054
31.25    ,0.9218882252252    ,0.9272371854086    ,0.9429030644719    ,0.9427403917883    ,0.9990518313599   ,0.9315656879244    ,0.999046417294    
31.328125,0.9224937035050935 ,0.9278174702640491 ,0.943355773355668  ,0.9431867451295498 ,0.9990547824722502,0.932143895389325  ,0.9990543303811394
31.40625 ,0.9230936735410656 ,0.9283922276733811 ,0.9438019398428531 ,0.9436485293481632 ,0.999057674176943 ,0.9326995213410757 ,0.9990625365766289
31.484375,0.9236881628836349 ,0.9289614740955925 ,0.944241598959582  ,0.9441251213622793 ,0.9990610162905761,0.9332324497545919 ,0.9990706740774047
31.5625  ,0.9242772055566251 ,0.9295252278278    ,0.9446753366039687 ,0.9446151728084188 ,0.9990652648357687,0.9337427928196438 ,0.9990784104315438
31.640625,0.9248608420564939 ,0.9300835090495965 ,0.9451042397146041 ,0.9451166712458335 ,0.9990707543008333,0.9342308880378708 ,0.9990855054890065
31.71875 ,0.9254391193052484 ,0.9306363398615086 ,0.9455298178131047 ,0.9456270318649149 ,0.9990776491222039,0.9346972925904656 ,0.9990918564857882
31.796875,0.9260120905594276 ,0.9311837443194817 ,0.9459539003062356 ,0.9461432151460343 ,0.9990859220147388,0.9351427750568024 ,0.9990975177532775
31.875   ,0.9265798152749    ,0.931725748463     ,0.9463785151412    ,0.9466618646903    ,0.9990953622712   ,0.9355683045936    ,0.9991026919822   
31.953125,0.927142358928878  ,0.9322623803385459 ,0.9468057553471033 ,0.9471794584831748 ,0.9991056121937417,0.9359750377162017 ,0.9991076948651044
32.03125 ,0.9276997928010946 ,0.9327936700166055 ,0.9472376405896711 ,0.9476924662415875 ,0.9991162252397093,0.9363643028548226 ,0.9991128994519876
32.109375,0.9282521937148683 ,0.9333196496036996 ,0.947675981111831  ,0.9481975052390319 ,0.9991267360335491,0.9367375828835831 ,0.9991186699347837
32.1875  ,0.9287996437407062 ,0.9338403532484875 ,0.9481222512986    ,0.9486914871210376 ,0.9991367306388125,0.9370964958485563 ,0.9991252962710749
32.265625,0.9293422298645405 ,0.9343558171424997 ,0.9485774796057792 ,0.9491717487176506 ,0.9991459056989481,0.9374427741434536 ,0.9991329408111843
32.34375 ,0.9298800436225344 ,0.9348660795161102 ,0.9490421607598281 ,0.9496361606856532 ,0.9991541071439008,0.9377782424011868 ,0.9991416059661126
32.421875,0.9304131807057118 ,0.9353711806289792 ,0.9495161949840302 ,0.9500832089628092 ,0.9991613427635418,0.938104794387361  ,0.9991511283328985
32.5     ,0.9309417405362    ,0.9358711627565    ,0.9499988576397    ,0.9505120453865    ,0.9991677674232   ,0.9384243691935    ,0.9991612002134   
32.578125,0.9314658258188075 ,0.9363660701714613 ,0.9504888011054596 ,0.9509225053998926 ,0.9991736442933139,0.9387389270362477 ,0.9991714148685041
32.65625 ,0.9319855420712867 ,0.9368559491217828 ,0.9509840890678843 ,0.9513150924284742 ,0.9991792894040759,0.9390504249739804 ,0.9991813279299437
32.734375,0.9325009971349602 ,0.9373408478042969 ,0.9514822617512082 ,0.9516909301687129 ,0.999185009470477 ,0.9393607928496568 ,0.9991905247851722
32.8125  ,0.9330123006709    ,0.9378208163345938 ,0.95198042900865   ,0.9520516856515563 ,0.9991910438463875,0.9396719097651249 ,0.9991986828715125
32.890625,0.933519563643733  ,0.938295906713703  ,0.9524753867873094 ,0.9523994673779604 ,0.9991975205351618,0.9399855813822418 ,0.9992056187969389
32.96875 ,0.9340228977969851 ,0.9387661727905946 ,0.9529637512709953 ,0.9527367040468844 ,0.9992044335912554,0.9403035183328999 ,0.9992113128413446
33.046875,0.9345224151236223 ,0.9392316702212331 ,0.95344210409984   ,0.9530660103332917 ,0.9992116454607557,0.9406273160003251 ,0.9992159072375686
33.125   ,0.9350182273349    ,0.9396924564238    ,0.9539071415017    ,0.9533900467612    ,0.999218913486    ,0.9409584359192    ,0.9992196790058   
33.203125,0.9355104453307845 ,0.9401485905302983 ,0.9543558199649412 ,0.9537113809471269 ,0.9992259356700556,0.9412981890110299 ,0.9992229922969433
33.28125 ,0.9359991786760625 ,0.9406001333332672 ,0.9547854912681758 ,0.9540323573370703 ,0.9992324075623125,0.941647720852264  ,0.9992262384878898
33.359375,0.9364845350845286 ,0.9410471472290365 ,0.9551940202210895 ,0.9543549820436453 ,0.9992380803115558,0.9420079991414877 ,0.9992297741259757
33.4375  ,0.9369666199151687 ,0.9414896961554812 ,0.9555798793653062 ,0.954680828526825  ,0.999242809827275 ,0.9423798035041375 ,0.9992338669684437
33.515625,0.937445535682854  ,0.9419278455252537 ,0.9559422160567611 ,0.9550109687145031 ,0.9992465885824721,0.94276371774258   ,0.9992386587944349
33.59375 ,0.9379213815866383 ,0.9423616621538805 ,0.9562808887574743 ,0.9553459327786836 ,0.9992495545975437,0.9431601246109351 ,0.9992441506839703
33.671875,0.9383942530580316 ,0.9427912141821578 ,0.9565964709355542 ,0.9556856992368175 ,0.9992519760345835,0.9435692031597455 ,0.9992502125767481
33.75    ,0.9388642413332    ,0.9432165709934    ,0.9568902225951    ,0.9560297154474    ,0.9992542139398   ,0.9439909286684    ,0.9992566148163   
33.828125,0.9393314330490468 ,0.9436378031247992 ,0.9571640310895998 ,0.9563769469597811 ,0.9992566693148697,0.944425075148067  ,0.9992630757309016
33.90625 ,0.9397959098678516 ,0.9440549821734469 ,0.95742032439765   ,0.9567259526741781 ,0.9992597232573109,0.9448712203714899 ,0.9992693167260555
33.984375,0.9402577481305566 ,0.9444681806966051 ,0.9576619613763183 ,0.9570749814448862 ,0.9992636799906207,0.9453287533520487 ,0.9992751152539049
34.0625  ,0.9407170185420562 ,0.9448774721071499 ,0.9578921046301437 ,0.9574220846627313 ,0.999268722025675 ,0.9457968841675938 ,0.9992803465715312
34.140625,0.9411737858890518 ,0.9452829305636681 ,0.9581140824201707 ,0.9577652385731471 ,0.9992748845952203,0.9462746559929704 ,0.9992850072578139
34.21875 ,0.9416281087922266 ,0.9456846308567421 ,0.9583312465120899 ,0.9581024696314125 ,0.9992820532413086,0.9467609591818757 ,0.9992892166794187
34.296875,0.9420800394943882 ,0.946082648290799  ,0.9585468329632364 ,0.9584319761062416 ,0.9992899845804718,0.9472545472069649 ,0.9992931964045553
34.375   ,0.9425296236857    ,0.9464770585634    ,0.9587638325914    ,0.9587522394078    ,0.9992983464677   ,0.9477540542496    ,0.9992972313195   
34.453125,0.9429769003662198 ,0.9468679376414016 ,0.9589848772712193 ,0.9590621192237558 ,0.9993067706701596,0.9482580142030852 ,0.9993016192718376
34.53125 ,0.9434219017473648 ,0.9472553616358969 ,0.9592121472892735 ,0.9593609274666586 ,0.999314909241332 ,0.9487648808375844 ,0.9993066179413649
34.609375,0.9438646531930581 ,0.9476394066768353 ,0.9594473038350265 ,0.9596484771902476 ,0.999322485362993 ,0.9492730488602324 ,0.9993123980256438
34.6875  ,0.9443051731998937 ,0.9480201487865313 ,0.9596914493325437 ,0.9599251040061625 ,0.9993293305436812,0.9497808755872063 ,0.9993190106786375
34.765625,0.9447434734184164 ,0.9483976637551921 ,0.9599451168519455 ,0.9601916589966201 ,0.9993354025049156,0.9502867029418814 ,0.9993263746732927
34.84375 ,0.9451795587135289 ,0.948772027017907  ,0.9602082883162141 ,0.9604494736307031 ,0.9993407814303374,0.9507888794848632 ,0.9993342854007548
34.921875,0.9456134272659776 ,0.9491555240493361 ,0.9604804397349908 ,0.960700298671926  ,0.9993456459193669,0.9512857821846458 ,0.9993424441477703
35.      ,0.9460450707133    ,0.9495601446381    ,0.9607606103383    ,0.9609462204056    ,0.9993502333518   ,0.9517758376406    ,0.9993505027282   
35.078125,0.9464744743298529 ,0.9499614173746417 ,0.961047491294993  ,0.9611895586906462 ,0.9993547918857824,0.9522575424791199 ,0.9993581160451207
35.15625 ,0.9469016172462047 ,0.9503593445277727 ,0.9613395287642289 ,0.9614327522554937 ,0.9993595325598101,0.9527294826579101 ,0.9993649939268476
35.234375,0.9473264727052497 ,0.9507539292874266 ,0.9616350353814417 ,0.9616782372853329 ,0.9993645897691958,0.9531903514267785 ,0.9993709438086069
35.3125  ,0.9477490083556938 ,0.9511451757954624 ,0.961932303942625  ,0.9619283256597563 ,0.9993699967964063,0.9536389657183    ,0.9993758974573625
35.390625,0.9481691865793941 ,0.9515330891742555 ,0.9622297170586233 ,0.962185089169699  ,0.9993756803759896,0.954074280759869  ,0.9993799176708562
35.46875 ,0.9485869648524079 ,0.951917675556629  ,0.9625258468741922 ,0.9624502556907102 ,0.9993814749566532,0.9544954027269548 ,0.9993831842446789
35.546875,0.9490022961369139 ,0.9522989421173208 ,0.9628195395867899 ,0.9627251226222373 ,0.9993871539309874,0.9549015992864418 ,0.999385961932806 
35.625   ,0.9494151293014    ,0.9526768971093    ,0.9631099804011    ,0.9630104919746    ,0.9993924722366   ,0.9552923079042    ,0.9993885560411   
35.703125,0.9498254095672941 ,0.953051549903334  ,0.9633967356836702 ,0.9633066303270366 ,0.9993972128397216,0.9556671418268404 ,0.9993912632210638
35.78125 ,0.9502330789780618 ,0.9534229110327773 ,0.9636797703674922 ,0.9636132555673868 ,0.9994012290237992,0.9560258936761953 ,0.9993943256642454
35.859375,0.9506380768892022 ,0.9537909922424261 ,0.9639594400235543 ,0.9639295509321651 ,0.9994044751791905,0.9563685366264897 ,0.999397896149897 
35.9375  ,0.9510403404738875 ,0.9541558065390313 ,0.96423645841495   ,0.9642542054349937 ,0.9994070207966188,0.956695223170175  ,0.9994020194248563
36.015625,0.951439805242386  ,0.9545173682443694 ,0.9645118426781008 ,0.9645854784241396 ,0.9994090452372668,0.9570062815068234 ,0.9994066325363522
36.09375 ,0.9518364055705774 ,0.9548756930460423 ,0.964786839481111  ,0.9649212847748633 ,0.9994108141036679,0.9573022096208804 ,0.9994115835044023
36.171875,0.9522300752346261 ,0.9552307980459229 ,0.9650628365330448 ,0.9652592961813937 ,0.9994126411110503,0.9575836671477161 ,0.9994166646417878
36.25    ,0.9526207479469    ,0.9555827018019    ,0.9653412645852    ,0.9655970532237    ,0.9994148417484   ,0.9578514651532    ,0.9994216544363   
36.328125,0.953008357890568  ,0.9559314243627965 ,0.9656234955654203 ,0.9659320823686927 ,0.999417686330796 ,0.9581065539800723 ,0.9994263605803477
36.40625 ,0.9533928402472905 ,0.9562769872914281 ,0.9659107426664313 ,0.966262011869704  ,0.9994213600746836,0.9583500093428304 ,0.9994306566808266
36.484375,0.9537741317169284 ,0.9566194136766523 ,0.9662039680858632 ,0.9665846806419345 ,0.9994259365898766,0.9585830168696626 ,0.9994345063907065
36.5625  ,0.9541521710218187 ,0.9569587281306938 ,0.9665038036743937 ,0.9668982346229625 ,0.9994313688971125,0.9588068553167562 ,0.9994379709489   
36.640625,0.9545268993959796 ,0.9572949567721677 ,0.966810489031564  ,0.967201205840145  ,0.999437499150867 ,0.959022878693115  ,0.9994411990099655
36.71875 ,0.9548982610526141 ,0.9576281271937266 ,0.967123830629807  ,0.9674925703653899 ,0.9994440851634274,0.9592324975471032 ,0.999444400680247 
36.796875,0.9552662036292223 ,0.9579582684152134 ,0.9674431843882331 ,0.9677717824987895 ,0.9994508391119772,0.9594371596789667 ,0.9994478103544194
36.875   ,0.9556306786051    ,0.9582854108217    ,0.9677674628465    ,0.9680387837986    ,0.9994574719036   ,0.9596383305454    ,0.9994516448036   
36.953125,0.9559916416908566 ,0.9586095860898868 ,0.9680951667446904 ,0.9682939869329744 ,0.9994637358863929,0.9598374736258176 ,0.9994560637214361
37.03125 ,0.9563490531850422 ,0.9589308271020109 ,0.9684244394922922 ,0.9685382356612383 ,0.9994694590342812,0.9600360310188383 ,0.999461139456132 
37.109375,0.9567028782982272 ,0.9592491678525129 ,0.9687531417775942 ,0.9687727435033509 ,0.9994745653035383,0.9602354045277306 ,0.999466841057547 
37.1875  ,0.9570530874404812 ,0.9595646433467125 ,0.9690789424728062 ,0.9689990147697    ,0.9994790782855375,0.9604369374870375 ,0.9994730353247875
37.265625,0.9573996564716752 ,0.959877289496392  ,0.9693994211270511 ,0.969218752509168  ,0.9994831081339297,0.9606418975675342 ,0.9994795046688821
37.34375 ,0.957742566914247  ,0.9601871430138758 ,0.9697121767153406 ,0.9694337585809211 ,0.9994868245435164,0.9608514607810539 ,0.9994859788075453
37.421875,0.9580818061251796 ,0.9604942413067495 ,0.9700149369893298 ,0.9696458314021519 ,0.999490420835463 ,0.9610666968850362 ,0.9994921750304707
37.5     ,0.9584173674286    ,0.9607986223754    ,0.9703056627543    ,0.969856666972     ,0.9994940755868   ,0.9612885563697    ,0.9994978404133   
37.578125,0.9587492502075827 ,0.9611003247160579 ,0.9705826416834625 ,0.9700677685101496 ,0.9994979185089878,0.9615178591829137 ,0.9995027891059421
37.65625 ,0.9590774599551414 ,0.9613993872281851 ,0.9708445668581663 ,0.9702803695006609 ,0.9995020064135516,0.9617552853237531 ,0.9995069287238274
37.734375,0.959402008285075  ,0.9616958491302823 ,0.9710905960476282 ,0.970495374127768  ,0.9995063132610835,0.9620013674131122 ,0.9995102717444567
37.8125  ,0.9597229129022813 ,0.9619897498807688 ,0.9713203887885562 ,0.9707133180730813 ,0.9995107358018   ,0.9622564853177062 ,0.9995129303430375
37.890625,0.9600401975336762 ,0.9622811291067336 ,0.9715341195100949 ,0.9709343514811164 ,0.999515113612197 ,0.9625208628791541 ,0.9995150958447668
37.96875 ,0.9603538918209993 ,0.9625700265366508 ,0.9717324662226796 ,0.971158244649036  ,0.9995192598693484,0.9627945667737664 ,0.9995170064803977
38.046875,0.9606640311759386 ,0.9628564819395123 ,0.9719165755830295 ,0.9713844157247087 ,0.9995229974008591,0.9630775074966347 ,0.9995189089759665
38.125   ,0.9609706565992    ,0.963140535065     ,0.9720880063725    ,0.9716119784936    ,0.9995261936886   ,0.9633694424412    ,0.9995210204074   
38.203125,0.9612738144661033 ,0.9634222255862516 ,0.9722486545396993 ,0.9718398072365556 ,0.9995287887331327,0.9636699810163089 ,0.999523496556131 
38.28125 ,0.9615735562786016 ,0.9637015930409273 ,0.9724006638837407 ,0.9720666147320367 ,0.9995308109369977,0.9639785917151227 ,0.9995264117765922
38.359375,0.9618699383876251 ,0.9639786767711677 ,0.9725463271503045 ,0.9722910387977192 ,0.9995323782265709,0.9642946110299345 ,0.9995297533485207
38.4375  ,0.962163021687425  ,0.9642535158590062 ,0.9726879827448562 ,0.9725117323406437 ,0.9995336841364437,0.9646172540783687 ,0.9995334307899375
38.515625,0.962452871283611  ,0.9645261490566441 ,0.9728279124126222 ,0.9727274517559703 ,0.9995349711100536,0.9649456267866334 ,0.9995372980829216
38.59375 ,0.962739556138718  ,0.9647966147120601 ,0.9729682450894969 ,0.9729371386624336 ,0.9995364953933797,0.9652787394547782 ,0.9995411846309211
38.671875,0.9630231486967744 ,0.9650649506867526 ,0.9731108717054548 ,0.9731399903942921 ,0.9995384892699186,0.96561552151018   ,0.9995449293830317
38.75    ,0.9633037244908    ,0.965331194269     ,0.9732573750508    ,0.9733355153413    ,0.9995411267831   ,0.9659548372395    ,0.9995484121348   
38.828125,0.9635813617358567 ,0.9655953820804103 ,0.9734089779286034 ,0.9735235701167411 ,0.9995444984637957,0.9662955022778142 ,0.9995515766172122
38.90625 ,0.9638561409098961 ,0.9658575499792328 ,0.9735665117712546 ,0.9737043765758313 ,0.9995485990412422,0.9666363006209093 ,0.9995544414847531
38.984375,0.964128144327035  ,0.9661177329600111 ,0.9737304067634982 ,0.9738785178379163 ,0.9995533299451091,0.9669760019225185 ,0.9995570974463421
39.0625  ,0.9643974557051875 ,0.966375965053025  ,0.9739007033218375 ,0.9740469136494313 ,0.9995585159594063,0.9673133788345437 ,0.9995596911865999
39.140625,0.9646641597314026 ,0.9666322792250668 ,0.9740770836377742 ,0.9742107765570981 ,0.9995639330933933,0.9676472241465324 ,0.9995623989793773
39.21875 ,0.9649283416286476 ,0.9668867072840173 ,0.9742589209291266 ,0.9743715514119742 ,0.9995693429579382,0.9679763674887774 ,0.999565394630282 
39.296875,0.9651900867257686 ,0.9671392797896934 ,0.9744453431259947 ,0.974530841623296  ,0.9995745279713298,0.96829969136645   ,0.9995688173128033
39.375   ,0.9654494800355    ,0.9673900259735    ,0.974635307005     ,0.9746903262809    ,0.9995793217128   ,0.9686161463057    ,0.9995727448402   
39.453125,0.9657066058408342 ,0.967638973668563  ,0.97482767829385   ,0.9748516727340155 ,0.9995836296734638,0.9689247649070168 ,0.9995771769522608
39.53125 ,0.9659615472953726 ,0.9678861492530149 ,0.9750213130360523 ,0.9750164494247859 ,0.9995874373637484,0.969224674616468  ,0.9995820314776211
39.609375,0.9662143860372576 ,0.9681315776070525 ,0.9752151355354759 ,0.9751860437252743 ,0.9995908049048812,0.9695151090468209 ,0.9995871540495427
39.6875  ,0.9664652018213062 ,0.9683752820852938 ,0.9754082084891312 ,0.9753615892110125 ,0.9995938495130937,0.969795417701775  ,0.9995923397737687
39.765625,0.9667140721701427 ,0.9686172845044316 ,0.9755997914406155 ,0.9755439062588427 ,0.999596719271837 ,0.9700650739802166 ,0.9995973632718712
39.84375 ,0.9669610720472656 ,0.9688576051469945 ,0.9757893844220297 ,0.9757334590980117 ,0.9995995629664735,0.9703236813647008 ,0.9996020121562742
39.921875,0.967206273552541  ,0.969096262779315  ,0.9759767545467583 ,0.9759303315214568 ,0.9996025013061562,0.9705709777218114 ,0.9996061184740512
40.      ,0.9674497456429997 ,0.9693332746834    ,0.9761619443193    ,0.9761342224452001 ,0.9996056044906999,0.9708068376698996 ,0.9996095830458003
\end{filecontents}

\begin{tikzpicture}
  \begin{axis}[width=\linewidth, height=0.5\linewidth]

    \addplot[smooth] table[x = Time, y = lbound, col sep=comma] {low_density_stats.dat};
    \addplot[smooth] table[x = Time, y = ubound, col sep=comma] {low_density_stats.dat};

    %\addplot[pattern=north west lines, pattern color=brown!50]fill between[of=F and G];

    \addplot+[smooth, no marks] table[x = Time, y = p1, col sep=comma] {low_density_stats.dat};
    \addplot+[smooth, no marks] table[x = Time, y = p2, col sep=comma] {low_density_stats.dat};
    \addplot+[smooth, no marks] table[x = Time, y = p3, col sep=comma] {low_density_stats.dat};
    \addplot+[smooth, no marks] table[x = Time, y = p4, col sep=comma] {low_density_stats.dat};
    \addplot+[smooth, no marks] table[x = Time, y = p5, col sep=comma] {low_density_stats.dat};

  \end{axis}

\end{tikzpicture}

\end{figure}
