\section{Numerical Results}
Here we detail the results of several investigations into coupled \qd{} behavior with the model presented thusfar.
Our algorithm reliably handles tens of thousands of \qds{} and can simulate ten picoseconds of system dynamics in two days on a single processor.
We perform simulations of systems of \qds{} randomly distributed throughout a cubic simulation volume at various densities.
\Cref{table:parameters} gives approximate values of the system parameters.

\begin{table}
  \begin{ruledtabular}
    \begin{tabular}{lll}
      Quantity                 & Symbol         & Value                        \\ \hline
      Transition frequency     & $\omega_0$     & $\SI{1500}{\milli\eV}/\hbar$ \\
      Transition dipole moment & $\abs{\vb{d}}$ & \SI{10}{\elementarycharge\bohr} \\
      Relaxation times         & $T_{1}, T_{2}$ & \SIlist{10;20}{\pico\second} \\
      Laser frequency          & $\omega_L$     & $\SI{1500}{\milli\eV}/\hbar$ \\ \hline
      Speed of light           & $c$            & \SI{299.792458}{\micro\meter\per\pico\second} \\
      Reduced Planck constant  & $\hbar$        & \SI{0.65821193}{\milli\eV \pico\second} \\
      Vacuum permeability      & $\mu_0$        & \SI{2.0133545e-4}{\milli\eV \pico\second\squared \per \elementarycharge \per \micro\meter}
    \end{tabular}
  \end{ruledtabular}
  \caption{\label{table:parameters}Rough simulation parameters.}
\end{table}

\begin{figure}
  \usetikzlibrary{pgfplots.groupplots}

\begin{filecontents}{ipr_1024.dat}
0.000000000000000000e+00 0.000000000000000000e+00
2.500000000000000139e-02 3.381020394210842994e-10
5.000000000000000278e-02 1.840404700598902155e-09
7.499999999999999722e-02 8.033211570338677499e-09
1.000000000000000056e-01 2.158837546033763599e-08
1.250000000000000000e-01 4.766930931281512795e-08
1.499999999999999944e-01 9.100822592755797302e-08
1.749999999999999889e-01 1.532972601608461466e-07
2.000000000000000111e-01 2.369038891636334604e-07
2.250000000000000056e-01 3.422420040461067459e-07
2.500000000000000000e-01 4.681486278444030313e-07
2.750000000000000222e-01 6.117639690802426783e-07
2.999999999999999889e-01 7.692248697121172041e-07
3.250000000000000111e-01 9.360082609832526962e-07
3.499999999999999778e-01 1.107389336343087352e-06
3.750000000000000000e-01 1.278841105706958123e-06
4.000000000000000222e-01 1.446443866589749994e-06
4.249999999999999889e-01 1.607074230902281672e-06
4.500000000000000111e-01 1.758516512770412835e-06
4.749999999999999778e-01 1.899458037815876868e-06
5.000000000000000000e-01 2.029400021924402178e-06
5.250000000000000222e-01 2.148513335702715091e-06
5.500000000000000444e-01 2.257463509529235995e-06
5.749999999999999556e-01 2.357237091006256224e-06
5.999999999999999778e-01 2.448984700587075736e-06
6.250000000000000000e-01 2.533892668434706286e-06
6.500000000000000222e-01 2.613087782425260047e-06
6.750000000000000444e-01 2.687574294546078366e-06
6.999999999999999556e-01 2.758199257206928739e-06
7.249999999999999778e-01 2.825640380475780145e-06
7.500000000000000000e-01 2.890410412151062786e-06
7.750000000000000222e-01 2.952872560571615651e-06
8.000000000000000444e-01 3.013262491573676788e-06
8.249999999999999556e-01 3.071713496050759717e-06
8.499999999999999778e-01 3.128282391075395616e-06
8.750000000000000000e-01 3.182974453097651285e-06
9.000000000000000222e-01 3.235766212925875879e-06
9.250000000000000444e-01 3.286625301406704766e-06
9.499999999999999556e-01 3.335526797607415994e-06
9.749999999999999778e-01 3.382465748892702490e-06
1.000000000000000000e+00 3.427465741071434452e-06
1.024999999999999911e+00 3.470583607187958941e-06
1.050000000000000044e+00 3.511910568371667471e-06
1.074999999999999956e+00 3.551570282031525291e-06
1.100000000000000089e+00 3.589714412759225162e-06
1.125000000000000000e+00 3.626516425651871156e-06
1.149999999999999911e+00 3.662164324548305636e-06
1.175000000000000044e+00 3.696853022110277096e-06
1.199999999999999956e+00 3.730776944928715019e-06
1.225000000000000089e+00 3.764123359608620087e-06
1.250000000000000000e+00 3.797066771237455826e-06
1.274999999999999911e+00 3.829764608610482582e-06
1.300000000000000044e+00 3.862354283488923443e-06
1.324999999999999956e+00 3.894951602124952991e-06
1.350000000000000089e+00 3.927650421283162413e-06
1.375000000000000000e+00 3.960523379467704302e-06
1.399999999999999911e+00 3.993623496010775820e-06
1.425000000000000044e+00 4.026986413490241806e-06
1.449999999999999956e+00 4.060633059392408650e-06
1.475000000000000089e+00 4.094572516820035868e-06
1.500000000000000000e+00 4.128804918072026195e-06
1.524999999999999911e+00 4.163324205460391253e-06
1.550000000000000044e+00 4.198120637795288435e-06
1.574999999999999956e+00 4.233182956060645846e-06
1.600000000000000089e+00 4.268500155741672755e-06
1.625000000000000000e+00 4.304062844353098716e-06
1.649999999999999911e+00 4.339864189782357065e-06
1.675000000000000044e+00 4.375900487067329453e-06
1.699999999999999956e+00 4.412171387983962120e-06
1.725000000000000089e+00 4.448679848966397704e-06
1.750000000000000000e+00 4.485431858933632346e-06
1.774999999999999911e+00 4.522436009863706544e-06
1.800000000000000044e+00 4.559702970323995311e-06
1.824999999999999956e+00 4.597244916378692009e-06
1.850000000000000089e+00 4.635074966307262817e-06
1.875000000000000000e+00 4.673206656210550369e-06
1.899999999999999911e+00 4.711653483786305829e-06
1.925000000000000044e+00 4.750428537845814009e-06
1.949999999999999956e+00 4.789544222175189450e-06
1.975000000000000089e+00 4.829012074641881799e-06
2.000000000000000000e+00 4.868842675949305171e-06
2.024999999999999911e+00 4.909045637700856432e-06
2.049999999999999822e+00 4.949629656082667889e-06
2.075000000000000178e+00 4.990602615678350548e-06
2.100000000000000089e+00 5.031971727404880787e-06
2.125000000000000000e+00 5.073743685204100407e-06
2.149999999999999911e+00 5.115924827499502073e-06
2.174999999999999822e+00 5.158521291627601521e-06
2.200000000000000178e+00 5.201539151873570827e-06
2.225000000000000089e+00 5.244984534391898198e-06
2.250000000000000000e+00 5.288863704942846702e-06
2.274999999999999911e+00 5.333183127787400474e-06
2.299999999999999822e+00 5.377949496222951542e-06
2.325000000000000178e+00 5.423169736981257524e-06
2.350000000000000089e+00 5.468850992031626043e-06
2.375000000000000000e+00 5.515000582160395348e-06
2.399999999999999911e+00 5.561625957198031503e-06
2.424999999999999822e+00 5.608734637735455358e-06
2.450000000000000178e+00 5.656334152962828876e-06
2.475000000000000089e+00 5.704431978694173705e-06
2.500000000000000000e+00 5.753035478895046975e-06
2.524999999999999911e+00 5.802151853255077887e-06
2.549999999999999822e+00 5.851788092389384466e-06
2.575000000000000178e+00 5.901950941449496044e-06
2.600000000000000089e+00 5.952646872081981955e-06
2.625000000000000000e+00 6.003882062003817327e-06
2.649999999999999911e+00 6.055662380899514904e-06
2.674999999999999822e+00 6.107993380893590035e-06
2.700000000000000178e+00 6.160880289657636941e-06
2.725000000000000089e+00 6.214328003987298205e-06
2.750000000000000000e+00 6.268341081792612804e-06
2.774999999999999911e+00 6.322923730495101299e-06
2.799999999999999822e+00 6.378079790043738803e-06
2.825000000000000178e+00 6.433812709043208386e-06
2.850000000000000089e+00 6.490125512807016703e-06
2.875000000000000000e+00 6.547020762395468554e-06
2.899999999999999911e+00 6.604500504117649662e-06
2.924999999999999822e+00 6.662566209181697979e-06
2.950000000000000178e+00 6.721218703464957450e-06
2.975000000000000089e+00 6.780458087564042446e-06
3.000000000000000000e+00 6.840283647454193511e-06
3.024999999999999911e+00 6.900693756160813930e-06
3.049999999999999822e+00 6.961685766915248763e-06
3.075000000000000178e+00 7.023255898226969315e-06
3.100000000000000089e+00 7.085399111335767899e-06
3.125000000000000000e+00 7.148108980325855455e-06
3.149999999999999911e+00 7.211377555236663656e-06
3.174999999999999822e+00 7.275195218258388719e-06
3.200000000000000178e+00 7.339550533118528730e-06
3.225000000000000089e+00 7.404430087596097082e-06
3.250000000000000000e+00 7.469818329078076885e-06
3.274999999999999911e+00 7.535697392971850428e-06
3.299999999999999822e+00 7.602046923803167100e-06
3.325000000000000178e+00 7.668843888807312312e-06
3.350000000000000089e+00 7.736062383853666014e-06
3.375000000000000000e+00 7.803673431631442947e-06
3.399999999999999911e+00 7.871644772088808376e-06
3.424999999999999822e+00 7.939940645274338668e-06
3.450000000000000178e+00 8.008521566842553147e-06
3.475000000000000089e+00 8.077344096680833668e-06
3.500000000000000000e+00 8.146360601298495430e-06
3.524999999999999911e+00 8.215519010817831994e-06
3.549999999999999822e+00 8.284762571616208562e-06
3.575000000000000178e+00 8.354029595936942817e-06
3.600000000000000089e+00 8.423253209975797504e-06
3.625000000000000000e+00 8.492361102189517619e-06
3.649999999999999911e+00 8.561275273927596019e-06
3.674999999999999822e+00 8.629911794571501135e-06
3.700000000000000178e+00 8.698180563810557070e-06
3.725000000000000089e+00 8.765985083825913668e-06
3.750000000000000000e+00 8.833222244512264551e-06
3.774999999999999911e+00 8.899782125118072332e-06
3.799999999999999822e+00 8.965547815996009404e-06
3.825000000000000178e+00 9.030395264488172771e-06
3.850000000000000089e+00 9.094193149239730364e-06
3.875000000000000000e+00 9.156802787619114370e-06
3.899999999999999911e+00 9.218078081236211543e-06
3.924999999999999822e+00 9.277865504890006262e-06
3.950000000000000178e+00 9.336004144650314474e-06
3.975000000000000089e+00 9.392325791127793463e-06
4.000000000000000000e+00 9.446655094339862294e-06
4.025000000000000355e+00 9.498809786956169052e-06
4.049999999999999822e+00 9.548600983021659702e-06
4.075000000000000178e+00 9.595833559687683145e-06
4.099999999999999645e+00 9.640306629723561019e-06
4.125000000000000000e+00 9.681814113000420177e-06
4.150000000000000355e+00 9.720145415417292059e-06
4.174999999999999822e+00 9.755086224117219193e-06
4.200000000000000178e+00 9.786419428103919696e-06
4.224999999999999645e+00 9.813926173748597701e-06
4.250000000000000000e+00 9.837387065026901786e-06
4.275000000000000355e+00 9.856583518654113071e-06
4.299999999999999822e+00 9.871299284769548266e-06
4.325000000000000178e+00 9.881322144262650112e-06
4.349999999999999645e+00 9.886445794415237368e-06
4.375000000000000000e+00 9.886471935267523992e-06
4.400000000000000355e+00 9.881212569889203864e-06
4.424999999999999822e+00 9.870492532880296987e-06
4.450000000000000178e+00 9.854152262633386963e-06
4.474999999999999645e+00 9.832050834546190998e-06
4.500000000000000000e+00 9.804069274282900537e-06
4.525000000000000355e+00 9.770114172616113993e-06
4.549999999999999822e+00 9.730121626254535638e-06
4.575000000000000178e+00 9.684061532556215157e-06
4.599999999999999645e+00 9.631942270291703190e-06
4.625000000000000000e+00 9.573815803548763395e-06
4.650000000000000355e+00 9.509783251869389012e-06
4.674999999999999822e+00 9.440000976700373830e-06
4.700000000000000178e+00 9.364687242437207473e-06
4.724999999999999645e+00 9.284129519924327889e-06
4.750000000000000000e+00 9.198692511452020992e-06
4.775000000000000355e+00 9.108826989143907829e-06
4.799999999999999822e+00 9.015079553561676159e-06
4.825000000000000178e+00 8.918103436446811125e-06
4.849999999999999645e+00 8.818670491233561448e-06
4.875000000000000000e+00 8.717684537443539175e-06
4.900000000000000355e+00 8.616196250843619758e-06
4.924999999999999822e+00 8.515419820661262670e-06
4.950000000000000178e+00 8.416751628676204938e-06
4.974999999999999645e+00 8.321791243200236249e-06
5.000000000000000000e+00 8.232365064525862253e-06
5.025000000000000355e+00 8.150553007950367412e-06
5.049999999999999822e+00 8.078718666893478298e-06
5.075000000000000178e+00 8.019543462894863570e-06
5.099999999999999645e+00 7.976065362302192652e-06
5.125000000000000000e+00 7.951722822751921757e-06
5.150000000000000355e+00 7.950404727257625107e-06
5.174999999999999822e+00 7.976507171554569897e-06
5.200000000000000178e+00 8.034998093165270430e-06
5.224999999999999645e+00 8.131490870351941458e-06
5.250000000000000000e+00 8.272328178115465752e-06
5.275000000000000355e+00 8.464677569155207651e-06
5.299999999999999822e+00 8.716640453140583542e-06
5.325000000000000178e+00 9.037376380791918573e-06
5.349999999999999645e+00 9.437244803676622662e-06
5.375000000000000000e+00 9.927966779887335141e-06
5.400000000000000355e+00 1.052280943392939344e-05
5.424999999999999822e+00 1.123679636040163937e-05
5.450000000000000178e+00 1.208694758961497556e-05
5.474999999999999645e+00 1.309255321366316858e-05
5.500000000000000000e+00 1.427548530759208123e-05
5.525000000000000355e+00 1.566055337624155420e-05
5.549999999999999822e+00 1.727590921580114507e-05
5.575000000000000178e+00 1.915350780230177900e-05
5.599999999999999645e+00 2.132963160694626749e-05
5.625000000000000000e+00 2.384548658838220314e-05
5.650000000000000355e+00 2.674787901824461980e-05
5.674999999999999822e+00 3.008998324868891912e-05
5.700000000000000178e+00 3.393221151155628897e-05
5.724999999999999645e+00 3.834319782406564979e-05
5.750000000000000000e+00 4.340090902933669247e-05
5.775000000000000355e+00 4.919389687402567733e-05
5.799999999999999822e+00 5.582270575438903412e-05
5.825000000000000178e+00 6.340145126215225059e-05
5.849999999999999645e+00 7.205958482459757196e-05
5.875000000000000000e+00 8.194385942184999501e-05
5.900000000000000355e+00 9.322051041211988284e-05
5.924999999999999822e+00 1.060776636958014364e-04
5.950000000000000178e+00 1.207279805608178162e-04
5.974999999999999645e+00 1.374115442882690129e-04
6.000000000000000000e+00 1.563989876386554043e-04
6.025000000000000355e+00 1.779948523254017752e-04
6.049999999999999822e+00 2.025411611370681909e-04
6.075000000000000178e+00 2.304211701012414369e-04
6.099999999999999645e+00 2.620632516353107750e-04
6.125000000000000000e+00 2.979448396904708744e-04
6.150000000000000355e+00 3.385963442823760813e-04
6.174999999999999822e+00 3.846049154956003146e-04
6.200000000000000178e+00 4.366179063064233937e-04
6.224999999999999645e+00 4.953458499823907950e-04
6.250000000000000000e+00 5.615647324527816599e-04
6.275000000000000355e+00 6.361173045056280236e-04
6.299999999999999822e+00 7.199131452141581660e-04
6.325000000000000178e+00 8.139271595745932193e-04
6.349999999999999645e+00 9.191961735978745190e-04
6.375000000000000000e+00 1.036813283305771565e-03
6.400000000000000355e+00 1.167919624946724697e-03
6.424999999999999822e+00 1.313693267071955805e-03
6.450000000000000178e+00 1.475334985406596836e-03
6.474999999999999645e+00 1.654050772240552762e-03
6.500000000000000000e+00 1.851031055266060552e-03
6.525000000000000355e+00 2.067426755934826700e-03
6.549999999999999822e+00 2.304322500867509935e-03
6.575000000000000178e+00 2.562707504401924748e-03
6.599999999999999645e+00 2.843444854851292711e-03
6.625000000000000000e+00 3.147240146588531177e-03
6.650000000000000355e+00 3.474610586962521494e-03
6.674999999999999822e+00 3.825855851846530348e-03
6.700000000000000178e+00 4.201032047327631774e-03
6.724999999999999645e+00 4.599930141212649551e-03
6.750000000000000000e+00 5.022060145114380647e-03
6.775000000000000355e+00 5.466642151613310303e-03
6.799999999999999822e+00 5.932605065565188344e-03
6.825000000000000178e+00 6.418593527470541418e-03
6.849999999999999645e+00 6.922983131876544148e-03
6.875000000000000000e+00 7.443903623827027340e-03
6.900000000000000355e+00 7.979269343535421236e-03
6.924999999999999822e+00 8.526815816875399009e-03
6.950000000000000178e+00 9.084141085998956658e-03
6.974999999999999645e+00 9.648750163179328765e-03
7.000000000000000000e+00 1.021810088567150047e-02
7.025000000000000355e+00 1.078964945319233620e-02
7.049999999999999822e+00 1.136089403635549139e-02
7.075000000000000178e+00 1.192941503868675049e-02
7.099999999999999645e+00 1.249291085555314859e-02
7.125000000000000000e+00 1.304922827528647543e-02
7.150000000000000355e+00 1.359638698591449611e-02
7.174999999999999822e+00 1.413259796133174359e-02
7.200000000000000178e+00 1.465627578424140778e-02
7.224999999999999645e+00 1.516604520518402902e-02
7.250000000000000000e+00 1.566074242850051465e-02
7.275000000000000355e+00 1.613941175387613447e-02
7.299999999999999822e+00 1.660129828684622430e-02
7.325000000000000178e+00 1.704583746841571365e-02
7.349999999999999645e+00 1.747264216924202321e-02
7.375000000000000000e+00 1.788148805664039470e-02
7.400000000000000355e+00 1.827229788108904879e-02
7.424999999999999822e+00 1.864512525175903573e-02
7.450000000000000178e+00 1.900013838488753132e-02
7.474999999999999645e+00 1.933760422100810200e-02
7.500000000000000000e+00 1.965787322160726658e-02
7.525000000000000355e+00 1.996136507653397188e-02
7.549999999999999822e+00 2.024855548246192252e-02
7.575000000000000178e+00 2.051996409118309397e-02
7.599999999999999645e+00 2.077614367488024236e-02
7.625000000000000000e+00 2.101767051388565843e-02
7.650000000000000355e+00 2.124513597940922219e-02
7.674999999999999822e+00 2.145913925930280700e-02
7.700000000000000178e+00 2.166028115738602086e-02
7.724999999999999645e+00 2.184915888530385847e-02
7.750000000000000000e+00 2.202636175938275584e-02
7.775000000000000355e+00 2.219246771225638279e-02
7.799999999999999822e+00 2.234804052942293562e-02
7.825000000000000178e+00 2.249362772363637519e-02
7.849999999999999645e+00 2.262975896416962590e-02
7.875000000000000000e+00 2.275694498343874481e-02
7.900000000000000355e+00 2.287567688949646363e-02
7.924999999999999822e+00 2.298642581915191427e-02
7.950000000000000178e+00 2.308964287295068446e-02
7.974999999999999645e+00 2.318575927949062532e-02
8.000000000000000000e+00 2.327518674258610698e-02
8.025000000000000355e+00 2.335831793049822011e-02
8.050000000000000711e+00 2.343552707180802594e-02
8.074999999999999289e+00 2.350717062728551257e-02
8.099999999999999645e+00 2.357358801180233637e-02
8.125000000000000000e+00 2.363510234420298714e-02
8.150000000000000355e+00 2.369202120692290997e-02
8.175000000000000711e+00 2.374463740026628092e-02
8.199999999999999289e+00 2.379322967935249347e-02
8.224999999999999645e+00 2.383806346426767600e-02
8.250000000000000000e+00 2.387939151631241971e-02
8.275000000000000355e+00 2.391745457522923349e-02
8.300000000000000711e+00 2.395248195404663799e-02
8.324999999999999289e+00 2.398469208968434171e-02
8.349999999999999645e+00 2.401429304877187795e-02
8.375000000000000000e+00 2.404148298913459658e-02
8.400000000000000355e+00 2.406645057831360152e-02
8.425000000000000711e+00 2.408937537116508446e-02
8.449999999999999289e+00 2.411042814917547725e-02
8.474999999999999645e+00 2.412977122440155067e-02
8.500000000000000000e+00 2.414755871135420076e-02
8.525000000000000355e+00 2.416393677020737055e-02
8.550000000000000711e+00 2.417904382481648670e-02
8.574999999999999289e+00 2.419301075904059517e-02
8.599999999999999645e+00 2.420596109470529028e-02
8.625000000000000000e+00 2.421801115455015155e-02
8.650000000000000355e+00 2.422927021321213958e-02
8.675000000000000711e+00 2.423984063912237863e-02
8.699999999999999289e+00 2.424981803007635398e-02
8.724999999999999645e+00 2.425929134483719232e-02
8.750000000000000000e+00 2.426834303308740212e-02
8.775000000000000355e+00 2.427704916560495876e-02
8.800000000000000711e+00 2.428547956652226530e-02
8.824999999999999289e+00 2.429369794912247457e-02
8.849999999999999645e+00 2.430176205655940058e-02
8.875000000000000000e+00 2.430972380858890253e-02
8.900000000000000355e+00 2.431762945537082124e-02
8.925000000000000711e+00 2.432551973901950562e-02
8.949999999999999289e+00 2.433343006372560083e-02
8.974999999999999645e+00 2.434139067488635261e-02
9.000000000000000000e+00 2.434942684772879382e-02
9.025000000000000355e+00 2.435755908573753944e-02
9.050000000000000711e+00 2.436580332907367988e-02
9.074999999999999289e+00 2.437417117315380383e-02
9.099999999999999645e+00 2.438267009726740805e-02
9.125000000000000000e+00 2.439130370318125240e-02
9.150000000000000355e+00 2.440007196340597193e-02
9.175000000000000711e+00 2.440897147865973490e-02
9.199999999999999289e+00 2.441799574394451500e-02
9.224999999999999645e+00 2.442713542241795391e-02
9.250000000000000000e+00 2.443637862613231432e-02
9.275000000000000355e+00 2.444571120240290538e-02
9.300000000000000711e+00 2.445511702451431263e-02
9.324999999999999289e+00 2.446457828521868619e-02
9.349999999999999645e+00 2.447407579138710654e-02
9.375000000000000000e+00 2.448358925799895211e-02
9.400000000000000355e+00 2.449309759957526630e-02
9.425000000000000711e+00 2.450257921713422729e-02
9.449999999999999289e+00 2.451201227865591400e-02
9.474999999999999645e+00 2.452137499113439881e-02
9.500000000000000000e+00 2.453064586222603732e-02
9.525000000000000355e+00 2.453980394978012847e-02
9.550000000000000711e+00 2.454882909737131358e-02
9.574999999999999289e+00 2.455770215437253629e-02
9.599999999999999645e+00 2.456640517910199314e-02
9.625000000000000000e+00 2.457492162376247180e-02
9.650000000000000355e+00 2.458323650015078674e-02
9.675000000000000711e+00 2.459133652526020869e-02
9.699999999999999289e+00 2.459921024611001067e-02
9.724999999999999645e+00 2.460684814332428361e-02
9.750000000000000000e+00 2.461424271324548149e-02
9.775000000000000355e+00 2.462138852843897760e-02
9.800000000000000711e+00 2.462828227678258430e-02
9.824999999999999289e+00 2.463492277938441818e-02
9.849999999999999645e+00 2.464131098786670471e-02
9.875000000000000000e+00 2.464744996159249149e-02
9.900000000000000355e+00 2.465334482568259586e-02
9.925000000000000711e+00 2.465900271071825092e-02
9.949999999999999289e+00 2.466443267521956431e-02
9.974999999999999645e+00 2.466964561207560247e-02
1.000000000000000000e+01 2.467465414018863573e-02
1.002500000000000036e+01 2.467947248274215966e-02
1.005000000000000071e+01 2.468411633349863396e-02
1.007499999999999929e+01 2.468860271271210854e-02
1.009999999999999964e+01 2.469294981413359263e-02
1.012500000000000000e+01 2.469717684480047387e-02
1.015000000000000036e+01 2.470130385918641036e-02
1.017500000000000071e+01 2.470535158939832915e-02
1.019999999999999929e+01 2.470934127307072500e-02
1.022499999999999964e+01 2.471329448055642475e-02
1.025000000000000000e+01 2.471723294308322480e-02
1.027500000000000036e+01 2.472117838337234072e-02
1.030000000000000071e+01 2.472515235025389210e-02
1.032499999999999929e+01 2.472917605869618535e-02
1.034999999999999964e+01 2.473327023647389766e-02
1.037500000000000000e+01 2.473745497874509439e-02
1.040000000000000036e+01 2.474174961150046176e-02
1.042500000000000071e+01 2.474617256477201765e-02
1.044999999999999929e+01 2.475074125628968363e-02
1.047499999999999964e+01 2.475547198613571934e-02
1.050000000000000000e+01 2.476037984270585204e-02
1.052500000000000036e+01 2.476547862019073606e-02
1.055000000000000071e+01 2.477078074754766879e-02
1.057499999999999929e+01 2.477629722889332364e-02
1.059999999999999964e+01 2.478203759505925169e-02
1.062500000000000000e+01 2.478800986598559547e-02
1.065000000000000036e+01 2.479422052356109601e-02
1.067500000000000071e+01 2.480067449452973485e-02
1.069999999999999929e+01 2.480737514300777533e-02
1.072499999999999964e+01 2.481432427224377485e-02
1.075000000000000000e+01 2.482152213524577850e-02
1.077500000000000036e+01 2.482896745396353730e-02
1.080000000000000071e+01 2.483665744678233839e-02
1.082499999999999929e+01 2.484458786411447226e-02
1.084999999999999964e+01 2.485275303196188279e-02
1.087500000000000000e+01 2.486114590328867766e-02
1.090000000000000036e+01 2.486975811713504000e-02
1.092500000000000071e+01 2.487858006536037767e-02
1.094999999999999929e+01 2.488760096686109050e-02
1.097499999999999964e+01 2.489680894915025436e-02
1.100000000000000000e+01 2.490619113695287032e-02
1.102500000000000036e+01 2.491573374764543755e-02
1.105000000000000071e+01 2.492542219294880540e-02
1.107499999999999929e+01 2.493524118649827898e-02
1.109999999999999964e+01 2.494517485661254530e-02
1.112500000000000000e+01 2.495520686345326089e-02
1.115000000000000036e+01 2.496532051984334810e-02
1.117500000000000071e+01 2.497549891472660918e-02
1.119999999999999929e+01 2.498572503823870417e-02
1.122499999999999964e+01 2.499598190731505687e-02
1.125000000000000000e+01 2.500625269065101172e-02
1.127500000000000036e+01 2.501652083184742470e-02
1.130000000000000071e+01 2.502677016948353164e-02
1.132499999999999929e+01 2.503698505300597926e-02
1.134999999999999964e+01 2.504715045315005287e-02
1.137500000000000000e+01 2.505725206591677065e-02
1.140000000000000036e+01 2.506727640894396147e-02
1.142500000000000071e+01 2.507721090942096470e-02
1.144999999999999929e+01 2.508704398268099198e-02
1.147499999999999964e+01 2.509676510078125491e-02
1.150000000000000000e+01 2.510636485050359717e-02
1.152500000000000036e+01 2.511583498033856734e-02
1.155000000000000071e+01 2.512516843624037696e-02
1.157499999999999929e+01 2.513435938588595361e-02
1.159999999999999964e+01 2.514340323155275086e-02
1.162500000000000000e+01 2.515229661163820296e-02
1.165000000000000036e+01 2.516103739109072993e-02
1.167500000000000071e+01 2.516962464101188748e-02
1.169999999999999929e+01 2.517805860788972783e-02
1.172499999999999964e+01 2.518634067283866823e-02
1.175000000000000000e+01 2.519447330148856870e-02
1.177500000000000036e+01 2.520245998501215168e-02
1.180000000000000071e+01 2.521030517293302090e-02
1.182499999999999929e+01 2.521801419850316917e-02
1.184999999999999964e+01 2.522559319723724838e-02
1.187500000000000000e+01 2.523304901951420667e-02
1.190000000000000036e+01 2.524038913805745632e-02
1.192500000000000071e+01 2.524762155115689924e-02
1.194999999999999929e+01 2.525475468269584856e-02
1.197499999999999964e+01 2.526179727987743970e-02
1.200000000000000000e+01 2.526875830979527504e-02
1.202500000000000036e+01 2.527564685592801133e-02
1.205000000000000071e+01 2.528247201558605009e-02
1.207499999999999929e+01 2.528924279955865537e-02
1.209999999999999964e+01 2.529596803490593085e-02
1.212500000000000000e+01 2.530265627200050868e-02
1.215000000000000036e+01 2.530931569684598606e-02
1.217500000000000071e+01 2.531595404941242369e-02
1.219999999999999929e+01 2.532257854895953939e-02
1.222499999999999964e+01 2.532919582686381579e-02
1.225000000000000000e+01 2.533581186755491654e-02
1.227500000000000036e+01 2.534243195797192955e-02
1.230000000000000071e+01 2.534906064576996493e-02
1.232499999999999929e+01 2.535570170637884910e-02
1.234999999999999964e+01 2.536235811901364767e-02
1.237500000000000000e+01 2.536903205135009748e-02
1.240000000000000036e+01 2.537572485274491169e-02
1.242500000000000071e+01 2.538243705560210006e-02
1.244999999999999929e+01 2.538916838443394317e-02
1.247499999999999964e+01 2.539591777215091661e-02
1.250000000000000000e+01 2.540268338298936820e-02
1.252500000000000036e+01 2.540946264146253658e-02
1.255000000000000071e+01 2.541625226670089777e-02
1.257499999999999929e+01 2.542304831156277922e-02
1.259999999999999964e+01 2.542984620582820765e-02
1.262500000000000000e+01 2.543664080283363632e-02
1.265000000000000036e+01 2.544342642897039555e-02
1.267500000000000071e+01 2.545019693545445133e-02
1.269999999999999929e+01 2.545694575183840569e-02
1.272499999999999964e+01 2.546366594074126952e-02
1.275000000000000000e+01 2.547035025341910439e-02
1.277500000000000036e+01 2.547699118575018698e-02
1.280000000000000071e+01 2.548358103425330326e-02
1.282499999999999929e+01 2.549011195195140592e-02
1.284999999999999964e+01 2.549657600366793744e-02
1.287500000000000000e+01 2.550296522056546070e-02
1.290000000000000036e+01 2.550927165369671484e-02
1.292500000000000071e+01 2.551548742632427785e-02
1.294999999999999929e+01 2.552160478467189325e-02
1.297499999999999964e+01 2.552761614698942014e-02
1.300000000000000000e+01 2.553351415052228582e-02
1.302500000000000036e+01 2.553929169614397102e-02
1.305000000000000071e+01 2.554494199032975726e-02
1.307499999999999929e+01 2.555045858412600610e-02
1.309999999999999964e+01 2.555583540880638355e-02
1.312500000000000000e+01 2.556106680789614616e-02
1.315000000000000036e+01 2.556614756520968201e-02
1.317500000000000071e+01 2.557107292868943810e-02
1.319999999999999929e+01 2.557583862974626907e-02
1.322499999999999964e+01 2.558044089793514581e-02
1.325000000000000000e+01 2.558487647081236424e-02
1.327500000000000036e+01 2.558914259881852477e-02
1.330000000000000071e+01 2.559323704524328266e-02
1.332499999999999929e+01 2.559715808118518537e-02
1.334999999999999964e+01 2.560090447561477744e-02
1.337500000000000000e+01 2.560447548065418813e-02
1.340000000000000036e+01 2.560787081218911243e-02
1.342500000000000071e+01 2.561109062608598469e-02
1.344999999999999929e+01 2.561413549013693419e-02
1.347499999999999964e+01 2.561700635217769792e-02
1.350000000000000000e+01 2.561970450451757711e-02
1.352500000000000036e+01 2.562223154522999882e-02
1.355000000000000071e+01 2.562458933652179971e-02
1.357499999999999929e+01 2.562677996075189008e-02
1.359999999999999964e+01 2.562880567453170583e-02
1.362500000000000000e+01 2.563066886136151565e-02
1.365000000000000036e+01 2.563237198340370671e-02
1.367500000000000071e+01 2.563391753294342809e-02
1.369999999999999929e+01 2.563530798402048269e-02
1.372499999999999964e+01 2.563654574491817245e-02
1.375000000000000000e+01 2.563763311196257805e-02
1.377500000000000036e+01 2.563857222525135413e-02
1.380000000000000071e+01 2.563936502681777099e-02
1.382499999999999929e+01 2.564001322165262650e-02
1.384999999999999964e+01 2.564051824213956290e-02
1.387500000000000000e+01 2.564088121619629429e-02
1.390000000000000036e+01 2.564110293956113965e-02
1.392500000000000071e+01 2.564118385248075754e-02
1.394999999999999929e+01 2.564112402108531183e-02
1.397499999999999964e+01 2.564092312366464746e-02
1.400000000000000000e+01 2.564058044203090456e-02
1.402500000000000036e+01 2.564009485810581079e-02
1.405000000000000071e+01 2.563946485582041523e-02
1.407499999999999929e+01 2.563868852835179610e-02
1.409999999999999964e+01 2.563776359081820777e-02
1.412500000000000000e+01 2.563668739824342957e-02
1.415000000000000036e+01 2.563545696886827124e-02
1.417500000000000071e+01 2.563406901249468128e-02
1.419999999999999929e+01 2.563251996384217565e-02
1.422499999999999964e+01 2.563080602046789030e-02
1.425000000000000000e+01 2.562892318510313905e-02
1.427500000000000036e+01 2.562686731178786745e-02
1.430000000000000071e+01 2.562463415557043184e-02
1.432499999999999929e+01 2.562221942498248486e-02
1.434999999999999964e+01 2.561961883693138903e-02
1.437500000000000000e+01 2.561682817318404470e-02
1.440000000000000036e+01 2.561384333785453674e-02
1.442500000000000071e+01 2.561066041511892225e-02
1.444999999999999929e+01 2.560727572642170977e-02
1.447499999999999964e+01 2.560368588643687632e-02
1.450000000000000000e+01 2.559988785692790883e-02
1.452500000000000036e+01 2.559587899788214396e-02
1.455000000000000071e+01 2.559165711501748541e-02
1.457499999999999929e+01 2.558722050305097667e-02
1.459999999999999964e+01 2.558256798398898801e-02
1.462500000000000000e+01 2.557769893979320100e-02
1.465000000000000036e+01 2.557261333886802143e-02
1.467500000000000071e+01 2.556731175578568632e-02
1.469999999999999929e+01 2.556179538391588346e-02
1.472499999999999964e+01 2.555606604045275454e-02
1.475000000000000000e+01 2.555012616375228890e-02
1.477500000000000036e+01 2.554397880263729881e-02
1.480000000000000071e+01 2.553762759782023453e-02
1.482499999999999929e+01 2.553107675535602750e-02
1.484999999999999964e+01 2.552433101235654020e-02
1.487500000000000000e+01 2.551739559530100468e-02
1.490000000000000036e+01 2.551027617120206828e-02
1.492500000000000071e+01 2.550297879224280861e-02
1.494999999999999929e+01 2.549550983429011222e-02
1.497499999999999964e+01 2.548787593003901814e-02
1.500000000000000000e+01 2.548008389737500476e-02
1.502500000000000036e+01 2.547214066370457816e-02
1.505000000000000071e+01 2.546405318699678302e-02
1.507499999999999929e+01 2.545582837434254286e-02
1.509999999999999964e+01 2.544747299886166575e-02
1.512500000000000000e+01 2.543899361573617388e-02
1.515000000000000036e+01 2.543039647829670180e-02
1.517500000000000071e+01 2.542168745508036412e-02
1.519999999999999929e+01 2.541287194869107716e-02
1.522499999999999964e+01 2.540395481754285922e-02
1.525000000000000000e+01 2.539494030144200457e-02
1.527500000000000036e+01 2.538583195200610002e-02
1.530000000000000071e+01 2.537663256907012055e-02
1.532499999999999929e+01 2.536734414403225488e-02
1.534999999999999964e+01 2.535796781136905043e-02
1.537500000000000000e+01 2.534850380922959445e-02
1.540000000000000036e+01 2.533895145018574477e-02
1.542500000000000071e+01 2.532930910310297185e-02
1.544999999999999929e+01 2.531957418694251796e-02
1.547499999999999964e+01 2.530974317724619255e-02
1.550000000000000000e+01 2.529981162582123008e-02
1.552500000000000036e+01 2.528977419416100422e-02
1.555000000000000071e+01 2.527962470065264755e-02
1.557499999999999929e+01 2.526935618173262177e-02
1.559999999999999964e+01 2.525896096663312468e-02
1.562500000000000000e+01 2.524843076541989639e-02
1.565000000000000036e+01 2.523775676954137515e-02
1.567500000000000071e+01 2.522692976405497070e-02
1.569999999999999929e+01 2.521594025050414439e-02
1.572499999999999964e+01 2.520477857894754328e-02
1.575000000000000000e+01 2.519343508793324762e-02
1.577500000000000036e+01 2.518190025058647183e-02
1.580000000000000071e+01 2.517016482515260084e-02
1.582499999999999929e+01 2.515822000812416304e-02
1.584999999999999964e+01 2.514605758798585042e-02
1.587500000000000000e+01 2.513367009765067953e-02
1.590000000000000036e+01 2.512105096361877399e-02
1.592500000000000071e+01 2.510819464981373442e-02
1.594999999999999929e+01 2.509509679431631834e-02
1.597499999999999964e+01 2.508175433707870300e-02
1.600000000000000000e+01 2.506816563697127770e-02
1.602499999999999858e+01 2.505433057647769857e-02
1.605000000000000071e+01 2.504025065269869024e-02
1.607499999999999929e+01 2.502592905332713669e-02
1.610000000000000142e+01 2.501137071639169401e-02
1.612500000000000000e+01 2.499658237298347727e-02
1.614999999999999858e+01 2.498157257198853690e-02
1.617500000000000071e+01 2.496635168631412133e-02
1.619999999999999929e+01 2.495093190010143347e-02
1.622500000000000142e+01 2.493532717664130594e-02
1.625000000000000000e+01 2.491955320682024549e-02
1.627499999999999858e+01 2.490362733811047474e-02
1.630000000000000071e+01 2.488756848425561358e-02
1.632499999999999929e+01 2.487139701600367056e-02
1.635000000000000142e+01 2.485513463327341036e-02
1.637500000000000000e+01 2.483880421959209356e-02
1.639999999999999858e+01 2.482242967942733039e-02
1.642500000000000071e+01 2.480603575963855056e-02
1.644999999999999929e+01 2.478964785623342590e-02
1.647500000000000142e+01 2.477329180786800103e-02
1.650000000000000000e+01 2.475699367783683325e-02
1.652499999999999858e+01 2.474077952642638953e-02
1.655000000000000071e+01 2.472467517574773732e-02
1.657499999999999929e+01 2.470870596939924932e-02
1.660000000000000142e+01 2.469289652936590071e-02
1.662500000000000000e+01 2.467727051288726853e-02
1.664999999999999858e+01 2.466185037196333418e-02
1.667500000000000071e+01 2.464665711832528250e-02
1.669999999999999929e+01 2.463171009674172685e-02
1.672500000000000142e+01 2.461702676951170801e-02
1.675000000000000000e+01 2.460262251486798266e-02
1.677499999999999858e+01 2.458851044203127362e-02
1.680000000000000071e+01 2.457470122543059521e-02
1.682499999999999929e+01 2.456120296051732960e-02
1.685000000000000142e+01 2.454802104324819989e-02
1.687500000000000000e+01 2.453515807523561423e-02
1.689999999999999858e+01 2.452261379623168688e-02
1.692500000000000071e+01 2.451038504526661943e-02
1.694999999999999929e+01 2.449846575153775963e-02
1.697500000000000142e+01 2.448684695586179211e-02
1.700000000000000000e+01 2.447551686298238965e-02
1.702499999999999858e+01 2.446446092493074348e-02
1.705000000000000071e+01 2.445366195514580426e-02
1.707499999999999929e+01 2.444310027268626825e-02
1.710000000000000142e+01 2.443275387563073608e-02
1.712500000000000000e+01 2.442259864240227049e-02
1.714999999999999858e+01 2.441260855928679627e-02
1.717500000000000071e+01 2.440275597227934204e-02
1.719999999999999929e+01 2.439301186090611639e-02
1.722500000000000142e+01 2.438334613156557806e-02
1.725000000000000000e+01 2.437372792741783478e-02
1.727499999999999858e+01 2.436412595178059914e-02
1.730000000000000071e+01 2.435450880171605853e-02
1.732499999999999929e+01 2.434484530818289971e-02
1.735000000000000142e+01 2.433510487907184286e-02
1.737500000000000000e+01 2.432525784125102314e-02
1.739999999999999858e+01 2.431527577756011776e-02
1.742500000000000071e+01 2.430513185475358454e-02
1.744999999999999929e+01 2.429480113823983059e-02
1.747500000000000142e+01 2.428426088953667544e-02
1.750000000000000000e+01 2.427349084246692781e-02
1.752499999999999858e+01 2.426247345416483042e-02
1.755000000000000071e+01 2.425119412714908662e-02
1.757499999999999929e+01 2.423964139917589300e-02
1.760000000000000142e+01 2.422780709753543857e-02
1.762500000000000000e+01 2.421568645519888568e-02
1.764999999999999858e+01 2.420327818644092688e-02
1.767500000000000071e+01 2.419058452013450600e-02
1.769999999999999929e+01 2.417761118946584048e-02
1.772500000000000142e+01 2.416436737737721885e-02
1.775000000000000000e+01 2.415086561767812445e-02
1.777499999999999858e+01 2.413712165240174123e-02
1.780000000000000071e+01 2.412315424660449806e-02
1.782499999999999929e+01 2.410898496239952116e-02
1.785000000000000142e+01 2.409463789471201489e-02
1.787500000000000000e+01 2.408013937170767893e-02
1.789999999999999858e+01 2.406551762341831158e-02
1.792500000000000071e+01 2.405080242265772786e-02
1.794999999999999929e+01 2.403602470243930422e-02
1.797500000000000142e+01 2.402121615490528453e-02
1.800000000000000000e+01 2.400640881664227802e-02
1.802499999999999858e+01 2.399163464568006277e-02
1.805000000000000071e+01 2.397692509563347646e-02
1.807499999999999929e+01 2.396231069242221040e-02
1.810000000000000142e+01 2.394782061903942896e-02
1.812500000000000000e+01 2.393348231384339816e-02
1.814999999999999858e+01 2.391932108763113879e-02
1.817500000000000071e+01 2.390535976461270751e-02
1.819999999999999929e+01 2.389161835219049540e-02
1.822500000000000142e+01 2.387811374404294820e-02
1.825000000000000000e+01 2.386485946075728518e-02
1.827499999999999858e+01 2.385186543177578861e-02
1.830000000000000071e+01 2.383913782193312556e-02
1.832499999999999929e+01 2.382667890548388276e-02
1.835000000000000142e+01 2.381448698973636161e-02
1.837500000000000000e+01 2.380255639006091314e-02
1.839999999999999858e+01 2.379087745722099564e-02
1.842500000000000071e+01 2.377943665742328697e-02
1.844999999999999929e+01 2.376821670479272244e-02
1.847500000000000142e+01 2.375719674526696745e-02
1.850000000000000000e+01 2.374635259024555564e-02
1.852499999999999858e+01 2.373565699757284111e-02
1.855000000000000071e+01 2.372507999686753258e-02
1.857499999999999929e+01 2.371458925547504162e-02
1.860000000000000142e+01 2.370415048068027705e-02
1.862500000000000000e+01 2.369372785324462549e-02
1.864999999999999858e+01 2.368328448692282734e-02
1.867500000000000071e+01 2.367278290781998223e-02
1.869999999999999929e+01 2.366218554745123590e-02
1.872500000000000142e+01 2.365145524270968400e-02
1.875000000000000000e+01 2.364055573582544631e-02
1.877499999999999858e+01 2.362945216736721255e-02
1.880000000000000071e+01 2.361811155509140059e-02
1.882499999999999929e+01 2.360650325167382851e-02
1.885000000000000142e+01 2.359459937451414274e-02
1.887500000000000000e+01 2.358237520105152876e-02
1.889999999999999858e+01 2.356980952356732897e-02
1.892500000000000071e+01 2.355688495774187555e-02
1.894999999999999929e+01 2.354358820012327630e-02
1.897500000000000142e+01 2.352991023023408984e-02
1.900000000000000000e+01 2.351584645373339805e-02
1.902499999999999858e+01 2.350139678412760738e-02
1.905000000000000071e+01 2.348656566121706118e-02
1.907499999999999929e+01 2.347136200550383606e-02
1.910000000000000142e+01 2.345579910868292317e-02
1.912500000000000000e+01 2.343989446122367662e-02
1.914999999999999858e+01 2.342366951915423462e-02
1.917500000000000071e+01 2.340714941282797876e-02
1.919999999999999929e+01 2.339036260138587647e-02
1.922500000000000142e+01 2.337334047755671232e-02
1.925000000000000000e+01 2.335611692787279217e-02
1.927499999999999858e+01 2.333872785438678016e-02
1.930000000000000071e+01 2.332121066420169211e-02
1.932499999999999929e+01 2.330360373382064920e-02
1.935000000000000142e+01 2.328594585579809792e-02
1.937500000000000000e+01 2.326827567506363145e-02
1.939999999999999858e+01 2.325063112297134221e-02
1.942500000000000071e+01 2.323304885696992131e-02
1.944999999999999929e+01 2.321556371364920032e-02
1.947500000000000142e+01 2.319820818305346963e-02
1.950000000000000000e+01 2.318101191180585094e-02
1.952499999999999858e+01 2.316400124218363082e-02
1.955000000000000071e+01 2.314719879392043364e-02
1.957499999999999929e+01 2.313062309505353048e-02
1.960000000000000142e+01 2.311428826731866740e-02
1.962500000000000000e+01 2.309820377113978862e-02
1.964999999999999858e+01 2.308237421423173924e-02
1.967500000000000071e+01 2.306679922730119070e-02
1.969999999999999929e+01 2.305147340911736439e-02
1.972500000000000142e+01 2.303638634259424708e-02
1.975000000000000000e+01 2.302152268236964255e-02
1.977499999999999858e+01 2.300686231356007666e-02
1.980000000000000071e+01 2.299238058020425246e-02
1.982499999999999929e+01 2.297804858112996104e-02
1.985000000000000142e+01 2.296383352992013155e-02
1.987500000000000000e+01 2.294969917472871462e-02
1.989999999999999858e+01 2.293560627286255621e-02
1.992500000000000071e+01 2.292151311425300736e-02
1.994999999999999929e+01 2.290737608696842217e-02
1.997500000000000142e+01 2.289315027761365529e-02
2.000000000000000000e+01 2.287879009847454750e-02
2.002499999999999858e+01 2.286424993306641201e-02
2.005000000000000071e+01 2.284948479119426265e-02
2.007499999999999929e+01 2.283445096448017547e-02
2.010000000000000142e+01 2.281910667312276111e-02
2.012500000000000000e+01 2.280341269475618174e-02
2.014999999999999858e+01 2.278733296617036289e-02
2.017500000000000071e+01 2.277083514932410832e-02
2.019999999999999929e+01 2.275389115311817270e-02
2.022500000000000142e+01 2.273647760313972366e-02
2.025000000000000000e+01 2.271857625223700286e-02
2.027499999999999858e+01 2.270017432541866775e-02
2.030000000000000071e+01 2.268126479373010901e-02
2.032499999999999929e+01 2.266184657231357785e-02
2.035000000000000142e+01 2.264192463932347341e-02
2.037500000000000000e+01 2.262151007311670112e-02
2.039999999999999858e+01 2.260062000647261593e-02
2.042500000000000071e+01 2.257927749768282380e-02
2.044999999999999929e+01 2.255751131947066404e-02
2.047500000000000142e+01 2.253535566794533790e-02
2.050000000000000000e+01 2.251284979496600766e-02
2.052499999999999858e+01 2.249003756807598994e-02
2.055000000000000071e+01 2.246696696368762070e-02
2.057499999999999929e+01 2.244368949974861333e-02
2.060000000000000142e+01 2.242025961517540508e-02
2.062500000000000000e+01 2.239673400414726387e-02
2.064999999999999858e+01 2.237317091391849422e-02
2.067500000000000071e+01 2.234962941548411519e-02
2.069999999999999929e+01 2.232616865672302706e-02
2.072500000000000142e+01 2.230284710810670826e-02
2.075000000000000000e+01 2.227972181102593624e-02
2.077499999999999858e+01 2.225684763899403798e-02
2.080000000000000071e+01 2.223427658168239049e-02
2.082499999999999929e+01 2.221205706171088673e-02
2.085000000000000142e+01 2.219023329345858747e-02
2.087500000000000000e+01 2.216884469296192400e-02
2.089999999999999858e+01 2.214792534702555993e-02
2.092500000000000071e+01 2.212750354924274826e-02
2.094999999999999929e+01 2.210760140950319627e-02
2.097500000000000142e+01 2.208823454295068267e-02
2.100000000000000000e+01 2.206941184304254955e-02
2.102499999999999858e+01 2.205113534262081315e-02
2.105000000000000071e+01 2.203340016544910740e-02
2.107499999999999929e+01 2.201619456994983667e-02
2.110000000000000142e+01 2.199950008506522045e-02
2.112500000000000000e+01 2.198329173772724243e-02
2.114999999999999858e+01 2.196753836944744659e-02
2.117500000000000071e+01 2.195220303883231236e-02
2.119999999999999929e+01 2.193724350544693988e-02
2.122500000000000142e+01 2.192261278929747564e-02
2.125000000000000000e+01 2.190825979908333707e-02
2.127499999999999858e+01 2.189413002143424103e-02
2.130000000000000071e+01 2.188016626225186209e-02
2.132499999999999929e+01 2.186630943055513712e-02
2.135000000000000142e+01 2.185249935438556348e-02
2.137500000000000000e+01 2.183867561786162481e-02
2.139999999999999858e+01 2.182477840789845761e-02
2.142500000000000071e+01 2.181074935882424889e-02
2.144999999999999929e+01 2.179653238317886751e-02
2.147500000000000142e+01 2.178207447668526961e-02
2.150000000000000000e+01 2.176732648599268363e-02
2.152499999999999858e+01 2.175224382781133781e-02
2.155000000000000071e+01 2.173678714896345315e-02
2.157499999999999929e+01 2.172092291733949271e-02
2.160000000000000142e+01 2.170462393474955134e-02
2.162500000000000000e+01 2.168786976374726880e-02
2.164999999999999858e+01 2.167064706153249212e-02
2.167500000000000071e+01 2.165294981545038203e-02
2.169999999999999929e+01 2.163477947585781130e-02
2.172500000000000142e+01 2.161614498372590662e-02
2.175000000000000000e+01 2.159706269169553439e-02
2.177499999999999858e+01 2.157755617895291272e-02
2.180000000000000071e+01 2.155765596183179325e-02
2.182499999999999929e+01 2.153739910347492875e-02
2.185000000000000142e+01 2.151682872760311491e-02
2.187500000000000000e+01 2.149599344252938263e-02
2.189999999999999858e+01 2.147494668328861836e-02
2.192500000000000071e+01 2.145374598067308416e-02
2.194999999999999929e+01 2.143245216729457864e-02
2.197500000000000142e+01 2.141112853156915824e-02
2.200000000000000000e+01 2.138983993157111616e-02
2.202499999999999858e+01 2.136865188109419306e-02
2.205000000000000071e+01 2.134762962100392181e-02
2.207499999999999929e+01 2.132683718900302630e-02
2.210000000000000142e+01 2.130633650118445022e-02
2.212500000000000000e+01 2.128618645871123080e-02
2.214999999999999858e+01 2.126644209257894103e-02
2.217500000000000071e+01 2.124715375900511002e-02
2.219999999999999929e+01 2.122836639753961085e-02
2.222500000000000142e+01 2.121011886296325644e-02
2.225000000000000000e+01 2.119244334126172552e-02
2.227499999999999858e+01 2.117536485886062358e-02
2.230000000000000071e+01 2.115890089295032958e-02
2.232499999999999929e+01 2.114306108956446120e-02
2.235000000000000142e+01 2.112784709437342048e-02
2.237500000000000000e+01 2.111325249972407711e-02
2.239999999999999858e+01 2.109926290974986976e-02
2.242500000000000071e+01 2.108585612360755826e-02
2.244999999999999929e+01 2.107300243521025018e-02
2.247500000000000142e+01 2.106066504596580333e-02
2.250000000000000000e+01 2.104880058531029274e-02
2.252499999999999858e+01 2.103735973212355823e-02
2.255000000000000071e+01 2.102628792842837005e-02
2.257499999999999929e+01 2.101552617514033591e-02
2.260000000000000142e+01 2.100501189846332509e-02
2.262500000000000000e+01 2.099467987398138252e-02
2.264999999999999858e+01 2.098446319459393999e-02
2.267500000000000071e+01 2.097429426753096710e-02
2.269999999999999929e+01 2.096410582501345710e-02
2.272500000000000142e+01 2.095383193279127046e-02
2.275000000000000000e+01 2.094340898060500708e-02
2.277499999999999858e+01 2.093277663873843730e-02
2.280000000000000071e+01 2.092187876528997861e-02
2.282499999999999929e+01 2.091066424931548387e-02
2.285000000000000142e+01 2.089908777608738555e-02
2.287500000000000000e+01 2.088711050166980332e-02
2.289999999999999858e+01 2.087470062557387693e-02
2.292500000000000071e+01 2.086183385170763821e-02
2.294999999999999929e+01 2.084849372962703506e-02
2.297500000000000142e+01 2.083467187018802139e-02
2.300000000000000000e+01 2.082036803137151532e-02
2.302499999999999858e+01 2.080559007258781304e-02
2.305000000000000071e+01 2.079035377763387149e-02
2.307499999999999929e+01 2.077468254884656459e-02
2.310000000000000142e+01 2.075860697697759841e-02
2.312500000000000000e+01 2.074216429359974218e-02
2.314999999999999858e+01 2.072539771473946163e-02
2.317500000000000071e+01 2.070835568623421649e-02
2.319999999999999929e+01 2.069109104308302152e-02
2.322500000000000142e+01 2.067366009661498333e-02
2.325000000000000000e+01 2.065612166438152500e-02
2.327499999999999858e+01 2.063853605881227349e-02
2.330000000000000071e+01 2.062096405155275716e-02
2.332499999999999929e+01 2.060346583064787823e-02
2.335000000000000142e+01 2.058609996816404206e-02
2.337500000000000000e+01 2.056892241566195267e-02
2.339999999999999858e+01 2.055198554461639196e-02
2.342500000000000071e+01 2.053533724821108436e-02
2.344999999999999929e+01 2.051902011998832723e-02
2.347500000000000142e+01 2.050307072384632487e-02
2.350000000000000000e+01 2.048751896804376291e-02
2.352499999999999858e+01 2.047238759474812389e-02
2.355000000000000071e+01 2.045769179430529175e-02
2.357499999999999929e+01 2.044343895166361377e-02
2.360000000000000142e+01 2.042962853010405963e-02
2.362500000000000000e+01 2.041625209502563251e-02
2.364999999999999858e+01 2.040329347825997966e-02
2.367500000000000071e+01 2.039072908089340982e-02
2.369999999999999929e+01 2.037852831001374376e-02
2.372500000000000142e+01 2.036665414263814575e-02
2.375000000000000000e+01 2.035506380739678448e-02
2.377499999999999858e+01 2.034370957280153339e-02
2.380000000000000071e+01 2.033253962841053558e-02
2.382499999999999929e+01 2.032149904380915750e-02
2.385000000000000142e+01 2.031053078857593397e-02
2.387500000000000000e+01 2.029957679503791024e-02
2.389999999999999858e+01 2.028857904502530743e-02
2.392500000000000071e+01 2.027748066080501191e-02
2.394999999999999929e+01 2.026622698040366280e-02
2.397500000000000142e+01 2.025476659759088000e-02
2.400000000000000000e+01 2.024305234726433630e-02
2.402499999999999858e+01 2.023104221777032030e-02
2.405000000000000071e+01 2.021870017303304540e-02
2.407499999999999929e+01 2.020599686879654550e-02
2.410000000000000142e+01 2.019291024920302074e-02
2.412500000000000000e+01 2.017942601192583216e-02
2.414999999999999858e+01 2.016553793258914901e-02
2.417500000000000071e+01 2.015124804156993854e-02
2.419999999999999929e+01 2.013656664913942088e-02
2.422500000000000142e+01 2.012151221744230140e-02
2.425000000000000000e+01 2.010611108087716203e-02
2.427499999999999858e+01 2.009039701905514230e-02
2.430000000000000071e+01 2.007441068933893402e-02
2.432499999999999929e+01 2.005819892855312259e-02
2.435000000000000142e+01 2.004181393593808605e-02
2.437500000000000000e+01 2.002531235169104085e-02
2.439999999999999858e+01 2.000875424725564317e-02
2.442500000000000071e+01 1.999220204564628248e-02
2.444999999999999929e+01 1.997571939082627701e-02
2.447500000000000142e+01 1.995936998691041908e-02
2.450000000000000000e+01 1.994321642804471745e-02
2.452499999999999858e+01 1.992731904047381417e-02
2.455000000000000071e+01 1.991173475808797472e-02
2.457499999999999929e+01 1.989651605215149097e-02
2.460000000000000142e+01 1.988170993536471365e-02
2.462500000000000000e+01 1.986735705886695513e-02
2.464999999999999858e+01 1.985349091948999598e-02
2.467500000000000071e+01 1.984013719257659084e-02
2.469999999999999929e+01 1.982731320359261765e-02
2.472500000000000142e+01 1.981502754935046684e-02
2.475000000000000000e+01 1.980327987704099887e-02
2.477499999999999858e+01 1.979206082646448039e-02
2.480000000000000071e+01 1.978135213812437510e-02
2.482499999999999929e+01 1.977112692661018439e-02
2.485000000000000142e+01 1.976135011593543878e-02
2.487500000000000000e+01 1.975197903032168203e-02
2.489999999999999858e+01 1.974296413128453706e-02
2.492500000000000071e+01 1.973424988884899672e-02
2.494999999999999929e+01 1.972577577225861911e-02
2.497500000000000142e+01 1.971747734335167063e-02
2.500000000000000000e+01 1.970928743333348609e-02
2.502499999999999858e+01 1.970113738244061788e-02
2.505000000000000071e+01 1.969295832010492520e-02
2.507499999999999929e+01 1.968468246267257235e-02
2.510000000000000142e+01 1.967624440487363360e-02
2.512500000000000000e+01 1.966758238119536636e-02
2.514999999999999858e+01 1.965863947377237408e-02
2.517500000000000071e+01 1.964936474389239116e-02
2.519999999999999929e+01 1.963971426561221187e-02
2.522500000000000142e+01 1.962965204146700088e-02
2.525000000000000000e+01 1.961915078233951101e-02
2.527499999999999858e+01 1.960819253574922968e-02
2.530000000000000071e+01 1.959676914966560976e-02
2.532499999999999929e+01 1.958488256155179949e-02
2.535000000000000142e+01 1.957254490586138318e-02
2.537500000000000000e+01 1.955977843605268690e-02
2.539999999999999858e+01 1.954661526084770090e-02
2.542500000000000071e+01 1.953309689763962562e-02
2.544999999999999929e+01 1.951927364933209522e-02
2.547500000000000142e+01 1.950520381426889757e-02
2.550000000000000000e+01 1.949095274168244352e-02
2.552499999999999858e+01 1.947659174827594611e-02
2.555000000000000071e+01 1.946219691393915804e-02
2.557499999999999929e+01 1.944784777694938752e-02
2.560000000000000142e+01 1.943362595091667824e-02
2.562500000000000000e+01 1.941961368727235585e-02
2.564999999999999858e+01 1.940589240818554884e-02
2.567500000000000071e+01 1.939254123554307388e-02
2.569999999999999929e+01 1.937963554171734729e-02
2.572500000000000142e+01 1.936724554774432380e-02
2.575000000000000000e+01 1.935543499380764851e-02
2.577499999999999858e+01 1.934425990575595888e-02
2.580000000000000071e+01 1.933376747997276462e-02
2.582499999999999929e+01 1.932399510678740537e-02
2.585000000000000142e+01 1.931496955060692244e-02
2.587500000000000000e+01 1.930670630205597732e-02
2.589999999999999858e+01 1.929920911477005230e-02
2.592500000000000071e+01 1.929246973617964461e-02
2.594999999999999929e+01 1.928646783856594743e-02
2.597500000000000142e+01 1.928117115304122178e-02
2.600000000000000000e+01 1.927653580577468745e-02
2.602499999999999858e+01 1.927250685229034421e-02
2.605000000000000071e+01 1.926901900216464181e-02
2.607499999999999929e+01 1.926599752325478324e-02
2.610000000000000142e+01 1.926335931126196485e-02
2.612500000000000000e+01 1.926101410767278727e-02
2.614999999999999858e+01 1.925886584626549172e-02
2.617500000000000071e+01 1.925681410618850301e-02
2.619999999999999929e+01 1.925475564745937965e-02
2.622500000000000142e+01 1.925258600324939243e-02
2.625000000000000000e+01 1.925020110214515495e-02
2.627499999999999858e+01 1.924749889277115392e-02
2.630000000000000071e+01 1.924438094306760508e-02
2.632499999999999929e+01 1.924075398673671597e-02
2.635000000000000142e+01 1.923653139003508178e-02
2.637500000000000000e+01 1.923163451344061040e-02
2.639999999999999858e+01 1.922599394432780717e-02
2.642500000000000071e+01 1.921955057898968103e-02
2.644999999999999929e+01 1.921225653470776465e-02
2.647500000000000142e+01 1.920407587567500227e-02
2.650000000000000000e+01 1.919498513963563760e-02
2.652499999999999858e+01 1.918497365549908656e-02
2.655000000000000071e+01 1.917404364591841076e-02
2.657499999999999929e+01 1.916221011250960823e-02
2.660000000000000142e+01 1.914950050527297609e-02
2.662500000000000000e+01 1.913595418138085841e-02
2.664999999999999858e+01 1.912162166243615985e-02
2.667500000000000071e+01 1.910656370278985411e-02
2.669999999999999929e+01 1.909085018474728987e-02
2.672500000000000142e+01 1.907455885971279935e-02
2.675000000000000000e+01 1.905777395708863065e-02
2.677499999999999858e+01 1.904058468497215120e-02
2.680000000000000071e+01 1.902308364882689812e-02
2.682499999999999929e+01 1.900536521567904730e-02
2.685000000000000142e+01 1.898752385249308922e-02
2.687500000000000000e+01 1.896965246780621048e-02
2.689999999999999858e+01 1.895184078577986583e-02
2.692500000000000071e+01 1.893417378124535172e-02
2.694999999999999929e+01 1.891673020339987993e-02
2.697500000000000142e+01 1.889958121397577670e-02
2.700000000000000000e+01 1.888278916424735235e-02
2.702499999999999858e+01 1.886640653219117877e-02
2.705000000000000071e+01 1.885047503881654568e-02
2.707499999999999929e+01 1.883502495914512326e-02
2.710000000000000142e+01 1.882007464018823836e-02
2.712500000000000000e+01 1.880563023416163826e-02
2.714999999999999858e+01 1.879168565191543031e-02
2.717500000000000071e+01 1.877822273694989413e-02
2.719999999999999929e+01 1.876521165699640906e-02
2.722500000000000142e+01 1.875261150583615311e-02
2.725000000000000000e+01 1.874037110439537301e-02
2.727499999999999858e+01 1.872842998649045312e-02
2.730000000000000071e+01 1.871671955115226244e-02
2.732499999999999929e+01 1.870516436050979892e-02
2.735000000000000142e+01 1.869368355934969764e-02
2.737500000000000000e+01 1.868219239038144963e-02
2.739999999999999858e+01 1.867060377718994177e-02
2.742500000000000071e+01 1.865882994579163429e-02
2.744999999999999929e+01 1.864678405468913128e-02
2.747500000000000142e+01 1.863438180332909269e-02
2.750000000000000000e+01 1.862154298899713900e-02
2.752499999999999858e+01 1.860819298323948545e-02
2.755000000000000071e+01 1.859426410027468016e-02
2.757499999999999929e+01 1.857969683185509147e-02
2.760000000000000142e+01 1.856444092558349676e-02
2.762500000000000000e+01 1.854845628652869796e-02
2.764999999999999858e+01 1.853171368538896607e-02
2.767500000000000071e+01 1.851419526008400013e-02
2.769999999999999929e+01 1.849589480168840547e-02
2.772500000000000142e+01 1.847681781969792020e-02
2.775000000000000000e+01 1.845698138570162766e-02
2.777499999999999858e+01 1.843641375929820289e-02
2.780000000000000071e+01 1.841515380365427168e-02
2.782499999999999929e+01 1.839325020282586362e-02
2.785000000000000142e+01 1.837076049643138892e-02
2.787500000000000000e+01 1.834774995080633297e-02
2.789999999999999858e+01 1.832429028949790137e-02
2.792500000000000071e+01 1.830045830784939737e-02
2.794999999999999929e+01 1.827633439967840320e-02
2.797500000000000142e+01 1.825200102483361120e-02
2.800000000000000000e+01 1.822754114848977244e-02
2.802499999999999858e+01 1.820303668284567120e-02
2.805000000000000071e+01 1.817856696224762897e-02
2.807499999999999929e+01 1.815420728209327442e-02
2.810000000000000142e+01 1.813002753035696316e-02
2.812500000000000000e+01 1.810609093888842364e-02
2.814999999999999858e+01 1.808245297926933115e-02
2.817500000000000071e+01 1.805916042484389389e-02
2.819999999999999929e+01 1.803625059747274509e-02
2.822500000000000142e+01 1.801375081350104282e-02
2.825000000000000000e+01 1.799167803966893187e-02
2.827499999999999858e+01 1.797003876505489164e-02
2.830000000000000071e+01 1.794882909101285992e-02
2.832499999999999929e+01 1.792803503639005952e-02
2.835000000000000142e+01 1.790763305088229063e-02
2.837500000000000000e+01 1.788759072520578267e-02
2.839999999999999858e+01 1.786786768248407528e-02
2.842500000000000071e+01 1.784841663171957485e-02
2.844999999999999929e+01 1.782918456050234038e-02
2.847500000000000142e+01 1.781011404169829601e-02
2.850000000000000000e+01 1.779114462596337667e-02
2.852499999999999858e+01 1.777221429045032150e-02
2.855000000000000071e+01 1.775326091285853863e-02
2.857499999999999929e+01 1.773422373931008572e-02
2.860000000000000142e+01 1.771504481486143104e-02
2.862500000000000000e+01 1.769567034604431488e-02
2.864999999999999858e+01 1.767605196651271526e-02
2.867500000000000071e+01 1.765614787851985112e-02
2.869999999999999929e+01 1.763592384620487827e-02
2.872500000000000142e+01 1.761535401911580243e-02
2.875000000000000000e+01 1.759442156857876985e-02
2.877499999999999858e+01 1.757311912309894239e-02
2.880000000000000071e+01 1.755144899361631150e-02
2.882499999999999929e+01 1.752942318365200564e-02
2.885000000000000142e+01 1.750706318417775617e-02
2.887500000000000000e+01 1.748439955756969644e-02
2.889999999999999858e+01 1.746147131965258142e-02
2.892500000000000071e+01 1.743832513310018817e-02
2.894999999999999929e+01 1.741501432962929211e-02
2.897500000000000142e+01 1.739159778213669391e-02
2.900000000000000000e+01 1.736813865124155198e-02
2.902499999999999858e+01 1.734470303333788405e-02
2.905000000000000071e+01 1.732135853977760653e-02
2.907499999999999929e+01 1.729817283797027705e-02
2.910000000000000142e+01 1.727521218649606671e-02
2.912500000000000000e+01 1.725253999631523941e-02
2.914999999999999858e+01 1.723021544982290473e-02
2.917500000000000071e+01 1.720829220846645594e-02
2.919999999999999929e+01 1.718681723766818034e-02
2.922500000000000142e+01 1.716582977553489028e-02
2.925000000000000000e+01 1.714536046893699178e-02
2.927499999999999858e+01 1.712543069684640748e-02
2.930000000000000071e+01 1.710605209705664750e-02
2.932499999999999929e+01 1.708722630808168913e-02
2.935000000000000142e+01 1.706894493334025487e-02
2.937500000000000000e+01 1.705118973014092820e-02
2.939999999999999858e+01 1.703393302090839417e-02
2.942500000000000071e+01 1.701713831945410316e-02
2.944999999999999929e+01 1.700076116030382542e-02
2.947500000000000142e+01 1.698475011435191320e-02
2.950000000000000000e+01 1.696904797040872312e-02
2.952499999999999858e+01 1.695359305789603810e-02
2.955000000000000071e+01 1.693832068299079216e-02
2.957499999999999929e+01 1.692316464765643291e-02
2.960000000000000142e+01 1.690805881902540317e-02
2.962500000000000000e+01 1.689293871502522623e-02
2.964999999999999858e+01 1.687774307166677729e-02
2.967500000000000071e+01 1.686241535733456684e-02
2.969999999999999929e+01 1.684690520037122538e-02
2.972500000000000142e+01 1.683116969766240353e-02
2.975000000000000000e+01 1.681517457436286847e-02
2.977499999999999858e+01 1.679889516774201244e-02
2.980000000000000071e+01 1.678231721165615586e-02
2.982499999999999929e+01 1.676543740228178536e-02
2.985000000000000142e+01 1.674826373021413503e-02
2.987500000000000000e+01 1.673081556896207495e-02
2.989999999999999858e+01 1.671312351480868166e-02
2.992500000000000071e+01 1.669522897848622900e-02
2.994999999999999929e+01 1.667718353413670876e-02
2.997500000000000142e+01 1.665904803642641319e-02
3.000000000000000000e+01 1.664089152141648431e-02
3.002499999999999858e+01 1.662278991168268660e-02
3.005000000000000071e+01 1.660482455033891053e-02
3.007499999999999929e+01 1.658708059235972782e-02
3.010000000000000142e+01 1.656964528486033067e-02
3.012500000000000000e+01 1.655260617063188805e-02
3.014999999999999858e+01 1.653604925079939919e-02
3.017500000000000071e+01 1.652005714400732461e-02
3.019999999999999929e+01 1.650470727961851625e-02
3.022500000000000142e+01 1.649007016208313067e-02
3.025000000000000000e+01 1.647620774254592718e-02
3.027499999999999858e+01 1.646317193186173575e-02
3.030000000000000071e+01 1.645100328647694510e-02
3.032499999999999929e+01 1.643972989549384936e-02
3.035000000000000142e+01 1.642936649347629882e-02
3.037500000000000000e+01 1.641991381891728594e-02
3.039999999999999858e+01 1.641135823396912621e-02
3.042500000000000071e+01 1.640367161541545285e-02
3.044999999999999929e+01 1.639681152204181327e-02
3.047500000000000142e+01 1.639072163767004223e-02
3.050000000000000000e+01 1.638533248396706679e-02
3.052499999999999858e+01 1.638056239146714124e-02
3.055000000000000071e+01 1.637631871236586503e-02
3.057499999999999929e+01 1.637249925346484156e-02
3.060000000000000142e+01 1.636899390321358561e-02
3.062500000000000000e+01 1.636568642297432305e-02
3.064999999999999858e+01 1.636245636888343513e-02
3.067500000000000071e+01 1.635918110813404727e-02
3.069999999999999929e+01 1.635573789134407707e-02
3.072500000000000142e+01 1.635200594119261275e-02
3.075000000000000000e+01 1.634786851711437020e-02
3.077499999999999858e+01 1.634321491609974294e-02
3.080000000000000071e+01 1.633794237045902917e-02
3.082499999999999929e+01 1.633195780553397397e-02
3.085000000000000142e+01 1.632517942265424016e-02
3.087500000000000000e+01 1.631753807621639379e-02
3.089999999999999858e+01 1.630897841737109777e-02
3.092500000000000071e+01 1.629945978166217821e-02
3.094999999999999929e+01 1.628895680259198966e-02
3.097500000000000142e+01 1.627745973883951147e-02
3.100000000000000000e+01 1.626497450802657369e-02
3.102499999999999858e+01 1.625152242623402812e-02
3.105000000000000071e+01 1.623713965777346876e-02
3.107499999999999929e+01 1.622187638587991101e-02
3.110000000000000142e+01 1.620579572010813832e-02
3.112500000000000000e+01 1.618897236193675773e-02
3.114999999999999858e+01 1.617149105417303376e-02
3.117500000000000071e+01 1.615344484476254061e-02
3.119999999999999929e+01 1.613493319861421249e-02
3.122500000000000142e+01 1.611605999448270654e-02
3.125000000000000000e+01 1.609693144589564515e-02
3.127499999999999858e+01 1.607765398673319593e-02
3.130000000000000071e+01 1.605833216257865348e-02
3.132499999999999929e+01 1.603906656908016259e-02
3.135000000000000142e+01 1.601995187703641327e-02
3.137500000000000000e+01 1.600107498289817312e-02
3.139999999999999858e+01 1.598251332008220754e-02
3.142500000000000071e+01 1.596433336393861604e-02
3.144999999999999929e+01 1.594658935884249104e-02
3.147500000000000142e+01 1.592932229183807030e-02
3.150000000000000000e+01 1.591255913213026132e-02
3.152499999999999858e+01 1.589631235056663747e-02
3.155000000000000071e+01 1.588057972759983338e-02
3.157499999999999929e+01 1.586534445279576960e-02
3.160000000000000142e+01 1.585057551272044118e-02
3.162500000000000000e+01 1.583622835882740262e-02
3.164999999999999858e+01 1.582224584105942911e-02
3.167500000000000071e+01 1.580855938800539151e-02
3.169999999999999929e+01 1.579509040912769474e-02
3.172500000000000142e+01 1.578175189064305595e-02
3.175000000000000000e+01 1.576845015260348523e-02
3.177499999999999858e+01 1.575508673159214251e-02
3.180000000000000071e+01 1.574156035098579418e-02
3.182499999999999929e+01 1.572776893922388924e-02
3.185000000000000142e+01 1.571361165529413378e-02
3.187500000000000000e+01 1.569899088099314943e-02
3.189999999999999858e+01 1.568381413996587648e-02
3.192500000000000071e+01 1.566799590516795501e-02
3.194999999999999929e+01 1.565145925893163606e-02
3.197500000000000142e+01 1.563413737256842337e-02
3.200000000000000000e+01 1.561597477661797601e-02
3.202499999999999858e+01 1.559692839677793985e-02
3.204999999999999716e+01 1.557696833580023837e-02
3.207500000000000284e+01 1.555607838682051755e-02
3.210000000000000142e+01 1.553425626901244749e-02
3.212500000000000000e+01 1.551151358263710284e-02
3.214999999999999858e+01 1.548787548592050686e-02
3.217499999999999716e+01 1.546338010251911352e-02
3.220000000000000284e+01 1.543807767353162427e-02
3.222500000000000142e+01 1.541202947356310957e-02
3.225000000000000000e+01 1.538530651515179065e-02
3.227499999999999858e+01 1.535798807030387680e-02
3.229999999999999716e+01 1.533016004154457573e-02
3.232500000000000284e+01 1.530191321811096711e-02
3.235000000000000142e+01 1.527334145492780307e-02
3.237500000000000000e+01 1.524453981375795825e-02
3.239999999999999858e+01 1.521560270629709194e-02
3.242499999999999716e+01 1.518662207892045729e-02
3.245000000000000284e+01 1.515768567749789866e-02
3.247500000000000142e+01 1.512887542888571857e-02
3.250000000000000000e+01 1.510026597294101755e-02
3.252499999999999858e+01 1.507192337541434293e-02
3.254999999999999716e+01 1.504390404792114999e-02
3.257500000000000284e+01 1.501625389674826963e-02
3.260000000000000142e+01 1.498900771682280461e-02
3.262500000000000000e+01 1.496218884180989389e-02
3.264999999999999858e+01 1.493580905584581189e-02
3.267499999999999716e+01 1.490986876611256895e-02
3.270000000000000284e+01 1.488435743015344842e-02
3.272500000000000142e+01 1.485925422601903217e-02
3.275000000000000000e+01 1.483452894783751641e-02
3.277499999999999858e+01 1.481014310481653176e-02
3.279999999999999716e+01 1.478605119684579805e-02
3.282500000000000284e+01 1.476220213631927736e-02
3.285000000000000142e+01 1.473854078246982988e-02
3.287500000000000000e+01 1.471500955223375290e-02
3.289999999999999858e+01 1.469155007001820071e-02
3.292499999999999716e+01 1.466810481805998212e-02
3.295000000000000284e+01 1.464461874919364276e-02
3.297500000000000142e+01 1.462104082490502631e-02
3.300000000000000000e+01 1.459732544336647675e-02
3.302499999999999858e+01 1.457343372476903667e-02
3.304999999999999716e+01 1.454933462485691331e-02
3.307500000000000284e+01 1.452500585139059573e-02
3.310000000000000142e+01 1.450043456301790094e-02
3.312500000000000000e+01 1.447561783522945296e-02
3.314999999999999858e+01 1.445056288324483330e-02
3.317499999999999716e+01 1.442528703768404183e-02
3.320000000000000284e+01 1.439981747424109136e-02
3.322500000000000142e+01 1.437419070468643097e-02
3.325000000000000000e+01 1.434845184166790889e-02
3.327499999999999858e+01 1.432265365530774673e-02
3.329999999999999716e+01 1.429685544419893525e-02
3.332500000000000284e+01 1.427112174781426483e-02
3.335000000000000142e+01 1.424552093088234973e-02
3.337500000000000000e+01 1.422012367328475725e-02
3.339999999999999858e+01 1.419500140113485152e-02
3.342499999999999716e+01 1.417022469607094257e-02
3.345000000000000284e+01 1.414586172023038838e-02
3.347500000000000142e+01 1.412197669391445157e-02
3.350000000000000000e+01 1.409862846184859950e-02
3.352499999999999858e+01 1.407586918161689293e-02
3.354999999999999716e+01 1.405374316528451052e-02
3.357500000000000284e+01 1.403228590116749330e-02
3.360000000000000142e+01 1.401152327910907755e-02
3.362500000000000000e+01 1.399147103717847396e-02
3.364999999999999858e+01 1.397213444310547309e-02
3.367499999999999716e+01 1.395350821787212300e-02
3.370000000000000284e+01 1.393557670341563984e-02
3.372500000000000142e+01 1.391831427057740357e-02
3.375000000000000000e+01 1.390168595794694269e-02
3.377499999999999858e+01 1.388564832649606673e-02
3.379999999999999716e+01 1.387015051024204429e-02
3.382500000000000284e+01 1.385513543822103691e-02
3.385000000000000142e+01 1.384054119914095507e-02
3.387500000000000000e+01 1.382630251658486902e-02
3.389999999999999858e+01 1.381235230010930196e-02
3.392499999999999716e+01 1.379862323556411453e-02
3.395000000000000284e+01 1.378504937694810181e-02
3.397500000000000142e+01 1.377156770210836816e-02
3.400000000000000000e+01 1.375811959514257506e-02
3.402499999999999858e+01 1.374465221997992100e-02
3.404999999999999716e+01 1.373111975215152158e-02
3.407500000000000284e+01 1.371748443875381827e-02
3.410000000000000142e+01 1.370371746064637415e-02
3.412500000000000000e+01 1.368979957533954066e-02
3.414999999999999858e+01 1.367572152419447781e-02
3.417499999999999716e+01 1.366148419273237671e-02
3.420000000000000284e+01 1.364709851885745166e-02
3.422500000000000142e+01 1.363258514939450113e-02
3.425000000000000000e+01 1.361797385139566254e-02
3.427499999999999858e+01 1.360330269032856282e-02
3.429999999999999716e+01 1.358861699275814433e-02
3.432500000000000284e+01 1.357396811626671203e-02
3.435000000000000142e+01 1.355941205399582850e-02
3.437500000000000000e+01 1.354500790511396942e-02
3.439999999999999858e+01 1.353081624594147485e-02
3.442499999999999716e+01 1.351689743880910878e-02
3.445000000000000284e+01 1.350330991763998290e-02
3.447500000000000142e+01 1.349010848977814751e-02
3.450000000000000000e+01 1.347734269376041523e-02
3.452499999999999858e+01 1.346505525144923825e-02
3.454999999999999716e+01 1.345328065137269338e-02
3.457500000000000284e+01 1.344204389707654747e-02
3.460000000000000142e+01 1.343135945106661670e-02
3.462500000000000000e+01 1.342123040055618818e-02
3.464999999999999858e+01 1.341164786646494023e-02
3.467499999999999716e+01 1.340259067186735584e-02
3.470000000000000284e+01 1.339402528024114664e-02
3.472500000000000142e+01 1.338590600808981478e-02
3.475000000000000000e+01 1.337817551020609695e-02
3.477499999999999858e+01 1.337076553001715143e-02
3.479999999999999716e+01 1.336359790121196059e-02
3.482500000000000284e+01 1.335658578146319171e-02
3.485000000000000142e+01 1.334963509354596643e-02
3.487500000000000000e+01 1.334264614464900425e-02
3.489999999999999858e+01 1.333551539022105256e-02
3.492499999999999716e+01 1.332813730570834437e-02
3.495000000000000284e+01 1.332040632654185076e-02
3.497500000000000142e+01 1.331221881525473591e-02
3.500000000000000000e+01 1.330347501368430753e-02
3.502499999999999858e+01 1.329408093824076091e-02
3.504999999999999716e+01 1.328395017738307100e-02
3.507500000000000284e+01 1.327300555219607971e-02
3.510000000000000142e+01 1.326118060391786799e-02
3.512500000000000000e+01 1.324842087564765195e-02
3.514999999999999858e+01 1.323468496009511455e-02
3.517499999999999716e+01 1.321994528976808310e-02
3.520000000000000284e+01 1.320418865197184669e-02
3.522500000000000142e+01 1.318741641632425673e-02
3.525000000000000000e+01 1.316964446931305661e-02
3.527499999999999858e+01 1.315090285595601517e-02
3.529999999999999716e+01 1.313123513566611528e-02
3.532500000000000284e+01 1.311069746504370942e-02
3.535000000000000142e+01 1.308935742650518992e-02
3.537500000000000000e+01 1.306729262709943086e-02
3.539999999999999858e+01 1.304458909663805052e-02
3.542499999999999716e+01 1.302133951898968439e-02
3.545000000000000284e+01 1.299764133349133773e-02
3.547500000000000142e+01 1.297359474674627741e-02
3.550000000000000000e+01 1.294930069657059500e-02
3.552499999999999858e+01 1.292485881116311672e-02
3.554999999999999716e+01 1.290036540647465993e-02
3.557500000000000284e+01 1.287591156402657748e-02
3.560000000000000142e+01 1.285158132960547740e-02
3.562500000000000000e+01 1.282745007045375113e-02
3.564999999999999858e+01 1.280358302551523782e-02
3.567499999999999716e+01 1.278003407848464154e-02
3.570000000000000284e+01 1.275684477914539386e-02
3.572500000000000142e+01 1.273404363256964468e-02
3.575000000000000000e+01 1.271164567022586497e-02
3.577499999999999858e+01 1.268965231076232998e-02
3.579999999999999716e+01 1.266805151197032166e-02
3.582500000000000284e+01 1.264681820914599630e-02
3.585000000000000142e+01 1.262591502882345375e-02
3.587500000000000000e+01 1.260529326095576053e-02
3.589999999999999858e+01 1.258489406707002413e-02
3.592499999999999716e+01 1.256464989684219838e-02
3.595000000000000284e+01 1.254448608117215806e-02
3.597500000000000142e+01 1.252432256612519626e-02
3.600000000000000000e+01 1.250407574913746898e-02
3.602499999999999858e+01 1.248366037695441098e-02
3.604999999999999716e+01 1.246299146353039428e-02
3.607500000000000284e+01 1.244198618598852009e-02
3.610000000000000142e+01 1.242056571736488771e-02
3.612500000000000000e+01 1.239865695657643707e-02
3.614999999999999858e+01 1.237619411848721492e-02
3.617499999999999716e+01 1.235312015040694079e-02
3.620000000000000284e+01 1.232938794513679243e-02
3.622500000000000142e+01 1.230496132567374806e-02
3.625000000000000000e+01 1.227981578181268005e-02
3.627499999999999858e+01 1.225393894444916885e-02
3.629999999999999716e+01 1.222733078944886718e-02
3.632500000000000284e+01 1.220000356910072928e-02
3.635000000000000142e+01 1.217198147509777895e-02
3.637500000000000000e+01 1.214330004315368645e-02
3.639999999999999858e+01 1.211400531508024028e-02
3.642499999999999716e+01 1.208415277938033536e-02
3.645000000000000284e+01 1.205380611641370531e-02
3.647500000000000142e+01 1.202303577831745005e-02
3.650000000000000000e+01 1.199191743738403995e-02
3.652499999999999858e+01 1.196053033946322709e-02
3.654999999999999716e+01 1.192895560060384247e-02
3.657500000000000284e+01 1.189727448651443573e-02
3.660000000000000142e+01 1.186556671416090156e-02
3.662500000000000000e+01 1.183390881432760423e-02
3.664999999999999858e+01 1.180237259218523682e-02
3.667499999999999716e+01 1.177102372039435553e-02
3.670000000000000284e+01 1.173992049607600584e-02
3.672500000000000142e+01 1.170911278893562585e-02
3.675000000000000000e+01 1.167864120328531569e-02
3.677499999999999858e+01 1.164853647161173331e-02
3.679999999999999716e+01 1.161881909184534108e-02
3.682500000000000284e+01 1.158949921478112308e-02
3.685000000000000142e+01 1.156057678220509186e-02
3.687500000000000000e+01 1.153204191042317764e-02
3.689999999999999858e+01 1.150387550818506022e-02
3.692499999999999716e+01 1.147605011248805512e-02
3.695000000000000284e+01 1.144853092074341223e-02
3.697500000000000142e+01 1.142127699321237623e-02
3.700000000000000000e+01 1.139424259567486768e-02
3.702499999999999858e+01 1.136737864912659192e-02
3.704999999999999716e+01 1.134063425083489327e-02
3.707500000000000284e+01 1.131395822952913598e-02
3.710000000000000142e+01 1.128730069679509569e-02
3.712500000000000000e+01 1.126061455693937174e-02
3.714999999999999858e+01 1.123385693875405232e-02
3.717499999999999716e+01 1.120699051445262634e-02
3.720000000000000284e+01 1.117998467404015747e-02
3.722500000000000142e+01 1.115281652668928883e-02
3.725000000000000000e+01 1.112547170516173071e-02
3.727499999999999858e+01 1.109794495394861996e-02
3.729999999999999716e+01 1.107024048704903507e-02
3.732500000000000284e+01 1.104237210698786123e-02
3.735000000000000142e+01 1.101436308233337026e-02
3.737500000000000000e+01 1.098624578679524913e-02
3.739999999999999858e+01 1.095806110884401698e-02
3.742499999999999716e+01 1.092985764612583975e-02
3.745000000000000284e+01 1.090169070420141280e-02
3.747500000000000142e+01 1.087362112385547100e-02
3.750000000000000000e+01 1.084571396518769354e-02
3.752499999999999858e+01 1.081803708023991277e-02
3.754999999999999716e+01 1.079065960842017467e-02
3.757500000000000284e+01 1.076365043092002323e-02
3.760000000000000142e+01 1.073707662116483302e-02
3.762500000000000000e+01 1.071100192844245090e-02
3.764999999999999858e+01 1.068548533101638942e-02
3.767499999999999716e+01 1.066057969332060047e-02
3.770000000000000284e+01 1.063633055930645645e-02
3.772500000000000142e+01 1.061277511075009103e-02
3.775000000000000000e+01 1.058994131535251695e-02
3.777499999999999858e+01 1.056784728491794167e-02
3.779999999999999716e+01 1.054650085889405862e-02
3.782500000000000284e+01 1.052589942316668921e-02
3.785000000000000142e+01 1.050602996842858565e-02
3.787500000000000000e+01 1.048686938672653036e-02
3.789999999999999858e+01 1.046838499919346480e-02
3.792499999999999716e+01 1.045053530245064824e-02
3.795000000000000284e+01 1.043327091604467904e-02
3.797500000000000142e+01 1.041653570853325224e-02
3.800000000000000000e+01 1.040026807561847573e-02
3.802499999999999858e+01 1.038440234016796242e-02
3.804999999999999716e+01 1.036887024110840286e-02
3.807500000000000284e+01 1.035360247607101242e-02
3.810000000000000142e+01 1.033853026143606406e-02
3.812500000000000000e+01 1.032358687302590942e-02
3.814999999999999858e+01 1.030870913112245067e-02
3.817499999999999716e+01 1.029383879482492770e-02
3.820000000000000284e+01 1.027892383289777910e-02
3.822500000000000142e+01 1.026391954109554146e-02
3.825000000000000000e+01 1.024878947964688418e-02
3.827499999999999858e+01 1.023350620868773188e-02
3.829999999999999716e+01 1.021805180424626043e-02
3.832500000000000284e+01 1.020241814247407550e-02
3.835000000000000142e+01 1.018660694526495394e-02
3.837500000000000000e+01 1.017062958598657918e-02
3.839999999999999858e+01 1.015450665966601690e-02
3.842499999999999716e+01 1.013826732748119089e-02
3.845000000000000284e+01 1.012194845071113375e-02
3.847500000000000142e+01 1.010559353422471086e-02
3.850000000000000000e+01 1.008925150404217132e-02
3.852499999999999858e+01 1.007297534741705310e-02
3.854999999999999716e+01 1.005682064709088885e-02
3.857500000000000284e+01 1.004084404389391470e-02
3.860000000000000142e+01 1.002510166355010182e-02
3.862500000000000000e+01 1.000964754439805961e-02
3.864999999999999858e+01 9.994532102788037178e-03
3.867499999999999716e+01 9.979800672049725374e-03
3.870000000000000284e+01 9.965492149278052866e-03
3.872500000000000142e+01 9.951637781731142648e-03
3.875000000000000000e+01 9.938260121433598396e-03
3.877499999999999858e+01 9.925372172752178895e-03
3.879999999999999716e+01 9.912976753287018258e-03
3.882500000000000284e+01 9.901066083556842035e-03
3.885000000000000142e+01 9.889621615699369442e-03
3.887500000000000000e+01 9.878614105972530202e-03
3.889999999999999858e+01 9.868003930242735464e-03
3.892499999999999716e+01 9.857741636118100628e-03
3.895000000000000284e+01 9.847768719943760657e-03
3.897500000000000142e+01 9.838018611772825564e-03
3.900000000000000000e+01 9.828417846656558113e-03
3.902499999999999858e+01 9.818887396370566678e-03
3.904999999999999716e+01 9.809344131998166960e-03
3.907500000000000284e+01 9.799702384869636573e-03
3.910000000000000142e+01 9.789875571073738611e-03
3.912500000000000000e+01 9.779777843357385578e-03
3.914999999999999858e+01 9.769325733619273547e-03
3.917499999999999716e+01 9.758439749432459495e-03
3.920000000000000284e+01 9.747045889122975953e-03
3.922500000000000142e+01 9.735077041793493979e-03
3.925000000000000000e+01 9.722474241372921697e-03
3.927499999999999858e+01 9.709187747075916086e-03
3.929999999999999716e+01 9.695177926695844328e-03
3.932500000000000284e+01 9.680415923608069739e-03
3.935000000000000142e+01 9.664884093342254387e-03
3.937500000000000000e+01 9.648576200820898827e-03
3.939999999999999858e+01 9.631497374796847352e-03
3.942499999999999716e+01 9.613663821516616609e-03
3.945000000000000284e+01 9.595102305132840320e-03
3.947500000000000142e+01 9.575849407568596053e-03
3.950000000000000000e+01 9.555950585560728933e-03
3.952499999999999858e+01 9.535459047085101045e-03
3.954999999999999716e+01 9.514434473391951072e-03
3.957500000000000284e+01 9.492941616293024826e-03
3.960000000000000142e+01 9.471048802989186569e-03
3.962500000000000000e+01 9.448826382726090850e-03
3.964999999999999858e+01 9.426345150691180924e-03
3.967499999999999716e+01 9.403674784907325654e-03
3.970000000000000284e+01 9.380882331386741074e-03
3.972500000000000142e+01 9.358030771499279213e-03
3.975000000000000000e+01 9.335177703442585490e-03
3.977499999999999858e+01 9.312374166860870606e-03
3.979999999999999716e+01 9.289663636234377769e-03
3.982500000000000284e+01 9.267081204550550039e-03
3.985000000000000142e+01 9.244652974285647723e-03
3.987500000000000000e+01 9.222395667803386793e-03
3.989999999999999858e+01 9.200316464082715015e-03
3.992499999999999716e+01 9.178413063447419043e-03
3.995000000000000284e+01 9.156673976582863778e-03
3.997500000000000142e+01 9.135079028987273461e-03
\end{filecontents}

\begin{tikzpicture}
  \begin{groupplot}[
      group style={
          group name=my plots,
          group size=1 by 2,
          xlabels at=edge bottom,
          xticklabels at=edge bottom,
          vertical sep=2pt,
          every axis yticklabel/.style={/pgf/number format/fixed}
      },
      width=\columnwidth,
      height=0.5\columnwidth,
      xlabel={Time (\si{\pico\second})},
      %xmin=0, xmax=40,
      %ytick align=outside,
      %xtick align=outside
  ]
  \nextgroupplot[ylabel = {Population}, tick label style={/pgf/number format/fixed}]
  \addplot graphics[xmin=0, xmax=40, ymin=0, ymax=1]{figures/rendered_1024.png};
  \nextgroupplot[ylabel = {IPR}, yticklabel style={/pgf/number format/fixed}] 
  \addplot[smooth, thick] table {ipr_1024.dat};
  \end{groupplot}

\end{tikzpicture}

  \caption{\label{fig:population dynamics}Population dynamics of 512 interacting \qds{}.
    The majority of the \qds{} follow a trajectory prescribed by the incident pulse; several pairs of dots, however, become strongly correlated due to the small distance between particles.
  }
\end{figure}

\begin{figure}
  \usetikzlibrary{pgfplots.groupplots}

\begin{tikzpicture}
  %\begin{axis}[
      %width=\columnwidth,
      %xlabel={Time (\si{\pico\second})},
      %xmin=-2.5, xmax=32.5,
      %ymin = -0.05, ymax = 1.05,
      %%ytick align=outside,
      %%xtick align=outside
  %]
    %\addplot graphics[xmin=0, xmax=30, ymin=0.064480053660309997, ymax=1]{figures/full_neighborhood.png};

  %\end{axis}
  \begin{groupplot}[
      group style={
          group name=my plots,
          group size=1 by 2,
          xlabels at=edge bottom,
          xticklabels at=edge bottom,
          vertical sep=2pt,
      },
      width=\columnwidth,
      xlabel={Time (\si{\pico\second})},
      %xmin=0, xmax=40,
      %ytick align=outside,
      %xtick align=outside
  ]
  \nextgroupplot[ylabel={Population (full system)}]
  \addplot graphics[xmin=0, xmax=30, ymin=0.064480053660309997, ymax=1]{figures/full_neighborhood.png};

  \nextgroupplot[ylabel={Population (reduced system)}] 
  \addplot graphics[xmin=0, xmax=30, ymin=0.069909905907700007, ymax=1]{figures/empty_neighborhood.png};
  \end{groupplot}

\end{tikzpicture}

\end{figure}

\begin{figure}
  \begin{filecontents}{low_density_stats.dat}
time       lbound               ubound               p1                   p2                   p3                  p4                   p5
0.         1.                   1.                   1.                   1.                   1.                  1.                   1.
0.078125   0.9999999999999      0.9999999999998946   0.9999999999999      0.9999999999999      0.9999999999999     0.9999999999999      0.9999999999999
0.15625    0.9999999999996766   0.9999999999997039   0.9999999999997039   0.9999999999997039   0.9999999999997039  0.9999999999997039   0.9999999999997039
0.234375   0.9999999999989145   0.9999999999989125   0.9999999999989145   0.9999999999989145   0.9999999999989125  0.9999999999989145   0.9999999999989125
0.3125     0.9999999999970063   0.9999999999970499   0.9999999999969937   0.9999999999970001   0.9999999999970938  0.9999999999970001   0.9999999999970938
0.390625   0.99999999999245     0.9999999999927423   0.999999999992752    0.99999999999265     0.99999999999305    0.9999999999925521   0.99999999999305
0.46875    0.9999999999825251   0.9999999999831289   0.999999999983229    0.9999999999832015   0.9999999999843     0.9999999999827249   0.999999999984382
0.546875   0.9999999999612627   0.9999999999627056   0.9999999999632002   0.9999999999631002   0.9999999999661752  0.9999999999617992   0.9999999999662054
0.625      0.9999999999168      0.99999999992        0.9999999999217      0.9999999999214      0.9999999999293     0.999999999918       0.9999999999294
0.703125   0.9999999998256843   0.999999999832929    0.9999999998375467   0.9999999998368531   0.9999999998555663  0.9999999998285041   0.9999999998558663
0.78125    0.9999999996423422   0.9999999996581422   0.9999999996697118   0.9999999996681102   0.9999999997106547  0.9999999996487445   0.9999999997112508
0.859375   0.9999999992791107   0.9999999993125065   0.999999999339728    0.9999999993364441   0.9999999994292796  0.9999999992928211   0.9999999994305672
0.9375     0.999999998569725    0.9999999986390062   0.9999999987001875   0.9999999986931375   0.9999999988897376  0.9999999985987187   0.9999999988923937
1.015625   0.9999999972029845   0.9999999973439773   0.9999999974765635   0.9999999974620872   0.9999999978672954  0.999999997262927    0.9999999978725185
1.09375    0.9999999946040414   0.9999999948860445   0.9999999951645633   0.9999999951355742   0.999999995952236   0.9999999947255962   0.9999999959624032
1.171875   0.9999999897246853   0.9999999902794305   0.9999999908502152   0.9999999907926893   0.9999999924063944  0.9999999899667134   0.9999999924261191
1.25       0.9999999806772      0.9999999817522      0.9999999828966      0.9999999827847      0.9999999859178     0.9999999811514      0.9999999859555
1.328125   0.9999999641058444   0.9999999661596316   0.9999999684107173   0.9999999681956588   0.9999999741833879  0.9999999650209497   0.9999999742544897
1.40625    0.9999999341203234   0.9999999379919086   0.9999999423434203   0.999999941936007    0.9999999532134328  0.9999999358618914   0.9999999533456156
1.484375   0.9999998805127724   0.9999998877176602   0.9999998959969445   0.999999895235224    0.9999999161864744  0.9999998837832101   0.9999999164292046
1.5625     0.9999997858163875   0.9999997990588125   0.99999981458        0.9999998131732561   0.9999998515939688  0.9999997918798688   0.9999998520345937
1.640625   0.9999996205227129   0.9999996445691579   0.9999996732637487   0.9999996706974601   0.9999997402786032  0.9999996316262906   0.9999997410687443
1.71875    0.9999993354131078   0.9999993785644281   0.9999994309118304   0.9999994262858226   0.999999550778725   0.9999993555024804   0.9999995521787812
1.796875   0.9999988494369534   0.999998925977659    0.9999990202588946   0.9999990120170591   0.9999992321243149  0.99999888535668     0.9999992345760081
1.875      0.9999980308254      0.9999981650469      0.999998332755       0.9999983182406      0.9999987028665     0.9999980943075      0.9999987071089
1.953125   0.9999966680912614   0.9999969008178453   0.9999971955368236   0.9999971702685125   0.9999978346397712  0.9999967790055864   0.9999978418943024
2.03125    0.999994426148486    0.9999948251885695   0.9999953369727758   0.999995293482036    0.9999964279399992  0.9999946177490157   0.9999964401989296
2.109375   0.9999907809010966   0.9999914575729711   0.99999233592475     0.999992261915205    0.9999941770377875  0.9999911081837475   0.999994197508629
2.1875     0.9999849231996499   0.9999860581307249   0.9999875482287812   0.9999874236997938   0.9999906200385063  0.9999854760468125   0.999990653818025
2.265625   0.9999756199722132   0.9999775028612659   0.9999800019172024   0.999979794730624    0.9999850690845915  0.9999765435555782   0.9999851241645275
2.34375    0.9999610165564773   0.9999641066693219   0.999968250414368    0.9999679095543047   0.9999765146326617  0.9999625425946477   0.9999766033784648
2.421875   0.9999383597919819   0.9999433768154811   0.9999501704220667   0.9999496158906328   0.999963496744935   0.9999408538091639   0.9999636380325826
2.5        0.9999036164126      0.9999116751277      0.9999226886779      0.9999217965613      0.9999439356015     0.9999076482459      0.9999441578559
2.578125   0.9998509559287367   0.9998637632172755   0.9998814194949128   0.9998800002057994   0.9999149132101003  0.9998574035177563   0.9999152586500555
2.65625    0.9997720619329703   0.9997922011488258   0.9998201934104148   0.9998179604529672   0.9998723988821985  0.999782262040454    0.9998729293454305
2.734375   0.9996552313587024   0.9996865672792986   0.9997304570512395   0.9997269828478313   0.9998109129012488  0.9996711954468688   0.9998117176878228
2.8125     0.9994842184356062   0.9995324660277938   0.9996005260670563   0.9995951804461313   0.9997231262587625  0.9995089375550875   0.9997243324854812
2.890625   0.9992367803661268   0.9993102924433263   0.9994146777213414   0.9994065436156955   0.9995993999024834  0.9992746496217709   0.9996011858835544
2.96875    0.9988828865807867   0.9989937288633953   0.9991520783172664   0.9991398381285891   0.9994272748815274  0.9989402873965453   0.9994298870327328
3.046875   0.9983825644733001   0.998547960999227    0.9987855538430577   0.998767338902892    0.999190935172475   0.9984686511718573   0.9991947088364997
3.125      0.9976833742329      0.9979276202989      0.9982802309919      0.9982534257575      0.9988706777735     0.9978111197006      0.9988760621717
3.203125   0.9967175353372095   0.9970744872736295   0.9975920999346912   0.9975530922530047   0.9984424390343205  0.9969050976892144   0.9984500262217978
3.28125    0.9953987694584641   0.9959150276202571   0.996666579574732    0.9966104488341602   0.9978774410358398  0.9956712459817781   0.9978879982489891
3.359375   0.993618980804713    0.9943578798974713   0.9954371993460109   0.9953573360308129   0.9971420354165149  0.9940106145988868   0.9971565395432571
3.4375     0.9912449640562062   0.9922914678207688   0.9938245454621187   0.9937121989406187   0.9961978315914313  0.9918018595785375   0.9962175036541
3.515625   0.9881154122026002   0.9895832537542455   0.9917356507524706   0.9915794075223294   0.9950021993872545  0.9888987942821125   0.995028535939921
3.59375    0.9840330502335469   0.9860774374394672   0.9890640296693398   0.9888492320821516   0.9935092294113211  0.9851285988180914   0.9935440266756203
3.671875   0.9787684104187114   0.9815922022295206   0.9856905670694373   0.9853986928637383   0.9916712154872249  0.9802910797627862   0.9917165809829128
3.75       0.9720671475764      0.9759240651306      0.9814854545973      0.981093490141       0.9894406908299     0.9741594270168      0.9894990363475
3.828125   0.9636412417677562   0.9688495115798114   0.9763113246628029   0.9757911788057045   0.9867730032590429  0.9664829396424272   0.9868470124472211
3.90625    0.9531782830854844   0.9601298573257531   0.9700276559084242   0.969345675318825    0.9836293575181367  0.9569921728563906   0.9837219213149399
3.984375   0.940350226836327    0.949518443833311    0.9624964154839586   0.9616130741384014   0.9799801894266061  0.9454068768705962   0.98009430324919
4.0625     0.9248259809316125   0.9367702515743875   0.9535887681843438   0.9524586106098876   0.97580867518195    0.931446939039875    0.9759472933459937
4.140625   0.9062879653314677   0.9216537952707667   0.9431925334936083   0.9417644506184915   0.9711141281132818  0.9148462973822669   0.9712799734755783
4.21875    0.8844523825700101   0.9039648930312609   0.9312199270412773   0.9294378326836258   0.9659150037942922  0.8953694669815132   0.9661103340706507
4.296875   0.8590924201980394   0.8835416035653595   0.9176150069673171   0.9154189608348738   0.9602512303053388  0.8728299321343997   0.9604775667913488
4.375      0.8300630074565      0.8602793376968      0.9023601800649      0.8996879701553      0.9541856062409     0.8471092426488      0.954443435519
4.453125   0.797325145606242    0.8341449204843179   0.8854811277603589   0.8822702839408716   0.9478040651546377  0.8181752719767977   0.9480925294732134
4.53125    0.7609673305668829   0.8051882565480345   0.867049596642357    0.8632397627559274   0.9412146848819641  0.7860978195251908   0.9415312818588633
4.609375   0.7212213095005034   0.7735502715516245   0.8471836577248939   0.8427192079612615   0.9345454135103701  0.7510596411414816   0.934885730233649
4.6875     0.6784694917851125   0.7394659964531938   0.82604525953935     0.8208780127347      0.92794058035555    0.7133611413503063   0.9282980904025937
4.765625   0.6332418414808125   0.7032620171742071   0.8038351503348283   0.797927018798053    0.9215563463187204  0.6734173812542057   0.9219223003192204
4.84375    0.5862010330170928   0.6653480043942827   0.7807854938508507   0.7741109047024156   0.9155553157632312  0.6317467401821804   0.9159187565203296
4.921875   0.538115968158292    0.6262026077899774   0.757150717242194    0.7496986645210709   0.9101005751399077  0.5889514480089559   0.9104485066577732
5.         0.4898252400173      0.5863545679869      0.7331972809006      0.7249729029761      0.9053494407646     0.5456911552103      0.9056671764503
5.078125   0.44219354330636235  0.5463603955121606   0.7091931341258738   0.7002187578403213   0.9014471928507287  0.5026515860651977   0.9017189012269529
5.15625    0.39606508879327046  0.5067803158115719   0.6853976116061203   0.6757132547739158   0.8985210507141719  0.4605109680063548   0.8987305066699179
5.234375   0.35221856968858267  0.468154339101631    0.6620524421363999   0.6517158118065453   0.896674614381025   0.41990723128475876  0.8968061496002859
5.3125     0.3113280358256625   0.43098026432631875  0.6393743975897314   0.6284604559323188   0.89598296787035    0.38140886752626246  0.896022594953625
5.390625   0.27393319416960527  0.3956951795750097   0.6175499290259521   0.6061501168200264   0.8964886185594099  0.345491842720054    0.8964252789769228
5.46875    0.2404213408663085   0.36266161768042804  0.5967319410816742   0.5849531483534757   0.8981984391310984  0.31252417182835535  0.898025295748743
5.546875   0.21102159395354841  0.33215902149818805  0.5770386668365752   0.565002022066357    0.9010817834869621  0.28275881647021744  0.9007974456378228
5.625      0.1858106261409      0.3043806422394      0.5585544438436      0.5463939611987      0.9050699621794     0.2563346293822      0.9046794988543
5.703125   0.16472791822276858  0.27943549870748224  0.5413320689128489   0.5291931533260952   0.9100572722788027  0.23328427319008108  0.9095728436492537
5.78125    0.14759781793144613  0.2573546251950235   0.5253963332217032   0.5134341033904251   0.9159037660833914  0.21354748903087037  0.9153446946336875
5.859375   0.13415543392527152  0.23810056935337312  0.510748311181313    0.4991256685019188   0.9224398941346678  0.1969878238979159   0.9218320153566097
5.9375     0.12407355741493126  0.22157897775599375  0.49736998965388124  0.48625534339858123  0.929473052545925   0.1834109248416625   0.9288472416768312
6.015625   0.11698827884848356  0.20765112571917943  0.4852288724764208   0.474793432907348    0.9367958997933525  0.17258272799007054  0.9361857706203071
6.09375    0.1125216051042453   0.19614637758586093  0.4742822657190273   0.46469683917499766  0.944196090989293   0.1642462275342656   0.9436350005414844
6.171875   0.11030005869975097  0.18687377000618566  0.46448103134755026  0.4559122932882453   0.9514668391549044  0.1581359289162365   0.9509844895029012
6.25       0.1099688542827      0.1796321547372      0.4557726809548      0.4483789602564      0.9584175001163     0.1539895036076      0.9580365729666
6.328125   0.1112017380133709   0.17421858094172218  0.4481037582534835   0.4420304317399074   0.9648832457866809  0.15155652026275185  0.9646165956086266
6.40625    0.11370691988851328  0.17043481410291486  0.4414215241530125   0.4367961859259047   0.9707328896146742  0.15060440321852733  0.9705818178316382
6.484375   0.11722972983031267  0.16809206067620872  0.4356750080126682   0.43260263481771233  0.975874081024693   0.15092195446866408  0.9758280953807934
6.5625     0.121552710518675    0.16701408811551874  0.4308155223749812   0.42937389574011875  0.9802553839087125  0.15232087705301248  0.9802936204468126
6.640625   0.12649385004151126  0.16703899958208604  0.4267967567764817   0.42703241924466434  0.9838651444049776  0.15463576892735253  0.9839593345232877
6.71875    0.13180978585417188  0.16801994865697187  0.4235745710494484   0.4254995844053281   0.9867274610521164  0.1577230361436922   0.9868460273668945
6.796875   0.1368780131693621   0.16982507221793966  0.421106602505772    0.4246963408501117   0.988895914006053   0.16145912074147775  0.9890085476895826
6.875      0.1422785092828      0.1723368893859      0.4193517873265      0.4245439409772      0.9904459241644     0.1657383669683      0.9905278893642
6.953125   0.14792819647124347  0.17545137229778826  0.4182698774556434   0.42496477112151365  0.9914666712007429  0.17047077333946464  0.9915021243226685
7.03125    0.15376102510100467  0.17907684796776094  0.4178210129420086   0.42588326138748045  0.9920534089530983  0.17557980533789375  0.992037200116261
7.109375   0.1597249852486237   0.18313284569146102  0.4179653884012985   0.4272268330389255   0.9923008181560509  0.1810003799397031   0.9922385165659727
7.1875     0.16577945997360624  0.1875489659493      0.41866303296544377  0.42892683099213746  0.9922977849912374  0.18667708212655     0.9922039804285062
7.265625   0.17189294936866595  0.19229451640393272  0.4198737070047285   0.4309193865137794   0.9921237428312617  0.19256263451916888  0.9920189620318984
7.34375    0.1780411614590508   0.19729025855536328  0.4215569068073485   0.4331461602853594   0.9918465058117696  0.198616614442943    0.9917532970893821
7.421875   0.18416214779896534  0.20248470496492743  0.4236719604097726   0.43555492652728905  0.9915213789526988  0.20480439606172238  0.9914602308341399
7.5        0.1901331409475      0.2078378972888      0.4261781936326      0.4380999722317      0.9911912566695     0.2110962864516      0.9911770164347
7.578125   0.19607878823998018  0.21331550438418964  0.42903514444649815  0.4407422999496698   0.9908874103685527  0.21746682194394695  0.9909267635188014
7.65625    0.20203731004541325  0.21888818032252108  0.432202805295975    0.44344963602296955  0.9906307001389812  0.22389419244274453  0.9907210837702415
7.734375   0.20799740720795026  0.22453098356229353  0.43564187605218335  0.4461962573295664   0.9904330054767396  0.23035976516493747  0.9905630891668722
7.8125     0.21394991787624376  0.23022284920088126  0.43931401410473125  0.4489626583354      0.9902987360736251  0.2368476842659625   0.9904503508004062
7.890625   0.21988742915547912  0.23594610808663216  0.4431820720267711   0.4517350856920482   0.9902263422448175  0.24334452786704788  0.9903775097410021
7.96875    0.2258039639363719   0.24168604842453753  0.44721031680497736  0.454504970454868    0.9902097856393484  0.24983900875541018  0.9903383319111813
8.046875   0.2316947266001067   0.24743051697817509  0.45136462754678763  0.45726828849907414  0.9902399516047358  0.2563217090715103   0.9903271046512756
8.125      0.2375558949216      0.2531695578812      0.4556126706628      0.4600248782044      0.9903059870935     0.262784842427       0.9903393723524
8.203125   0.24338444851371596  0.25910597262184604  0.45992405287754795  0.46277774185454856  0.9903965378826947  0.2692220392697212   0.9903720926738201
8.28125    0.2491780265059227   0.265045134167543    0.4642704530884875   0.4655323536803641   0.9905008442182632  0.27562815290367815  0.9904233561592125
8.359375   0.2549348089312525   0.2709449128319876   0.46862573419509046  0.46829599356210805  0.9906096424240007  0.2819990845511249   0.9904918459179316
8.4375     0.2605837799151375   0.27680140701073125  0.4729660358580625   0.47107712154826875  0.9907158177653125  0.2883316264165187   0.9905762193395251
8.515625   0.26596298415945857  0.2826115879583829   0.4772698487152713   0.47388480455135046  0.9908147641355589  0.2946233219125029   0.9906745727337896
8.59375    0.271311695191857    0.28837315095951793  0.48151807012661796  0.47672820319374765  0.9909044288690445  0.3008723422414148   0.9907841078908493
8.671875   0.2766290118364014   0.294084388021792    0.4856940411534851   0.4796161237739408   0.9909850530576904  0.3070773784464419   0.9909010646309178
8.75       0.2819143404577      0.2997440799078      0.4897835641456      0.4825566377651      0.9910586535059     0.3132375479105      0.9910209246521
8.828125   0.2871673221533725   0.3053514053356017   0.49377490020374726  0.48555676914976253  0.9911283254660618  0.3193523141697719   0.9911388384561317
8.90625    0.2923877739185602   0.31090586522899455  0.49765874581053754  0.48862224818277344  0.9911974690506788  0.3254214188112227   0.9912501866646438
8.984375   0.29757564166902994  0.31640721999389104  0.5014281880371292   0.49175732887327334  0.9912690519067175  0.331444824171192    0.9913511650017766
9.0625     0.3027309632116438   0.321855437939075    0.5050786380734      0.494964666448825    0.9913450139146938  0.33742266555144373  0.9914392809429687
9.140625   0.30785383945980543  0.3272506531227288   0.5086077431470336   0.4982452503649033   0.991425896741987   0.3433552116971161   0.9915136684832524
9.21875    0.3129444124033484   0.33259313108524213  0.512015277313064    0.5015983879370781   0.9915107453673461  0.34924283235209763  0.991575161580875
9.296875   0.31800284854201616  0.33788324111982293  0.5153030120612836   0.5050217333517634   0.9915972856723405  0.35508597179464624  0.991626110259366
9.375      0.3230293266852      0.3431214339125      0.5184745680715      0.5085113566911      0.9916823388358     0.3608851273669      0.9916699683601
9.453125   0.328024029195293    0.3483082235595546   0.5215352498806413   0.5120618475539424   0.9917623964531644  0.3666408321217679   0.9917107210303562
9.53125    0.3329871359097149   0.353444173130793    0.5244918655809985   0.5156664479089976   0.9918342560783782  0.3723536408300282   0.9917522466824539
9.609375   0.33791882011511537  0.35852988308870865  0.5273525339473634   0.5193172089655937   0.9918956090463183  0.37802411869936536  0.9917977182654139
9.6875     0.34281924606851877  0.3635659820003625   0.5301264816766125   0.5230051669807625   0.9919454821818563  0.38365283226365     0.991849140880875
9.765625   0.34768856766028594  0.36855311908473004  0.5328238335743172   0.5267205331644714   0.9919844605542941  0.3892403419935816   0.9919070987455174
9.84375    0.35252692790101325  0.3734919582280586   0.5354553986480922   0.5304528930682609   0.9920146554595328  0.3947871962653859   0.9919707485916289
9.921875   0.3573344589846518   0.37838317317407294  0.538032455131922    0.5341914111041869   0.9920394241601718  0.4002939263960207   0.9920380549880999
10.        0.362111282738       0.3832274436555      0.5405665374377      0.5379250361294      0.9920628885126     0.4057610425161      0.9921062228835
10.078125  0.36685751131299715  0.3880254522811636   0.5430692279800113   0.5416427043221002   0.9920893319557536  0.4111890301031819   0.9921722504957401
10.15625   0.37157324801643987  0.392777882031086    0.5455519566869798   0.5453335358965985   0.9921225731498124  0.4165783470402297   0.9922335068211884
10.234375  0.3762585882004435   0.39748541424307154  0.5480258108475594   0.5489870225254019   0.9921654170533544  0.42192942110016246  0.9922882353660757
10.3125    0.38091362016104374  0.4021487269933812   0.5505013577272875   0.5525932026848313   0.9922192701779438  0.42724264778212495  0.9923358994575625
10.390625  0.3855384260081182   0.4067684937971664   0.5529884821349631   0.5561428224866406   0.9922839787886216  0.4325183884484671   0.9923773121087502
10.46875   0.3901330824864703   0.41134538256383435  0.5554962408511609   0.5596274799121663   0.9923579116967454  0.43775696872708586  0.9924145301230234
10.546875  0.39469766173525994  0.41588005475607925  0.5580327355143144   0.563039750737643    0.9924382692210171  0.4429586771556318   0.9924505317551741
10.625     0.3992322319824      0.4203731647084      0.5606050052616      0.5663732947835      0.9925215635671     0.4481237640516      0.9924887331273
10.703125  0.4037368581738812   0.42482535906928764  0.5632189400869609   0.569622941485622    0.9926041890797419  0.453252440600252    0.9925324249470944
10.78125   0.4082116025437969   0.42923727633641334  0.5658792155508391   0.5727847541392532   0.9926829878910695  0.45834487815352193  0.9925842235844641
10.859375  0.4126565251309756   0.433609546459345    0.5685892491761452   0.5758560724777502   0.992755719244901   0.46340120773546145  0.9926456274116648
10.9375    0.41707168425055624  0.4379427904913      0.5713511785206999   0.57883553360155     0.9928213585668625  0.46842151975033747  0.9927167511684812
11.015625  0.421457136927898    0.44223762027078745  0.5741658606410546   0.5817230715545981   0.9928801819693183  0.4734058638878528   0.9927962812009642
11.09375   0.42581293930232417  0.4464946381249656   0.5770328923714226   0.5845198961339594   0.9929336282643985  0.4783542492184375   0.9928816578005046
11.171875  0.4301391470089324   0.45071443658412585  0.5799506505561152   0.5872284518055612   0.9929839676452265  0.48326664446915213  0.9929694536196865
11.25      0.4344358155412      0.4548975981046      0.582916351148       0.5898523578186      0.9930338377627     0.488142978469       0.9930558854867
11.328125  0.4387030006013341   0.45904469479754995  0.5859261258350449   0.5923963308618949   0.9930857289355334  0.49298314074957417  0.9931373759196923
11.40625   0.44294075843929614  0.46315628816727034  0.5889751146474624   0.594866091808921    0.9931415072737719  0.49778698228312657  0.9932110736765594
11.484375  0.44714914618191137  0.4672329288615839   0.5920575728228695   0.5972682582667176   0.9932020566742317  0.5025543163413837   0.9932752506355588
11.5625    0.4513282221533687   0.47127515644195     0.5951669900189562   0.599610224834125    0.9932670994147688  0.5072849194541312   0.9933295137646062
11.640625  0.4554780461834173   0.4752834991804065   0.5982962198365503   0.6019000330925527   0.9933352241925656  0.5119785324473417   0.993374802083918
11.71875   0.4595986799039242   0.4792584738917195   0.6014376174949195   0.6041462334708398   0.9934041151703453  0.5166348615398773   0.9934131741434086
11.796875  0.4636901870282282   0.4832005858104404   0.6045831834096005   0.6063577412191634   0.9934709418181005  0.5212535794798349   0.9934474256693558
11.875     0.4677526336118      0.4873142647384      0.6077247103709      0.6085436887706      0.9935328427253     0.5258343267026      0.9934806040213
11.953125  0.4717860882898117   0.4914381612318979   0.6108539319885845   0.6107132768056358   0.9935874213759821  0.5303767124943786   0.9935155016715592
12.03125   0.47579062248822973  0.49552872207311877  0.613962670077479    0.6128756263216468   0.9936331705613352  0.5348803161518391   0.9935542125956531
12.109375  0.4797663106053974   0.4995862267036934   0.6170429786918843   0.6150396339770419   0.9936697546333872  0.5393446881270525   0.9935978232772711
12.1875    0.4837132301614      0.5036109520237124   0.6200872825907312   0.6172138329037375   0.9936981029832936  0.5437693511583312   0.9936462861474562
12.265625  0.487631461913092    0.5076031723374983   0.6230885080172104   0.6194062610857479   0.9937202997178689  0.548153801386007    0.9936984919306333
12.34375   0.491521089934632    0.5115631593220304   0.626040203800396    0.6216243392703922   0.9937392882295875  0.552497509462493    0.9937525238317946
12.421875  0.4953822016617745   0.5154911820137695   0.6289366509593837   0.6238747602160737   0.993758439584911   0.5567999216672984   0.9938060463821995
12.5       0.4992148879023      0.5193875068128      0.63177295916        0.6261633909056      0.9937810554106     0.5610604610461      0.9938567600436
12.578125  0.5030192428129104   0.523252397501074    0.6345451485872593   0.6284951891474467   0.9938098858280712  0.565278528592005    0.9939028427681201
12.65625   0.5067953638466828   0.5270861152708227   0.6372502160291134   0.63087413575665     0.9938467394481149  0.5694535044974836   0.9939433031390649
12.734375  0.510543351673126    0.5308889187605795   0.6398861841844257   0.6333031832899425   0.9938922463548509  0.573584749500605    0.9939781856325667
12.8125    0.5142633100756688   0.5346610640949062   0.642452133488325    0.6357842220362687   0.9939458091297625  0.57767160635795     0.9940085941807875
12.890625  0.5179553458312016   0.5384028049241574   0.6449482159937681   0.6383180637195657   0.9940057458027141  0.581713401470186    0.994036531108976
12.96875   0.5216195685762696   0.5421143924625359   0.6473756511135117   0.6409044431090015   0.9940695973141414  0.5857094466916922   0.9940645795042382
13.046875  0.5252560906652103   0.545796075519494    0.6497367033177361   0.6435420374440524   0.9941345458002935  0.589659041349566    0.994095483010478
13.125     0.5288650270257      0.5494481005248      0.652034642134       0.6462285033372      0.9941978730508     0.5935614744955      0.9941316936679
13.203125  0.5324464950161933   0.5530707115429788   0.6542736850805702   0.6489605305310305   0.994257383413453   0.5974160274116874   0.9941749631490792
13.28125   0.5360006142900102   0.5566641502787024   0.6564589244343867   0.6517339116262845   0.9943117229475179  0.601221976384036    0.9942260449014305
13.359375  0.5395275066706949   0.5602286560696161   0.6585962389749408   0.654543626655606    0.9943605452226171  0.6049785957541558   0.9942845558259444
13.4375    0.5430272960413812   0.5637644658695625   0.6606921921171938   0.6573839411139313   0.9944045006672875  0.6086851612519687   0.9943490193705875
13.515625  0.5465001082516546   0.5672718142205406   0.6627539180587917   0.6602485158423922   0.9944450561784105  0.6123409536079112   0.9944170819198667
13.59375   0.5499460710440344   0.570750933216486    0.6647889977868344   0.6631305269509      0.9944841798904992  0.6159452624347891   0.994485866041918
13.671875  0.5533653140005566   0.5742020524585959   0.666805326979398    0.6660227937707708   0.994523947897643   0.6194973903659973   0.9945524021953247
13.75      0.5567579685116      0.5776253990066      0.668810977992       0.6689179126804      0.9945661417648     0.6229966574298      0.9946140684541
13.828125  0.5601241677641825   0.5810211973256116   0.6708140582539955   0.6718083945111836   0.9946119060418644  0.6264424056336362   0.9946689675000832
13.90625   0.563464046751739    0.5843896692335695   0.6728225674883852   0.6746868031517063   0.9946615237657243  0.6298340037309548   0.9947161814923429
13.984375  0.5667777423006846   0.5877310338501591   0.6748442562269857   0.6775458929120138   0.9947143472015124  0.6331708521364295   0.9947558663462516
14.0625    0.5700653931139875   0.591045507549825    0.6768864881086437   0.6803787421937563   0.9947688943455562  0.636452387956075    0.9947891739216125
14.140625  0.573327139827289    0.5943333039224656   0.6789561084145038   0.6831788810408519   0.9948230936153837  0.6396780900937522   0.9948180189957047
14.21875   0.5765631250758719   0.5975946337419616   0.6810593212383655   0.6859404102077391   0.9948746343277601  0.6428474843981211   0.9948447330215476
14.296875  0.5797734935676983   0.6008297049454566   0.6832015775580979   0.688658109503506    0.9949213631404877  0.6459601488117392   0.9948716644161124
14.375     0.5829583921611      0.6040387226244      0.6853874763419      0.691327533309       0.9949616594179     0.6490157184854      0.9949007926528
14.453125  0.5861179699419936   0.6072218890280491   0.6876206806146443   0.6939450913668684   0.99499472652806    0.6520138908228763   0.9949334196893814
14.53125   0.5892523782989812   0.6103794035793414   0.6899038501776062   0.6965081131746297   0.9950207505898571  0.6549544304220632   0.9949699880234523
14.609375  0.5923617709921316   0.6135114629038454   0.6922385924242922   0.6990148945608758   0.9950409005725908  0.6578371738802314   0.9950100525287263
14.6875    0.5954463042144125   0.616618260870025    0.6946254323738813   0.7014647253445813   0.9950571701424374  0.66066203443595     0.9950524069176375
14.765625  0.5985061366424073   0.619699988640862    0.6970638027297081   0.7038578972879161   0.9950720877242032  0.6634290064188901   0.9950953397462304
14.84375   0.6015414294755281   0.6227568347343656   0.6995520544231586   0.706195691901093    0.9950883426400828  0.6661381694843672   0.9951369735554828
14.921875  0.6045523464624649   0.6257889850917553   0.7020874877112365   0.7084803480456403   0.9951083883998857  0.6687896926100734   0.9951756275500255
15.        0.6075390539139      0.6287966231513      0.704666403536       0.7107150096439      0.9951340871753     0.6713838378352      0.9952101412055
15.078125  0.6105017207024379   0.631779929926099    0.7072841744405487   0.7129036542122341   0.9951664518760499  0.6739209637221826   0.9952401037072309
15.15625   0.6134405182489062   0.6347390840819578   0.7099353339404516   0.7150510033316898   0.9952055254485266  0.6764015285266758   0.9952659507540039
15.234375  0.6163556204968721   0.6376742620162815   0.7126136828743276   0.7171624165431066   0.9952504140077507  0.6788260930560318   0.9952889130679511
15.3125    0.6192472038761313   0.6405856379330875   0.7153124108537937   0.719243770556125    0.9952994650012374  0.6811953232043      0.9953108261109562
15.390625  0.6221154472571383   0.6434733839140777   0.7180242305937176   0.7213013260011212   0.995350558128838   0.6835099921446736   0.9953338337185632
15.46875   0.6249605318975179   0.6463376699845305   0.7207415225712804   0.7233415842830335   0.9954014589578242  0.6857709821653992   0.9953600358552812
15.546875  0.6277826413832558   0.6491786641716392   0.7234564871699501   0.7253711373898754   0.9954501759765457  0.6879792861301144   0.9953911396053745
15.625     0.6305819615666      0.6519965325558      0.7261613012309      0.7273965137415      0.9954952625779     0.6901360085443      0.9954281714865
15.703125  0.6333586805016149   0.6547914393144041   0.7288482757457857   0.7294240233534198   0.9955360158664023  0.6922423662055353   0.9954712985113774
15.78125   0.6361129883810086   0.6575635467575071   0.7315100113091305   0.731459605705789    0.9955725423715797  0.6942996884152047   0.9955197870071094
15.859375  0.6388450774748748   0.6603130153565446   0.734139547891153    0.7335086837665727   0.9956056835385251  0.6963094167222392   0.9955721052290665
15.9375    0.6415551420727438   0.6630400037667687   0.7367305055268125   0.7355760275845687   0.9956368174626999  0.6982731041715062   0.9956261520659125
16.015625  0.6442433784314053   0.6657446688445936   0.7392772126139554   0.737665630768728    0.9956675737275044  0.7001924140230155   0.9956795736593828
16.09375   0.6469099847283563   0.6684271656606907   0.741774818695661    0.7397806029922297   0.9956995121072125  0.7020691179048806   0.9957301158271383
16.171875  0.649555161021294    0.6710876475112931   0.7442193888794598   0.7419230813796475   0.9957338210645945  0.7039050933646029   0.9957759550381481
16.25      0.6521791092156      0.6737262659286      0.7466079773718      0.7440941633099      0.9957710876864     0.7057023207768      0.9958159550602
16.328125  0.6547820330371129   0.6763431706923537   0.7489386780380816   0.7462938627364493   0.995811177814914   0.7074628795677711   0.9958498094227561
16.40625   0.6573641380120616   0.6789385098439328   0.7512106503859609   0.7485210916324273   0.9958532459184547  0.7091889437189328   0.9958780492331282
16.484375  0.6599256314513219   0.6815124297051588   0.7534241198989847   0.7507736676410356   0.9958958720330804  0.7108827765095798   0.9959019183163137
16.5625    0.662466722439375    0.6840650749020125   0.7555803522731375   0.7530483483762063   0.9959373015989125  0.7125467244687125   0.9959231394051562
16.640625  0.6649876218247016   0.6865965883952989   0.7576816017227301   0.7553408922012459   0.9959757468649115  0.7141832105060709   0.9959436125183541
16.71875   0.667488542211368    0.6891071115191352   0.7597310341866399   0.7576461446403032   0.9960096984979656  0.7157947262051532   0.9959650969743367
16.796875  0.6699696979488616   0.6915967840267491   0.7617326269446593   0.7599581488912068   0.9960381947224889  0.7173838232673698   0.9959889300588604
16.875     0.6724313051185      0.694065744145       0.7636910467914      0.7622702782569      0.996061002819      0.7189531041088      0.9960158281055
16.953125  0.6748735815136084   0.6965141286367211   0.7656115095592244   0.7645753876617463   0.9960786827749406  0.7205052116251615   0.9960458010028939
17.03125   0.6772967466118694   0.6989420728702727   0.7674996243559656   0.7668659808310961   0.9960925227053751  0.7220428181526961   0.9960781914958304
17.109375  0.679701021537347    0.701349710896879    0.7693612263894586   0.7691343891923779   0.9961043569628466  0.7235686136693439   0.9961118294810309
17.1875    0.682086629010575    0.7037371755324688   0.7712022026872563   0.7713729581122251   0.9961162971260438  0.7250852932946562   0.9961452723170625
17.265625  0.6844537932838078   0.7061045984453418   0.773028315330771    0.7735742357554969   0.9961304200474851  0.7265955441638849   0.9961770882071719
17.34375   0.6868027400609211   0.7084521102464625   0.7748450270338015   0.7757311596378478   0.9961484636785055  0.7281020317644922   0.9962061331582429
17.421875  0.6891336964007101   0.7107798405817375   0.776657333958334    0.7778372358628921   0.9961715794450942  0.7296073858388238   0.996231773890386
17.5       0.6914468906008      0.7130879182249      0.7784696106028      0.7798867060937      0.9962001800086     0.7311141859681      0.9962540188904
17.578125  0.6937425520640613   0.7153764711695582   0.7802854713772476   0.7818746975151739   0.99623390505235    0.7326249469638095   0.9962735358336846
17.65625   0.6960209111454577   0.7176456267185125   0.7821076531248875   0.7837973514026132   0.9962717080716984  0.7341421041991508   0.9962915530608414
17.734375  0.6982821989792518   0.7198955115705028   0.7839379223656895   0.7856519263879502   0.9963120473473833  0.7356679990232186   0.9963096623755419
17.8125    0.7005266472891063   0.7221262519016562   0.7857770103965375   0.7874368731644563   0.9963531475639374  0.7372048643975688   0.9963295568990187
17.890625  0.7027544881789043   0.724337973442675    0.7876245786540832   0.7891518780968169   0.996393287692789   0.7387548109004393   0.9963527483218896
17.96875   0.7049659539079985   0.7265308015500375   0.7894792159146126   0.7907978740483931   0.9964310674189547  0.7403198132394477   0.9963803109904196
18.046875  0.7071612766510068   0.7287048612718592   0.7913384679777816   0.7923770176698697   0.9964656090230675  0.7419016974078562   0.9964126954010861
18.125     0.7093406882442      0.7308602774071      0.7931988995454      0.7938926333426      0.9964966633119     0.7435021286142      0.9964496418048
18.203125  0.7115044199197825   0.7329971745601407   0.7950561870161      0.7953491249736308   0.9965246048671724  0.745122600103277    0.9964902077640159
18.28125   0.7136527020320882   0.7351156771902047   0.7969052399508758   0.7967518578214507   0.996550320704243   0.7467644229769641   0.9965329045627165
18.359375  0.7157857637751667   0.7372159096555075   0.7987403480609148   0.7981070134467596   0.9965750142035873  0.7484287171102671   0.9965759195075666
18.4375    0.7179038328958249   0.7392979962546562   0.8005553497017      0.799421421774125    0.996599959998925   0.750116403242275    0.9966173872994187
18.515625  0.7200071354038492   0.7413620612652736   0.8023438171348204   0.8007023749831439   0.9966262530779726  0.751828196310803    0.9966556660966938
18.59375   0.7220958952814767   0.7434082289800211   0.8040992532057618   0.8019574285800063   0.9966545955591485  0.7535646000832898   0.9966895737385304
18.671875  0.7241703341937066   0.7454366237423505   0.8058152936375623   0.8031941954662954   0.9966851574828914  0.7553259031207624   0.9967185468297692
18.75      0.7262306712019      0.747447369982       0.807485908866       0.8044201391061      0.9967175348501     0.757112176104       0.9967426986383
18.828125  0.7282771224814576   0.7494405922508812   0.8091055992586428   0.8056423719924104   0.9967508112859484  0.7589232705337601   0.9967627687883382
18.90625   0.7303099010455468   0.7514164152607922   0.8106695776748867   0.8068674655097594   0.9967837120235234  0.7607588188082414   0.9967799756239828
18.984375  0.7323292164752069   0.7533749639222129   0.8121739336176699   0.8081012770008613   0.9968148233456795  0.7626182356745047   0.9967957978891292
19.0625    0.7343352746566      0.7553163633855312   0.8136157737416      0.80934879934595     0.9968428397661813  0.764500721038075    0.99681172345975
19.140625  0.7363282775262632   0.7572407390840086   0.8149933341405697   0.8106140377035326   0.996866796850191   0.766405264113712    0.9968290074648085
19.21875   0.7383084228241906   0.7591482167787532   0.8163060606590946   0.8118999172469219   0.9968862502909109  0.7683306488936649   0.9968484796292344
19.296875  0.7402759038532539   0.761038922605213    0.8175546544483303   0.8132082247570244   0.9969013711523854  0.770275460903983    0.996870431564346
19.375     0.7422309092474      0.7629129831204      0.8187410810172      0.8145395859022      0.9969129413907     0.7722380952212      0.9968946006936
19.453125  0.7441736227446378   0.7647705253504227   0.8198685421711123   0.8158934788961331   0.9969222504693668  0.7742167657137748   0.9969202509130252
19.53125   0.746104222966932    0.7666116768378195   0.8209414113861977   0.8172682840698023   0.99693091025415    0.7762095154719617   0.9969463338051711
19.609375  0.7480228832042265   0.7684365656871239   0.8219651342926749   0.8186613677665172   0.9969406186884732  0.7782142283876332   0.9969717009430338
19.6875    0.7499297712033937   0.7702453206077687   0.8229460970796625   0.8200691978365875   0.9969529108567562  0.7802286418374375   0.9969953297085813
19.765625  0.7518250489594664   0.7720380709543658   0.8238914666333097   0.8214874870035748   0.9969689376421822  0.7822503604208293   0.9970165233960062
19.84375   0.7537088725108859   0.7738149467624461   0.8248090071394625   0.8229113594625688   0.9969893071375305  0.7842768706964844   0.9970350513308804
19.921875  0.7555813917363687   0.7755760787790322   0.8257068786582082   0.8243355352950462   0.99701401317506    0.786305556852979    0.997051205390715
20.        0.7574427501544      0.777321598488       0.8265934237592      0.825754526707       0.9970424607203     0.7883337172411      0.997065763778
20.078125  0.7592930847248841   0.7790516381291691   0.8274769487259229   0.8271628396826464   0.9970735818718925  0.7903585816867407   0.997079868715139
20.15625   0.7611325256539562   0.7807663307108992   0.8283655060266406   0.8285551744597976   0.997106021519332   0.792377329491       0.9970948392164336
20.234375  0.7629611962023446   0.7824658100160685   0.8292666847421007   0.829926618243359    0.9971383607490866  0.7943871080175902   0.9971119508558675
20.3125    0.764779212498       0.78415021060185     0.8301874154058125   0.8312728238149437   0.9971693405865688  0.7963850517526687   0.9971322197415438
20.390625  0.7665866833557794   0.7858196677931774   0.8311337952763155   0.8325901681301038   0.9971980494718418  0.7983683017180861   0.9971562269122173
20.46875   0.768383710104554    0.7874743176698062   0.8321109394359437   0.8338758856291915   0.9972240447156898  0.800334025105829    0.9971840123448664
20.546875  0.7701703864251445   0.7891142970492422   0.8331228622856245   0.8351281718059356   0.9972473898543381  0.802279434996038    0.9972150559460254
20.625     0.7719467981996      0.7907397434643      0.8341723930661      0.8363462535224      0.9972686042676     0.8042018100152      0.997248348339
20.703125  0.7737130233767792   0.7923507951372473   0.8352611279456336   0.8375304236455732   0.9972885362118781  0.806098513784516    0.99728253939281
20.78125   0.7754691318545015   0.7939475909510765   0.8363894200421914   0.8386820387487351   0.9973081830590148  0.8079670140091344   0.9973161397617992
20.859375  0.7772151853828411   0.795530270418927    0.8375564075548473   0.8398034798107992   0.997328490950011   0.8098049010540834   0.9973477422888243
20.9375    0.7789512374898563   0.7970989736523313   0.8387600789218125   0.8408980770978188   0.997350168940475   0.8116099058565875   0.997376227299225
21.015625  0.7806773334330936   0.7986538413288461   0.839997372739354    0.8419700015841863   0.9973735496198128  0.8133799170285605   0.9974009189900325
21.09375   0.7823935101792195   0.8001950146603336   0.8412643090386868   0.8430241264016665   0.9973985197652594  0.8151129970025602   0.9974216686518875
21.171875  0.7840997964137676   0.801722635361402    0.8425561474707179   0.8440658628421001   0.9974245323087773  0.8168073970876433   0.9974388528927302
21.25      0.7857962125826      0.8032368456202      0.8438675670667      0.8451009763055      0.9974506969004     0.8184615713028      0.997453289188
21.328125  0.7874827709673206   0.8047377880689796   0.845192861501661    0.8461353883235391   0.9974759329875711  0.8200741888692716   0.997466084629897
21.40625   0.7891594757950929   0.8062256057573516   0.8465261432541374   0.8471749713000358   0.9974991588562414  0.8216441452497757   0.9974784444098266
21.484375  0.7908263233835867   0.8077004421259625   0.8478615497339699   0.848225342929537    0.9975194841994929  0.8231705716332416   0.9974914726313671
21.5625    0.7924833023218812   0.809162440981375    0.8491934443536313   0.8492916673324812   0.9975363734223438  0.8246528427737125   0.9975059985881125
21.640625  0.7941303936861446   0.8106117464706641   0.85051660566055     0.8503784697956779   0.9975497520453991  0.8260905831059979   0.997522456641322
21.71875   0.7957675712904836   0.8120485030568414   0.851826398019111    0.8514894716250906   0.997560038293407   0.8274836710632851   0.9975408382491937
21.796875  0.7973948019705509   0.8134728554917051   0.8531189179421821   0.8526274509935096   0.9975680945757058  0.8288322415430139   0.9975607221822012
21.875     0.7990120459004      0.8148849487883      0.8543911109733      0.8537941348578      0.997575106913      0.8301366864673      0.9975813756836
21.953125  0.8006192569388983   0.8162849281898114   0.8556408550269055   0.8549901260022732   0.9975824122329743  0.831397653401714    0.9976019075058425
22.03125   0.8022163830045672   0.8176729391355343   0.8568670072587461   0.8562148680891735   0.9975913018624836  0.8326160422063227   0.9976214453454133
22.109375  0.8038033664769132   0.8190491272225839   0.8580694127941869   0.8574666503359902   0.9976028331528656  0.8337929996989937   0.9976393065542121
22.1875    0.8053801446208125   0.8204136381629125   0.8592488750288312   0.8587426520366187   0.9976176794469875  0.8349299123312875   0.9976551326610062
22.265625  0.8069466500331282   0.8217666177352583   0.8604070885841976   0.8600390257667088   0.997636041902858   0.836028396884825    0.9976689648283229
22.34375   0.8085028111066578   0.8231082117313437   0.8615465373816765   0.8613510167213516   0.9976576362127688  0.8370902892133555   0.9976812477165343
22.421875  0.8100485525116279   0.8244385658962223   0.8626703616184639   0.8626731142978269   0.9976817547918637  0.8381176310721843   0.9976927615433416
22.5       0.8115837956897      0.8257578258637      0.8637821986072      0.8639992308428      0.9977073926681     0.8391126550949      0.9977044943052
22.578125  0.8131084593604051   0.8270661370852822   0.8648860034916022   0.8653229014228563   0.997733415113008   0.840077767994853    0.9977174761538855
22.65625   0.8146224600363093   0.8283636447549844   0.8659858566713469   0.8666374976343952   0.997758738683186   0.8410155320945336   0.9977326041189648
22.734375  0.8161257125451652   0.8296504937287655   0.8670857653600913   0.8679364478657844   0.9977824957774419  0.8419286453006138   0.9977504867752776
22.8125    0.8176181305577312   0.8309268284407937   0.8681894670228375   0.8692134560887188   0.9978041562396063  0.8428199196727876   0.9977713348839312
22.890625  0.8190996271187619   0.8321927928162776   0.8693002424632451   0.8704627112128286   0.9978235872877205  0.8436922587492394   0.9977949161596853
22.96875   0.8205701151800758   0.8334485301822782   0.8704207460705501   0.8716790792907094   0.9978410437686891  0.8445486338167774   0.9978205815007594
23.046875  0.8220295081339197   0.8346941831775276   0.8715528601575585   0.8728582714246886   0.9978570925844896  0.8453920593332144   0.9978473581024997
23.125     0.8234777203455      0.835929893661       0.8726975794908      0.8739969810524      0.9978724861047     0.8462255677233      0.9978740938694
23.203125  0.8249146676827833   0.837155802622483    0.8738549310003666   0.8750929853967461   0.9978880076645273  0.8470521837888807   0.9978996292543882
23.28125   0.826340268042832    0.8383720500933876   0.8750239323337086   0.876145207189307    0.9979043164950282  0.847874898981029    0.9979229684335923
23.359375  0.8277544418733981   0.8395787750599694   0.876202591449443    0.8771537342474593   0.9979218190048362  0.8486966457911379   0.9979434222200141
23.4375    0.82915711268785     0.8407761153794749   0.8773879478124874   0.8781197961310687   0.9979405883665126  0.84952027252025     0.9979607001702062
23.515625  0.8305482075721022   0.8419642076989208   0.8785761541210746   0.8790456987442686   0.9979603457723681  0.8503485186862608   0.9979749380450134
23.59375   0.8319276576826087   0.8431431873776977   0.8797625958493579   0.8799347194123992   0.9979805060066914  0.8511839913216672   0.9979866576375562
23.671875  0.8332953987332942   0.844313188412749    0.8809420443250876   0.880790966549261    0.9980002789579241  0.8520291424069835   0.9979966671519314
23.75      0.8346513714692      0.8454743433677      0.8821088376725      0.8816192094393      0.9980188092442     0.8528862476721      0.9980059199203
23.828125  0.8359955221261904   0.8466267833043578   0.8832570827393237   0.8824246848947818   0.9980353297895632  0.8537573869818964   0.998015355736146
23.90625   0.8373278028735671   0.8477706377171382   0.8843808702115757   0.8832128884961039   0.998049303010507   0.854644426502814    0.9980257514065024
23.984375  0.8386481722374461   0.8489060344691189   0.8854744944927632   0.8839893587669337   0.998060525544672   0.8555490028302373   0.9980376049869373
24.0625    0.8399565955041438   0.85003309972905     0.8865326696492812   0.8847594629484813   0.9980691787535813  0.8564725092300687   0.998051071978825
24.140625  0.8412530450990156   0.8511519579095911   0.8875507328036635   0.885528192978126    0.9980758164318632  0.8574160841271075   0.9980659626582539
24.21875   0.8425375009409797   0.8522627316047672   0.8885248267878985   0.8862999798748923   0.9980812916966805  0.8583806019519875   0.9980817992316773
24.296875  0.8438099507680354   0.8533655415266331   0.8894520546622819   0.8870785339648136   0.998086635120984   0.8593666664291796   0.9980979214541253
24.375     0.8450703904334      0.8544605064407      0.8903305997859      0.8878667173226      0.9980929041412     0.8603746063716      0.9981136213877
24.453125  0.8463188241696963   0.8555477430988656   0.8911598064970513   0.888666453450826    0.9981010283147904  0.861404474021119    0.9981282834028548
24.53125   0.8475552648183922   0.8566273661700367   0.8919402180472852   0.8894786776554195   0.9981116754106313  0.8624560459515765   0.998141505064939
24.609375  0.8487797340236813   0.8576994881681875   0.8926735701326399   0.8903033298894112   0.9981251595917752  0.863528826533913    0.9981531781866068
24.6875    0.8499922623893812   0.8587642193775      0.8933627401982313   0.8911393900208687   0.9981414057519875  0.8646220539351499   0.9981635164240937
24.765625  0.8511928895960233   0.8598216677749126   0.8940116544646681   0.8919849537223529   0.9981599746711806  0.8657347086059853   0.9981730250675486
24.84375   0.852381664479082    0.8608719389507125   0.8946251563485937   0.892837345464125    0.9981801436310758  0.8668655241906094   0.9981824185639391
24.921875  0.85355864506567     0.861915136026738    0.8952088415303742   0.893693263531902    0.9982010281671725  0.8680130007679273   0.998192500140579
25.        0.8547238985703      0.8629513595742      0.8957688662472      0.8945489506637      0.9982217241989     0.8691754203155      0.9982040241856
25.078125  0.8558775013493106   0.8639807075318177   0.8963117364803518   0.8954003828307001   0.9982414468937304  0.8703508642642638   0.9982175647967954
25.15625   0.8570195388130836   0.8650032751233828   0.8968440864468445   0.8962434679467656   0.9982596437150805  0.8715372329888469   0.9982334126915031
25.234375  0.8581501052974736   0.8660191547792621   0.8973724552036461   0.8970742459056686   0.9982760639118282  0.8727322670626355   0.9982515177596814
25.3125    0.8592693038943999   0.8670284360598375   0.8979030702048437   0.8978890813185625   0.99829077438125    0.8739335700787688   0.9982714867773438
25.390625  0.8603772462418763   0.8680312055843645   0.898441646307068    0.8986848406757485   0.9983041210287504  0.8751386328256804   0.9982926365516765
25.46875   0.8614740522752953   0.8690275469647547   0.8989932080345016   0.8994590463460868   0.9983166439098109  0.8763448585834415   0.9983140935710398
25.546875  0.8625598499409106   0.8700175407468287   0.8995619418877222   0.9002100008550439   0.9983289620334873  0.8775495892937502   0.9983349236335187
25.625     0.8636347748718      0.871001264359       0.9001510842036      0.9009368761636      0.9983416484871     0.8787501323435      0.9983542701737
25.703125  0.8646989700297787   0.8719787920704198   0.9007628485510044   0.9016397641873807   0.9983551177572181  0.879943787689268    0.9983714788360847
25.78125   0.8657525853129734   0.8729501949575266   0.9013983949738695   0.9023196864672383   0.9983695446256946  0.8811278750475453   0.9983861884222249
25.859375  0.8667957771320465   0.8739155408816837   0.9020578416562943   0.90297856262936     0.9983848282920539  0.8822997608688047   0.9983983741537472
25.9375    0.8678287079561      0.8748748944768      0.9027403177915125   0.9036191390595      0.9984006074227687  0.8834568848158437   0.9984083372626062
26.015625  0.8688515458318464   0.8758283171471796   0.9034440547488415   0.9042448808853957   0.9984163230358876  0.8845967854726896   0.99841664383197
26.09375   0.869864463875786    0.8767758670760867   0.904166511058811    0.9048598319279297   0.9984313179782055  0.885717125014629    0.9984240240928384
26.171875  0.8708676397439172   0.8777175992440818   0.9049045253490527   0.9054684486425051   0.9984449556066298  0.886815712587023    0.9984312496571867
26.25      0.8718612550807      0.8786957212975      0.9056544902589      0.906075415163       0.9984567371612     0.8878905261498      0.9984390094162
26.328125  0.8728454949484445   0.8796817885327566   0.9064125395293509   0.9066854473696764   0.998466397677966   0.888939732565619    0.9984478045848688
26.40625   0.8738205472421656   0.8806582548654625   0.9071747399812445   0.9073030943580305   0.9984739640569953  0.8899617057284367   0.9984578797027812
26.484375  0.8747866020906567   0.8816249801458812   0.9079372799632016   0.9079325458026297   0.998479765364347   0.8909550425484144   0.9984692000176715
26.5625    0.8757438512472813   0.8825818385839      0.9086966460725187   0.9085774534539375   0.9984843935235376  0.891918576635       0.9984814776529063
26.640625  0.876692487473997    0.8835287197585321   0.9094497805299956   0.9092407744111112   0.998488620835485   0.8928513895416108   0.9984942407066075
26.71875   0.8776327039204063   0.8844655295278961   0.9101942124826625   0.9099246428941938   0.9984932878938952  0.8937528194582188   0.9985069323072469
26.796875  0.8785646935025841   0.8853921908395411   0.9109281577042642   0.9106302760184178   0.998499180245259   0.8946224672671944   0.9985190218402903
26.875     0.8794886482832      0.8863086444448      0.9116505825748      0.9113579176408      0.9985069138217     0.8954601998961      0.9985301088389
26.953125  0.8804047588573122   0.8872148495165516   0.9123612298292257   0.912106822715905    0.998516847512385   0.8962661509318376   0.9985400016216724
27.03125   0.8813132137465266   0.8881107841741696   0.9130606052805835   0.912875282861039    0.9985290365530414  0.897040718483675    0.9985487573556538
27.109375  0.8822141988052772   0.8889964459126709   0.9137499264524227   0.9136606920803829   0.9985432335442     0.8977845603053873   0.998556676992974
27.1875    0.8831078966414313   0.8898718519357063   0.9144310357889626   0.91445964984        0.9985589360091688  0.898498586218725    0.9985642563118062
27.265625  0.8839944860552313   0.8907370393900631   0.9151062826900609   0.9152680970907597   0.9985754718188471  0.8991839479025893   0.9985721017566795
27.34375   0.8848741414997617   0.8915920654961782   0.9157783800353766   0.9160814793938906   0.9985921077935438  0.8998420261385883   0.9985808256783453
27.421875  0.8857470325647584   0.8924370075708449   0.9164502420355898   0.9168949301142126   0.9986081633033859  0.9004744156360953   0.9985909389421446
27.5       0.8866133234873      0.8932719629333      0.9171248111018      0.9177034657727      0.9986231102891     0.9010829075813      0.9986027591784
27.578125  0.8874731726920423   0.8940970486918358   0.9178048819581146   0.918502185099147    0.9986366438001169  0.9016694700844172   0.9986163501822057
27.65625   0.8883267323626086   0.8949124014002774   0.9184929313676844   0.9192864631662554   0.9986487124260492  0.9022362267285023   0.9986315026252156
27.734375  0.8891741480473397   0.895718176581186    0.9191909616248075   0.9200521321883532   0.9986595049247203  0.9027854334418232   0.9986477592683523
27.8125    0.8900155583001126   0.8965145481093938   0.9199003653587687   0.920795641168775    0.9986693967700062  0.9033194539448313   0.9986644804181812
27.890625  0.8908510943592759   0.8973017074528643   0.9206218182491207   0.9215141875150914   0.9986788669778598  0.9038407340416504   0.9986809387371085
27.96875   0.8916808798646032   0.8980798627699328   0.9213552049949203   0.9222058149977125   0.9986884003444008  0.9043517750405422   0.9986964277658812
28.046875  0.8925050306150172   0.8988492378637433   0.922099582353724    0.9228694739561029   0.9986983923632741  0.9048551066065303   0.9987103664282901
28.125     0.8933236543669      0.8996100709985      0.9228531813751      0.9235050413498      0.9987090732728     0.9053532593555      0.9987223826709
28.203125  0.8941368506745426   0.9003626135839264   0.9236134491289226   0.9241133000912486   0.9987204641044997  0.9058487375044365   0.9987323630544919
28.28125   0.8949447107723242   0.9011071287388938   0.9243771283838781   0.924695878953364    0.9987323719032258  0.9063439918945219   0.9987404609084195
28.359375  0.8957473175004914   0.9018573710993464   0.9251403719224675   0.9252551561321818   0.998744424471426   0.9068413936983892   0.9987470625805511
28.4375    0.8965447452728812   0.9026419665181062   0.9258988865204812   0.9257941312301312   0.9987561382020188  0.9073432091141812   0.9987527182130626
28.515625  0.897337060087988    0.9034212095160548   0.926648100217576    0.9263162718483847   0.9987670069736306  0.9078515753346866   0.9987580491699355
28.59375   0.8981243195824585   0.9041950893519204   0.9273833453765281   0.926825342135832    0.9987765965863517  0.908368478067336    0.9987636478359391
28.671875  0.8989065731272955   0.9049635946169873   0.9281000492539149   0.9273252214509838   0.9987846283996258  0.908895730853119    0.9987699864146639
28.75      0.8996838619647      0.9057267133476      0.9287939234179      0.9278197217056      0.9987910377796     0.9094349564164      0.9987773494833
28.828125  0.9004562193870669   0.9064844331368034   0.9294611433712987   0.9283124119744596   0.9987959973155093  0.9099875702479817   0.9987858007637361
28.90625   0.9012236709558444   0.9072367412455171   0.930098510173907    0.9288064585525313   0.9987999007630555  0.9105547665958984   0.9987951886120929
28.984375  0.9019862347592158   0.9079836247101173   0.9307035866662279   0.9293044878550087   0.9988033102798954  0.9111375070135743   0.9988051881374048
29.0625    0.9027439217082875   0.908725070447775    0.9312748020908624   0.9298084784072312   0.9988068756340188  0.9117365115814312   0.9988153717392563
29.140625  0.9034967358697668   0.9094610653574895   0.9318115203611471   0.9303196867310792   0.9988112386582083  0.9123522528896056   0.9988252952371226
29.21875   0.9042446748332094   0.9101915964170352   0.9323140689223001   0.9308386102759265   0.9988169385584281  0.9129849528406054   0.9988345843835672
29.296875  0.9049877301125073   0.9109166507747851   0.9327837270002359   0.931364988713881    0.9988243333925277  0.9136345822973713   0.9988430067838489
29.375     0.9057258875777      0.9116362158364      0.9332226739009      0.9318978430709      0.9988335502119     0.9143008635744      0.9988505169911
29.453125  0.9064591279166536   0.9123502793462435   0.9336338998930699   0.9324355503227478   0.9988444714987349  0.9149832757391282   0.9988572673220953
29.53125   0.9071874271228508   0.9130588294631188   0.9340210839229953   0.9329759493805743   0.9988567594816687  0.9156810626571757   0.9988635828929031
29.609375  0.9079107570076307   0.9137618548307004   0.9343884439148314   0.933516472907563    0.9988699136991527  0.9163932436930027   0.9988699054973921
29.6875    0.908629085733375    0.9144593446436999   0.9347405666537      0.9340542981772625   0.9988833518212562  0.9171186269378813   0.9988767162125438
29.765625  0.9093423783644667   0.9151512887082138   0.9350822251241907   0.9345865093243544   0.9988965001421032  0.9178558248172732   0.9988844501329367
29.84375   0.9100505974337781   0.915837677498918    0.9354181916868562   0.9351102628352      0.9989088788560281  0.9186032718965891   0.998893417855896
29.921875  0.9107537035197426   0.916518502212841    0.9357530555800808   0.9356229480370152   0.9989201684331629  0.9193592446794607   0.9989037470754785
30.        0.9114516558314      0.9171937548195      0.9360910529303      0.9361223346506      0.9989302468526     0.9201218831666      0.9989153541556
\end{filecontents}

\begin{filecontents}{high_density_stats.dat}
time        lbound               ubound               p1                  p2                   p3                  p4                  p5                  p6                  p7                  p8                  p9                  p10                 p11                 p12                 p13                 p14                  p15                 p16                  p17                  p18                 p19                 p20                  p21                  p22                 p23                 p24                  p25                 p26                 p27                 p28                 p29                 p30                 p31                  p32                  p33                 p34                 p35                 p36                 p37                 p38                 p39                 p40                 p41
0.          1.                   1.                   1.                  1.                   1.                  1.                  1.                  1.                  1.                  1.                  1.                  1.                  1.                  1.                  1.                  1.                   1.                  1.                   1.                   1.                  1.                  1.                   1.                   1.                  1.                  1.                   1.                  1.                  1.                  1.                  1.                  1.                  1.                   1.                   1.                  1.                  1.                  1.                  1.                  1.                  1.                  1.                  1.
0.1171875   0.9999999999997936   0.9999999999998568   0.9999999999998568  0.9999999999998568   1.                  0.9999999999998568  0.9999999999998568  1.                  0.9999999999998568  0.9999999999998568  0.9999999999998568  0.9999999999999     0.9999999999998568  0.9999999999997936  0.9999999999998568  0.9999999999997936   0.9999999999998568  0.9999999999997936   0.9999999999997936   0.9999999999998568  1.                  0.9999999999997936   0.9999999999997936   0.9999999999997936  0.9999999999998568  0.9999999999998568   0.9999999999999     0.9999999999997936  0.9999999999998568  0.9999999999998568  0.9999999999998568  0.9999999999998568  0.9999999999997936   0.9999999999997936   0.9999999999999     0.9999999999998568  0.9999999999998568  1.                  0.9999999999998568  0.9999999999998568  0.9999999999998568  0.9999999999998568  0.9999999999998568
0.234375    0.9999999999989145   0.9999999999989125   0.9999999999989125  0.9999999999989145   0.9999999999999     0.9999999999990014  0.9999999999989091  0.9999999999999     0.9999999999992013  0.9999999999989125  0.9999999999989091  0.9999999999995     0.9999999999990111  0.9999999999989145  0.9999999999990014  0.9999999999989013   0.9999999999989091  0.9999999999986126   0.9999999999989145   0.9999999999989091  0.9999999999999     0.9999999999989145   0.9999999999989145   0.9999999999989145  0.9999999999990014  0.9999999999989145   0.999999999999498   0.9999999999989145  0.9999999999989125  0.9999999999989125  0.9999999999989091  0.9999999999989125  0.9999999999989145   0.9999999999989145   0.9999999999995     0.9999999999990111  0.9999999999989125  0.9999999999999     0.9999999999990146  0.9999999999989091  0.9999999999989091  0.9999999999990014  0.9999999999990111
0.3515625   0.9999999999951122   0.9999999999953123   0.9999999999954076  0.9999999999952361   0.9999999999999     0.9999999999957422  0.9999999999956362  0.9999999999999     0.9999999999976688  0.9999999999954016  0.9999999999956423  0.9999999999990941  0.9999999999963375  0.9999999999953362  0.9999999999959421  0.999999999994911    0.9999999999955123  0.999999999993375    0.9999999999951062   0.9999999999955361  0.9999999999999     0.9999999999951122   0.9999999999953123   0.9999999999953123  0.9999999999959375  0.9999999999952063   0.9999999999989702  0.9999999999952361  0.9999999999953362  0.9999999999954362  0.9999999999955123  0.9999999999954016  0.9999999999953123   0.9999999999952123   0.9999999999990941  0.9999999999963436  0.9999999999954362  0.9999999999999     0.9999999999962375  0.9999999999956423  0.9999999999954362  0.9999999999958076  0.9999999999963375
0.46875     0.9999999999820305   0.9999999999831016   0.9999999999838274  0.9999999999830289   0.9999999999997     0.9999999999857749  0.9999999999850804  0.9999999999995     0.9999999999939734  0.9999999999839055  0.999999999985175   0.999999999998025   0.999999999988725   0.9999999999835234  0.99999999998675    0.9999999999810594   0.9999999999843039  0.9999999999757055   0.9999999999821305   0.9999999999848039  0.9999999999997     0.9999999999820305   0.9999999999832015   0.9999999999833055  0.99999999998675    0.9999999999828251   0.9999999999978288  0.9999999999830289  0.9999999999835234  0.9999999999842001  0.9999999999843039  0.9999999999839055  0.9999999999831289   0.9999999999827289   0.999999999998025   0.9999999999886304  0.9999999999841094  0.9999999999995     0.9999999999884485  0.9999999999851805  0.9999999999840999  0.9999999999859789  0.999999999988725
0.5859375   0.9999999999410497   0.9999999999454015   0.999999999948639   0.9999999999451014   0.9999999999992828  0.9999999999573296  0.9999999999537951  0.9999999999987984  0.9999999999824358  0.9999999999488435  0.9999999999548093  0.9999999999928422  0.9999999999687421  0.9999999999475373  0.9999999999610842  0.9999999999367076   0.9999999999507623  0.999999999921181    0.9999999999412467   0.999999999953078   0.9999999999992828  0.9999999999409623   0.9999999999459185   0.9999999999466076  0.9999999999611672  0.9999999999445044   0.9999999999939437  0.9999999999458059  0.9999999999474247  0.9999999999508592  0.9999999999507748  0.9999999999487388  0.9999999999459059   0.9999999999440842   0.9999999999929297  0.9999999999681422  0.9999999999496576  0.9999999999987984  0.9999999999677376  0.9999999999547967  0.9999999999498607  0.9999999999579342  0.9999999999687593
0.703125    0.9999999998187915   0.9999999998350486   0.9999999998467656  0.999999999833528    0.9999999999981375  0.9999999998788099  0.9999999998652852  0.9999999999969794  0.9999999999459492  0.999999999847429   0.9999999998702727  0.9999999999790429  0.9999999999162296  0.9999999998431977  0.999999999891743   0.9999999998031979   0.9999999998548726  0.9999999997599274   0.9999999998189915   0.9999999998636162  0.9999999999981375  0.9999999998187334   0.9999999998368476   0.9999999998396727  0.9999999998918431  0.9999999998316655   0.9999999999812073  0.9999999998374102  0.9999999998423351  0.9999999998562351  0.9999999998550727  0.9999999998471291  0.9999999998369531   0.9999999998304602   0.9999999999791063  0.9999999999144306  0.9999999998507423  0.9999999999969794  0.9999999999132682  0.9999999998699727  0.9999999998512423  0.9999999998806735  0.9999999999163296
0.8203125   0.9999999994694116   0.9999999995237817   0.9999999995625984  0.9999999995176586   0.9999999999950875  0.9999999996680219  0.9999999996229884  0.9999999999910794  0.9999999998445775  0.9999999995648067  0.9999999996411436  0.9999999999471717  0.9999999997778243  0.9999999995517682  0.9999999997074398  0.9999999994178486   0.9999999995904904  0.9999999992967169   0.9999999994688054   0.9999999996198787  0.9999999999950875  0.9999999994691062   0.9999999995296005   0.9999999995405325  0.9999999997077333  0.9999999995127462   0.9999999999463574  0.9999999995344201  0.999999999548361   0.9999999995969134  0.9999999995912923  0.9999999995638057  0.9999999995306014   0.9999999995094345   0.9999999999471698  0.9999999997729152  0.9999999995764476  0.9999999999910794  0.9999999997697046  0.9999999996402382  0.9999999995781484  0.9999999996735398  0.9999999997781296
0.9375      0.9999999985051937   0.99999999867485     0.9999999987948187  0.9999999986525625   0.9999999999868563  0.9999999991135313  0.9999999989782125  0.9999999999779062  0.9999999995821938  0.9999999988014563  0.99999999903575    0.9999999998634812  0.999999999413975   0.9999999987630688  0.9999999992253125  0.9999999983457876   0.9999999988825374  0.9999999980125562   0.9999999985000875   0.9999999989724124  0.9999999999868563  0.9999999985034937   0.9999999986922375   0.9999999987290312  0.9999999992260625  0.9999999986407625   0.999999999860775   0.9999999987135313  0.999999998751625   0.9999999989058376  0.9999999988846437  0.9999999987985687  0.9999999986970876   0.9999999986320125   0.9999999998635812  0.999999999401925   0.9999999988385063  0.9999999999779124  0.999999999393125   0.9999999990330499  0.9999999988434563  0.9999999991291875  0.9999999994148749
1.0546875   0.9999999959218612   0.9999999964272194   0.9999999967781344  0.9999999963531537   0.9999999999657769  0.9999999976852636  0.9999999973064749  0.9999999999435324  0.99999999891613    0.9999999967976836  0.9999999974723224  0.9999999996391219  0.9999999984727703  0.9999999966894724  0.9999999979863856  0.9999999954591471   0.9999999970363156  0.9999999945760024   0.9999999959037174   0.9999999972955009  0.9999999999657769  0.9999999959189655   0.9999999964772824   0.9999999965912452  0.9999999979884794  0.9999999963273973   0.9999999996518477  0.9999999965524419  0.9999999966534797  0.9999999971097927  0.9999999970421958  0.9999999967892033  0.9999999964951262   0.9999999963051546   0.9999999996395165  0.9999999984442328  0.9999999969073734  0.9999999999435324  0.9999999984207928  0.9999999974649431  0.999999996921433   0.9999999977277536  0.999999998475164
1.171875    0.9999999892052024   0.9999999906430173   0.9999999916252199  0.9999999904147268   0.9999999999147055  0.9999999940902206  0.9999999930833207  0.9999999998568609  0.9999999972371216  0.9999999916791708  0.9999999935305341  0.9999999990902638  0.999999996100809   0.9999999913854104  0.9999999948734827  0.9999999879167102   0.9999999923460656  0.9999999856577506   0.9999999891467276   0.999999993058409   0.9999999999146753  0.9999999892010961   0.9999999907832758   0.9999999911136879  0.9999999948790059  0.9999999903629769   0.9999999991360898  0.9999999910162737  0.9999999912794038  0.9999999925584072  0.9999999923618862  0.9999999916554313  0.9999999908399402   0.999999990308126    0.9999999990912585  0.999999996033295   0.9999999919871191  0.9999999998567491  0.9999999959721692  0.9999999935106697  0.9999999920259556  0.9999999942025168  0.999999996107027
1.2890625   0.9999999722286026   0.999999976163175    0.9999999788094678  0.999999975498146    0.9999999997946636  0.9999999852658984  0.9999999826920364  0.9999999996582568  0.9999999930992826  0.9999999789516839  0.999999983839763   0.9999999978103254  0.9999999903030111  0.9999999781819411  0.9999999872489915  0.9999999687644681   0.9999999807394896  0.9999999631367994   0.999999972053227    0.9999999826223688  0.9999999997945606  0.999999972224301    0.9999999765464951   0.9999999774547058  0.9999999872630932  0.9999999754041443   0.9999999978863726  0.9999999772110941  0.9999999778829913  0.9999999813172585  0.9999999807809804  0.9999999788878962  0.9999999767102581   0.9999999752736408   0.9999999978127068  0.9999999901410513  0.9999999797809194  0.9999999996581568  0.9999999899859118  0.9999999837882472  0.9999999798840166  0.9999999855545446  0.9999999903186111
1.40625     0.9999999304862852   0.9999999408889609   0.9999999477846406  0.9999999390430898   0.9999999995186617  0.9999999641768766  0.9999999577997977  0.9999999991918211  0.9999999832421227  0.9999999481441953  0.9999999606362532  0.9999999948333266  0.9999999765707868  0.9999999461897711  0.9999999690831735  0.9999999214588993   0.9999999527712039  0.9999999077580485   0.9999999299871734   0.999999957585414   0.9999999995183797  0.9999999304856562   0.9999999419133462   0.9999999443027664  0.999999969118518   0.9999999388959648   0.9999999949778625  0.9999999436982718  0.9999999453784453  0.9999999542741633  0.999999952876857   0.999999947978225   0.9999999423548555   0.9999999385956718   0.9999999948389672  0.9999999761808586  0.9999999502980336  0.9999999991915938  0.9999999757983594  0.9999999605060734  0.999999950563871   0.9999999648990859  0.9999999766092289
1.5234375   0.9999998305975164   0.9999998572532084   0.9999998746836515  0.9999998523366895   0.9999999988982577  0.9999999151540896  0.9999998997736257  0.9999999981399397  0.999999960508059   0.9999998755601663  0.9999999065705714  0.9999999880477705  0.9999999450325823  0.9999998707401014  0.999999927023196   0.9999998077411307   0.9999998871683242  0.9999997752263763   0.9999998292359135   0.9999998991164685  0.9999999988976876  0.9999998306104291   0.9999998599312996   0.999999865983283   0.9999999271094792  0.9999998521716573   0.9999999884628515  0.9999998645075586  0.9999998686128448  0.9999998909415599  0.9999998874300935  0.9999998751406994  0.9999998610571579   0.9999998515062656   0.9999999880610381  0.9999999441058159  0.9999998809784034  0.9999999981394396  0.9999999431891478  0.9999999062501704  0.9999998816446067  0.999999916913129   0.999999945124539
1.640625    0.9999995978994042   0.9999996642373783   0.9999997070565145  0.9999996516078157   0.9999999975396231  0.9999998043728868  0.9999997681923452  0.9999999958667486  0.9999999094701297  0.9999997091257001  0.9999997840452017  0.9999999731865135  0.9999998747240477  0.9999996975632952  0.999999832425193   0.9999995415925768   0.9999997374262253  0.9999994666068032   0.9999995943193257   0.9999997662642726  0.9999999975383198  0.9999995979466626   0.9999996710792943   0.9999996858979486  0.9999998326302209  0.9999996516361925   0.9999999742761478  0.9999996823774148  0.9999996921585195  0.9999997466183036  0.9999997380570407  0.9999997080947489  0.9999996738176238   0.9999996502218775   0.9999999732166398  0.9999998725777345  0.99999972236182    0.9999999958657354  0.9999998704435037  0.9999997832775945  0.9999997239863156  0.999999808543345   0.9999998749373924
1.7578125   0.9999990700776531   0.9999992306576517   0.9999993330113914  0.9999991992580126   0.9999999946506658  0.9999995610733352  0.9999994780251655  0.9999999910097956  0.9999997977115798  0.9999993377534572  0.9999995140718297  0.9999999416745895  0.9999997224552297  0.9999993107472461  0.999999625758422   0.9999989349949877   0.9999994048989792  0.9999987671019213   0.9999990609752536   0.9999994726566988  0.9999999946479154  0.9999990701960535   0.9999992477268532   0.9999992828968921  0.9999996262319916  0.9999992002372322   0.999999944119106   0.9999992747308859  0.9999992974218304  0.9999994266901184  0.9999994063793642  0.9999993352876702  0.9999992541088609   0.9999991973696409   0.9999999417407639  0.9999997176495266  0.9999993692026327  0.9999999910077453  0.9999997128205358  0.9999995122816817  0.9999993730592668  0.9999995706950868  0.9999997229366933
1.875       0.9999979041832      0.9999982826326      0.9999985208919     0.9999982068845      0.9999999886913     0.9999990418323     0.999998855888      0.9999999810006     0.9999995593288     0.9999985314586     0.9999989358175     0.9999998764006     0.9999994019674     0.9999984700032     0.9999991871667     0.9999975884242      0.9999986866187     0.9999972233256      0.999997881756       0.999998841663      0.9999999886853     0.9999979044221      0.9999983241925      0.9999984052419     0.9999991882312     0.9999982110949      0.9999998816928     0.999998386886      0.9999984380916     0.9999987369551     0.9999986900014     0.9999985257159     0.9999983384824      0.9999982056019      0.9999998765424     0.9999993915842     0.9999986042078     0.9999999809963     0.9999993809628     0.9999989317534     0.9999986131246     0.9999990634291     0.9999994030233
1.9921875   0.9999953960212813   0.9999962650066484   0.9999968053816304  0.9999960873702014   0.9999999767579234  0.999997965070478   0.9999975592332447  0.9999999609990559  0.9999990647381644  0.9999968282955772  0.9999977319946316  0.9999997447926635  0.9999987463851479  0.9999966919932117  0.9999982829452537  0.9999946766219696   0.9999971776445895  0.9999939054049052   0.9999953423992914   0.9999975231861513  0.9999999767454426  0.9999953964040312   0.9999963637215501   0.9999965452788664  0.9999982852731271  0.9999961010594266   0.9999997562711119  0.9999965053840807  0.9999966177194294  0.9999972909876784  0.9999971851713245  0.999996815270591   0.9999963944892838   0.9999960912811332   0.9999997450886142  0.99999872471519    0.9999969922265194  0.9999999609907374  0.9999987020002713  0.9999977230141505  0.9999970123069215  0.999998012225157   0.9999987486373081
2.109375    0.9999901404976568   0.9999920854960485   0.9999932799823453  0.9999916800023712   0.9999999535399943  0.9999957955318409  0.9999949326694733  0.9999999219502752  0.999998066605293   0.9999933283661211  0.9999952963911733  0.9999994876718299  0.9999974432732178  0.9999930336507292  0.999996471955231   0.9999885426576858   0.9999940948941904  0.9999869611365821   0.9999900159680458   0.9999948449089823  0.9999999535146907  0.9999901408746621   0.9999923141827066   0.999992709777787   0.9999964769078267  0.9999917192280249   0.9999995118547873  0.9999926260905648  0.9999928655713841  0.9999943436693728  0.9999941112019809  0.9999932995931713  0.999992377855617    0.9999917035789573   0.9999994882721025  0.9999973994941472  0.9999936883097105  0.9999999219344153  0.9999973522559256  0.9999952770780008  0.9999937323578005  0.9999958956730683  0.999997447944545
2.2265625   0.999979413548316    0.9999836586125398   0.9999862325654275  0.9999827567372195   0.9999999096298435  0.9999915485754783  0.9999897625023468  0.9999998481607845  0.9999961057718838  0.9999863320867786  0.9999905078142447  0.9999990001285496  0.9999949269729865  0.9999857107396205  0.9999929485840905  0.9999759532887487   0.9999879711815512  0.9999728077136669   0.9999791324439299   0.999989556491095   0.9999999095800107  0.999979413158064    0.9999841752821839   0.9999850140052402  0.9999929588353127  0.9999828607815392   0.9999990492911294  0.9999848448707438  0.9999853408237797  0.9999885033187762  0.9999880055847912  0.9999862701749092  0.9999843016938947   0.9999828400878733   0.9999990013131514  0.9999948411913819  0.9999871022909236  0.9999998481314644  0.9999947456655042  0.999990467394788   0.9999871964113026  0.9999917554076312  0.9999949363950152
2.34375     0.9999580845566758   0.9999671215897891   0.9999725293499329  0.9999651657928157   0.9999998289227859  0.9999834724465492  0.999979874698868   0.9999997126752821  0.9999923541498031  0.9999727288155179  0.9999813601110827  0.9999981009276351  0.999990207931086   0.9999714513454563  0.9999862904734063  0.9999507774600711   0.9999761445271483  0.9999447146259852   0.9999574674472789   0.9999794072222562  0.9999998288269359  0.9999580808783141   0.9999682600156695   0.9999699905869852  0.9999863111175523  0.9999654268314476   0.9999981983817391  0.9999696619754165  0.9999706594194625  0.999977253533282   0.9999762151904055  0.9999725990491688  0.9999684999129664   0.9999654112484305   0.9999981032032601  0.9999900447244695  0.9999743350103367  0.9999997126235094  0.9999898568950398  0.9999812777964328  0.9999745309108624  0.9999838878845476  0.9999902264100579
2.4609375   0.9999167673109417   0.9999355361561749   0.9999466145883998  0.9999313986488871   0.9999996848208086  0.9999685554659842  0.9999615053478478  0.9999994706388828  0.9999853636445498  0.9999470041949529  0.9999643830026096  0.9999964894749039  0.9999816147853979  0.9999444427261757  0.9999740718574326  0.9999017191757963   0.9999539407682244  0.999890404352326    0.9999154492216992   0.9999604781241369  0.9999996846416225  0.9999167533119144   0.9999379826770831   0.9999414578582762  0.9999741123055057  0.9999320250608644   0.9999966772984359  0.9999408459647584  0.9999427933537838  0.999956192095844   0.9999540820676684  0.9999467392561999  0.9999384151772176   0.9999320516759523   0.9999964937292841  0.9999813132081169  0.9999502687148135  0.9999994705507139  0.9999809541392811  0.9999642198936426  0.999950665878984   0.9999693668741156  0.999981650022668
2.578125    0.999838785791502    0.9998768218828724   0.9998989538287499  0.9998682799533327   0.9999994349834872  0.9999417987519404  0.9999283618252806  0.9999990514958282  0.999972680182881   0.9998996955628965  0.9999337781575379  0.9999936874229455  0.9999664244995017  0.9998946862484358  0.9999523006450611  0.9998085613284242   0.9999134226848208  0.999788136293784    0.9998360456464757   0.9999261735687203  0.9999994346568793  0.9998387437526318   0.9998819504131017   0.9998887417471412  0.9999523777477983  0.999869726798584    0.9999940369667528  0.999887654807534   0.9998913439518079  0.9999178737133682  0.9999136977267903  0.9998991686885149  0.9998826833258078   0.9998699003842635   0.9999936951600165  0.999965883549288   0.9999061620725868  0.9999990513514093  0.9999652162520609  0.9999334637050448  0.9999069463457892  0.9999433396460788  0.9999664898242651
2.6953125   0.9996953748387862   0.9997706003320342   0.9998137195587159  0.9997533857087629   0.9999990144799261  0.9998952009626305  0.9998702965248113  0.9999983463594078  0.9999502709306417  0.999815096159917   0.999880197626344   0.9999889583733849  0.9999403624173612  0.999805540857988   0.9999146450381383  0.9996361634791285   0.9998415671134535  0.9996005545618496   0.9996898288004683   0.9998657742515387  0.9999990139006661  0.9996952632827707   0.9997810882507238   0.9997940023526131  0.9999147880254166  0.999756614588563    0.9999895882527513  0.9997921739030146  0.999798950850279   0.9998501363774379  0.9998420882121201  0.9998140756390473  0.9997822329061012   0.9997571955694237   0.9999889720645098  0.9999394214101481  0.999827579970362   0.9999983461344569  0.9999382159687792  0.9998796078787189  0.9998290883150066  0.9998980457647917  0.9999404801444031
2.8125      0.9994383758795438   0.9995835865396437   0.9996655191217814  0.9995497135789437   0.9999983274787313  0.9998164294617875  0.9997715460659188  0.9999971942521187  0.999911707656775   0.9996680103751563  0.9997891192908125  0.9999812075086313  0.9998969721694625  0.9996502311599625  0.9998514407657938  0.9993252273258812   0.999717749830325   0.999265399610025    0.9994274442001375   0.9997624763278438  0.9999983264787375  0.9994381033374062   0.999604512589925    0.9996284029117938  0.9998516987293687  0.9995566928706313   0.9999823115301688  0.9996255263463812  0.9996375882741     0.99973381373275    0.9997187106564938  0.9996660852610625  0.9996060922809312   0.9995582766321188   0.999981231086925   0.9998953865793437  0.9996914983704438  0.999997193919625   0.999893270086025   0.9997880434206     0.9996943234712999  0.9998215346463437  0.9998971784088375
2.9296875   0.9989895668770573   0.999263180921413    0.9994150240538614  0.9991980951985675   0.9999972381170028  0.9996872130081949  0.9996085616493958  0.9999953690055129  0.9998470598072068  0.9994194219345688  0.99963884007525    0.9999688771316985  0.9998268889612337  0.9993871513309646  0.999748520212003   0.9987786511739815   0.9995104886674567  0.9986820654430052   0.9989685784684923   0.999590907212757   0.9999972364368724  0.9989889418847828   0.9993039212148692   0.9993469070645568  0.9997489729229911  0.9992127307026474   0.9999707537210921  0.9993427738461353  0.9993635513892664  0.999539808425932   0.9995122125283743  0.9994158855086798  0.9993056310686126   0.99921659522561     0.999968916644308   0.9998243042494384  0.9994624904461147  0.9999953685451675  0.9998206927468762  0.9996369312124883  0.9994676429797881  0.9996961167338717  0.999827240144395
3.046875    0.9982257553480106   0.9987290388758661   0.999003510619617   0.9986069060219711   0.9999955619345445  0.9994815926172937  0.9993475813976739  0.9999925621794415  0.9997414828617619  0.9990110881287443  0.9993982240355858  0.9999498520046528  0.9997171007604044  0.9989539467313605  0.9995859841584968  0.997842234003628    0.9991735422127082  0.9976930280044135   0.9981864970305189   0.9993142328012403  0.9999955591874921  0.9982243959726954   0.9988064344824279   0.9988816411699476  0.9995867569128449  0.9986367133955113   0.9999529257589255  0.9988764996345771  0.998911087734907   0.9992256449674867  0.9991765510910907  0.9990047628150011  0.9988071009235686   0.9986454628936752   0.9999499164254922  0.9997130295925738  0.9990880509609281  0.9999925615986861  0.9997070412521573  0.9993949306984001  0.99909720053187    0.9994966808253043  0.9997176819573436
3.1640625   0.9969589680511352   0.9978626814096443   0.9983465923427346  0.997638859238807    0.9999930601227957  0.9991643090001988  0.9989422710631242  0.9999883756716597  0.9995735435743446  0.9983593455266132  0.9990245118875685  0.9999213849808936  0.9995503471008977  0.9982606344863719  0.9993371424058041  0.9962787074682342   0.9986417009494414  0.9960595122685703   0.9968874218721857   0.9988811613327869  0.9999930557522035  0.9969561468803333   0.9980061473254366   0.9981340527630165  0.9993384252007016  0.9976978640571388   0.9999262177367811  0.9981293882639115  0.99818492091796    0.9987318482691653  0.9986468088751169  0.9983483320958985  0.9980026278117907   0.997716569359303    0.9999214871624779  0.9995441588109942  0.9984933748234651  0.9999883750356431  0.9995345115645421  0.99901898788811    0.9985091907993363  0.9991891456705871  0.9995512818221374
3.28125     0.9949116281999969   0.9964957049061766   0.9973278270990571  0.996095138649382    0.9999894395944704  0.9986897711172922  0.9983319597848812  0.9999823246431546  0.9993134187748282  0.9973488178888078  0.9984617067639079  0.9998800813681024  0.9993048741373547  0.9971824484668703  0.9989679661300274  0.9937347765238227   0.9978269365451055  0.9934315619280212   0.9947845848714117   0.9982233960002312  0.9999894328285969  0.9949060219605047   0.9967551607793484   0.996966560264389   0.9989700366440344  0.9962087200595109   0.9998873454284899  0.9969666141512477  0.9970523694489266  0.9979787997386703  0.9978353677626141  0.9973301552450382  0.9967402758526289   0.9962468530292649   0.9998802390322062  0.9992958084185133  0.9975762775210313  0.9999823241351656  0.9992807116865577  0.9984527010989523  0.9976028852756843  0.9987294716756141  0.999306334703382
3.3984375   0.9916876731341774   0.994397863974997    0.9957933827354237  0.9936978885061813   0.9999843622555119  0.9980022036679007  0.9974411620226452  0.9999738470585451  0.9989212734430155  0.9958272266429782  0.9976402578291972  0.9998220045352308  0.9989548180559737  0.9955536378443793  0.9984374921067142  0.9897022851630275   0.9966159532335053  0.9893140753779343   0.9914678910191904   0.9972543532025724  0.9999843520682712  0.9916769839234252   0.9948555287248741   0.9951950084006493  0.9984407405280841  0.9939105487428269   0.9998323328948547  0.995209498197517   0.9953363653919695  0.99686506693276    0.9966294795920436  0.9957964593506503  0.9948144432519499   0.9939851206720448   0.9998222411048348  0.9989420358126144  0.9962033954982036  0.9999738470659626  0.9989190949516105  0.9976259923874607  0.9962469481806596  0.9980638020725625  0.9989570347531492
3.515625    0.9867422691368983   0.991267185006302    0.9935494533426572  0.9900731537373909   0.999977466639145   0.9970376536971716  0.996181108361869   0.9999623464220645  0.9983463989346981  0.9936030296702713  0.9964789009703644  0.9997429322113449  0.9984715013961353  0.9931640232378088  0.9976996533842286  0.9834770450824372   0.9948705248701194  0.9830324904538433   0.9863719134721004   0.9958701741511576  0.999977451726933   0.9867226926250849   0.9920542674147613   0.9925839373224259  0.9977046049993679  0.9904604343317109   0.9997565256887151  0.9926325941741752  0.9928113415577209  0.9952685033575464  0.9948916073160559  0.9935537020298822  0.9919585877738093   0.9906008533166838   0.9997432772303414  0.9984541817161168  0.9942092447478844  0.9999623476474191  0.9984203448436287  0.9964569531955321  0.9942785819755172  0.9971303761939512  0.9984747675471757
3.6328125   0.9793551726653723   0.9867251622452372   0.9903644645490814  0.9847376870153154   0.9999684040476777  0.9957284776930948  0.9944537874211637  0.9999472534538043  0.9975280644425935  0.990447992022643   0.9948894958777965  0.9996387585288299  0.9978258718000735  0.989760550324414   0.9967069344191677  0.9741224157427613   0.9924321639945696  0.9737046355989338   0.9787474854309096   0.9939541128893281  0.9999683828399031  0.9793207129113837   0.9880440991973718   0.9888472671933485  0.9967142630408786  0.9854235252792446   0.9996546141770123  0.9889669324619403  0.9892045077188746  0.9930515480801315  0.992464070152407   0.990371122275435   0.9878429396071501   0.9856787361544891   0.9996392473418835  0.9978033518664741  0.9913999938019099  0.9999472570747607  0.9977549403920022  0.9948567154001459  0.9915073146934886  0.9958637935311396  0.9978305406775941
3.75        0.9686157168414      0.9803207129107      0.9859780623995     0.9770944218616      0.9999568893863     0.9940106702861     0.9921584305343     0.9999281016588     0.9963994407546     0.9861067988392     0.992785437175      0.9995060641376     0.9969921723863     0.9850561373361     0.9954160255135     0.9604492523963      0.9891324905899     0.9602285554513      0.967645310983       0.9913851758974     0.9999568601061     0.9685573851119      0.9824725920298      0.9836572288828     0.9954265475429     0.9782749094466      0.9995206209726     0.9839123012396     0.9842052143807     0.9900719772848     0.9891793456771     0.9859904342717     0.9820801728252      0.978723291942       0.9995067363779     0.9969641046243     0.9875637266249     0.9999281094967     0.9968969784057     0.992737938326      0.9877251474041     0.9942019501082     0.996998640818
3.8671875   0.953430358796857    0.9715465391941182   0.9801182169240675  0.9664425016601357   0.9999427638333824  0.9918339134055466  0.9891994416425411  0.9999046440765212  0.9948962756247054  0.9803152246761887  0.990093670240886   0.9993428660845155  0.9959527077380693  0.9787475320398572  0.9937952792269898  0.9410286087841425   0.9848100098355753  0.9412991802374803   0.9519232613918004   0.9880513429225887  0.9999427246261385  0.953335322984725    0.9749614317968011   0.9766639227204283  0.9938099150548287  0.9684159356304556   0.9993477831564398  0.9771608800466959  0.9774846054733345  0.9861997538577691  0.9848767248028812  0.9801442317882825  0.9742416484422163   0.9691781076603343   0.999343762517085   0.9959191770003915  0.9824883438062154  0.9999046587489844  0.9958290881427562  0.9900269533782937  0.9827241484265781  0.9920955026925683  0.9959613807172895
3.984375    0.9325656771886136   0.9598704653227035   0.9725263665910476  0.952008023553228    0.9999260625391602  0.9891735682788154  0.9854936444367236  0.9998769579828548  0.992971998402546   0.97282703643712    0.9867695474533281  0.9991494372070688  0.9947032843706403  0.9705425075003488  0.9918332976384892  0.9142569606049175   0.9793328364836456  0.9154683715773467   0.9302912546539703   0.9838668245792109  0.999926011678108   0.9324163979327056   0.9651354900673388   0.9675265089976037  0.9918529854974707  0.9552089817774311   0.999128275602291   0.9684330461164887  0.9687263571029386  0.9813395552000209  0.9794248502410655  0.9725833264312426  0.9638855462888252   0.956462988771183    0.9991505947417375  0.9946647957511684  0.9759869878557085  0.9998769830336138  0.9945479752318416  0.9866788067715208  0.976321330131036   0.9895190941564042  0.9947145156821626
4.1015625   0.9047390673898911   0.9447819529926211   0.9629885042166887  0.9330004490768011   0.9999070763811051  0.9860431268747463  0.9809737492483519  0.9998455333462406  0.9906201467878619  0.9634478102362438  0.9828128573308136  0.9989290138568541  0.9932586071547854  0.9601945981395382  0.9895475010470488  0.8784930598832209   0.9726253644477667  0.8812616195327224   0.9014070988400905   0.9787917392730429  0.9999070125436715  0.9045122982200235   0.9526582985539715   0.9559546959388487  0.9895730503786124  0.9380322284022026   0.9988528098570587  0.9575237832603859  0.9576669853049838  0.9754571412462522  0.9727481421893748  0.9631112186150144  0.9505959332338877   0.9400295738263462   0.9989304589231293  0.9932158391868755  0.9679290907323304  0.9998455732196195  0.9930698294536985  0.9826935257332606  0.9683888910363718  0.9864829237653474  0.9932726139453487
4.21875     0.8687642305518194   0.9258504956144312   0.9513679128214367  0.9086963173235171   0.9998863937341539  0.9825050639607601  0.9755846444526945  0.9998113577631539  0.9879027884408297  0.9520715769945539  0.9782825721852538  0.9986882828937812  0.9916566652861218  0.9475411157434984  0.986991171788243   0.8322775903264311   0.9646953245885547  0.8373589126224765   0.8640312911145859   0.9728515697201328  0.9998863163207211  0.8684294463427351   0.9372673081069687   0.9417562379960835  0.9870230514381437  0.9163518465202539   0.9985103089736196  0.9443552205495038  0.9441413521352819  0.9686062143655289  0.9648535768061812  0.9516215844642016  0.9340289233211366   0.9194315744137382   0.9986900237898031  0.9916099077983641  0.9582731973326586  0.999811417584386   0.991434594062561   0.9781311287154414  0.9588860194852554  0.9830427502421445  0.9916734207657063
4.3359375   0.8237474298079347   0.9027886441661502   0.9376336427809236  0.8785461654627058   0.9998649077543469  0.9786777628352686  0.9692699693734874  0.9997759080084567  0.9849794888061214  0.938714349330329   0.9733074319721083  0.9984375471486664  0.9899610054428071  0.9325381258497806  0.9842573441752248  0.774629080171742    0.9556565846366923  0.7828359718041175   0.8172368374458157   0.9661530966918087  0.9998648171908233  0.8232639137601152   0.9188002639690064   0.924883350215925   0.9842954179230984  0.8898037450069186   0.9980880995847767  0.9290281047434757  0.9281263756605738  0.9609511803145925  0.9558532141645733  0.938132733121623   0.9139593466397329   0.8944006918727019   0.9984395666783642  0.989909288264114   0.9470961936612576  0.9997759931442803  0.9897079880623967  0.9731224487747214  0.9478870034503698  0.9793062448801849  0.9899801242055921
4.453125    0.7693136510094314   0.8755092129462994   0.9218785667045281  0.8422928531930117   0.9998437778806252  0.9747363395139877  0.961948617760788   0.9997410798831516  0.9821257563883404  0.9235378787520028  0.9680895866486983  0.9981903564385269  0.988259822074034   0.9152852654155442  0.9814780852923439  0.7053825543200775   0.9457428729008857  0.7174452816730678   0.7606518905868676   0.9588934718684977  0.9998436758267171  0.768625518589059    0.8972046206676817   0.9054692897602374  0.9815213342271415  0.8582720238866224   0.9975733965999446  0.9118632324629001  0.9097745355743627  0.952780606344768   0.9459777188928477  0.9228122185289368  0.8903197570298939   0.8649149764083099   0.9981926069588191  0.9881998942765331  0.9346130681160345  0.9997411951673947  0.9879802490261401  0.9678729180035798  0.9355999621041283  0.9754338603420563  0.9882804660518758
4.5703125   0.7058226073324841   0.8441650792791843   0.9043224007445079  0.8000813957757426   0.9998243412312552  0.9709057682295774  0.953484067913598   0.9997090665638065  0.9797247559883624  0.9068579549470324  0.9628991875851797  0.9979624574407067  0.986661042701576   0.8960343438514016  0.9788181734792476  0.6255069794497695   0.9353085618379234  0.6418962660954101   0.6946919996700297   0.9513597972589583  0.9998242306120009  0.7048520949978088   0.8725258253997641   0.8838467185786092  0.9788644780773381  0.8219484373702484   0.9969572515373357  0.8934235361067961  0.8894293548173593  0.9445064409280901  0.935577083755405   0.9059857073119585  0.863223977670448    0.8312448174773528   0.9979648605379838  0.986586549986152   0.9211822278179793  0.999709215352901   0.9863608568481437  0.9626571074068986  0.9223714560141257  0.9716330139275604  0.9866818887557347
4.6875      0.6345198386373875   0.8091618157325375   0.8852982154298689  0.7525347638132125   0.9998079793574688  0.9674456324771062  0.9436530378810375  0.9996821142328499  0.978215425704125   0.8891341545781625  0.9580589335388813  0.9977701978381812  0.9852830885900937  0.8751785789205374  0.976462872533125   0.5373152133271812   0.9248145752731125  0.5580746209686313   0.620722543070525    0.9439179311928813  0.999807864164      0.6331616826991062   0.8448766947838188   0.8605421290458938  0.9765089565828187  0.7813581605972688   0.9962422735754625  0.8745086566174625  0.8676178080558687  0.93664529740355    0.9251066959288125  0.8881267960002125  0.8329698819522062   0.7939642256370812   0.9977726499890063  0.9851844402614812  0.9072936064715063  0.9996822974707688  0.9849690284807062  0.9578029254685875  0.9086744530638375  0.9681452531198     0.9853024036601374
4.8046875   0.5575667049345894   0.7711381092655069   0.8652247446645309  0.7007703341146954   0.9997959565919928  0.964626971927255   0.9321231406907884  0.9996622703160799  0.9779895917614387  0.8709415235537546  0.9539189304442299  0.997628644069275   0.9842417617552748  0.8532233365250895  0.9746003454943142  0.4444798557655385   0.9147998980579672  0.4691274895403505   0.5410900956654441   0.9369909489974474  0.9997958414574822  0.5556899891092901   0.8143967102953816   0.8362450549878668  0.9746419913750668  0.7373425561097623   0.9954544245721941  0.8561160762483914  0.8450189238316839  0.9297802370004133  0.9150992880932881  0.8698280927929268  0.8000201122984014   0.7539215362830136   0.9976310282981772  0.984107167356394   0.8935397946673805  0.9996624857455826  0.9839205367902588  0.9536657572661605  0.8950800781040348  0.9652269791211808  0.9842576183565369
4.921875    0.4779113162409258   0.7309153763527481   0.8445699635115074  0.646340260923545    0.99978925549788    0.9627029685481339  0.9184514633820045  0.9996511413309438  0.9792546042931669  0.8529281889570172  0.9508236063563578  0.9975496099282928  0.9836346962265745  0.8307431574730374  0.9734001907313254  0.3517992596686051   0.9058414360340192  0.3793399600012992   0.4589812259602061   0.9310293294012119  0.9997891451203126  0.475378056391809    0.7812141390061955   0.8117557839005994  0.9734327997180516  0.6909983855270979   0.9946557593720986  0.8393657851569724  0.8224124501131898  0.9245043152363918  0.906125556487051   0.8517574793104165  0.7649637788222805   0.7121724274585717   0.9975518117860044  0.9834525886880029  0.8805733799598258  0.9996513832622747  0.9833123436961141  0.9505942791755599  0.8822164464830577  0.9631250991430362  0.9836453226266179
5.0390625   0.39899965117124614  0.6894243470587836   0.8238129212651724  0.5910956244753724   0.9997884397196509  0.9618765460893163  0.9021170160332896  0.9996496414179394  0.9819122115245947  0.835765329140055   0.9490736864452798  0.997539904819086   0.9835258432514435  0.8083324989694315  0.9729905555457345  0.2647167996100268   0.8985065598889341  0.29374839615801007  0.3781077972284973   0.926476056244306   0.9997883382597811  0.3957089390750991   0.7454237313984461   0.7879193896603145  0.9730100336248898  0.6435825331828952   0.9939491164895473  0.8253888341740923  0.8006149816429314  0.9213503955377287  0.8987477680022294  0.8346060131380985  0.7284656514200465   0.6698864521625243   0.9975418291041052  0.9832883825666126  0.8690558241117169  0.9996499011557117  0.9832074840755414  0.948891072451905   0.8707199181659757  0.9620496231336082  0.9835300425477101
5.15625     0.32437183408930637  0.647620864341379    0.8034103765212134  0.536994487240793    0.9997935722683304  0.9622692448855031  0.8825989515881922  0.9996578853397172  0.9855272639212328  0.8200966730645977  0.9488873066822852  0.9976002890897602  0.9839332184057757  0.7865583408585812  0.9734371815597562  0.1886654130572181   0.8933033016509758  0.2174910147765517   0.30226781553132126  0.9237300225700539  0.9997934828578102  0.32033244386572357  0.70708849834629     0.7655559844210453  0.9734409892783087  0.5963988338845938   0.9934619855082789  0.815184184693168   0.7804124751443352  0.9207150597609727  0.8934712515075469  0.8190347932721219  0.6912111758792977   0.6282435246171407   0.9976018728733446  0.9836399345665571  0.8596040558571102  0.9996581521569352  0.9836232966780148  0.9487723998520554  0.8611848621268633  0.9621457987552438  0.9839307305860766
5.2734375   0.2572255846034458   0.6064046333225639   0.7837728517783956  0.48589073467344635  0.9998042070879968  0.9638967466071477  0.8595069946755649  0.9996751816097162  0.9894368942951343  0.8064937357675555  0.9503651999655531  0.997725428650878   0.9848232095565479  0.7659216576308769  0.97472834663897    0.1283627783820464   0.8906329302037376  0.1549937514949577   0.23487138112243972  0.9231102287697525  0.9998041315512282  0.25264886496960454  0.6662674195498778   0.7453967091102989  0.974716405674715   0.5506867672790865   0.9933045124328812  0.8094536558005091  0.7624974321212727  0.9227875432163242  0.8906983470103798  0.8056272552108806  0.6538556193256428   0.5883375756382484   0.9977266474195499  0.9844833979557023  0.8527404874337468  0.9996754442262764  0.9845260939395815  0.9503325330397994  0.8541174687372604  0.9634690264849782  0.9848148672291813
5.390625    0.2000412991766553   0.566553044733896    0.7652521866277084  0.4393468648203906   0.999819455816997   0.9666567299801638  0.8327570349241424  0.9997001026200357  0.9929699520026541  0.795420846998829   0.953465469949184   0.9979047348957603  0.9861137832288995  0.7468323296499757  0.9767696526253979  0.08719996411811141  0.8907488759857395  0.10920268202877244  0.1785348905457725   0.9248215672653666  0.999819394641037   0.19543357317479115  0.6230640355957746   0.7280324994375611  0.9767447316674741  0.5075298456347646   0.9935191403544786  0.8084325059111854  0.7474167543472541  0.9274971380356638  0.8906885311365113  0.794851294649017   0.6169840913532784   0.5511011143751734   0.9979056010479465  0.9857463508407984  0.848850747185318   0.9997003507077115  0.9858343952931998  0.9535186898705775  0.8498976460511519  0.9659665698914844  0.9861013582470027
5.5078125   0.15434514823389234  0.5286782433224659   0.7481403134723434  0.3985067981530157   0.9998381148715834  0.9703337597604391  0.8027615636323927  0.9997307115791333  0.9956607841304761  0.7872115404840521  0.9579935925450286  0.9981238510768864  0.987686893756661   0.7295991578374905  0.9793915836471953  0.06684368269443387  0.8937252217851301  0.08112377679056834  0.13482680643591619  0.9289214772785599  0.9998380673920095  0.15057051999823493  0.5776853500362303   0.7138790657633931  0.9793588755254137  0.467795407572906    0.9940577552797062  0.8117437380148381  0.7355333702172239  0.9344937393405063  0.8935275687663     0.7870331624534256  0.5810866912709212   0.5172582054481377   0.9981244060966853  0.9873171317826942  0.8481517014377612  0.9997309369294769  0.9874309205082841  0.9581226624769597  0.848751173168536   0.9694707664132711  0.9876726487281547
5.625       0.1206455327145      0.493210551155       0.7326771854558     0.3640470796641      0.9998588259047     0.9746231915546     0.770577366807      0.9997648088068     0.997340658771      0.7820558650266     0.963612550732      0.998366585892      0.989407760508      0.7144339337475     0.9823705033778     0.06712102804008     0.8994376751207     0.06983773145339     0.104203338561       0.9352871301545     0.9998587906278     0.1189310265995      0.530497176786       0.7031580629929     0.9823363818676     0.4321093558575      0.9948074173779     0.8183149382561     0.7270022554108     0.9431746208446     0.8991081352708     0.7823430462872     0.546550238971       0.4873064615111      0.9983668889237     0.9890617711705     0.8506710071586     0.9997650060706     0.9891811178337     0.9637940810427     0.8507326896074     0.9737084393187     0.9893938560147
5.7421875   0.09853112713311929  0.4604064717361149   0.7190645229325955  0.33620352453010793  0.9998802387861541  0.9791717419178716  0.7379385268331138  0.9998001725970312  0.9980883300500312  0.7799970978930939  0.969875821744481   0.9986169467839067  0.9911463823720403  0.7014663980981292  0.9854604889961988  0.08619336735619126  0.9075601982824633  0.07294821476290829  0.08612075262063881  0.9435868071965297  0.999880213752071   0.10039786385487401  0.4820614703741898   0.6958921727741619  0.9854308099240626  0.4008616351650986   0.9956441488526636  0.826399175728795   0.7217601986247224  0.952763732476395   0.9071245983905452  0.7807911431997504  0.5136654497351115   0.46152297878053716  0.9986170629163259  0.9908450215501808  0.8562389729247606  0.9998003393929897  0.9909539514494649  0.9700773732578908  0.8557192444710062  0.9783295622431075  0.9911343667057038
5.859375    0.08688273866997405  0.43037577664135046  0.7074811362644339  0.3148529885369616   0.9999011504219886  0.9836275375584423  0.7071218192678911  0.999834816483322   0.9981170795628715  0.780936120494915   0.9762818245502612  0.9988608146907898  0.9927957804559635  0.6907655230069061  0.9884299725228544  0.12096656187930388  0.9175808440445716  0.08724260643779307  0.07927158207379352  0.9532669111489093  0.9999011335335923  0.0940059878340547   0.4331436735526774   0.691911283150853   0.9884085593841335  0.3742329554598924   0.9964729914657138  0.8337257905072873  0.7195287980888705  0.9624359975060504  0.9170852219771901  0.7822328432027937  0.4826462784289368   0.4399859560592229   0.9988608055605352  0.992550331819576   0.8644939338760402  0.9998349530214623  0.9926399013284394  0.9764696124182551  0.8634141555023415  0.9829541456133638  0.9927863485921726
5.9765625   0.08413600987215629  0.40312012033890704  0.6980950037074577  0.29962138307827757  0.9999206020624282  0.9876890746909303  0.6806475033459296  0.9998671392697438  0.9976787715644415  0.7846420234968634  0.982342670147787   0.9990870297721824  0.994282601561148   0.6823618702767774  0.9910949446667932  0.16763725089163997  0.9288398356832167  0.10932079578332958  0.08187538748524258  0.9635740711549248  0.9999205913307424  0.09815662852939949  0.3846838484537546   0.6908679265041584  0.9910828689774667  0.3522314216938045   0.9972350248052518  0.8377719237928742  0.7198312826952826  0.9714601475538946  0.9283450903795405  0.7863820606559454  0.45365660339386377  0.4226033978121113   0.9990869475444828  0.9940936931696726  0.8749031188230539  0.9998672477462264  0.9941620892400992  0.9824903690611709  0.8733615826922037  0.9872301166285286  0.9942756740061073
6.09375     0.08853718343236744  0.3785752568315547   0.6910672807197594  0.28999007132216953  0.9999379284818938  0.9911424070758804  0.6608825390931586  0.9998960057313054  0.9970063027625852  0.7907680451451194  0.9876542185094109  0.9992879293637313  0.9955685271112695  0.6762658061740102  0.9933407537171     0.22225027000779699  0.9405903982135454  0.1360548887669352   0.09196048179593018  0.9736322660356891  0.9999379221771352  0.11086003021063441  0.3377335788327163   0.6922618132615626  0.9933366446342976  0.3347291453703648   0.9978957987752367  0.8361094081482163  0.7220248867250547  0.9793179063582297  0.940161402017636   0.7928323313953696  0.42683854367217183  0.40914230541630153  0.9992878139549031  0.9954283845898977  0.8868010756746992  0.9998960895754484  0.9954788213074265  0.9877494833257157  0.8849729047940087  0.9908888647447336  0.9955635530078102
6.2109375   0.09579236705149859  0.3566499168841603   0.6865443956303083  0.28538081438678414  0.9999527617515338  0.9938793877156596  0.6496414816584574  0.9999207806910259  0.9962787475425391  0.7988725209754959  0.9919512869583632  0.999459391208558   0.9966441640499523  0.6724771561151537  0.9951276701464002  0.2811519733549493   0.9520778315190175  0.16485850143326355  0.10759304322732727  0.9825797720117124  0.9999527584690723  0.12996939719142467  0.29336927248909445  0.695474614010159   0.9951287752584688  0.3214925140328535   0.9984371581063701  0.826769536413839   0.725350598710267   0.9857575356745312  0.9517686717333196  0.8010856219023094  0.4023366596944422   0.3992550211180121   0.999459270123688   0.9965409265036658  0.8994460469762725  0.9999208438838874  0.9965786519764891  0.9919971236592767  0.8975659764199112  0.993782786789842   0.996640480057214
6.328125    0.10413873363421365  0.3372562563526748   0.6846366221070425  0.2852103945972166   0.9999649989017031  0.9958953068099359  0.647857322365851   0.9999412648256653  0.9956030333456352  0.8084448949370086  0.9951339646547828  0.9996004136451137  0.9975192576243087  0.6709835257233276  0.9964803708241791  0.3412693615635068   0.9626258548966953  0.19379472436855896  0.12703394620258113  0.9897403982162524  0.999964997566386   0.1533795783257692   0.2525986552734686   0.6998161852207367  0.9964837733797217  0.3122040936996222   0.9988574150650963  0.8085840780420317  0.7290004062402221  0.9907635977079896  0.9624652181252688  0.810588482084538   0.3803132113492497   0.3925026130803878   0.9996003035808447  0.9974414346646765  0.9120907737429461  0.9999413113937248  0.9974711687558847  0.9951443601712373  0.9104170640907214  0.9958934462535194  0.99751635508839
6.4453125   0.11147033815307199  0.32032902927352463  0.6853848835521725  0.28891624432135454  0.9999747462824946  0.9972693091800973  0.6553544582281229  0.999957608120935   0.9950209341092289  0.8189367121640666  0.9972588162685171  0.9997123983246713  0.9982132963924215  0.6717469338090396  0.9974662962844306  0.4002080569568793   0.9717140596420061  0.22155774765711844  0.14882362875918326  0.9947729726890195  0.9999747461149009  0.17917551814105642  0.21627854342456934  0.7045818223531678  0.9974697657572468  0.306477767116694    0.9991704532718404  0.7814753875941974  0.7321986482659542  0.9944640310968367  0.9716961607518164  0.8207739871715276  0.360951514191447    0.3883773157008942   0.9997123075913353  0.9981527201782873  0.924058854621388   0.9999576417638618  0.9981772932881909  0.997250934000016   0.922822598625173   0.99730915280874    0.9982109254588518
6.5625      0.12009390472181876  0.3058328864094437   0.6887214031059188  0.2959616151144875   0.9999822540463     0.9981339420050813  0.6707471469028625  0.9999702141806688  0.9945384941572188  0.829796707150425   0.9984995676290624  0.9997983697965875  0.9987490980629875  0.6746812830051375  0.9981704903352312  0.45621548985915     0.979029623565125   0.24736960848680623  0.17180779658873124  0.9977357534883     0.999982254511375   0.2057259084449875   0.18505792960828124  0.7091171638395874  0.9981727928717438  0.3038707693696875   0.9993981801684187  0.74664906160545    0.7342895583808687  0.9970265061792688  0.9791152106185875  0.8311060204008688  0.3444469497511125   0.3863265407032625   0.9997983008303499  0.9987016500537187  0.9348132135360124  0.9999702380792688  0.9987219438425374  0.9984865229649812  0.934164376386025   0.9981820389395875  0.9987472194348437
6.6796875   0.12779110716986264  0.29375915783845896  0.6944331470699868  0.3058320635811449   0.9999878523632688  0.998641109562625   0.6915133338385282  0.9999796221718861  0.9941551259982048  0.8405077832131007  0.9990909098951493  0.9998622547509834  0.9991498303331267  0.679625956800609   0.9986741959369578  0.5080784566181807   0.9844819371399155  0.2708453243655153   0.1951218010609953   0.9990311324554012  0.9999878531150251  0.23172592486945412  0.1593527922421714   0.7128839764996496  0.998675016401292   0.30389651353252556  0.999561364066007   0.7066003771370652  0.7348189322865004  0.9985963092293442  0.984610459787017   0.8411205450178298  0.3309877993465727   0.38577998222290777  0.9998622063456524  0.9991142908067183  0.9440022419632901  0.9999796389298893  0.9991300739370964  0.9990784950762631  0.9439686909930946  0.998681017425703   0.999148484443547
6.796875    0.13603985705747704  0.2841149191796048   0.7021386965143203  0.3180338792130595   0.9999918975277041  0.9989314081188003  0.7143001579230059  0.9999864211309318  0.993872166855605   0.8506223469739542  0.9992724141983776  0.9999082559454106  0.9994383103045482  0.6863227559080775  0.9990417330888203  0.5550134817972254   0.9881789434672933  0.2918736354575544   0.21815115053180362  0.9992484237781171  0.9999918983574078  0.2561987466185799   0.13935089115231053  0.7155174407017987  0.9990414430479897  0.3060408395451372   0.9996756110860108  0.66484364753726    0.7335956387805972  0.9992967650452179  0.9882882667921351  0.850457617995168   0.3207301273480972   0.386179511879262    0.999908225037151   0.9994144014466825  0.9514734952800675  0.999986432778578   0.9994253995520025  0.9992605760351851  0.9519486137696811  0.9989553471700073  0.9994374945861725
6.9140625   0.1449248710824772   0.27690790246194236  0.7112883041922773  0.3320998094863457   0.999994732106154   0.9991126570878913  0.7354480329320514  0.9999911851453311  0.9936836099356138  0.8597917095319082  0.9992464184255085  0.9999403762553373  0.9996369606638363  0.6944038103613976  0.9993165181353697  0.5965776353503254   0.990374556734659   0.3105310241578796   0.24048228426268592  0.9989682857078422  0.9999947328996256  0.27846835922595514  0.12503887495573524  0.716864830057398   0.9993159010462762  0.3097822966785838   0.9997528687084727  0.6253654702992674  0.7307196836989219  0.9992724009601505  0.9904210398842724  0.8588788051363148  0.3137719135949633   0.3870097650794425   0.9999403591665447  0.99962374685348    0.9572535493458375  0.9999911932095071  0.999630333432011   0.9992335094961329  0.9580204708994035  0.9991158754158248  0.9996365800668803
7.03125     0.15418712867120546  0.2721308328324102   0.7211930795197757  0.3476001887761101   0.9999966597505102  0.9992511552592734  0.7516026190965632  0.999994422242368   0.9935689677801812  0.8677854960914445  0.9991565050015265  0.9999621355952617  0.9997673343403398  0.7033966493187485  0.9995240523992539  0.6325978400318437   0.9914029984241672  0.3270173929696765   0.26185342815162654  0.9986055927145993  0.9999966604543687  0.2981161175878281   0.11624158186727267  0.7169972712087812  0.9995239749436344  0.31461511035956485  0.9998036288835047  0.5919490010465494  0.7265682594246922  0.9987312709201593  0.9913743214611032  0.8662674090570828  0.31013133849128827  0.3878262627394836   0.9999621287580992  0.9997626492653852  0.9615022684600539  0.9999944278385382  0.9997658872065094  0.9991430507920875  0.9622905671143485  0.9992324290948414  0.9997672314194711
7.1484375   0.16335914003839877  0.2697481366586358   0.7310830001206802  0.36415235426958714  0.999997932869323   0.99937566108742    0.7602254384821666  0.9999965568368072  0.9935015180373633  0.8744982759870942  0.9990855220913055  0.9999764479924069  0.9998490223203886  0.7127493604982984  0.9996782547261966  0.6631047358718423   0.9916161872533851  0.34159719340540345  0.2821112951999691   0.9983441082770321  0.9999979334662905  0.31493148880469957  0.1126641515233634   0.7161904896089966  0.9996792979898365  0.320072224262532    0.9998366389892115  0.5675922605056948  0.7217390797589874  0.9979479918420106  0.9915322670706277  0.8726126948134089  0.3097327793490569   0.3882778830927807   0.9999764483015963  0.9998498728192824  0.9644560947826206  0.9999965607505795  0.999851087462274   0.9990750884949281  0.9650149817464619  0.9993407947244023  0.9998490416479128
7.265625    0.17118524141354086  0.2696869052323378   0.7401863613311579  0.38142121606129176  0.99999875013205    0.9994901812783807  0.7598813417729632  0.9999979246596885  0.9934641487785156  0.8799426489744355  0.9990669265315882  0.9999855984698974  0.9998984110910014  0.7218743759950517  0.9997877427722464  0.6882641026344533   0.9913355488076734  0.35454664035711     0.3011761686628164   0.9981719693782566  0.9999987506248138  0.3288643480346437   0.11393022788584602  0.7148762159930576  0.9997899266717389  0.3257455175466674   0.9998578705331314  0.5541717921671175  0.7169593901946996  0.9972159303685919  0.9912377242062856  0.877982788157361   0.31240220382231076  0.38812134065291015  0.9999856034267854  0.9999018170064181  0.9663755915198631  0.9999979274346162  0.9999021253898309  0.9990638929318811  0.9665414310749003  0.9994526180303664  0.9998984456728635
7.3828125   0.17913629619559723  0.2753685561004424   0.7478176465995225  0.3991085872203432   0.9999992605112364  0.9995890605676091  0.7503134801132083  0.9999987761533273  0.9934554432522887  0.8842302820014263  0.9991016479053274  0.9999912891854543  0.9999278881469137  0.7302050166728794  0.9998600123231038  0.7083122147905473   0.9908228256864132  0.3661208196987205   0.3190156786597028   0.9979845467119123  0.9999992609116051  0.33998311354089306  0.11961272460415526  0.7135717528502884  0.9998627878842806  0.3313008612706822   0.9998710758821391  0.5523506031683927  0.712976549045784   0.996768246871163   0.990755973156776   0.8824929729769486  0.31787182882285037  0.3872266179764232   0.9999912969119128  0.9999315847722189  0.967507889560856   0.999998778159641   0.9999316843870008  0.9991077959349522  0.9672456063740503  0.9995645568112519  0.9999278886648617
7.5         0.1873530323803      0.2828121560136      0.753458643764      0.4169348926727      0.9999995708911     0.999668091028      0.7323949967652     0.9999992918587     0.993479881435      0.8875442856092     0.9991741090455     0.999994733957      0.9999458168781     0.7372558042897     0.9999030690159     0.7235079644049      0.9902686390018     0.3765471362749      0.3356262837374      0.9976982691638     0.9999995712144     0.3484404748485      0.1292563508421      0.7127994221529     0.999905618986      0.3364867349815      0.9998793020463     0.5616308349112     0.7104494634874     0.996706724811      0.9902632429884     0.8862761809281     0.3257925739809      0.3855735788418      0.9999947431175     0.9999485408446     0.9680676318662     0.9999992933417     0.9999487533868     0.9991866337517     0.9674742113887     0.9996660629382     0.9999457779648
7.6171875   0.19497560458420427  0.29514042081333897  0.7568195214731894  0.43462074133491846  0.9999997549056889  0.9997284615206262  0.7080231363792835  0.9999995963044613  0.9935337046628668  0.8901073866105654  0.9992639752855391  0.9999967671191489  0.999957206541793   0.7426766156834853  0.9999252155016174  0.7341101462380163   0.9897942622745673  0.3860329871876007   0.3510209134189834   0.9973224615849254  0.9999997551671336  0.35444672027340046  0.14239290596922075  0.7130090709068017  0.9999268870344358  0.341136258098727    0.9998854287808825  0.5804669861075352  0.7098585468606982  0.9969758258788792  0.9898550469104891  0.8894607490588499  0.3357521292149005   0.3832409724586335   0.9999967768089598  0.9999585980416861  0.9682335919860726  0.9999995974287237  0.999958990258827   0.9992764713216129  0.9675037926811018  0.9997462078530109  0.9999571487631005
7.734375    0.200392583010133    0.3081917066144506   0.7578713342641978  0.45187606428729027  0.9999998614108712  0.9997746381828813  0.6799482820665272  0.9999997714002168  0.9936024540953384  0.892150442276483   0.9993530458549095  0.9999979406384132  0.9999647364474236  0.7462921059484607  0.9999340866679073  0.7403775972716523   0.9894613591657564  0.3947682630662079   0.3652211798417857   0.9969620480587194  0.9999998616244942  0.3582496717762918   0.1585512246752353   0.7145153299766929  0.9999346773676818  0.34516339313199357  0.9998911910020091  0.6064337072033911  0.7114464545363485  0.9973986157665512  0.9895663136849722  0.8921577933020605  0.3472962921362109   0.3803896257142927   0.9999979502889473  0.9999650284643733  0.9681543814202146  0.9999997722764203  0.9999655031315039  0.9993589762190184  0.9675198749969769  0.9997991231876056  0.9999646887163285
7.8515625   0.20620703384891406  0.3204942339184572   0.7568469338137078  0.4684012913367369   0.9999999217055312  0.9998102048224103  0.651492672043634   0.9999998698920591  0.9936722094610814  0.8938851359110483  0.9994284070207649  0.9999986049865683  0.999969751097042   0.7481207559921241  0.9999357927214748  0.7425802542581703   0.9892853992715126  0.40291595049152296  0.37825279070793844  0.996759388286988   0.9999999218830651  0.36011982032699036  0.1772637992764756   0.717457737828933   0.999935528645867   0.34855461157624884  0.9998964919538105  0.636491916529382   0.7151951054577962  0.9977614052181563  0.9893950485311911  0.8944577839049498  0.35995141713317963  0.3772418636727025   0.9999986142861836  0.9999694093223063  0.9679560586006617  0.9999998705924001  0.9999697945841877  0.9994248782880136  0.9676163357840558  0.9998265980308743  0.9999697328495333
7.96875     0.21237670693967423  0.33206368693711485  0.7542118498549015  0.48389978314762344  0.999999955186293   0.9998356831714407  0.626151821665571   0.9999999242534751  0.9937408278969336  0.8954835563515383  0.99948333067705    0.9999989752374039  0.999972977877      0.7483713521523743  0.9999345494980078  0.7410064462502133   0.9892499766940305  0.4106020010166883   0.3901431057144719   0.9968122353322797  0.999999955337243   0.3603392337502203   0.19807179553064927  0.7217874578197782  0.9999338352619953  0.3513574951445156   0.9999005161987438  0.6673690561154711  0.7208385418426188  0.9979098328883758  0.9893230094054898  0.8964338867893062  0.3732461641213711   0.37405920023432815  0.9999989840475868  0.9999724140005773  0.9677463573689766  0.9999999248290211  0.9999725816584704  0.9994727581099766  0.9678112894890422  0.9998367412552407  0.9999729880094445
8.0859375   0.21886993790782672  0.34291443339727895  0.7506098545154367  0.4980959879479591   0.9999999734955403  0.999849786099903   0.6071499832105651  0.9999999536934926  0.9938162733364203  0.897066115479413   0.9995169077777095  0.9999991800767631  0.9999749128306364  0.7474178438114553  0.9999327856785271  0.7359577341213172   0.9893204892613559  0.41791845524795523  0.4009200650399547   0.9971139082488182  0.999999973627409   0.3591930153660761   0.2205296535458852   0.727280238557523   0.9999319617520481  0.3536676703916144   0.9999033752767497  0.6959930705598704  0.7279068939590625  0.9978128706214578  0.9893296600524739  0.8981486811261871  0.38673112699027495  0.37112017964649613  0.9999991883559134  0.9999743518038079  0.9676134555000895  0.9999999541793708  0.9999743004534075  0.9995054871750602  0.9680733820977308  0.9998394770216157  0.9999749342377516
8.203125    0.22566237090982688  0.35305955675402023  0.7467900338006107  0.5107521900920442   0.9999999834091389  0.9998525007065311  0.5970608304990527  0.9999999695170034  0.993904172266573   0.8986980779104372  0.9995327899870784  0.9999992941242017  0.9999759616371239  0.7457560147143277  0.9999315496512752  0.7277337898878019   0.9894562584374923  0.4249392408190683   0.4106119475763874   0.9975478164427073  0.9999999835274899  0.3569624032065749   0.24420990256348396  0.7335720175757884  0.9999308318002288  0.35561526117228676  0.9999063428955423  0.7198727790768338  0.7357938646727494  0.9975662903790681  0.9893985696950558  0.8996609010213331  0.3999953167879719   0.3686999694236084   0.9999993018892338  0.9999754686019852  0.9676204244894544  0.9999999699358552  0.9999752940209287  0.9995267192458894  0.9683513985597492  0.9998415782016287  0.9999759770880675
8.3203125   0.23273436522538177  0.36251132968651967  0.7435237054033467  0.5216790312220151   0.9999999887579566  0.9998470642009328  0.59756373420886    0.9999999780032364  0.9939990246916192  0.900393944089834   0.9995371914246113  0.9999993589289717  0.9999764629013189  0.743947254290578   0.9999310340797269  0.716618959655102    0.9896202945750321  0.43173380206479817  0.4192475997864976   0.9979392095525323  0.9999999888670368  0.3539189295277299   0.26870832943089107  0.7402112201537698  0.9999306067627513  0.35735178210540747  0.999910486099281   0.7373477018927261  0.7438382653619257  0.9973361696918038  0.9895177637004434  0.9010296178809128  0.41267886829250233  0.36705292582850413  0.9999993662375564  0.9999760441543337  0.9677987161652128  0.9999999783728176  0.9999758327920943  0.9995389959804163  0.9686004316771194  0.9998442798514956  0.9999764662737132
8.4375      0.24006901566768749  0.37128187669166873  0.7415204159178626  0.5307392708556687   0.9999999916532938  0.9998390512742062  0.6093226303094937  0.9999999825633312  0.9940890416500062  0.9021280918645063  0.9995365247377812  0.9999993971016438  0.9999766827949562  0.7425559759944312  0.9999310754863     0.7028835905951187   0.9897858960963812  0.43836653496239375  0.42685693309358747  0.9981374738068375  0.9999999917564938  0.35031935178435     0.2936492853304125   0.7467203001091313  0.9999311643122375  0.35903814470412504  0.9999153572940437  0.747707941347525   0.751409144783375   0.9972716099777875  0.9896768374068062  0.9023145009690188  0.4244817267265375   0.36639890325668123  0.999999404027075   0.9999763482225     0.9681446773169312  0.9999999828951812  0.9999761170908813  0.9995437474474249  0.9687994260488687  0.9998447644611812  0.9999766766235813
8.5546875   0.24765071846026976  0.3793838916942055   0.7413521508858631  0.5378474746708074   0.9999999932360277  0.9998333022202305  0.6319250671240628  0.999999985096263   0.9941684053066219  0.9038494852562923  0.9995353776108408  0.9999994210102832  0.9999768138082012  0.7420878072994573  0.9999315140277282  0.6868065772934472   0.9899394671180901  0.4448864003565601   0.43347162158623853  0.9980871250063732  0.9999999933359358  0.3464012855227708   0.31869071381009983  0.7526584461285735  0.9999323070148122  0.36083422929573605  0.999919516713445   0.7512370962555815  0.7579842229416041  0.997429875482945   0.9898637903033218  0.9035721786652602  0.4351684332717984   0.3669136087163259   0.9999994276164625  0.9999765613857332  0.9686221086244214  0.9999999853970307  0.9999763011255044  0.9995423787012012  0.9689574030715498  0.9998401362941683  0.9999768011831965
8.671875    0.25546421138521197  0.3868313664718736   0.74339361541796    0.5429687675510998   0.9999999941149686  0.9998312164481522  0.6638435114395069  0.9999999865503227  0.9942425287980172  0.9054979288526576  0.9995355709959044  0.9999994372907844  0.9999769789792209  0.7429354275948774  0.9999323384783576  0.6687148813974361   0.9900794973796236  0.4513218652783396   0.4391260260040419   0.9978519366658986  0.9999999942133323  0.3423796039826709   0.34352842565364955  0.7576777147753935  0.9999338069198351  0.3628902920428818   0.9999222192696775  0.7492167343303587  0.7632127868120266  0.9977499243720732  0.990063704212951   0.9048499450237145  0.4445694270202213   0.36872287832580475  0.9999994436312424  0.9999767575209114  0.9691717648620499  0.9999999868265048  0.9999764995057313  0.9995372286478298  0.9691085896679992  0.9998308461267622  0.9999769621111768
8.7890625   0.26349399943295065  0.39364034296195427  0.747784305949248   0.5461180357406921   0.9999999946124961  0.9998306274085996  0.702453363095705   0.9999999874261055  0.9943204508469478  0.907019300154877   0.9995366155332425  0.9999994492724017  0.9999772387475638  0.7453378570425593  0.9999336242050281  0.6490261484097886   0.9902124587969197  0.4576892011167481   0.443858403929853    0.9975806268354598  0.9999999947107963  0.3384437222681204   0.3678992045743157   0.7615660935597712  0.9999354257889803  0.3653403203583455   0.9999241517336271  0.7438700801990394  0.7669565752818176  0.9980857213974164  0.9902600102233499  0.9061788205868212  0.45257948172310414  0.37190043262558814  0.9999994554014754  0.9999769580242088  0.9697262172505402  0.9999999876811632  0.9999767633858886  0.9995315612531793  0.9692993653618344  0.9998211041647391  0.9999772235641925
8.90625     0.27172406354854456  0.4029058840542477   0.7544157126905273  0.5473590565400258   0.9999999948989359  0.9998283804318571  0.7441962699027619  0.9999999880071336  0.9944033918705789  0.9083774609021132  0.9995371948091406  0.9999994586583586  0.9999776007518187  0.7493573283822969  0.9999353711427078  0.628276578111246    0.990347135888468   0.464005895942818    0.4477124336942344   0.997431809350525   0.9999999949983649  0.3347558720887906   0.3915824629244368   0.7642728761257173  0.9999369667796953  0.3682973166099328   0.9999265093861961  0.7381745619482079  0.7693048415607898  0.9982811824993023  0.9904377097765805  0.9075681434258875  0.45915397586397894  0.3764684846472344   0.9999994646256531  0.9999771866248961  0.9702265172790758  0.9999999882427352  0.9999770719296469  0.9995285697411211  0.9695717702443047  0.9998161660834398  0.9999775966820618
9.0234375   0.2801377612836961   0.4132149504278557   0.7629447845164608  0.5468029582426919   0.9999999950635108  0.9998235613426026  0.7849582135661015  0.9999999884081668  0.9944829575668273  0.9095611530125469  0.9995367709695285  0.9999994664997455  0.9999780323055751  0.7548759264838567  0.9999373923757521  0.6071208633252666   0.9904892950002853  0.47029471253419836  0.45073899765325764  0.9974958292855334  0.9999999951649435  0.33145041746751075  0.4144003388847871   0.76591389020657    0.9999383204311523  0.37185038015353505  0.9999295068565761  0.7354875754782026  0.7705622876023944  0.9982486670179618  0.9905870186832926  0.9090034806098761  0.4643036613329195   0.38240052589428075  0.9999994723383734  0.9999774677568245  0.9706362629927862  0.9999999886277963  0.9999773743303609  0.9995300956630209  0.9699488386125219  0.9998187098731437  0.999978043898918
9.140625    0.28871785854102006  0.42279628981630274  0.7728327132418685  0.5446075370834876   0.9999999951520464  0.9998187943514735  0.8206448094962674  0.9999999887052784  0.9945508468149482  0.9105850827984704  0.999536379509896   0.9999994734277795  0.9999784748418751  0.7616121927385818  0.9999393472056834  0.5863030748547388   0.9906384048041105  0.47657483640099246  0.4529980736517558   0.9977519822441522  0.999999995256468   0.3286341860294664   0.43621629727064337  0.7667562934131718  0.9999394536813424  0.3760633768061923   0.9999321059094761  0.7390026990159922  0.7712109112671055  0.9980121498121619  0.9907057532655571  0.910448857971289   0.4680884788950301   0.38962567301710493  0.9999994791599761  0.9999777932379583  0.9709492336865966  0.999999988910016   0.9999776483593268  0.9995359536014092  0.9704261328425416  0.9998271010234654  0.9999784954476822
9.2578125   0.2974466551448726   0.43160151035889077  0.7834058357274509  0.5409813185026037   0.9999999951880437  0.9998185413573123  0.8478506873409176  0.9999999889463255  0.9946083614392308  0.9114854530194292  0.9995381958423343  0.9999994796921977  0.999978862293535   0.7691559309064534  0.9999408924961474  0.5666098275883431   0.9907873621637664  0.4828515144181512   0.4545605091779515   0.9980814375375998  0.9999999952965054  0.32638771455603527  0.45693243626068736  0.7671847544525789  0.9999403509578212  0.3809749358452531   0.9999332957842774  0.7511243204777016  0.7718492566096192  0.9976945150003764  0.99079934511101    0.9118532821766062  0.4706108146281079   0.3980340777445279   0.9999994853428541  0.9999781203932437  0.9711885372278705  0.9999999891379934  0.9999779131136144  0.9995443354282851  0.9709718215315813  0.9998367968320033  0.9999788790872715
9.375       0.3063061694604      0.4396073371828      0.7939329046107     0.5361956720523      0.9999999951846     0.9998256866108     0.8644973860863     0.9999999891366     0.9946646726287     0.912311219305      0.9995442110285     0.9999994854182     0.9999791448211     0.7770175368584     0.99994183702       0.5488212161754      0.9909250955017     0.4891169745766      0.4555094260016      0.9983283383163     0.9999999952978     0.3247672468502      0.4764857943551      0.7676528423672     0.9999409730954     0.3865994960186      0.999933239074      0.7728675462526     0.7731151856267     0.9974563845898     0.9908783681071     0.913160487383      0.4720084447837      0.4074830395936      0.9999994910114     0.9999784133118     0.9713976448903     0.9999999893183     0.9999781895081     0.9995530519303     0.971535056509      0.9998437107398     0.9999791497881
9.4921875   0.31527834198816485  0.4468133547371436   0.8037105186084493  0.5306030961646073   0.9999999951503875  0.9998390410453604  0.8703313049362562  0.9999999892973688  0.9947260119910676  0.9131131258430251  0.9995548879538615  0.999999490815387   0.999979311325522   0.7846864328662901  0.9999421944293941  0.5336685483627853   0.9910408978681365  0.4953618340234752   0.45594102749787513  0.9983779891466587  0.9999999952691814  0.32380728512712026  0.4948439922934216   0.7686252293718776  0.9999412858705106  0.3929291283767817   0.9999330300003637  0.8033849768138915  0.7756007305400008  0.9974175646870458  0.9909544709593339  0.9143200352608184  0.47244730891522424  0.41780357676084984  0.9999994963643367  0.9999786744099973  0.9716260236384828  0.999999989470438   0.9999784667937669  0.9995608305356867  0.9720599855736773  0.9998469194404802  0.9999793052847122
9.609375    0.3230265538097183   0.45323947856132574  0.8121468100204698  0.5246520476121238   0.9999999950935057  0.9998537122972834  0.8671562067411679  0.9999999894411146  0.9947886462158432  0.9139329525174023  0.9995686594510362  0.9999994960629359  0.9999793972189985  0.7916918968405245  0.9999421229073006  0.5217979174254398   0.9911287585104053  0.5015838585721902   0.4559646396207755   0.9982132868631929  0.9999999952183584  0.32352348477963144  0.5120005412886411   0.7705175968854723  0.9999413360944777  0.39993588824780785  0.9999334408624345  0.839756880673922   0.7797692048223261  0.9976000290722878  0.9910362374069918  0.9152975461990237  0.472114196201045    0.4288072815152802   0.9999995015752964  0.9999789260276967  0.9719136099959946  0.9999999896080409  0.9999787190770572  0.999567867863869   0.972501164850902   0.9998487111360753  0.9999793840644866
9.7265625   0.330528627468691    0.458923283464462    0.818833062916864   0.518883999536124    0.9999999950215207  0.9998641114269905  0.8586290982255722  0.9999999895603515  0.9948439407679823  0.9147952847410857  0.9995825934701604  0.9999995011737887  0.9999794694377239  0.7976590440132688  0.9999418265696869  0.5137349464858759   0.991190024531311   0.5077843409108606   0.4557018839688343   0.9979220743800237  0.9999999951526192  0.3239156936486084   0.5279701133524792   0.7736416168807915  0.999941294742489   0.4075744857866518   0.9999340582212476  0.8772353323332471  0.7858850940229566  0.9979198885737675  0.9911265124754293  0.9160820062454373  0.47120942449944825  0.4402933079388286   0.9999995066630012  0.9999791711314098  0.9722786479723232  0.999999989724119   0.999978944266121   0.9995753338424965  0.9728358302537781  0.9998521037481815  0.9999794506520562
9.84375     0.33804628853060076  0.46391728550983746  0.8235938144025687  0.513902388205025    0.9999999949410445  0.9998673918968601  0.8494986432991352  0.9999999896703751  0.9948891974791929  0.9157035040041289  0.9995938288397359  0.9999995061455016  0.9999795960782039  0.8023530970933797  0.9999415010493742  0.5098510864088555   0.9912333971464101  0.5139572263769641   0.45528494117264295  0.9976527997087922  0.9999999950785695  0.32497096417356797  0.5427840143822578   0.7781619219166328  0.9999414060412422  0.4157851071856703   0.9999338893355602  0.9100731680613178  0.7939663031376476  0.9982311146603375  0.9912221451635274  0.9166886907607648  0.46993962303308984  0.45205532370215856  0.99999951162645    0.9999793896610141  0.9727123618700023  0.9999999898319789  0.9999791684836375  0.9995842080513203  0.973069506752468   0.999857922535364   0.9999795730998703
9.9609375   0.34557378643878817  0.4682862469947703   0.82650790223282    0.510315600624134    0.9999999948574343  0.9998648434735273  0.8443908461881545  0.9999999897746952  0.9949313461024982  0.9166407264786921  0.9996009261595102  0.999999511113266   0.9999798179741847  0.8057063079985959  0.9999413380209224  0.5103380064422557   0.9912722877524239  0.5200846695215634   0.4548539169439545   0.9975413043758523  0.9999999950013779  0.32666640326204865  0.5564860356018824   0.7840706723126288  0.9999418647813549  0.42449626051910005  0.9999326323291114  0.9328239011719598  0.8037663688101917  0.9983983384942077  0.9913161494029415  0.9171571201273463  0.4685107774336903   0.46388822665172497  0.9999995165909285  0.9999795748440365  0.9731818776311945  0.9999999899363952  0.9999794100738248  0.9995942551289931  0.9732334525712426  0.9998638089376455  0.9999797968794057
10.078125   0.3531053360527324   0.4721045504775836   0.82789573010306    0.5086688624559145   0.9999999947756503  0.9998603631873266  0.846508985541937   0.9999999898652187  0.9949792117351905  0.9175753221569934  0.9996043564065359  0.9999995162016329  0.9999801349342979  0.8078237865217047  0.9999415342562197  0.5151950745558832   0.9913205414514029  0.5261449022661472   0.4545533884110565   0.9976452379750317  0.9999999949256688  0.32897176320707305  0.5691287938438239   0.7911833927070402  0.9999427131256393  0.4336275556340223   0.9999310777984672  0.9417250997710015  0.8147913603902961  0.998362338443399   0.9914013532919904  0.9175444603020079  0.46712173849347166  0.4755944037974135   0.9999995216734454  0.9999797561669412  0.9736403643768107  0.9999999900288875  0.9999796550030559  0.9996038449916961  0.9733748488249737  0.9998659594436741  0.9999801239850642
10.1953125  0.3606351559556115   0.47545367095524027  0.8282718858967407  0.5093834308557627   0.9999999947017528  0.9998573324266126  0.8566892588222393  0.9999999899576115  0.9950335936199326  0.9184696801277065  0.9996059008794149  0.9999995213768489  0.9999805080482277  0.8089668732403369  0.9999422490130658  0.5242275174046158   0.9913881143966246  0.5321236145615975   0.45452829551714474  0.9979180498201012  0.9999999948574698  0.33185171323572205  0.5807706139457706   0.7991573604917895  0.9999438363613284  0.4430923548416874   0.9999302844133062  0.9357072998829017  0.8263522176959308  0.998166307028975   0.9914740505316857  0.9179157188969715  0.46595841301087443  0.4869893106276349   0.9999995268422291  0.999979973352547   0.9740412678663312  0.9999999901229186  0.9999798789272208  0.9996107681638964  0.9735427437788222  0.999862224004718   0.9999805095625015
10.3125     0.3681575130827875   0.4784197651888312   0.8282672705847937  0.5127156468279312   0.9999999946430937  0.9998563465111687  0.8731188107761     0.9999999900496063  0.9950870763080563  0.9192902395167374  0.9996074047247875  0.9999995265994187  0.9999808739429562  0.80951543736995    0.9999435177322062  0.5370502378669124   0.991478260879725   0.5380160209994562   0.454919436121525    0.998234385652225   0.9999999948036875  0.3352677629756125   0.5914729623242625   0.8075310895662687  0.9999450541269626  0.452800251193875    0.9999304058030187  0.9167141022400438  0.8376481168073999  0.997931025305975   0.9915361923366001  0.9183327457970937  0.46518883545190626  0.49790617812789373  0.9999995320575688  0.99998023345745    0.9743523618600062  0.9999999902176563  0.9999800849700938  0.9996135478997312  0.9737739363572501  0.9998541209132188  0.9999808812263187
10.4296875  0.37566676704669794  0.48272757524381643  0.8285305287398098  0.5187408808637414   0.9999999946071461  0.9998554531391921  0.8918230177005533  0.9999999901355471  0.9951336048516042  0.9200165856151966  0.9996096934952922  0.9999995319613267  0.9999811677419531  0.8099133957423275  0.999945192479571   0.5530986520398271   0.9915871779623454  0.543818050553247    0.4558588816045702   0.99845324014047    0.9999999947716378  0.33917983536880736  0.6012984090711831   0.8157805435339979  0.9999462260885236  0.4626593461176022   0.9999305832221222  0.8892201654465939  0.8478707164877789  0.9977924735865275  0.9915952443669296  0.9188442479186185  0.46495926516052005  0.5081997117118515   0.9999995374016891  0.9999805067419975  0.9745654721280413  0.9999999903065364  0.9999803037509238  0.9996124227753158  0.9740829067356932  0.9998461614306898  0.9999811715999555
10.546875   0.3831574052278121   0.4890336001302873   0.8296234564502539  0.5273627290609589   0.999999994598808   0.9998525263962714  0.9078094208512099  0.9999999902311865  0.9951761769154595  0.9206476420627723  0.9996123098849645  0.9999995375210954  0.9999813499183869  0.8106044546585928  0.9999469747267432  0.5716539271652306   0.9917061180047747  0.5495180344342255   0.45746561040262457  0.998487009550179   0.9999999947659454  0.343547506514583    0.6103090732804334   0.8233850327620157  0.9999472891131351  0.4725783096183317   0.9999300829758475  0.859051916015239   0.8563142349250635  0.9978335717024586  0.9916617506989528  0.9194787375549407  0.4653913796364182   0.5177487226055403   0.9999995429266205  0.9999807645501761  0.9746990711009684  0.999999990403424   0.999980554488723   0.9996093705394552  0.9744586212648616  0.9998429469139208  0.9999813465171861
10.6640625  0.39062406723680854  0.49497720610899143  0.8319275934469372  0.5383433960636942   0.9999999946186657  0.9998477349885019  0.9165016853661729  0.9999999903284421  0.9952235573280144  0.9212036247120933  0.9996141706525016  0.9999995431755203  0.9999814260317075  0.8119670076294712  0.9999485350672978  0.5918855065810913   0.991825071893846   0.5550994317259669   0.4598415565575423   0.9983424391006237  0.9999999947872641  0.3483309432584138   0.6185654908343973   0.8298938717853052  0.9999482149560973  0.4824682145322211   0.9999292273545366  0.8318788586359788  0.862473866359633   0.9980438957897366  0.9917455387630826  0.9202416665946245  0.46658055588853004  0.5264577150543533   0.9999995485342359  0.9999810074295895  0.9747930825140779  0.9999999905021563  0.9999808197373206  0.9996071368323164  0.9748689577229995  0.999846191600583   0.9999814169145841
10.78125    0.3980615658752766   0.500591646151343    0.8355784027214821  0.5513453762720149   0.9999999946640094  0.9998440372318446  0.914954593077157   0.9999999904250961  0.9952800068461227  0.9217233333637767  0.9996146466925188  0.9999995488340906  0.999981446786175   0.8142577379052742  0.9999496433656445  0.6129030175375728   0.9919365811928     0.5605516891008195   0.4630681400791969   0.9981140763713751  0.9999999948329094  0.35349157697454064  0.6261258363833563   0.8349843069774195  0.9999489528594969  0.49224414128241173  0.9999290789237696  0.8119026786165328  0.8661160026787657  0.9983266505443859  0.9918520200203242  0.9211171246725578  0.4685951616285516   0.5342575243225102   0.9999995541382461  0.9999812490378907  0.9748975282007476  0.9999999905992141  0.9999810672572556  0.9996078438858492  0.9752712441353906  0.999853847281382   0.9999814334982329
10.8984375  0.4054649116577614   0.5059114747203608   0.8404393631801339  0.5659717012421668   0.9999999947308871  0.9998452328421489  0.9024804097971045  0.9999999905338576  0.9953404428750064  0.9222573496938575  0.9996143104808414  0.999999554546202   0.9999814859433951  0.8175725274031596  0.999950227701826   0.6338060400486661   0.992038246764486   0.5658765144382896   0.46720326273709284  0.9979346673461686  0.9999999948989183  0.35899255538447195  0.633045432386658    0.8385021790448551  0.999949429171253   0.5018265520398016   0.9999302209218679  0.8011386360850237  0.8673079179444614  0.9985484259161443  0.9919799962785838  0.9220735577064112  0.47147672316240985  0.5411051471036168   0.9999995597830019  0.999981484370372   0.9750584166949978  0.9999999907066875  0.9999812882912702  0.999612001406177   0.9756256436407353  0.9998618798311705  0.9999814686144427
11.015625   0.4128293402841108   0.5109720268754034   0.8461217689956488  0.581796258907547    0.9999999948163896  0.9998530506182229  0.8806302982921923  0.9999999906422374  0.9953968115904912  0.9228585681145224  0.9996148295283735  0.9999995603395659  0.9999816081054962  0.8218309046939423  0.9999503443024003  0.6537276430936005   0.992133049448293   0.5710823755142633   0.47227876866951896  0.997908025123317   0.9999999949824658  0.3647990143722188   0.6393764852364171   0.8404791187683003  0.9999496119770153  0.5111424353829045   0.999932045635434   0.7993585494815376  0.8664011045097292  0.9986062418110441  0.9921218015806381  0.9230721591552986  0.4752408045914877   0.54698292267229     0.9999995654938021  0.9999816909259495  0.9753050101369787  0.999999990814027   0.9999815011746074  0.9996184776679501  0.9759073558806315  0.9998672518186713  0.999981589793329
11.1328125  0.42015033658443945  0.5158089534663279   0.8520481154232519  0.5983839593474917   0.999999994919069   0.9998655331884879  0.8526950663185271  0.9999999907520374  0.9954477679045776  0.9235720334925428  0.9996180517911444  0.9999995661177241  0.9999818434797872  0.826786869123248   0.999950112027876   0.6718754816658141   0.9922274667776745  0.5761741208910424   0.4782984725450853   0.9980579290582848  0.99999999508195    0.37087820902767815  0.6451679970662653   0.8411234184652422  0.9999495775838313  0.5201262265962608   0.9999335208190201  0.804527506820942   0.8639702504914505  0.9984783908721866  0.9922657268140611  0.9240760751788639  0.4798784275553245   0.5518972158098251   0.9999995711814245  0.9999818622386778  0.9756421221940814  0.9999999909213378  0.9999817190814376  0.9996253900153141  0.9761140445666178  0.9998698300991959  0.9999818320449527
11.25       0.4270450647606      0.5204578057355      0.8575488844593     0.6153058454842      0.9999999950377     0.9998780048216     0.8229341798511     0.9999999908712     0.9955000668839     0.9244261875071     0.9996248607945     0.9999995718246     0.999982179114      0.8320649800153     0.9999496780729     0.6875725599588      0.9923282184918     0.5811502675597      0.4852369814511      0.9983179129773     0.9999999951964     0.3771995382932      0.650465809099       0.840785615958      0.9999495071196     0.5287205143854      0.9999344310416     0.8135179751051     0.8607177768343     0.9982342887518     0.9923998247594     0.925058467213      0.4853578807972      0.5558767297925      0.9999995767933     0.9999820254055     0.9760495629378     0.9999999910373     0.9999819298125     0.9996313206148     0.9762663858677     0.9998718645354     0.9999821812898
11.3671875  0.4315390271044953   0.5249536621736002   0.8619758892764814  0.6321533489311687   0.9999999951701503  0.9998859501399898  0.7957218878303954  0.9999999909855561  0.9955601158338697  0.9254273099705453  0.9996345349574386  0.9999995775323002  0.9999825657759802  0.8372166133267618  0.9999492290141514  0.7002913882785902   0.9924389771260933  0.5860109918030328   0.49303959882067616  0.9985660517253715  0.9999999953233627  0.38373449181916347  0.6553127384691323   0.8399037585313256  0.9999496031769248  0.5368765501875709   0.9999355811662027  0.8229144455974695  0.8573604482897621  0.9979992757057938  0.9925156933159384  0.9260077863305762  0.49162680522158064  0.5589705469209497   0.999999582395606   0.9999822151212202  0.9764887255477619  0.999999991149175   0.9999821161994049  0.9996360961909594  0.9764017036725336  0.9998756105952292  0.9999825798583436
11.484375   0.43604675641008817  0.5293307898491459   0.8648118395343367  0.6485515543765181   0.999999995312602   0.9998877067555328  0.7747877410112753  0.9999999911004105  0.9956259406253836  0.9265582527040731  0.9996450321300249  0.9999995832766929  0.9999829358428178  0.8417878831471597  0.9999489998195187  0.7096712201728109   0.992558356645447   0.5907667892249142   0.5016235530317115   0.998686562238741   0.9999999954594284  0.39045654207085123  0.6597487766231638   0.8389366721784894  0.9999499886773868  0.544554579996711    0.9999377832489067  0.8297418924893173  0.8545181421856509  0.9978925932214429  0.9926108683040771  0.9269292291017935  0.4986144802347906   0.5612459610150072   0.9999995880213443  0.9999824366446105  0.9769142332885071  0.9999999912607216  0.9999822869676489  0.9996406806460593  0.9765626966538937  0.9998810859633696  0.9999829526143356
11.6015625  0.4405716995017987   0.5336223333719875   0.865756620228937   0.6641685466827971   0.999999995460411   0.9998850616093091  0.7626894080409381  0.9999999912192808  0.9956893420352285  0.9277816606381294  0.9996540417634014  0.999999588974698   0.9999832278778465  0.8453886009676925  0.9999492267648605  0.7155142371225112   0.9926799678948429  0.5954369452352419   0.5108806514574784   0.9986272195933108  0.9999999956001219  0.39734099987861904  0.6638113242549676   0.8382956727421659  0.9999506647459702  0.5517240202383259   0.9999407937288867  0.8319873527274334  0.8526242067105749  0.9979695694293427  0.9926890066385725  0.9278421413983083  0.506234271010149    0.5627861361902939   0.9999995935950465  0.9999826654771349  0.9772871389801637  0.9999999913765435  0.9999824726927159  0.9996463083953118  0.9767846725164099  0.9998857820566048  0.9999832389501496
11.71875    0.4451170899677914   0.5387162273062734   0.8647757712462539  0.6787189650962695   0.9999999956098579  0.9998814606787406  0.760563349841118   0.9999999913288211  0.9957454613260508  0.9290469602530789  0.9996601723555586  0.999999594579925   0.9999834110434171  0.8477513528407337  0.9999500465349165  0.7177676937884852   0.9927945168359664  0.6000392772255116   0.5206812739037219   0.9984243423081055  0.9999999957418344  0.40436484753096325  0.667535444354146    0.8382854867478     0.9999515495244664  0.5583634993381984   0.9999435680004555  0.8288639271005516  0.8518720627143852  0.9981962932859718  0.9927578887632891  0.928773994497839   0.514386218165218    0.5636876055541782   0.9999995990739039  0.9999828816737367  0.9775857973034258  0.9999999914843516  0.9999826903290921  0.9996534632672273  0.977085085852093   0.9998865669135336  0.9999834144739937
11.8359375  0.4496859551596532   0.5471393526026178   0.8621043192324686  0.6919630435615025   0.999999995758275   0.9998793245289311  0.7681086932254119  0.9999999914379093  0.9957981367266384  0.9302996554717164  0.9996635673654338  0.999999600150553   0.9999834983630226  0.8487714499426513  0.9999514078914133  0.7165051841789951   0.9928931342398964  0.6045830805512512   0.5308794445986117   0.9981837053823008  0.9999999958821172  0.41150656004880987  0.6709541201493182   0.8390635537562762  0.9999525495054775  0.5644607849412486   0.9999454910106167  0.8208472215554852  0.8522052293073031  0.9984684543951655  0.9928260853531156  0.9297523161020722  0.5229597609590152   0.5640576160096996   0.9999996045110467  0.999983091240257   0.9778113012068084  0.999999991591189   0.9999829236851911  0.9996613860677123  0.9774584333028506  0.999882275674047   0.9999834961270264
11.953125   0.4542811188548823   0.5553271530094064   0.8582074308571105  0.7037044925733636   0.9999999959041197  0.9998785459628428  0.783734838627265   0.9999999915460553  0.9958547221107336  0.9314910358573976  0.9996656080036803  0.9999996056973994  0.9999835403794708  0.8485217866013371  0.9999530697417275  0.7119167012910685   0.99297066237431    0.6090731590287921   0.5413185932929294   0.9980276979562539  0.9999999960195697  0.4187459225122967   0.6740985081320152   0.8406236948521983  0.9999535955150393  0.5700126132388973   0.9999469870614153  0.8095403783307441  0.8533497679022374  0.998664062698753   0.9928996072908349  0.9307964237982461  0.5318365831325181   0.5640113318627681   0.9999996099174913  0.9999833078985063  0.9779860394284289  0.9999999916979564  0.9999831452693188  0.9996684352992141  0.9778779300496574  0.9998749408732602  0.999983533668199
12.0703125  0.4589052017938507   0.5632532781416024   0.8537066108025254  0.7137900483650548   0.9999999960458713  0.9998772235626002  0.8048360040171939  0.9999999916433759  0.995916689802899   0.9325863447658612  0.9996679293398149  0.999999611125441   0.9999836019168409  0.84724081830352    0.9999546992249865  0.7043076995640285   0.9930276559756297  0.6135192592219495   0.5518375476810764   0.9980354241353027  0.999999996152747   0.42606384927457114  0.6769981787725416   0.8428060304824644  0.9999546219708602  0.5750244317820179   0.9999488858010988  0.7974012338302846  0.8548809468768164  0.9987033660705904  0.9929798186914561  0.931910846441924   0.5408935589792674   0.5636689260860789   0.9999996152075783  0.9999835246952851  0.9781459871382814  0.999999991794874   0.9999833486630249  0.9996730685442288  0.9783033700573143  0.9998686675421131  0.9999835918219301
12.1875     0.4635606220127562   0.5708939539462374   0.8492852560594812  0.7221113417295812   0.9999999961813938  0.9998738921980937  0.8281853391659688  0.9999999917411     0.995977512240375   0.9335697922044937  0.9996713881932938  0.9999996163796813  0.9999837349226     0.84529617293605    0.9999560084660125  0.6940994145099      0.9930703577761437  0.6179387068507876   0.5622762979107188   0.9982053190905062  0.9999999962798499  0.43344220961173124  0.6796813399845062   0.8453307835770062  0.9999555315139188  0.5795100616538      0.9999513281949812  0.787345976933625   0.8563108301903001  0.9985863862480437  0.993063366625225   0.9330819358072937  0.5500057589269564   0.5631526078818687   0.9999996203311312  0.9999837179056937  0.9783291752305625  0.9999999918916438  0.9999835482557375  0.9996748143698     0.9786927622319312  0.9998669684073438  0.9999837263398312
12.3046875  0.46824959619731743  0.5782281604418553   0.8455896896480639  0.7286080592734524   0.9999999963078072  0.9998693538282105  0.8504274834908039  0.9999999918359438  0.9960303442585845  0.9344454746058402  0.9996756240604134  0.9999996215001988  0.9999839601865766  0.8431301404843913  0.9999568453709545  0.6818215569771638   0.9931086787839126  0.622348443293531    0.5724811456345343   0.9984581117270014  0.999999996398126   0.4408636636743858   0.682175041062226    0.8478499029161619  0.999956203754943   0.5834912801334379   0.9999535520183795  0.7822603857900053  0.8571810981343716  0.9983902086478254  0.9931440774487791  0.9342784820644174  0.5590494538633061   0.5625836529364044   0.9999996253250308  0.9999838788415384  0.9785638597715275  0.9999999919864475  0.9999837504222797  0.999674614649388   0.9790142669485609  0.9998705366960822  0.9999839608347835
12.421875   0.47297414231699014  0.585237797381098    0.8431411654322556  0.7332715679136044   0.9999999964222681  0.999866568397037   0.868609630357881   0.9999999919224571  0.9960760440275633  0.9352341469818841  0.999679456934582   0.9999996264871094  0.9999842625356974  0.84119588691819    0.9999571998340481  0.6680939256152225   0.9931529526107502  0.6267556573427467   0.5823089900143673   0.9986777195342951  0.9999999965050492  0.44831151225568333  0.6845053568503738   0.8500078491575539  0.9999565605022235  0.5869973215118703   0.9999548282022025  0.7844829718842121  0.8571458896474736  0.9982283900384377  0.9932160024472497  0.9354562975061448  0.5679050417030455   0.5620795150320211   0.9999996301912665  0.999984029175979   0.9788599271732873  0.999999992072999   0.9999839398443409  0.9996743201126832  0.9792549344781661  0.9998770698255778  0.9999842759278293
12.5390625  0.4777360831685986   0.5919078377949758   0.8422711599567176  0.7361496956018486   0.9999999965227155  0.9998686244833601  0.8806707774340023  0.9999999920129655  0.9961218835498928  0.935966689909845   0.999681828125921   0.999999631256196   0.9999845978104249  0.8398943120703705  0.9999571561038224  0.6536003099156443   0.9932106664493433  0.6311579362909087   0.5916306825277304   0.9987690640806012  0.9999999965989592  0.45576956355304743  0.686697552389801    0.8515017913699855  0.9999566367369541  0.590064291632616    0.9999552691002661  0.7953364099305841  0.856032438366638   0.9981934429435655  0.9932763622957097  0.9365658736297386  0.5764598130155533   0.5617511027103941   0.9999996348552601  0.9999841977546737  0.979206037025918   0.9999999921626453  0.9999841017173904  0.9996756554955502  0.9794237566734919  0.9998831781937212  0.9999846186281647
12.65625    0.48253705047605394  0.5982264687608718   0.8430875088371852  0.7373541441352516   0.99999999660905    0.9998763389749368  0.8858174729664391  0.999999992101482   0.996172551810939   0.9366758219395548  0.9996826534193609  0.9999996357706641  0.9999849092280422  0.8395208851081609  0.9999568509469493  0.6390606466957093   0.9932843447396508  0.6355522759978204   0.6003335498307821   0.9987039659004594  0.9999999966794983  0.46322201957025705  0.6887762273538024   0.8521319983903063  0.9999565896974226  0.5927344909558375   0.9999556013155688  0.8147716529781844  0.8538718445820469  0.9983111983692406  0.9933272018986875  0.9375615255830048  0.5846104763269555   0.5617003014732812   0.999999639283032   0.9999843894262922  0.979573248419679   0.999999992250575   0.9999842483915649  0.9996792958835821  0.9795482683276461  0.9998868543784672  0.9999849274956
12.7734375  0.48737848939680667  0.6041852183884522   0.8454737729371113  0.7370703258589922   0.9999999966821252  0.9998873491202885  0.8847208646022099  0.9999999921865017  0.9962244488109496  0.93738796531409    0.9996830283268757  0.9999996400826585  0.9999851499207085  0.8402297570660778  0.9999564647263834  0.6252044801434645   0.9933712950560554  0.6399416296200914   0.6083232868909767   0.9985328308923128  0.9999999967478639  0.4706533841443717   0.69076543824383     0.8518351799103401  0.9999566261247498  0.5950556422496748   0.9999562705036017  0.8411872903655985  0.8508973718736043  0.9985293181387831  0.9933750876257591  0.9384100937973483  0.5922653764171514   0.562017810188164    0.9999996435220635  0.9999845860611049  0.9799239309322926  0.9999999923346078  0.9999844111652748  0.9996846120224925  0.9796658521055132  0.9998886283909755  0.9999851592279151
12.890625   0.492261663057035    0.6097790676003401   0.849119543962292   0.7355656776652599   0.9999999967433855  0.9998974329866679  0.8794560411329239  0.99999999227935    0.9962702694937068  0.9381170945390912  0.9996846479836624  0.9999996442115469  0.9999853035646463  0.8420194687871181  0.9999562182303574  0.6127442145054958   0.9934652059043203  0.6443309462168065   0.6155254335815182   0.9983572261430603  0.999999996805373   0.4780483939376106   0.6926887956120159   0.8506964640923361  0.9999568965674741  0.5970800225515284   0.999956921847578   0.8715088618202059  0.8475123104947193  0.9987448658908111  0.9934289465853497  0.9390973683314962  0.5993463568881772   0.5627813413339373   0.9999996475924823  0.9999847747215999  0.9802232097409592  0.9999999924249375  0.9999846054314627  0.9996902106366594  0.9798124390444868  0.9998906263153179  0.999985303964176
13.0078125  0.49718765613138066  0.6150065445645052   0.8535756492442125  0.7331894937582951   0.9999999967941     0.9999031124788448  0.8731095334277429  0.9999999923713188  0.9963081310900278  0.9388619022305185  0.9996888051051125  0.9999996481082181  0.9999853916872626  0.8447412407749555  0.9999563309204436  0.6023453625195982   0.9935588943919549  0.6487177328039374   0.6218865792053287   0.9982778479553334  0.9999999968534942  0.48539197323272604  0.6945695337091943   0.848939017221427   0.9999574323620665  0.5988635044529064   0.9999569158387803  0.9016121839758993  0.8442332748180789  0.9988564531553917  0.9934968657308073  0.9396308960047766  0.6057902378988776   0.5640542117998276   0.9999996514515367  0.9999849619972685  0.9804494891724407  0.999999992514463   0.9999848141610207  0.9996948955439792  0.9800120061766435  0.9998944028056072  0.9999853857997302
13.125      0.5021573761136      0.619869798569       0.8583255199989     0.7303582280311      0.9999999968349     0.9999036518004     0.869060781406      0.9999999924638     0.9963440875708     0.939606828517      0.9996955923324     0.9999996517753     0.9999854619978     0.8481276554514     0.9999569353793     0.5945940117985      0.9936471015996     0.6530889890504      0.6273753338997      0.998344192275      0.9999999968929     0.4926692128877      0.6964305503283      0.8468935075971     0.9999581623219     0.6004645177877      0.9999561561066     0.9270906457879     0.8416176214286     0.9988146622211     0.993583066407      0.9400385951506     0.611549898319       0.5658843291469      0.9999996551012     0.9999851560503     0.9806009347837     0.9999999926033     0.999985010295      0.9996984236482     0.9802699584895     0.9998993438116     0.9999854515167
13.2421875  0.507171551139542    0.6243746487772627   0.8628624405001603  0.7275255706481939   0.9999999968658871  0.999901064970153   0.8700842600452202  0.99999999256445    0.9963849902985683  0.9403266129988864  0.9997038011164674  0.9999996553022248  0.9999855643228249  0.8518368763893016  0.9999579988854077  0.589965183741049    0.9937282670449911  0.6574286024521867   0.6319830132110453   0.9985307926469967  0.999999996923793   0.49986537321614116  0.6982944147196823   0.8449523130841864  0.9999589797411014  0.601942948374714    0.9999552113403484  0.9442164826154188  0.8401853322827697  0.9986458692292159  0.9936862006956874  0.9403634856544293  0.616594966590398    0.5683035619423513   0.9999996586236312  0.9999853453999207  0.9806960514773554  0.9999999926991625  0.9999851872927066  0.9997015801347933  0.9805722586672836  0.9999029384271104  0.9999855515183885
13.359375   0.5122307236996768   0.628530601597649    0.8667621281277282  0.7251430137883249   0.999999996887677   0.9998982722297615  0.8775270828049523  0.999999992662725   0.9964304043710233  0.9409932461941898  0.9997115711578275  0.9999996587431109  0.9999857281425942  0.8555056462144968  0.9999593177226085  0.5887953534353965   0.9938046534681627  0.661726566785211    0.6357239342432964   0.9987512039326233  0.9999999969464771  0.5069659099505437   0.7001833415664895   0.8435152938339027  0.9999597970862946  0.6033589951790943   0.9999546947011152  0.950831837081637   0.8403460344702502  0.9984382540588229  0.9937996120188646  0.9406556739359675  0.6209121349089036   0.571327473695196    0.9999996620713304  0.9999855083437665  0.9807682825211772  0.9999999927927513  0.9999853589320201  0.9997055454744748  0.9808904178693031  0.999902716531726   0.9999857180818039
13.4765625  0.5173352397913953   0.6323508284243226   0.8697411168864753  0.7236196930259127   0.9999999969014547  0.9998969077923335  0.8908228325040817  0.9999999927616064  0.9964735503960022  0.941583684853326   0.9997173834176925  0.9999996620785369  0.999985952064535   0.8588033636389998  0.9999606031601945  0.5912588907540641   0.9938808597978567  0.6659793307189514   0.6386352215545985   0.9989020135135864  0.9999999969620812  0.5139565220118245   0.7021191299186664   0.8429348935587094  0.9999605523133823  0.604772010918396    0.9999545932417265  0.9469000926718665  0.8423408109577588  0.9982976728465064  0.9939134628178223  0.9409633190452419  0.6245051153847585   0.5749553974947391   0.9999996654282465  0.999985642642899   0.9808563279818238  0.999999992885844   0.9999855334742688  0.9997109832709316  0.9811907155830379  0.9998983275670359  0.9999859498513427
13.59375    0.5224852329792078   0.6358520939865242   0.8716934978090484  0.7232889301586969   0.9999999969085249  0.9998964491810891  0.9075233315026265  0.9999999928650555  0.9965100118529071  0.94208655089935    0.9997207941873704  0.9999996653318961  0.9999862066340852  0.8614796612651157  0.9999616089947414  0.5973467184285882   0.9939613922551547  0.6701816149292452   0.6407760509885188   0.9989144965126515  0.9999999969718711  0.520823219589246    0.7041230679222796   0.8434681950815821  0.9999611859894328  0.6062393532744719   0.999954373928232   0.9345300693733931  0.8462063443095523  0.9982957502841461  0.9940179409217624  0.9413244854510867  0.6273942551518039   0.5791708301801015   0.999999668714325   0.9999857744180117  0.980993113058725   0.9999999929831305  0.9999857011382203  0.9997174536162242  0.981444793358068   0.9998921170115547  0.9999862120138898
13.7109375  0.5276806030572749   0.6390546254088915   0.872701329285662   0.724385645613374    0.9999999969099407  0.9998952245685934  0.9238975614828738  0.9999999929619422  0.9965434305264744  0.9425062800753635  0.9997224917855895  0.999999668596022   0.9999864467806886  0.8633990290538683  0.999962220482713   0.606848889621064    0.994048316439148   0.6743205244248888   0.6422262967676146   0.9987882387253988  0.9999999969765783  0.5275524107802102   0.7062158061945658   0.8452423991119283  0.9999616362103914  0.6078152741413152   0.9999537334765024  0.9174180990966889  0.8517661272232189  0.9984358766230361  0.994106371149209   0.941761521447691   0.6296158221892793   0.5839421297018494   0.9999996720141032  0.9999859316516938  0.9811966725114121  0.9999999930746704  0.9999858518383441  0.9997235449998222  0.9816384181930733  0.9998877142591829  0.999986453553505
13.828125   0.532920990172469    0.6419819143199066   0.8730171803934107  0.7270361232867645   0.9999999969053689  0.9998923580547651  0.9359595370754452  0.999999993057175   0.9965812562820738  0.9428637663380813  0.999723693823185   0.9999996719015509  0.9999866299150098  0.8645579340286944  0.9999624595132526  0.6193475781543392   0.9941399554683166  0.678379857354795    0.6430845796813739   0.9985904964362589  0.9999999969758875  0.5341310045060286   0.70841720350334     0.8482382440923752  0.9999618729071601  0.6095498728427349   0.9999530096004948  0.8998381786137365  0.8586502713181304  0.9986533919243719  0.994177116802555   0.9422789909331512  0.6312209697344114   0.5892235056274714   0.9999996753557929  0.9999861150464434  0.9814655784399747  0.9999999931640436  0.999985995564721   0.9997276582603406  0.9817758221232386  0.9998876668105227  0.999986631724046
13.9453125  0.5382057469914171   0.6446604486599782   0.8730206939209405  0.7312596191286918   0.9999999968933448  0.9998890011371511  0.9405973690885436  0.9999999931517332  0.9966257581945346  0.9431933319978298  0.9997253100039778  0.9999996751892073  0.9999867346971896  0.8650822855630754  0.9999624279020514  0.6342277876219069   0.9942311644982145  0.6823498591763955   0.6434657419573904   0.9984222987959154  0.9999999969683538  0.5405465267724819   0.710746148165224    0.8522934498253018  0.999961942900614   0.6114881362678182   0.9999528312309619  0.8855010040917657  0.8663419767379943  0.9988490397130634  0.994233628696972   0.9428653974909217  0.632274387144644    0.5949562980138792   0.9999996786825125  0.9999863022359143  0.9817801368835241  0.9999999932536261  0.9999861539337831  0.9997289619033972  0.9818783941568661  0.9998918664799759  0.9999867301960247
14.0625     0.5435339125046375   0.6471193755747875   0.8731546464637313  0.7369793605416125   0.9999999968728626  0.9998878021190875  0.9364183218131438  0.9999999932361999  0.9966714127388437  0.943536667366625   0.9997274186269688  0.9999996784449438  0.999986772418475   0.8652057170385125  0.9999622494706187  0.6507094082057188   0.9943150802566312  0.6862309293857625   0.6434978015770187   0.9983686534183313  0.9999999969525563  0.5467872470228      0.7132203587880125   0.85712563857575    0.9999619748116688  0.6136690846800813   0.9999534022444938  0.8766657437692874  0.8742451041478688  0.9989384300950375  0.9942827041439125  0.9434981342407063  0.6328526490436313   0.60107053992655     0.9999996819774437  0.999986476794225   0.9821088462958875  0.9999999933346437  0.9999863328544375  0.9997279293567     0.981978206720675   0.99989788555765    0.9999867639519812
14.1796875  0.548904190493104    0.6493901000610506   0.8738493794170028  0.7440391228417282   0.9999999968440606  0.9998909846077116  0.9240601796430753  0.9999999933173063  0.9967118592165279  0.9439350474149091  0.9997293758117239  0.9999996817252982  0.9999867835355151  0.8652317537348666  0.9999620515045586  0.6678998606189583   0.9943856641223983  0.6900272060492355   0.6433184614872527   0.9984576539630832  0.99999999692875    0.5528423115418886   0.7158561694920949   0.8623716045802636  0.9999621232678898  0.6161250371823481   0.9999542689654481  0.873758764269366   0.8817638830332692  0.9988927958696517  0.9943316832433957  0.9441504204108194  0.6330422814656984   0.6074867914930039   0.9999996852898221  0.999986642614282   0.982418068661197   0.9999999934127106  0.9999865138755689  0.999726128737344   0.9821083886159215  0.9999028835493828  0.9999867725627344
14.296875   0.5543149341340314   0.6528468109623409   0.875447011080953   0.7522217164258896   0.9999999968092373  0.9998985218390627  0.9059418078382877  0.9999999933943742  0.9967477247150488  0.9444215002019581  0.9997304509791047  0.9999996850261867  0.9999868198104576  0.8654852435458575  0.9999619731583307  0.6848631927712132   0.994440049384348   0.6937383133329691   0.643071254425072    0.9986483533846566  0.9999999968989186  0.5587018813884977   0.7186683065704322   0.8676375142757741  0.9999624869863641  0.6188810067261177   0.9999549198178144  0.8755343064198098  0.8883832631660477  0.9987512164598186  0.9943856805785535  0.9447985914468805  0.6329375736310223   0.61411822029214     0.9999996886149243  0.9999868078639323  0.9826819655121339  0.9999999934883615  0.9999866767222176  0.9997253955310199  0.9822932607070913  0.9999055895509085  0.9999868061876149
14.4140625  0.5597641369400949   0.6572471263358496   0.8781377891795152  0.7612670173673161   0.9999999967715111  0.9999078426461906  0.885597334921337   0.9999999934600516  0.996785742191972   0.9450145929906555  0.999730537322911   0.9999996882776623  0.999986921902147   0.8662601240846843  0.999962157173021   0.7006973347691483   0.9944797817385961  0.6973596153563607   0.642901382931641    0.9988526705990775  0.9999999968659736  0.5643572727708481   0.7216696639159539   0.8725529319664533  0.999963062142908   0.6219542320089471   0.9999554071002899  0.879643505662777   0.8937374972749552  0.9985990256297707  0.9944458932874787  0.9454280906632996  0.6326381778310164   0.6208728879933205   0.9999996918864138  0.9999869658400752  0.9828897485909812  0.9999999935544345  0.999986822989026   0.9997268887766777  0.982541172852324   0.9999068829412464  0.9999869075215111
14.53125    0.5652494297644322   0.6618903825434063   0.8819198504654547  0.7708883691331578   0.9999999967340015  0.9999153419169727  0.8668202588635757  0.9999999935228734  0.9968302981194133  0.9457151257654258  0.9997304415182274  0.9999996914856907  0.999987104009382   0.8677713678091438  0.9999627017217328  0.7146064538210609   0.9945104728577476  0.700890592056086    0.6429512768003945   0.9989797458395703  0.9999999968328696  0.5698010979607414   0.7248710832041094   0.8768208072353085  0.9999637641925633  0.6253538525751194   0.9999562499670281  0.883399910192      0.8976556781832118  0.9985239209676289  0.9945095778309805  0.9460368584776024  0.6322465559267836   0.6276561897060648   0.9999996951069641  0.9999871024188992  0.9830481684534601  0.9999999936182765  0.9999869700588359  0.9997305855304813  0.9828421294919898  0.9999086365986859  0.9999870942405484
14.6484375  0.5696872568950568   0.6674062787945935   0.8865904869332251  0.78078674691818     0.9999999966987686  0.9999186042237356  0.8528536731423676  0.9999999935798607  0.9968778799676565  0.9465063737059044  0.9997315358892633  0.9999996947315705  0.9999873505329185  0.8701185732567859  0.9999635986759144  0.7259546435759604   0.9945399988191994  0.704340770136852    0.643355869097996    0.9989803957355524  0.999999996801536   0.5750274048516211   0.7282811430459303   0.8802559513022498  0.9999644936982621  0.6290807328618273   0.9999577705377514  0.8845199347638409  0.9001761456706505  0.9985720247868817  0.9945716299329587  0.9466354252203965  0.6318653453172517   0.6343733812630666   0.9999996983501969  0.9999872198296491  0.9831786349472434  0.9999999936778486  0.9999871255569659  0.9997354820243233  0.9831708341323283  0.9999118432310748  0.9999873499004535
14.765625   0.5738443261346335   0.6727536827922741   0.8917721147579376  0.7906625574108878   0.9999999966674284  0.9999177325718575  0.8458068063202029  0.9999999936265624  0.9969213159605773  0.9473577541656573  0.9997349746951977  0.999999698032773   0.9999876228126439  0.8732671319340243  0.9999647118224245  0.7342939347344287   0.9945759267000959  0.7077255941189481   0.6442376239642881   0.998869807897571   0.9999999967735554  0.5800318131834097   0.7319059603757937   0.8828060866881948  0.9999651881440772  0.6331274407344463   0.9999596400157125  0.8816654026906274  0.9015265457895539  0.9987256519549136  0.9946270882808942  0.9472437487121002  0.6315947257817153   0.6409321213596653   0.9999997016319263  0.9999873415664412  0.983309981000996   0.9999999937282841  0.9999872783392987  0.999740330247554   0.9834941636299058  0.9999155669280233  0.9999876299647815
14.8828125  0.5780945037367471   0.6779037716597931   0.8969701271031215  0.8002257704260017   0.9999999966418689  0.9999148821631938  0.8463759678218489  0.9999999936725497  0.9969584398010256  0.948231078025444   0.9997409642320552  0.9999997013547877  0.9999878725962202  0.8770503718710763  0.9999658243735938  0.7393701463572231   0.994623131071053   0.7110569908778336   0.6457014501892357   0.9987178619744926  0.9999999967504684  0.5848116452673046   0.7357490070993445   0.8845520899787814  0.9999658220984076  0.6374783863058127   0.9999612620339402  0.8747059154746073  0.902072217829723   0.9989133147707252  0.9946735429876388  0.9478855654757484  0.6315298699003805   0.6472449515443349   0.9999997049192184  0.9999874885165524  0.9834691612939607  0.9999999937780563  0.9999874172960386  0.99974442253481    0.9837807860724518  0.9999175685023342  0.999987881248711
15.         0.5824383300074      0.682829687757       0.9016536470524     0.8092057344279      0.9999999966242     0.9999124629605     0.8538551021968     0.9999999937129     0.9969944890343     0.9490880247016     0.9997485512512     0.9999997047396     0.9999880597112     0.8811928877684     0.999966730598      0.7411165445315      0.9946824674783     0.7143398247221      0.6478297816147      0.9986126818273     0.9999999967344     0.5893660486415      0.7398109446941      0.8856872263942     0.9999663709563     0.6421101274396      0.9999624651254     0.8647044005444     0.9022415615851     0.9990463167816     0.9947124885963     0.9485815990366     0.6317585545192      0.6532316390395      0.9999997082491     0.999987656283      0.9836727449462     0.9999999938231     0.9999875498527     0.9997479356783     0.9840100419463     0.9999161108736     0.9999880644781
15.1171875  0.5868759953004582   0.687507158547817    0.9053451794414344  0.8173617006095328   0.9999999966165811  0.99991143832313    0.8663790795354532  0.9999999937473504  0.9970353703125239  0.9498972533270637  0.9997560742281096  0.9999997082727689  0.999988169021279   0.8853518503636515  0.9999673120469784  0.7396436271510132   0.9947509481249861  0.7175779958935076   0.6506782445739226   0.998616098807348   0.9999999967272319  0.5936961074624768   0.7440894787254135   0.8864785593237231  0.9999667875214813  0.6469918482581551   0.9999636079620958  0.8536731531500141  0.9024418702875054  0.9990632559430443  0.9947490883194924  0.9493432404633808  0.6323590007002602   0.6588213151119999   0.9999997116998383  0.999987822658252   0.9839217611464206  0.999999993861981   0.9999876931085339  0.9997516491386776  0.984177482167165   0.9999115144335818  0.9999881689620914
15.234375   0.5914072910598225   0.6919150833996481   0.9077026561164666  0.8244937872475827   0.9999999966199696  0.9999109883454494  0.881338509111361   0.9999999937857375  0.9970801470233893  0.9506396723909295  0.9997620121183578  0.9999997119482424  0.9999882181838856  0.8891699736096254  0.9999675556631695  0.7352268598474123   0.9948232838872056  0.7207819076826594   0.654272402684427    0.9987332576425976  0.9999999967298807  0.5978049387807047   0.7485792347399348   0.8872167804781834  0.9999670241918694  0.6520860149205876   0.9999650562381549  0.844159862209414   0.902981516059546   0.9989602429642516  0.9947904390769154  0.950168201117612   0.6333979962121189   0.6639543531272438   0.999999715267182   0.9999879740816061  0.9842013954114551  0.9999999939039159  0.9999878497444284  0.9997562586218068  0.9842955024565021  0.9999062210960562  0.9999882148217836
15.3515625  0.5960315412057525   0.696036070401305    0.9085782820918328  0.8304529293018225   0.999999996633389   0.9999096842834267  0.8959008121092606  0.9999999938214639  0.9971221053417243  0.951310815155248   0.9997657243119451  0.999999715701258   0.9999882500163505  0.8923315923397169  0.9999675214274127  0.7282893370340158   0.9948941423013904  0.7239673826160872   0.6586060314892348   0.9989110891906058  0.9999999967414579  0.6016977693506812   0.7532716575593376   0.8881628701942227  0.9999670840521278  0.6573492087694801   0.9999666653086268  0.8387333228397442  0.9040134922982486  0.9987921702797504  0.9948430030018355  0.9510391756204553  0.6349293396136664   0.6685839424499371   0.9999997188918143  0.9999881161918565  0.9844855448809922  0.9999999939428998  0.9999880027558623  0.9997617576842932  0.9843890056669389  0.9999032757396755  0.9999882442990379
15.46875    0.6007475314409187   0.6998569075598617   0.9080429254735172  0.8351484182748281   0.99999999665395    0.9999071289734672  0.907557316065107   0.9999999938578961  0.9971569831143133  0.9519199210241601  0.9997676520263649  0.9999997195292313  0.999988312492407   0.8946124796253015  0.9999673031856633  0.7193767305493579   0.9949602205248359  0.7271468215959      0.663641226445707    0.9990657687949875  0.9999999967591805  0.6053819884155922   0.7581549373642195   0.8895007660985477  0.999967050620754   0.6627331312297687   0.9999679387339375  0.8394531772993906  0.9055111541678015  0.9986449764914274  0.994910183761125   0.9519258770747656  0.6369926304150358   0.6726773254419093   0.9999997225713352  0.9999882577680655  0.9847447781144992  0.9999999939812712  0.9999881371951711  0.9997673025883368  0.9844875078320048  0.9999043435141133  0.9999883054909352
15.5859375  0.6055534565154835   0.7033689545822529   0.9063716057169376  0.8385529092998409   0.999999996678178   0.9999046807897339  0.9146076242012906  0.9999999939027188  0.997188105363888   0.9524860612006546  0.9997688917951918  0.9999997234518454  0.9999884374964111  0.8959148886172457  0.9999670162744456  0.7091257221966228   0.9950213366856065  0.7303230912885524   0.6693104205428531   0.9991249507790984  0.9999999967795576  0.608867171880141    0.7632139673261101   0.8913043354800445  0.9999670611072634  0.6681857684984771   0.9999686384085793  0.8474114860495907  0.9072804661122156  0.9985931657061602  0.9949909632712455  0.9527900667075717  0.6396124186252939   0.6762166741001475   0.9999997263259407  0.9999883942637199  0.9849552471994666  0.9999999940266188  0.9999882597709251  0.9997716545200012  0.9846161344885334  0.9999087212223182  0.9999884327659091
15.703125   0.6104468871930637   0.7065684442829137   0.9039933665070871  0.8407058765537804   0.9999999967026179  0.9999046354781901  0.9165000721522232  0.9999999939474009  0.9972221605968001  0.953032303647006   0.9997704380971983  0.999999727404036   0.9999886288911463  0.8962817579542367  0.9999668039007871  0.6982299064742005   0.995080152253386   0.7334932458112718   0.675520135349078    0.9990635832543439  0.9999999967995928  0.6121650734687961   0.7684303382693193   0.8935245042536406  0.9999672374827531  0.6736526969701883   0.9999690011611353  0.8624224896181617  0.9090056576745701  0.9986638025722707  0.9950801060504393  0.9535925328724723  0.6427977204284211   0.6791995906231052   0.9999997300996234  0.9999885154810176  0.9851058580649206  0.9999999940711375  0.9999883900741371  0.9997739188386751  0.9847882304472915  0.9999140278320885  0.9999886311508432
15.8203125  0.6154247421992085   0.7094566862832044   0.9014157319140234  0.8417159676734821   0.9999999967256875  0.9999084793014612  0.9139589265386099  0.9999999939972303  0.9972608767315094  0.9535793345351397  0.9997725825553402  0.9999997312889353  0.9999888632003663  0.895887641285873   0.9999668256078549  0.6874064015781685   0.9951406857327098  0.7366570201358011   0.682156072466645    0.9989157410198344  0.9999999968177761  0.6152895789168594   0.7737823763320049   0.8959988211579719  0.9999676267409638  0.6790785023088463   0.9999693878992251  0.8829263636242459  0.9103185639570947  0.9988245431547398  0.9951698443328638  0.9543005198482033  0.6465419043592638   0.6816392218407865   0.9999997338048559  0.999988625606543   0.9852015766012033  0.9999999941191293  0.9999885357858347  0.999774111998785   0.9850017599764215  0.9999179879796302  0.9999888734532448
15.9375     0.6204832551649563   0.712040168898775    0.8991385279973062  0.8417605748637875   0.9999999967472938  0.9999155759592188  0.9088407717443375  0.9999999940560562  0.9972989663176625  0.9541400414416062  0.99977483766465    0.9999997350870126  0.9999891001664625  0.8950080373587875  0.999967206056975   0.6773636553038188   0.9952062512837     0.739819292294675    0.6890890222121062   0.9987571288547563  0.9999999968341875  0.61825662010145     0.7792452305770562   0.8984816912600251  0.999968192487825   0.6844082762826875   0.9999698324792875  0.906165856745675   0.9108775213539875  0.9990010278143     0.99525240720125    0.9548939896990938  0.6508229515691375   0.6835639810205812   0.9999997374246188  0.9999887449524938  0.9852619448830687  0.9999999941751062  0.9999886849834501  0.9997731988622938  0.9852404859326312  0.9999199342574062  0.9999891134430313
16.0546875  0.6256179451935532   0.7143305566901375   0.8975731743023058  0.8410790651737854   0.999999996768749   0.9999233334214866  0.9036985466520793  0.9999999941144169  0.997330842780667   0.9547162993881161  0.999776406331852   0.9999997388157683  0.999989298938979   0.8939726305729074  0.9999679669036126  0.6687694327820801   0.9952776755689733  0.7429834716886091   0.6961810497631006   0.9986674816337388  0.9999999968504615  0.6210840472271534   0.784791018929959    0.900689765626839   0.9999688548527733  0.6895891473900446   0.9999700326496561  0.9286619972365303  0.9104403977791966  0.9991143841038072  0.9953224606476211  0.9553693638535307  0.6556040920823171   0.6850168730841695   0.9999997409790182  0.999988887894942   0.9853153310639556  0.9999999942302303  0.9999888239552983  0.9997725786848596  0.9854794558458471  0.9999209354103342  0.9999893082557331
16.171875   0.6308236079469934   0.7163445811781933   0.8969815285103092  0.8399567058773969   0.9999999967919196  0.9999287721383515  0.9011136489626165  0.9999999941793697  0.9973576383271775  0.955298655395686   0.9997768679059696  0.9999997424392111  0.9999894351229295  0.8931107600484194  0.9999689940877201  0.6622189339859134   0.9953525304426801  0.7461442369580674   0.7032915070506045   0.9986917827180335  0.9999999968686811  0.6237914584979104   0.7903890416324584   0.9023542597542284  0.9999695378497574  0.6945717958331936   0.9999697941767217  0.9469313978500162  0.9089182553239598  0.9991199068230714  0.9953785877568953  0.9557399610633319  0.6608348154119724   0.6860544229598385   0.9999997444400663  0.9999890452035038  0.985390698575238   0.9999999942908312  0.9999889542196054  0.9997733402561874  0.9856930353399102  0.9999225599593742  0.9999894372852423
16.2890625  0.6360943310429844   0.7181038221590935   0.8974437705859405  0.8386997520226344   0.9999999968191389  0.9999303582946415  0.9029572695467991  0.9999999942516188  0.9973858593249433  0.9558689181213446  0.9997766178609998  0.999999745911612   0.9999895118343467  0.8926986621589965  0.9999700755599271  0.6582044807324728   0.9954257146315033  0.7492889076326501   0.7102825680879437   0.9988193731279768  0.9999999968913718  0.6263999879488734   0.7960060703976212   0.9032708343014635  0.9999701865332745  0.6993119016796029   0.9999692836960397  0.9582818635708089  0.9064005600458099  0.9990278575845828  0.9954233236474805  0.9560330783586627  0.6664522481594894   0.686745214165195    0.9999997477671965  0.9999891940815852  0.9855093821199299  0.9999999943585609  0.9999890856215132  0.9997757400351409  0.9858629985960952  0.9999253372914089  0.9999895085250812
16.40625    0.6414235185994773   0.7202071246804508   0.898858925340275   0.8376062374661922   0.9999999968524804  0.9999287834358906  0.9097990359097959  0.9999999943230695  0.9974193946182671  0.9564049894717758  0.9997767618106258  0.9999997492605133  0.9999895579805032  0.8929174532274976  0.9999709927467086  0.6570879218727164   0.995491183838568   0.7524060105096453   0.7170241378476656   0.9989907778668008  0.999999996920854   0.6289320531009938   0.8016067185744202   0.9033383096202258  0.9999707535537539  0.7037714766599898   0.9999688271319586  0.9614622552379407  0.9031485528587461  0.9988965963634859  0.9954618132016695  0.9562843938754219  0.6723828833252875   0.6871680560379437   0.9999997509864039  0.9999893226620352  0.9856793022197469  0.999999994425725   0.9999892174332844  0.9997792033969375  0.985984220461639   0.999928172393025   0.9999895520466586
16.5234375  0.6468039075016445   0.7236170823572725   0.900975746906405   0.8369388604857084   0.9999999968945573  0.9999261871632104  0.9206591985597349  0.9999999944019954  0.9974548407611702  0.9568867718250771  0.9997785164332855  0.9999997525430091  0.9999896132622402  0.8938286189262844  0.9999716083475051  0.6590753694513218   0.99554417539419    0.7554907791043238   0.723398093010545    0.9991282057792724  0.9999999969597125  0.6314110648917318   0.8071538936672233   0.9025797018550051  0.9999711869840937  0.70792003393448     0.999968564295804   0.9569641843543211  0.8995591304891741  0.9988021342542638  0.995499668371157   0.9565309551027996  0.6785446350626196   0.6874098124958243   0.9999997541540406  0.999989437838685   0.9858932697357049  0.9999999944997625  0.999989336336173   0.9997828128583748  0.9860662883892973  0.9999291803472315  0.9999896059680149
16.640625   0.6522275745654611   0.7269432460213875   0.9034468236614945  0.8369050467539256   0.9999999969477716  0.9999244892103318  0.9332121535298394  0.9999999944863646  0.9974859579086595  0.9573017526332845  0.999782485113799   0.9999997557617464  0.9999897083320807  0.8953704779236638  0.9999718991179781  0.6641939369581822   0.9955830759317214  0.7585413543397341   0.7293018601273605   0.9991741280447076  0.999999997010459   0.6338611029317126   0.8126093294748598   0.9011423425702317  0.9999714466386973  0.7117355554139119   0.9999683378631458  0.9468629374057053  0.8961063800536755  0.9987997313857354  0.995540874546574   0.9568043111714717  0.6848491789873122   0.6875629407175355   0.9999997572755804  0.9999895496780785  0.9861317755492452  0.9999999945798264  0.9999894356291696  0.9997859452153358  0.9861306158364267  0.9999272693208394  0.9999897006753622
16.7578125  0.6567749181840896   0.7301871496656097   0.9058955207670377  0.8376462488691894   0.999999997014044   0.9999241225767759  0.9444152224356013  0.9999999945697878  0.9975115615036736  0.9576489535073447  0.9997882571050355  0.9999997589003938  0.9999898516132794  0.8973753724686032  0.9999719281550253  0.672273120724154    0.9956102115965941  0.7615511432725284   0.7346513371802205   0.9991187461887799  0.9999999970747445  0.6363065596224271   0.8179341909998308   0.8992775082108754  0.9999715407589154  0.7152052230032381   0.999967972394186   0.934218999819888   0.89327113766152    0.9988967235712778  0.9955865346744545  0.9571252658711398  0.6912045234777305   0.6877228034213345   0.9999997603343934  0.9999896572321081  0.986369314175164   0.9999999946593748  0.9999895290489469  0.9997886415063505  0.9862040018920527  0.9999232066784155  0.9999898462227649
16.875      0.6597901780886      0.7333505083332      0.9079840607444     0.8392358187809      0.9999999970929     0.9999240817732     0.9513833287012     0.9999999946602     0.9975370204487     0.957940311023      0.9997945788155     0.999999761996      0.9999900284924     0.8996038756823     0.9999717947806     0.682937154398       0.9956312467355     0.7645075024188      0.7393831476624      0.9990031114549     0.9999999971522     0.6387717570738      0.8230897410626      0.897303385499      0.999971546041      0.7183258882085      0.9999675464375     0.922213813515      0.8914702653881     0.9990497706177     0.9956348797808     0.9575012959551     0.6975177441366      0.6879848318564      0.9999997633621     0.999989755811      0.9865822582908     0.9999999947458     0.999989638789      0.9997914827448     0.986310633106      0.9999193037943     0.9999900279827
16.9921875  0.6628304393069343   0.7364352094350819   0.9094716524710634  0.8416833343502905   0.9999999971815745  0.9999231413127982  0.9522231652303798  0.9999999947524997  0.9975675846392856  0.9581991436992745  0.9997999942148988  0.9999997650969683  0.9999902102463631  0.9017896419632753  0.999971604235306   0.6956157774215254   0.9956534015220384  0.7673988664670931   0.7434562067910987   0.9988969971177979  0.9999999972398254  0.6412805411759667   0.828038053771463    0.8955577969458054  0.9999715822620812  0.721104262768354    0.9999673145541617  0.9133007383144192  0.8909967319123742  0.9991873501300975  0.9956824647983042  0.9579270331450316  0.7036978030380875   0.6884416263084793   0.9999997664035755  0.9999898530148237  0.9867558857606403  0.9999999948350063  0.9999897720236021  0.9997950732489169  0.9864649536707971  0.999917924858083   0.9999902138342776
17.109375   0.6659066474614638   0.7394432998965449   0.9102538161325378  0.8449429207015796   0.9999999972749284  0.9999211940958761  0.9465539794613639  0.9999999948421979  0.9976018934628273  0.9584560212499063  0.999803570085113   0.999999768189118   0.9999903666538749  0.9036870465267476  0.9999714657211206  0.7095783050775474   0.9956831398389229  0.7702218358554437   0.7468525860887631   0.9988631176489635  0.9999999973326396  0.6438558577462022   0.8327427575388197   0.8943485381998715  0.9999717519480439  0.7235568202021152   0.9999674416675647  0.9086415875385971  0.8919793335754014  0.9992458217630343  0.9957260098159912  0.9583875169953063  0.7096583669001136   0.689180085283803    0.9999997694446681  0.9999899672157286  0.9868884730761295  0.999999994921907   0.9999899154859873  0.9997994894096807  0.9866675231801528  0.9999199272423537  0.9999903701663029
17.2265625  0.6690290348709922   0.7423769697658739   0.9103773834430834  0.8489233651131721   0.9999999973672234  0.9999196005566763  0.9355733863979571  0.9999999949357502  0.9976337889007408  0.9587429396197672  0.9998052839000833  0.9999997712353219  0.9999904778355413  0.9051136416009901  0.9999714925233041  0.7239908140250956   0.9957241679533271  0.7729798206486375   0.7495776788635764   0.9989257607324425  0.9999999974248949  0.6465193165348891   0.8371697894731153   0.8939092711182988  0.9999720885005671  0.7257094042512644   0.9999678269802522  0.9079727497807083  0.8943675484701454  0.9992009339946367  0.9957641123429232  0.9588633298636928  0.715320534323781    0.6902786537255639   0.9999997724467248  0.9999901069862809  0.9869913006039979  0.9999999950127296  0.9999900530588959  0.9998040730263346  0.9869050452591333  0.9999241844015703  0.9999904778133508
17.34375    0.6722069133825633   0.7452385334276102   0.910029497340775   0.8534983180127141   0.9999999974536266  0.9999202496827195  0.9216927616228358  0.9999999950264     0.9976600426564438  0.9590870852898945  0.9998058741348531  0.9999997742441055  0.9999905423563844  0.9059806367863587  0.9999717768357758  0.7379900664064533   0.9957764545522172  0.7756753531988547   0.7516596853893742   0.9990592893455672  0.9999999975115266  0.649290750018761    0.8412881430585173   0.8943677370074219  0.9999725510382086  0.7275965466120048   0.9999682522088531  0.9098845802759259  0.8979431652850384  0.9990792911738008  0.9957981294842743  0.9593363347373742  0.7206153824877486   0.6918047759680657   0.9999997754142266  0.9999902580893046  0.9870846738374423  0.9999999951013304  0.9999901815384601  0.9998077248513633  0.9871544791638289  0.9999285089833391  0.9999905391881281
17.4609375  0.6754484740476342   0.7480304098605741   0.909502805400946   0.8585154843453948   0.9999999975305187  0.9999240631431557  0.9078906587128031  0.9999999951121765  0.9976842907420096  0.9595055562862873  0.9998062884255404  0.9999997772302264  0.9999905782979489  0.906306802973853   0.99997234396521    0.7507621309250163   0.9958365903000822  0.7783059071367744   0.7531484468753025   0.9992017783845193  0.9999999975890359  0.6521877752046817   0.8450705911965071   0.8957308782242754  0.9999730650670487  0.7292605039528425   0.9999686746640696  0.9123789784385541  0.9023564562753212  0.9989445320572721  0.9958318751576412  0.9597946399763595  0.7254862479820485   0.6938126325567022   0.9999997783602091  0.9999903963196997  0.987191265536558   0.999999995185172   0.9999903051821148  0.9998095307405609  0.9873899278174587  0.9999312387777562  0.9999905746380099
17.578125   0.6787605985284397   0.7507551034567618   0.9091425666803653  0.8638044553320396   0.9999999975954625  0.9999301195549422  0.896987733259338   0.999999995198026   0.9977123798385162  0.9600021774395827  0.9998071013793736  0.9999997801554046  0.999990615422964   0.9062139847152803  0.9999731218653044  0.7616114357763737   0.9958993035492653  0.7808684162649328   0.7541136632476555   0.999286486370282   0.9999999976549615  0.6552253680236304   0.8484943669237649   0.8978886891309359  0.9999735711415054  0.7307500332201342   0.9999692548460618  0.9135165180839333  0.9071811328552374  0.998865567036582   0.9958702560171595  0.9602356428167486  0.7298906637719507   0.6963412328055486   0.9999997812483369  0.9999905108194349  0.9873288659272695  0.9999999952691688  0.9999904208882453  0.999809335934375   0.9875901929197447  0.9999322813359319  0.9999906126253975
17.6953125  0.6821486863360594   0.7534151867608146   0.9092842092264655  0.8691836740878578   0.9999999976464     0.9999361015334972  0.8910335195060413  0.9999999952769134  0.9977450021974905  0.9605670644527713  0.9998082671447889  0.9999997829741727  0.9999906815989393  0.9059053387981084  0.9999739592808747  0.7700096534438201   0.9959595270963488  0.7833666292632767   0.7546425327755071   0.9992751943389778  0.9999999977072044  0.6584154613245068   0.8515417848500436   0.9006356991697974  0.9999740424689801  0.732118933556651    0.9999701390909137  0.9119730592289301  0.911979499947011   0.9988829051690673  0.9959173895561299  0.9606664923296055  0.7338018827649836   0.6994129282208078   0.9999997840332817  0.9999906095304509  0.9875047640915737  0.9999999953468135  0.9999905183281856  0.9998079088689431  0.9877447331080352  0.9999328750669981  0.9999906791980818
17.8125     0.6856165012689499   0.7560132860996438   0.910191582004525   0.8744676933590125   0.9999999976814062  0.9999397546716126  0.8909294437776938  0.9999999953496937  0.9977768600914062  0.9611789758828562  0.9998093356013501  0.9999997856937687  0.9999907910244188  0.9056300919827688  0.9999746894544875  0.7756218240655562   0.9960142247837187  0.7858118603718375   0.754836850336675    0.9991768661198188  0.9999999977440562  0.6617665788800875   0.8542007877129876   0.9037063221679063  0.9999744662002125  0.7334243920042125   0.9999712496950187  0.9073829300412563  0.9163675249278938  0.9989896773031001  0.9959749459479813  0.9611019293155375  0.7372099270246937   0.7030324059233563   0.9999997867193687  0.9999907032477313  0.98771344496655    0.9999999954182499  0.9999905963077562  0.9998065861392125  0.9878562884037375  0.9999343361017     0.9999907893487937
17.9296875  0.6891660400995953   0.7585520705528831   0.912006366715306   0.8794759266105817   0.9999999976990833  0.9999402838165159  0.8963298926017198  0.9999999954199277  0.997803308407669   0.9618098868294066  0.9998099529528938  0.9999997883330809  0.9999909402850364  0.9056406785251808  0.9999751995462003  0.7783141035928107   0.996063303025878   0.788215961234109    0.7548095926996174   0.9990426967665956  0.9999999977639771  0.6652835188070104   0.8564654040084589   0.9068182342007517  0.9999748183778059  0.7347251779904256   0.9999723434362681  0.9004211689352473  0.920069336917629   0.9991363328225257  0.9960413611715675  0.9615600660284958  0.7401221100156723   0.7071862150674253   0.9999997893244726  0.9999907924955739  0.9879382485924245  0.9999999954870386  0.9999906716447262  0.999806605885105   0.9879394722258835  0.9999368303067135  0.999990941577885
18.046875   0.6927974267683418   0.7610342442785062   0.914717663022025   0.8840432930045693   0.9999999976993313  0.9999386788991197  0.9058013444522303  0.9999999954810178  0.9978260457557002  0.9624307516739237  0.9998102860353631  0.9999997908724038  0.9999911121865301  0.9061492943854967  0.9999754625799943  0.778149154224432    0.9961093019302285  0.7905849779609384   0.7546810066775311   0.998940469404087   0.9999999977668689  0.6689671000229543   0.8583361037993444   0.9097167546083536  0.9999750656731741  0.736079738287358    0.9999732997801372  0.892638212127762   0.9229521499479435  0.9992561128154802  0.9961122270395535  0.9620571459930586  0.7425629877088928   0.7118428713627843   0.9999997918296359  0.9999908745892105  0.9881564250893714  0.9999999955473696  0.9999907656568079  0.9998085207154229  0.9880158592145858  0.9999391533233694  0.9999911180086439
18.1640625  0.6965088358439435   0.7634625419797877   0.9181577185537512  0.8880318707436222   0.9999999976834266  0.9999367911823613  0.9171862922142586  0.9999999955361062  0.9978508334650457  0.9630171887474367  0.9998110672398651  0.9999997932943372  0.9999912831381892  0.9072910061536731  0.9999755211239872  0.7753722037324638   0.9961560368396025  0.7929209198551227   0.7545742136663376   0.9989212854077367  0.9999999977538954  0.6728139846515612   0.8598200404821053   0.9122130926327142  0.9999751965042821  0.7375442497651914   0.999974212552003   0.8860979812020815  0.9250356529042479  0.9992982392466692  0.9961817098280535  0.9626025700535191  0.7445737047709425   0.716953573122042    0.9999997942162326  0.9999909579427921  0.9883460006587768  0.9999999956014891  0.9999908837945142  0.9998120220800666  0.9881071399268186  0.9999396838894569  0.9999912916359893
18.28125    0.7002964495344296   0.7658397272116234   0.9220255012745562  0.8913417316504414   0.9999999976539461  0.9999359197872781  0.9280973296350336  0.9999999955869805  0.9978801979988351  0.9635538640190304  0.9998132111022617  0.9999997956295094  0.9999914308190594  0.9090995409224976  0.9999754492438758  0.7703876055291892   0.9962067591313445  0.7952287787663804   0.7546103722198453   0.9989952004513141  0.9999999977272945  0.6768165879878031   0.8609311665393883   0.9142103290376937  0.9999752508523726  0.7391706917472484   0.9999752008322453  0.8828958965216555  0.9264741480618367  0.9992522802659695  0.9962444589914343  0.9631954138336312  0.7462107162911227   0.7224535462631945   0.9999997965112359  0.9999910599751555  0.9884923510716804  0.9999999956514789  0.9999910135387133  0.9998162481655524  0.9882284605776344  0.9999377472610976  0.999991438006107
18.3984375  0.7041544528504635   0.7681685930966957   0.925934892863683   0.893919197504908    0.9999999976135217  0.9999359977176824  0.9364473926561039  0.9999999956281076  0.9979102926153252  0.9640366644951006  0.9998172377289523  0.9999997979123871  0.9999915413545172  0.9114998866830311  0.9999753278342136  0.763725755896874    0.9962625760601481  0.7975194081216708   0.7549035129538993   0.9991275327506419  0.999999997689695   0.6809630858850556   0.8616902115583766   0.9157129963566284  0.9999753150839367  0.7410050026981866   0.9999762111229452  0.884652731917916   0.9275145481658735  0.9991519132028732  0.9962972906514677  0.9638232985173286  0.7475438860119938   0.7282640145510755   0.9999997987476433  0.9999911883646808  0.9885925061636467  0.9999999956923374  0.9999911412540992  0.9998203532582906  0.9883839615073927  0.9999343068504489  0.9999915450219458
18.515625   0.7080750715787775   0.7704519647907064   0.9294795321502863  0.895761666609175    0.999999997564373   0.9999359524577075  0.9409120876438551  0.9999999956641611  0.9979357887782658  0.9644722664413178  0.9998228700980609  0.9999998001366693  0.9999916148407493  0.9143191309408402  0.9999752436093097  0.7560029166950013   0.9963217545117516  0.7998042209813957   0.7555552706971129   0.9992571512790773  0.9999999976429986  0.6852375273003344   0.8621245111475728   0.9168188146404048  0.9999754785982928  0.7430853869916354   0.9999770449275865  0.8920867641273407  0.9284379518481476  0.9990567651265502  0.996340032403266   0.9644639150325566  0.7486539889023427   0.7342947654461883   0.9999998009200839  0.999991329677065   0.9886559300568707  0.9999999957279513  0.9999912642887491  0.9998239714610895  0.9885658163353283  0.9999314277213033  0.9999916161192205
18.6328125  0.7120486573685602   0.7726927028034124   0.9323027282427869  0.8969189503240474   0.9999999975083941  0.9999348789477477  0.9412266070498502  0.999999995695775   0.9979565088104114  0.9648753026989969  0.9998290695752282  0.9999998022934435  0.9999916660272491  0.9173139459738217  0.9999752913128279  0.7478787048780983   0.9963801820075471  0.8020880213333565   0.7566498570938325   0.9993271915903486  0.9999999975890445  0.6896200569606868   0.8622676764675634   0.9176941354520812  0.9999757840653055  0.7454408363878947   0.9999776083187939  0.9047571512396749  0.9294959715701596  0.9990218716501128  0.9963752507948188  0.9650888914728321  0.7496296778486262   0.7404472514634506   0.9999998030196747  0.9999914606974505  0.988701606892458   0.9999999957594814  0.9999913835872067  0.9998272968791175  0.98875698265129    0.9999309043539906  0.9999916670505135
18.75       0.7160638232895      0.7748937062079      0.9341593301396     0.8974909076709      0.9999999974473     0.9999330373754     0.9382409016845     0.999999995719      0.99797753128       0.9652638795956     0.9998345042202     0.999999804411      0.999991718002      0.9202102568219     0.9999755538299     0.7400156257685      0.9964328077038     0.80436860061        0.7582496989694      0.9993127537478     0.9999999975294     0.6940872496865      0.8621590965424      0.9185374776495     0.9999762093454     0.7480899277824      0.9999780040738     0.9210508810059     0.9308544047503     0.9990701866725     0.9964070151784     0.9656691662151     0.7505640080121      0.7466181340247      0.9999998050746     0.9999915685088     0.988752375403      0.9999999957831     0.9999914908813     0.9998307620103     0.9889367429547     0.9999330793072     0.9999917194144
18.8671875  0.7201076315292766   0.7770579148204618   0.9349575584362471  0.8976203432096314   0.9999999973838499  0.9999318506012325  0.9336942947493913  0.999999995740563   0.9980027697934072  0.9656545596619022  0.9998381794614963  0.9999998065164563  0.9999917914628833  0.9227483964714165  0.9999760606400699  0.7330428155142489   0.9964755105630538  0.8066432122101115   0.7603921759443406   0.9992320055264898  0.99999999746685    0.698612554091928    0.8618432698393762   0.9195377128585805  0.9999766914653544  0.751039954013784    0.9999783454966314  0.9384511664303904  0.932555691949132   0.9991817039667065  0.9964392131912211  0.9661806833125924  0.7515506447841693   0.7527031448873702   0.9999998071149376  0.9999916556786844  0.9888282610029785  0.9999999958049062  0.9999915723961115  0.9998345570998266  0.9890872852147682  0.9999367188662546  0.9999917930795245
18.984375   0.7241658325491389   0.7791883094490901   0.934772533088585   0.8974799328754304   0.9999999973224639  0.9999328822439609  0.9297346748976208  0.9999999957606875  0.9980301272866409  0.9660580237792404  0.9998398662996674  0.9999998085962325  0.9999918956939258  0.9247259114342454  0.9999767550220549  0.7275242780124433   0.9965066838688543  0.8089135375455601   0.7630878094371222   0.9991356441051285  0.9999999974055716  0.7031668395217052   0.8613689677395536   0.9208345376689124  0.999977167880215   0.7542864377072961   0.999978599664136   0.954063259825972   0.93450796005473    0.9993042175216582  0.9964740872448302  0.9666091418806116  0.7526799092810273   0.7586011135080173   0.999999809132241   0.9999917288060899  0.9889408152208004  0.9999999958258007  0.999991627042319   0.9998383304030148  0.9891992911667271  0.9999399967519973  0.999991898488206
19.1015625  0.7282231523341298   0.7812879092363777   0.9338288507724658  0.8972534136377752   0.9999999972684657  0.9999365878536183  0.9282924257263844  0.9999999957778064  0.9980542746021163  0.9664764324684175  0.9998400938054112  0.9999998106317763  0.9999920257293857  0.9260304260539212  0.9999775032283911  0.7239313384440218   0.9965279398064242  0.8111829371789055   0.7663200977895      0.9990804927246918  0.9999999973511002  0.7077190363996577   0.8607882397940065   0.9224885778649933  0.9999776001399895  0.7578130688193048   0.9999786410420255  0.9652876216853318  0.9365030070354312  0.9993799202753546  0.9965115525712773  0.9669527458397126  0.7540348360326993   0.7642179901278242   0.9999998111127487  0.9999917904035324  0.9890901230930144  0.9999999958440078  0.9999916749307786  0.9998412849237189  0.989274841645462   0.9999418060320279  0.9999920313457615
19.21875    0.7322636216331156   0.783359765154686    0.9324558617648219  0.8971139377130805   0.9999999972275961  0.9999418180728016  0.9304879768867008  0.9999999958006945  0.9980733960210507  0.9669030729734837  0.9998397429239563  0.9999998126430499  0.9999921656171438  0.9266566478331704  0.9999781490806234  0.7226190285091718   0.9965436746223922  0.813449691515304    0.7700469938499657   0.9991007228659993  0.9999999973090101  0.7122368565423751   0.860155278633775    0.9244667885959016  0.9999779673961859  0.7615920937725579   0.999978457313839   0.9704588533425196  0.9382595246475164  0.9993743236341281  0.9965495321863421  0.967222368585932   0.7556874282161742   0.7694706850356118   0.9999998130804445  0.9999918457360851  0.9892652988465938  0.999999995867775   0.9999917427912288  0.9998426249221437  0.9893267962153421  0.9999424070453422  0.999992173678561
19.3359375  0.7362709384769651   0.7854069491231775   0.9310248398501373  0.8972044916907219   0.9999999972046266  0.9999464783054968  0.9362616603044054  0.9999999958283796  0.9980914024977696  0.9673243197867498  0.9998395156790117  0.9999998146566235  0.9999922951129467  0.9267044410697107  0.9999785812748941  0.7238058596037382   0.9965596624799999  0.8157049999193338   0.7742038294142076   0.9991910524577325  0.9999999972845125  0.7166875765784937   0.8595251705599197   0.9266460133714904  0.9999782483665578  0.7655851680707269   0.9999782052897526  0.9692571624536115  0.9394825561142012  0.9992931206889473  0.9965851345269271  0.9674391413082174  0.757695295055123    0.7742905440446547   0.9999998150665578  0.9999919113016519  0.9894482233490867  0.9999999958966842  0.9999918421169663  0.9998420747157077  0.9893748914845408  0.9999429693254924  0.9999923024668336
19.453125   0.7402288528737458   0.7874325389833952   0.9298807522995982  0.8976250040192048   0.9999999972028615  0.9999488675312271  0.9443553914077479  0.9999999958607572  0.9981129790945327  0.9677233887886537  0.9998396115347764  0.9999998166616669  0.9999923975692381  0.9263583962004351  0.9999787709713688  0.7275564669932595   0.9965812031679345  0.8179384606250959   0.7787073485605996   0.9993100071602283  0.999999997280799   0.7210388646427157   0.8589525657005516   0.9288341534548004  0.9999784196573316  0.7697446652547172   0.9999780104650186  0.9627585263848515  0.9399268738982617  0.9991788872349222  0.9966161690657734  0.9676300457255484  0.7600988443633222   0.7786262884425028   0.9999998170634938  0.9999920068575996  0.989619309581533   0.9999999959301562  0.9999919642515808  0.9998401395169717  0.9894397943725913  0.9999444335899418  0.9999924012769075
19.5703125  0.7441213105188208   0.7894396003554441   0.9292820321849196  0.8984282775829668   0.9999999972241937  0.9999487116359327  0.9526684803422648  0.9999999959054169  0.9981377632297029  0.9680849534535791  0.9998397911922096  0.9999998186373331  0.9999924671295028  0.9258527962342101  0.9999787623351934  0.733767982910366    0.9966115021936478  0.8201444071828944   0.783460447069285    0.9994012056273097  0.9999999972999885  0.7252596283616359   0.8584903081249208   0.9308050369605798  0.9999784777237645  0.7740154139562471   0.9999778383993686  0.9530980792664924  0.9394513824723033  0.9990896568045406  0.9966423711892881  0.9678225221560895  0.7629191830064574   0.7824462640368177   0.9999998190511937  0.9999921359471275  0.9897635039029821  0.9999999959755188  0.9999920940027802  0.9998379346905532  0.9895369442248066  0.9999466294376679  0.9999924676594633
19.6875     0.7470216613642624   0.7914311668766187   0.92935958378755    0.899623764600925    0.9999999972694062  0.9999471554652687  0.958885420290575   0.9999999959575062  0.9981609638451     0.9683995018346188  0.99983974980005    0.9999998205980938  0.9999925121359062  0.9254286255967312  0.9999786351036     0.7421635174762563   0.9966508566011     0.8223216268904563   0.7883572289601812   0.9994217391143375  0.9999999973428499  0.7293208607671625   0.8581880704562562   0.9323404406517001  0.9999784629582188  0.7783368119994563   0.9999775880480688  0.9428463320875875  0.9380538337620937  0.9990703171106687  0.9966658320031063  0.96803930442085    0.7661568513525062   0.7857398677165437   0.9999998210432375  0.9999922784813687  0.9898747061229688  0.99999999602845    0.9999922214270313  0.9998366745948313  0.98967204790265    0.9999483529689625  0.9999925126381125
19.8046875  0.7498728374483299   0.7934102204387711   0.9301018861832566  0.9011861707230553   0.9999999973378658  0.9999457767794635  0.9611736077307222  0.9999999960163303  0.9981794900799559  0.9686663828436861  0.9998395231111001  0.9999998225677383  0.9999925517288711  0.925290704508012   0.9999784733776846  0.7522981009301332   0.9966969016113143  0.8244674794016258   0.7932880602713663   0.999363432338361   0.9999999974088739  0.7331964599116485   0.85809103882678     0.9332713565815454  0.9999784569971533  0.7826452429368497   0.9999772741057207  0.934297793365534   0.9358794351517403  0.9991310774064844  0.9966904474320615  0.9682946089929767  0.7697914854589285   0.7885180503365573   0.9999998230606938  0.9999924062635834  0.9899573148225189  0.9999999960877313  0.9999923389216119  0.999837117689113   0.9898395567411006  0.9999483345839241  0.9999925544259447
19.921875   0.752701859080224    0.7953796729162791   0.9313676262197838  0.9030656043419836   0.9999999974277008  0.9999454119796535  0.9587229524149926  0.9999999960851865  0.998195782191788   0.9688948020274807  0.9998396358154087  0.9999998245297367  0.9999926064179553  0.9255728643127117  0.9999783577075176  0.7635821761070664   0.9967457691414279  0.8265744035676289   0.7981443885709917   0.9992565932336308  0.9999999974959634  0.7368639974926084   0.8582386926166149   0.9335107905371709  0.999978546553842   0.7868766981790968   0.999977065255261   0.9288981235901653  0.9332022496538926  0.9992434452948892  0.9967205881232503  0.9685924489031456  0.7737824632319958   0.7908128313607358   0.9999998250837814  0.9999925045681674  0.9900245054460652  0.999999996156551   0.9999924345569855  0.9998392838658147  0.9900245270696618  0.9999463233660126  0.9999926107971069
20.0390625  0.7555092503262401   0.7973423505318755   0.932921989719592   0.9051969231383238   0.9999999975348299  0.9999456860466719  0.951965382158609   0.9999999961556783  0.998214660210405   0.9691025085392306  0.9998408757009675  0.9999998264513155  0.9999926881889717  0.9263173666099987  0.9999783680042674  0.7753247355862544   0.9967936753087947  0.8286335650212954   0.8028231736861072   0.9991547340661394  0.9999999976003562  0.7403054120119493   0.8586637195872553   0.9330720682375102  0.9999787779658748  0.7909694897633553   0.9999770862356024  0.9269720514298718  0.9303834777575194  0.9993552737302126  0.9967594970451209  0.9689273141158781  0.778070547861613    0.7926758045112763   0.9999998270766983  0.9999925769006304  0.9900938478052989  0.999999996227103   0.9999924978930032  0.9998425853265646  0.9902072357174065  0.999943425361733   0.9999926928367859
20.15625    0.7582954569171187   0.7993009813912914   0.9344885714161281  0.9075070721672766   0.999999997653625   0.9999455577039805  0.942432768874847   0.9999999962277735  0.9982372714444313  0.9693124538567274  0.9998438254229641  0.9999998283350571  0.9999927961776359  0.927471911786968   0.9999785731699219  0.7867931897524467   0.9968383137727984  0.8306412693305992   0.8072308273569132   0.9991081740227117  0.999999997716      0.743507603125004    0.8593911013112164   0.9320694276156696  0.9999791366066867  0.79486692721265     0.9999772834310171  0.9277977319367343  0.9278147897900352  0.9994169976564625  0.996808027667125   0.9692869092887656  0.7825804943509688   0.794175664070889    0.9999998290387484  0.9999926328292477  0.9901816634377344  0.9999999962986461  0.9999925342861508  0.9998462576181845  0.9903691033459571  0.9999414165148336  0.9999928010670406
20.2734375  0.7610608441147424   0.8012581858291632   0.9358064874195221  0.909920009819786    0.9999999977772109  0.9999443912764137  0.9323267662974412  0.9999999963014685  0.9982597734829095  0.9695481744101788  0.9998484463109543  0.9999998301977485  0.99999291922058    0.9289041941556754  0.9999790006128195  0.7972817044509133   0.9968795324069362  0.8326003430143121   0.8112865901617019   0.9991397459233399  0.9999999978365482  0.7464629060093944   0.8604373966539799   0.9307010065312712  0.9999795647229361  0.7985198279292121   0.9999775234135933  0.9299628886513079  0.9258571919145179  0.9994058499389752  0.9968642040159478  0.9696561948239208  0.7872245332113393   0.795394837149798    0.9999998309833547  0.9999926787331195  0.9902978763749084  0.9999999963715639  0.9999925677225847  0.9998498036895772  0.9904981151540388  0.9999415439155277  0.9999929251721078
20.390625   0.7638056966576097   0.8032164682842003   0.9366819590020322  0.912359398560453    0.9999999978997535  0.9999426701706927  0.9239452385647138  0.9999999963692751  0.9982782899262354  0.9698289459656287  0.999853983536996   0.9999998320204451  0.9999930417612017  0.9304307179098571  0.9999796066375061  0.8061765684744312   0.9969190682637514  0.8345150910726582   0.8149252933716007   0.9992344981373148  0.9999999979559319  0.7491694275193772   0.8618102435597839   0.9292175398376249  0.9999799994511801  0.8018887340857296   0.9999777682899347  0.9318695116172186  0.9247861833236768  0.999336164486321   0.9969237702112393  0.9700217166120597  0.7919065917982236   0.7964253639760868   0.9999998328891839  0.9999927224704898  0.9904429249155994  0.999999996438777   0.9999926243665825  0.9998531767315776  0.9905921212338119  0.9999437158646044  0.9999930483662071
20.5078125  0.7665302203339768   0.8054148557289953   0.9370252457548234  0.9147498630138651   0.9999999980165688  0.9999417686622102  0.9191272205714278  0.9999999964337062  0.9982937443037618  0.9701658079432894  0.9998592668577203  0.9999998337701316  0.9999931492862928  0.9318540890682349  0.9999802779177257  0.8130082463284092   0.9969594915842722  0.8363863925676337   0.8180994693469308   0.9993481165072149  0.9999999980695622  0.7516312294702493   0.8635080924407555   0.9278825681547455  0.9999803991446822  0.8049457196072369   0.9999781359135721  0.932234862660848   0.9247523289892885  0.9992510345218697  0.9969815249691947  0.9703751787492889  0.7965270614564335   0.7973642215279909   0.9999998347229992  0.99999277998563    0.9906076437796913  0.9999999965022062  0.9999927134703309  0.999856612316108   0.9906590774919687  0.9999466718831183  0.9999931548614083
20.625      0.7692345447276      0.8078008369406      0.9368668669567     0.917018179581       0.9999999981247     0.9999429490808     0.9188482911524     0.999999996494      0.9983105745466     0.9705593473136     0.9998632402938     0.9999998354534     0.9999932323338     0.9330016714101     0.9999808737393     0.8174847215654      0.9970028165125     0.8382130223721      0.8207807883864      0.9994294598536     0.9999999981746     0.7538583484391      0.8655201788247      0.9269313019883     0.9999807421039     0.8076756878766      0.9999787367935     0.930461009273      0.9257633007437     0.9991999239183     0.9970329479779     0.9707154251115     0.800987872652       0.7983083299853      0.9999998364902     0.9999928680891     0.9907761042018     0.9999999965616     0.9999928257045     0.9998602542053     0.9907142362795     0.9999489697534     0.9999932362342
20.7421875  0.7719187268580534   0.8100367080237694   0.9363509046545258  0.9190958255385828   0.9999999982221063  0.9999463608765626  0.9230413247539211  0.999999996546063   0.9983311028623897  0.970999716080657   0.9998654183032307  0.999999837093843   0.9999932894937733  0.9337585231772673  0.9999812862157113  0.8195062330275583   0.9970493570258538  0.8399972130820337   0.8229608109643354   0.9994457607454756  0.9999999982690193  0.7558666457997142   0.8678267386563778   0.9265352838624931  0.9999810104043073  0.8100770831247105   0.9999795090787444  0.9267963731675124  0.9276895851199675  0.9992136699229603  0.9970755274588815  0.9710484041892045  0.8051976058014256   0.7993495116094678   0.9999998382118305  0.9999929881766916  0.9909304989598796  0.9999999966130689  0.9999929468865089  0.9998638507754881  0.9907752196941914  0.9999500158596817  0.9999932933554566
20.859375   0.7745827556451451   0.8121446825800154   0.9357071657366184  0.9209236190444602   0.9999999983071013  0.9999508183581878  0.9306598577975637  0.9999999965940749  0.9983527023119556  0.9714688393257143  0.9998660174273776  0.9999998386771713  0.999993328487347   0.9340888330983269  0.9999814808380985  0.8191641185230011   0.997097299602528   0.8417471255983142   0.8246510517278202   0.9993980104844149  0.999999998351416   0.7576774862195996   0.8703994661564313   0.9267787165689263  0.9999811845531985  0.8121619650430827   0.999980281031232   0.9222685267547859  0.9302920817891392  0.9992893424516295  0.997109325615425   0.9713851887063237  0.8090763475144609   0.8005696842234273   0.9999998398738609  0.9999931210173925  0.9910565531668444  0.9999999966601784  0.9999930676331792  0.999866747950392   0.9908565277691083  0.9999503621319339  0.9999933343633233
20.9765625  0.7772265571627566   0.8141511215339786   0.9352075142433801  0.9224579966575862   0.9999999983783873  0.9999545358830054  0.9399564062697475  0.9999999966368373  0.9983712449140605  0.9719442792574571  0.999865714705285   0.9999998401742918  0.9999933636264614  0.9340423325726832  0.9999814957642386  0.816725803660898    0.9971431756580631  0.8434728994515047   0.8258823571550262   0.9993191999962244  0.9999999984204859  0.7593172474346132   0.8732022108655354   0.9276500834543984  0.9999812601539498  0.8139554256964776   0.9999809494851027  0.9184142151440177  0.9332662957865863  0.9993922620020084  0.9971366179218428  0.9717385813909488  0.812560003699831    0.8020365653562407   0.999999841448517   0.9999932407547487  0.9911478088189112  0.9999999967023312  0.999993179317103   0.9998682061642757  0.9909651673147505  0.9999510845680807  0.9999933714752366
21.09375    0.7798500005084883   0.8160853625614438   0.9351142981615171  0.9236774709947664   0.9999999984350804  0.9999562733383227  0.9489127607301704  0.9999999966722249  0.9983862985113383  0.9724038644903016  0.9998652080179047  0.9999998415956344  0.9999934100660922  0.9337449214241781  0.9999814099677156  0.812606657329943    0.9971830604397454  0.8451810259593329   0.826703605403929    0.9992567401164063  0.9999999984754999  0.7608166690398875   0.8761919066421469   0.9290499275273875  0.9999812709502672  0.8154943598223117   0.9999815843056022  0.9168702445574837  0.9362952212957257  0.9994738160541523  0.9971607939662429  0.9721191647288118  0.8156038037521539   0.8038001389928906   0.9999998429454579  0.9999933328949258  0.9912075031046429  0.999999996737425   0.9999932683225616  0.9998678525199696  0.9910987056770734  0.9999527744967023  0.9999934180273687
21.2109375  0.7824529039634975   0.8179783962067537   0.9356301153066472  0.9245875359433753   0.9999999984769     0.9999560492105648  0.95572929391601    0.9999999967047063  0.9984015422959595  0.972830044986141   0.9998648597301592  0.9999998429644139  0.9999934779737724  0.933375514163468   0.9999813091846114  0.8073301203488963   0.9972140390379033  0.8468737175913406   0.8271797443514575   0.9992481792621375  0.9999999985163859  0.762210053218742    0.8793197207082267   0.9308128755501025  0.9999812909027149  0.8168256327067496   0.9999823090045314  0.9189185134699289  0.9391016896621639  0.9994959153455791  0.9971849753731629  0.9725317839985863  0.818184772013134    0.8058900970206273   0.9999998443862764  0.9999933977818856  0.9912475142223997  0.9999999967693033  0.9999933229773883  0.9998659836474313  0.9912462847078772  0.9999549595847144  0.9999934849009083
21.328125   0.7850350411034824   0.8198614399105568   0.9368579703201277  0.9252227198786241   0.9999999985043241  0.9999549288491828  0.959261827180739   0.999999996732051   0.998420054765711   0.9732130193985657  0.9998646319459202  0.9999998442597109  0.999993569530679   0.9331334479441931  0.9999812697703009  0.8014801694990212   0.9972353749782749  0.84855348275178     0.8273891912891543   0.9993024395924958  0.9999999985433304  0.7635343357433372   0.8825324045909648   0.9327396388089594  0.9999814038283336  0.8180037231011872   0.9999830908050584  0.9250939074755784  0.9414911300689651  0.9994498353187978  0.9972108916055611  0.9729733537640746  0.820303000907205    0.8083144143090667   0.9999998457516859  0.9999934412257503  0.9912847469925161  0.9999999967965876  0.9999933477350048  0.9998635198206948  0.9913922158164866  0.9999564166135104  0.9999935761377133
21.4453125  0.7875961466041184   0.8217644681650855   0.9387791565663091  0.9256450954902065   0.999999998518025   0.9999541070307666  0.9592974558475605  0.9999999967534563  0.9984404986864986  0.9735519832631516  0.9998643218317781  0.9999998454470193  0.999993678972166   0.9332023383646743  0.9999813561907284  0.7956524604206069   0.9972489316606279  0.850226600456904    0.8274206258689657   0.9993957752634527  0.999999998557225   0.76482805084103     0.8857738207611632   0.9346331522235523  0.9999816582689423  0.8190879501166128   0.9999837479119014  0.9349580631003589  0.943377386968599   0.9993602235038005  0.9972384313011481  0.97343255735741    0.8219816300584696   0.8110591533926257   0.9999998470096111  0.9999934690981096  0.991336097512967   0.9999999968180616  0.9999933657443957  0.9998616216463494  0.991520998175282   0.9999561304629028  0.999993686149318
21.5625     0.7901359215261625   0.8241905840955626   0.94125317657375    0.9259388740601125   0.99999999851875    0.9999540225811937  0.9565890744133313  0.9999999967733     0.9984589640210125  0.9738542943880251  0.9998639092979125  0.9999998465387687  0.9999937943198626  0.9337168009183437  0.9999816136852188  0.7904103819913437   0.9972586833442437  0.8519008071224687   0.8273692237601624   0.9994841503356875  0.9999999985586562  0.766130218454575    0.8889866090109375   0.936332456250075   0.999982040254775   0.8201394207642563   0.9999841483298     0.9471134814539063  0.9447872297155687  0.9992725833124187  0.9972660078063937  0.9738915455545499  0.8232655129115125   0.8140895253254312   0.9999998481716125  0.9999934927574062  0.9914135411707687  0.9999999968380062  0.9999934035061749  0.99986118377815    0.9916221925733001  0.9999541475159063  0.9999938013553374
21.6796875  0.7926540378140541   0.827495767355865    0.9440401270781149  0.9262008893630193   0.9999999985069582  0.9999541777549599  0.9526222722491732  0.9999999967890508  0.99847395185621    0.9741328393413446  0.9998637611084864  0.99999984756266    0.9999939005052207  0.934737902618823   0.999982047474615   0.7862498852719001   0.9972694961849583  0.8535799774592249   0.8273324126383745   0.9995255553799234  0.9999999985479563  0.7674791867910441   0.892113946214381    0.9377384552653975  0.9999824833272025  0.8212178562326843   0.9999843670266291  0.9594807632304636  0.9458430317775852  0.999230744161609   0.9972915528037798  0.9743292619996142  0.8242186320583763   0.8173521608556663   0.9999998492637853  0.9999935328855712  0.9915208160376102  0.9999999968543508  0.9999934732362563  0.9998624893637715  0.9916936595760952  0.9999516543795237  0.9999939059912513
21.796875   0.7951501417275768   0.8305962916056862   0.9468418281661569  0.926527425601827    0.9999999984828448  0.9999537550149389  0.949163479248215   0.999999996801324   0.9984881550815683  0.9744022962029917  0.9998645352044009  0.9999998485054056  0.9999939844882494  0.9362416662907428  0.9999826024033055  0.7835739898793065   0.9972856285659771  0.8552623494528334   0.8274052912157739   0.9995009931655756  0.9999999985255178  0.7689114665132334   0.8951013436360592   0.938827202601617   0.9999829058079482  0.8223784675451228   0.9999845936552162  0.9697888506991208  0.9467271707959254  0.9992556272638097  0.9973137079468495  0.9747256467561807  0.8249204012630644   0.8207784739576821   0.999999850274793   0.9999936091856123  0.991652656620937   0.9999999968673187  0.9999935704074389  0.9998652041854527  0.9917421990064879  0.9999502107032384  0.9999939886213438
21.9140625  0.7976238560734897   0.8334445474227465   0.9493543004503847  0.9269992507342497   0.9999999984472188  0.9999524922537334  0.9477151357730982  0.9999999968151281  0.9985049606980855  0.9746752288249145  0.9998668293902855  0.999999849355178   0.9999940412453234  0.9381226666067448  0.9999831674398874  0.7826757376598049   0.9973094937810306  0.8569442775700148   0.8276759304669912   0.9994237573808957  0.9999999984919047  0.7704605949535779   0.8978984188627496   0.9396486129776354  0.9999832463935193  0.8236690531494985   0.9999848649675737  0.9761519685311151  0.9476352815213771  0.9993357214379822  0.9973327146455025  0.9750657687459937  0.8254610613482464   0.8242889311152645   0.9999998511937374  0.9999937233061804  0.9917967384182453  0.9999999968818234  0.9999936840893774  0.9998686801364586  0.9917814386587019  0.9999507094517498  0.9999940460701734
22.03125    0.8000747813817187   0.8360038286022891   0.9513217532780984  0.9276682325327149   0.9999999984020289  0.9999510853591859  0.9490470385346578  0.9999999968272749  0.9985242087770749  0.9749589619454782  0.9998707900336125  0.9999998501604961  0.9999940770389203  0.9402121009895672  0.9999836148277961  0.783727415127232    0.9973411153501555  0.8586245874279765   0.8282208540178703   0.9993326336972125  0.9999999984487539  0.7721560690626351   0.9004605756454235   0.9403111178704749  0.999983476422032   0.825127482128707    0.9999850086098836  0.9775629588009672  0.94872784096985    0.9994336684042633  0.9973506295159305  0.9753429492005976  0.825936433926054    0.8277979794645796   0.9999998520662962  0.9999938544818765  0.9919376699730883  0.9999999968952711  0.999993803738939   0.999872356601718   0.9918276613660062  0.9999528343973125  0.9999940840748875
22.1484375  0.8025024963989216   0.8382503854249718   0.9525816602599377  0.9285493244833378   0.9999999983503562  0.9999507734607159  0.952956438027561   0.9999999968402186  0.9985425452248343  0.9752539715996648  0.9998759301848839  0.9999998509832018  0.9999941070177376  0.9423071263458297  0.9999838555981325  0.7867739158358109   0.9973784332928598  0.8603041244513259   0.8291010438574326   0.9992726861108608  0.9999999983991904  0.7740223845096054   0.9027505249645895   0.9409553275994964  0.9999835913600905  0.8267797080679109   0.9999848748908233  0.9741487742786709  0.9500900914426273  0.9995053125890193  0.9973707633171917  0.9755601675828071  0.8264423384081392   0.8312193398603347   0.9999998529515893  0.9999939741513284  0.992061766794239   0.9999999969094064  0.9999939159628908  0.9998760067563317  0.9918948748187753  0.999955392479422   0.9999941154357781
22.265625   0.8049065583551813   0.8401747822898186   0.9530920436693706  0.9296203197787228   0.999999998296723   0.9999524151788732  0.9583368041290178  0.9999999968580764  0.9985577487033241  0.9755541353652448  0.9998812824943552  0.999999851839057   0.9999941481261478  0.9442061798359979  0.9999838773478833  0.7917290646790875   0.9974182967132246  0.8619814145345179   0.8303588139972282   0.9992733379804816  0.9999999983475341  0.7760782158459961   0.9047395844385223   0.9417217999470766  0.9999836010591716  0.8286384317939886   0.9999845558961125  0.967106955416788   0.9517087614228078  0.9995215168202993  0.9973965476634481  0.9757294440769655  0.827069001013075    0.8344713422155787   0.999999853865559   0.9999940642885462  0.9921611379469794  0.9999999969284986  0.9999940025538637  0.9998796942491699  0.9919906348245336  0.9999572316989602  0.9999941558622317
22.3828125  0.8072865034484827   0.841782467107394    0.9529355632926584  0.9308291348968041   0.9999999982460502  0.9999557713830162  0.9635338018136572  0.9999999968767193  0.9985714443147278  0.975848723407137   0.9998858099641958  0.9999998527460692  0.9999942115160804  0.9457438457854281  0.9999837417944751  0.7983754384624081   0.997457729430504   0.8636502839820018   0.8320158284488482   0.9993345762985707  0.9999999982986065  0.7783357691075142   0.9064087006118171   0.9427191513014955  0.9999835327969735  0.8307024973994269   0.9999843141151771  0.9583441644453584  0.9534706425660522  0.9994819234318497  0.9974302588811099  0.9758693653513969  0.8278957883566703   0.8374819730130321   0.9999998548249878  0.9999941223252106  0.9922358632271977  0.9999999969487124  0.9999940504336924  0.9998835034872557  0.9921138753717753  0.9999580295234985  0.9999942176555335
22.5        0.8095810637055      0.8430935071895      0.9522997250692     0.9321059403862      0.9999999982029     0.9999595583761     0.966871154527      0.9999999968996     0.9985869155409     0.9761255921809     0.9998888440701     0.9999998537758     0.9999942981263     0.9468188526182     0.9999835488247     0.8063701192939      0.9974949735567     0.8653026640755      0.8340724205257      0.9994278595192     0.9999999982569     0.7808003336914      0.9077491489496      0.9439987293082     0.9999834408063     0.8329570666186      0.9999842629022     0.9499341720106     0.9551834976264     0.9994140871479     0.9974720867019     0.9760013382327     0.8289865825126      0.8401933197536      0.999999855897      0.9999941541112     0.9922936509216     0.9999999969728     0.9999940656359     0.9998872553586     0.9922553653987     0.9999583700573     0.9999943034088
22.6171875  0.8117646949475035   0.8451742927596889   0.9514374672158519  0.9333764535059794   0.9999999981707809  0.9999622490920934  0.9671732603223219  0.9999999969284683  0.9986049368009968  0.976374765946931   0.9998903091978795  0.9999998549893756  0.9999943984009105  0.9474100288394824  0.9999833951289558  0.8152599605562456   0.997529926146581   0.866933203954933    0.8365082243461913   0.999510914219834   0.9999999982258749  0.783470054828939    0.9087628785634176   0.9455408531499856  0.9999834031198567  0.8353745736570156   0.9999842382517756  0.9435682809287103  0.956615002867713   0.9993587967927167  0.99751990621195    0.9761454250092892  0.8303860804598266   0.842565132862255    0.9999998571388693  0.9999941691884618  0.9923471489251633  0.9999999970026623  0.9999940736790409  0.9998904259104596  0.9924006028389714  0.9999591009975612  0.999994403668703
22.734375   0.8136913559194147   0.8478243847630638   0.9506164243246661  0.9345733286050417   0.9999999981523521  0.9999630356235437  0.9641197404400839  0.999999996958598   0.9986229123224124  0.9765915133068778  0.999890627693123   0.999999856381135   0.9999944955845331  0.9475773140950408  0.9999833517828268  0.8245094619513175   0.9975638383627692  0.8685402068573671   0.8392839901562784   0.999548786256302   0.9999999982080466  0.786335941546378    0.9094624821172608   0.9472554145876971  0.9999834953728086  0.8379164180924508   0.9999840449143322  0.9401627009632367  0.9575415512130967  0.9993492128161128  0.9975698468289834  0.9763166863083504  0.8321172502785062   0.8445772791930031   0.9999998585450995  0.9999941835595725  0.992409793960506   0.9999999970337979  0.9999941026652087  0.9998923476134061  0.9925342210130902  0.999960520048783   0.9999945008640362
22.8515625  0.8156571222823635   0.850429911453559    0.9500673764490226  0.9356439995005589   0.9999999981496264  0.9999622557238983  0.9583341715952662  0.9999999969939328  0.998638268885426   0.9767781581422005  0.999890393309038   0.9999998579407243  0.9999945720141782  0.9474477842940235  0.9999834626973299  0.833541903070263    0.9975984435746659  0.8701222550182465   0.8423443526450737   0.9995306211183865  0.9999999982054502  0.7893821178981919   0.9098707853692222   0.9489968323672806  0.9999837538060335  0.8405353099230808   0.9999837370821878  0.9397214236829735  0.9577966606540248  0.9993943253037311  0.9976174558866254  0.9765228195858601  0.8341801120135445   0.8462309280274402   0.9999998601044006  0.9999942195454761  0.9924915099983622  0.9999999970694627  0.9999941654803143  0.9998925822983824  0.9926445433466133  0.9999620545207996  0.9999945775470227
22.96875    0.8177386959675438   0.8529550127886516   0.9499422748026266  0.9365545718118585   0.9999999981635196  0.9999610040490258  0.9512008566290507  0.9999999970337734  0.9986516449150227  0.9769441744336703  0.9998900310016117  0.999999859698593   0.9999946170144812  0.947189546274932   0.9999837474638414  0.8417905108428266   0.9976349038690048  0.871675296782482    0.8456212707333337   0.999474022163489   0.9999999982189196  0.7925863171458462   0.9100200653494601   0.9505905139274985  0.9999841538792258  0.843178140065186    0.9999835886017234  0.9414643185469164  0.9573089166484335  0.999474850773675   0.9976590390829242  0.9767635578674836  0.8365519351439493   0.8475483864815125   0.9999998618442969  0.9999942941551414  0.9925955842921539  0.9999999971093789  0.9999942574166001  0.9998912345591164  0.9927269075165102  0.999962690941268   0.9999946242480797
23.0859375  0.8199334308104201   0.8553648817309311   0.9502898961077849  0.9372900950723307   0.9999999981938142  0.9999602648053624  0.9444821753294638  0.9999999970739533  0.9986659278300948  0.977104529098973   0.9998896496656707  0.9999998616575302  0.9999946329171913  0.9469786960951507  0.9999841945781144  0.8487527057557799   0.9976730271504821  0.8731941213984434   0.8490378570858518   0.9994151786036001  0.9999999982480924  0.7959206094688008   0.9099509204261742   0.9518656841216132  0.9999846218651902  0.845789214410431    0.9999837609048392  0.9441546323545561  0.9561217316861025  0.9995533357266876  0.9976927038098525  0.9770318631429661  0.8391888611700845   0.8485715822639867   0.9999998637670785  0.9999944047346232  0.9927176609328939  0.9999999971493563  0.9999943663466513  0.9998890009413319  0.9927847852515157  0.99996181180418    0.999994643336878
23.203125   0.8222365963056854   0.8576275260456951   0.9510532424004228  0.9378520233706094   0.9999999982385446  0.9999602420756106  0.9398541407182208  0.999999997118725   0.998682541719529   0.9772766559668592  0.9998891545873753  0.9999998637545714  0.9999946351734303  0.946965830450594   0.9999847495325455  0.854038879937409    0.9977110882232625  0.8746763882305941   0.8525123502933427   0.9993900395192146  0.9999999982911875  0.799352343679962    0.9097108290730269   0.9526881722625239  0.9999850717397714  0.8483136624731303   0.9999841025830178  0.946516202566911   0.9543911148512325  0.9995931514271739  0.9977187515406699  0.9773165291105121  0.8420288721644852   0.849359275011152    0.9999998658135133  0.9999945285594627  0.9928471241321838  0.999999997193026   0.999994480144145   0.9998869064044772  0.99282841660003    0.999959773988192   0.9999946481758509
23.3203125  0.8245685478847942   0.8597154181127712   0.952088220128237   0.9382541379388347   0.9999999982942427  0.9999603771020827  0.9384848009919311  0.9999999971654259  0.9986997372697765  0.9774767668853833  0.9998885116273852  0.9999998659355201  0.9999946456662933  0.9472485819060973  0.9999853161981924  0.8574086622543977   0.9977463383436544  0.8761241833729548   0.8559620325665172   0.9994161139031937  0.9999999983446394  0.8028452731064045   0.9093524463232148   0.9529867760815572  0.999985441058386   0.8507008156303665   0.9999843609957056  0.9476217344977584  0.9523621777303203  0.9995767703468905  0.9977393138867643  0.9776055043896134  0.8449959388708812   0.8499831588228544   0.9999998679318345  0.9999946388490075  0.9929704059075358  0.999999997238425   0.9999945853018382  0.9998858668619589  0.9928714398226578  0.9999577636622807  0.9999946584732439
23.4375     0.8268983931456563   0.8616069465105876   0.953198902448425   0.9385184913696812   0.99999999835655    0.9999599882301687  0.9407554302006563  0.9999999972112438  0.9987149438190063  0.9777163589052187  0.9998879581462188  0.9999998681931938  0.9999946829177625  0.9478553330327125  0.9999857837012562  0.8587910724456187   0.9977760091150125  0.8775413441077751   0.859306943956075    0.9994839348985625  0.99999999840415    0.806360826224375    0.9089316975106749   0.9527681298621687  0.9999857036938062  0.8529073469577875   0.9999845008451125  0.9471544059641438  0.9503292772975688  0.9995142736644875  0.9977574074320312  0.97788911580355    0.8480051000927312   0.8505230939691313   0.9999998701142437  0.9999947212273125  0.99307517206995    0.9999999972827437  0.9999946661899125  0.9998863185965812  0.9929265960472     0.9999570223853375  0.9999946929978938
23.5546875  0.8292767401980449   0.8632875928146758   0.9541816351202038  0.9386726972734191   0.9999999984213206  0.9999589859597163  0.9461820377115661  0.9999999972602823  0.9987280154063846  0.9779997258716931  0.9998879964873667  0.9999998705001403  0.9999947545804168  0.9487429491468466  0.9999860694703749  0.8582879665468756   0.9977984169641136  0.87893009593422     0.8624732818532614   0.9995622195460173  0.9999999984658322  0.8098594747509276   0.9085057375585569   0.9521173041556147  0.9999858593325246  0.8548999756498145   0.9999847419566533  0.945484625548558   0.9485876911654617  0.9994380669820251  0.9977757874877353  0.9781624249816071  0.850968155257326    0.8510617698341166   0.9999998723358768  0.9999947758933074  0.9931540908095957  0.9999999973294866  0.9999947133655134  0.9998881106989714  0.9930018427487274  0.9999580208929051  0.9999947617060277
23.671875   0.8316967976543531   0.8647507704505129   0.9548689077350472  0.9387490717158812   0.999999998485051   0.9999580734188636  0.9535486335156237  0.9999999973086885  0.9987412769003052  0.9783230405269702  0.9998891550236227  0.9999998727872608  0.9999948552295282  0.9498085541415239  0.999986152713794   0.8561613440806408   0.9978137302723531  0.8802911411808573   0.8653964007446292   0.9996138667326121  0.9999999985262126  0.8133021463134368   0.9081308502617249   0.951184163984159   0.9999859184117095  0.8566575634539716   0.9999852672606201  0.9435494369514412  0.9473860215124783  0.99938668953768    0.9977960348599129  0.9784261778494306  0.8537996009453094   0.8516791485517704   0.9999998745327681  0.9999948091505835  0.9932070393766698  0.9999999973756564  0.9999947338015458  0.9998906962959062  0.9930980522005187  0.9999601914387923  0.9999948608617393
23.7890625  0.8337818161526124   0.8659982794090108   0.9551644102605071  0.9387851141313815   0.9999999985452953  0.9999582624876392  0.9612199120451309  0.9999999973555279  0.9987565019558223  0.9786751976273173  0.9998916611022272  0.9999998750138247  0.9999949690792642  0.9509126382534429  0.9999860773588901  0.8528052876765487   0.9978240887914089  0.8816267162182428   0.8680233451066602   0.9996146202343537  0.9999999985830282  0.816651625893781    0.907860365322471    0.9501582381507224  0.9999858998573078  0.8581724670384393   0.9999859643121235  0.9425666182729192  0.9468879956182478  0.9993859305241042  0.9978181980965973  0.9786861611055463  0.8564224153506378   0.8524470642716656   0.9999998766655889  0.9999948285826563  0.993241102093375   0.9999999974199467  0.9999947490805283  0.9998934837471054  0.9932089579830327  0.9999623982959209  0.9999949743642429
23.90625    0.8358034934416867   0.8670403503799196   0.9550627879992172  0.9388241312726172   0.9999999986002734  0.9999600814055766  0.9675606728663335  0.999999997403775   0.9987727409856242  0.9790401946357304  0.9998952365006234  0.9999998771845633  0.999995075870518   0.951908670918532   0.9999859256693805  0.8487044310471515   0.9978330133528164  0.8829423693241149   0.8703148609735265   0.9995650769603516  0.9999999986346196  0.8198738901760851   0.9077426712795141   0.9492369671647242  0.9999858406449798  0.8594510560227258   0.9999865553178648  0.9436554314215742  0.9471502124083101  0.9994366144963953  0.997841085327989   0.9789511759992188  0.858773295413496    0.8534243525063008   0.9999998787385859  0.9999948462496571  0.9932683979962594  0.9999999974652538  0.9999947805468743  0.9998961273610671  0.993323309398704   0.9999637685694984  0.9999950810324828
24.0234375  0.837794846867777    0.8678952733548716   0.954650777985646   0.9389147344911617   0.9999999986487328  0.9999630951476576  0.9713582759133039  0.9999999974496451  0.99878768301555    0.979400477878176   0.9998991568693819  0.9999998792866451  0.9999951583211563  0.952673287657723   0.9999857835085412  0.8443835393048931   0.9978443149990572  0.8842449971855068   0.872246841110642    0.9994902145113753  0.9999999986799987  0.8229393217261098   0.9078194001179729   0.9485933111411147  0.9999858001262789  0.8605133588221494   0.9999869212156924  0.9474644076121295  0.9481191946013325  0.9995149140107464  0.9978630451755139  0.9792301736257973  0.8608069806893472   0.8546528556206097   0.9999998807414291  0.9999948797308301  0.9933024360369905  0.9999999975084063  0.999994836225086   0.9998985945333857  0.9934285191574894  0.999964250369627   0.9999951637537734
24.140625   0.8397557076233195   0.8689472840092917   0.9540900059451421  0.9391083376066249   0.999999998689575   0.9999661200081527  0.9721346681778119  0.9999999974934362  0.9988006346138233  0.9797404788311358  0.9999025564982439  0.9999998812793005  0.9999952090520058  0.9531312027777106  0.9999857186646992  0.8403555565687583   0.9978609012216709  0.8855394761931484   0.8738111660289627   0.9994269113821287  0.9999999987181646  0.825823755224684    0.9081238543979879   0.9483486345597043  0.9999858446332437  0.8613918538185745   0.9999872190284217  0.9539131772508379  0.9496474969923381  0.9995847309901863  0.9978828930060885  0.9795292892050896  0.8624993543597992   0.8561545953795167   0.999999882637186   0.999994942974583   0.9933541291836685  0.9999999975493236  0.9999949116389856  0.9999010007192137  0.9935146493598501  0.999964499260918   0.9999952158207102
24.2578125  0.8416859751905951   0.8707269871772447   0.953585103569778   0.9394543230857967   0.9999999987221246  0.9999679816394494  0.9702554045809627  0.9999999975374622  0.9988132871433347  0.9800495455521857  0.9999048101952487  0.9999998831520185  0.999995233900286   0.9532701987034147  0.9999857766007016  0.837076179116425    0.9978839123679758  0.8868278557228058   0.8750159107245619   0.9994058575935765  0.999999998748562   0.8285093151475605   0.9086797401600987   0.9485556084897823  0.9999860168171495  0.8621294782741505   0.9999876611722346  0.9621332491367658  0.9515252320442628  0.9996157457320147  0.9979005955788629  0.9798495312683089  0.8638490918527708   0.8579303250800007   0.9999998844155809  0.9999950345475082  0.993428678607431   0.9999999975904059  0.9999949993234435  0.9999033543902143  0.9935775034900647  0.9999652239452123  0.9999952424671188
24.375      0.843484481851       0.8724567674021      0.9533434220626     0.9399930805747      0.9999999987459     0.9999682555774     0.9667950075806     0.9999999975781     0.9988274732563     0.980323635175      0.9999057787793     0.9999998849282     0.9999952493038     0.953143573553      0.9999859813183     0.8349108467904      0.997912492734      0.8881116601835      0.8758848973139      0.9994368822575     0.9999999987711     0.8309850126346      0.9095002588188      0.9491938610468     0.9999863110395     0.8627769779904      0.9999882138555     0.9706423688252     0.9535207494283     0.9995980918233     0.9979174365975     0.9801857239674     0.8648777259781      0.8599595779135      0.9999998860985     0.9999951373729     0.9935242484504     0.9999999976284     0.9999950931822     0.999905406067      0.9936198931377     0.9999665333992     0.9999952582784
24.4921875  0.845087838083095    0.8741355092725115   0.9535340667006931  0.9407477719996389   0.9999999987613122  0.9999674532091746  0.9631908897867046  0.9999999976168059  0.998842882666475   0.9805654360573915  0.999905790339277   0.9999998866100935  0.9999952746731702  0.9528598678471552  0.9999863288189959  0.8341161270305042   0.9979442595366995  0.8893939375640235   0.8764565877058267   0.999504834219307   0.9999999987859689  0.8332470782792192   0.9105875992102107   0.9501786798843161  0.9999866743397078  0.8633897634886991   0.999988621557379   0.9777174736802735  0.9554227485309272  0.9995465037752554  0.9979356002576927  0.9805269949150365  0.8656281077794272   0.8622022218928398   0.9999998876910116  0.999995231458589   0.9936327871591896  0.9999999976644562  0.9999951844573018  0.9999067190003195  0.99365072692545    0.9999678073679581  0.9999952817029172
24.609375   0.8466062266579866   0.8757625829491789   0.9542540369090655  0.941716509232176    0.9999999987686639  0.999966546404842   0.960789910472342   0.9999999976548125  0.9988576033207145  0.9807829628337308  0.9999053901874474  0.9999998881787373  0.9999953237469973  0.9525620735169636  0.9999867756641713  0.834833735240534    0.9979762722356421  0.8906782526947115   0.8767823276843214   0.9995777807087454  0.9999999987935605  0.8352990170019317   0.9119328572378783   0.951380684787912   0.9999870331703328  0.86402447052708     0.9999887177949668  0.981862288439791   0.9570760343722697  0.9994921628381573  0.997957342995297   0.9808587167505822  0.8661613573035861   0.8646014225110443   0.9999998891766624  0.9999953060578387  0.9937426911356009  0.999999997700025   0.9999952597774491  0.9999069265888246  0.9936822427486068  0.9999681874110642  0.9999953277394317
24.7265625  0.8480471351652543   0.8773378426170683   0.9555079210446039  0.9428673153947785   0.9999999987683315  0.9999662220255409  0.960432573706201   0.9999999976892987  0.9988706294883232  0.9809870246874713  0.9999050266979157  0.9999998896425556  0.9999953990250581  0.9524004283549266  0.9999872385754143  0.8370933052469451   0.9980061247099248  0.8919659010927173   0.8769239737132284   0.9996231719986551  0.9999999987943627  0.8371513807298672   0.9135163955177492   0.9526524645527081  0.9999873267290109  0.8647354418682993   0.9999886244873938  0.9822260701472474  0.9584044507644045  0.9994665947306979  0.9979840852161812  0.9811654520036975  0.8665525095786606   0.8670878107385028   0.9999998905623482  0.9999953601334295  0.9938424539579979  0.9999999977325674  0.9999953089286737  0.9999060126102738  0.9937262649850477  0.9999672947536717  0.9999954009993385
24.84375    0.8494205956417109   0.8788616155336015   0.9572050812545664  0.9441381573447868   0.9999999987605789  0.9999664412683554  0.9622088978005484  0.999999997722275   0.9988830446575     0.9811882208691516  0.9999048656176969  0.9999998910337844  0.9999954909934321  0.9525039997809422  0.9999876175391742  0.8408198241029414   0.9980327507184453  0.893254872041979    0.8769509588930203   0.9996237972512734  0.9999999987884266  0.8388212643432719   0.9153086387348899   0.953857028147604   0.9999875253678211  0.8655713523908874   0.9999886062540039  0.9788412325216344  0.9594170146908657  0.9994858037035407  0.9980157845999508  0.9814342131868984  0.8668851599392031   0.8695845635250304   0.9999998918793633  0.9999953957527242  0.9939241653895757  0.9999999977635985  0.9999953349525196  0.9999044273618766  0.9937906472362539  0.9999656053673133  0.9999954927176492
24.9609375  0.8507388407896295   0.880334682128817    0.9591749908047669  0.9454431286101502   0.9999999987449125  0.9999666173134287  0.9654606371632577  0.9999999977542467  0.9988965144417243  0.9813942093897787  0.9999048301263317  0.9999998923622435  0.9999955817313718  0.952956392957461   0.9999878360136157  0.8458428878301242   0.9980566567201418  0.8945417639130135   0.8769368886046446   0.999585664571718   0.9999999987753359  0.8403315400327431   0.917271282704793    0.9548927346740802  0.9999876279584738  0.8665721907999949   0.9999887325611406  0.9726068690858672  0.9601965067153838  0.9995425183213934  0.998050841046351   0.9816572822551208  0.8672454903157143   0.8720130344685648   0.9999998931393155  0.9999954163061126  0.9939857977839593  0.9999999977941251  0.9999953535738613  0.9999029497893284  0.9938769796136258  0.9999641818237431  0.9999955842038324
25.078125   0.8520158628393963   0.8817582479434195   0.9611979159164274  0.946683904769327    0.9999999987213554  0.999966232223871   0.9690213421502376  0.9999999977828125  0.9989111574109609  0.9816079064918779  0.9999048122549558  0.9999998936196939  0.9999956518038325  0.9537800799488122  0.999987874860296   0.8519068316528884   0.9980795234789898  0.8958241310112266   0.8769558024549458   0.9995341282608706  0.9999999987547867  0.8417098551367601   0.919358876874646    0.9557101512764746  0.9999876497541358  0.8677667891452514   0.9999887923762075  0.9650284291587492  0.9608731818143784  0.9996102233200275  0.9980865752005182  0.9818339382317981  0.8677161075389022   0.8742985292654107   0.9999998943348377  0.9999954312230284  0.9940316132384214  0.9999999978219135  0.9999953812562802  0.9999023605005052  0.9939802308437589  0.999963959401051   0.9999956554420879
25.1953125  0.853266890846544    0.8831339084584303   0.9630451052372674  0.9477638636089171   0.9999999986905187  0.9999653766292241  0.9716088117746062  0.9999999978104699  0.998925502740945   0.9818270284468309  0.9999048912377699  0.999999894823575   0.9999956886847537  0.9549322023625082  0.9999877780098985  0.858682650765741    0.9981033599968078  0.8971004249574676   0.8770782936773206   0.9995004954264022  0.9999999987273323  0.8429874286053594   0.9215207224377558   0.956317889679402   0.9999876159349153  0.8691710576128953   0.9999885831289358  0.9578005669050097  0.9615892229236553  0.9996567261536047  0.9981201004063799  0.9819706941653665  0.8683701412504748   0.876375807775995    0.9999998954814279  0.9999954568784416  0.9940706152655991  0.9999999978489688  0.9999954248902336  0.999903105600102   0.99409039178392    0.9999651228415571  0.9999956942909367
25.3125     0.8545078063111687   0.8844636080500937   0.9645206744714563  0.9486016490753062   0.9999999986546062  0.9999647699916313  0.9722393193731937  0.99999999783705    0.998938446575825   0.9820450439339188  0.9999053949248875  0.9999998960106999  0.9999956933955687  0.9563125479303938  0.9999876274783249  0.8657839244507001   0.998129558522175   0.8983679897267187   0.8773677287076125   0.999506008435675   0.9999999986949499  0.8441976912305376   0.9237030121497688   0.9567765974756938  0.9999875640604188  0.870787038605875    0.9999881895899313  0.9523646396093625  0.9624614674705312  0.9996603280686625  0.9981492590351125  0.9820799936028375  0.8692660313237812   0.8781939125630812   0.9999998966146     0.9999955073130062  0.9941135694085312  0.99999999787555    0.99999548290815    0.9999051355541437  0.9941955767575374  0.9999670329150062  0.9999957018252937
25.4296875  0.8557545204178888   0.8857495943982497   0.96549691503398    0.9491416038585365   0.999999998616497   0.9999652377911299  0.9705341623072318  0.9999999978606366  0.998950574681274   0.9822532524988034  0.9999067486125737  0.9999998971941508  0.9999956813790876  0.9577821255854009  0.9999875063872944  0.8727891990294094   0.9981582266125472  0.8996220635879341   0.8778768408557049   0.9995512385426256  0.9999999986603918  0.8453748239632772   0.9258511241598684   0.9571825459243367  0.999987544843584   0.8726028434987612   0.999987911109433   0.9495702631763089  0.9635503503829156  0.9996205259131291  0.9981732745397447  0.9821776789012908  0.870443384345304    0.8797199707926306   0.9999998977490804  0.9999955829799642  0.9941695443230002  0.9999999978998313  0.9999955524911823  0.9999079963711944  0.9942856085659721  0.9999687367750227  0.9999956920192364
25.546875   0.8570223386655154   0.8869943697182372   0.9659359939478749  0.9493597661908175   0.9999999985799937  0.9999670527934358  0.9668328450390268  0.9999999978839311  0.9989632560369696  0.9824434334932073  0.9999092001759062  0.999999898370024   0.9999956760707184  0.9591887750283661  0.9999874733858153  0.879272082646441    0.9981880540623429  0.9008573011270429   0.8786449668027199   0.9996160562132332  0.9999999986272321  0.8465522561461342   0.9279119716112626   0.9576442463348057  0.9999876098905658  0.8745934807499997   0.9999879031485773  0.949526546192689   0.9648417651206872  0.9995580494742605  0.9981928846642224  0.9822798232611768  0.8719202005469573   0.8809416883107941   0.9999998988809436  0.9999956703910114  0.9942430514958965  0.999999997924074   0.9999956313330852  0.9999111010147681  0.994354989284084   0.9999696565633874  0.9999956864199412
25.6640625  0.8583253399795459   0.8882006402392423   0.9658937614616829  0.9492652371720741   0.9999999985483624  0.9999696562719907  0.962092462955788   0.9999999979058142  0.9989768331730792  0.9826103851441017  0.999912597937294   0.9999998995563812  0.9999956981730097  0.9603949173073543  0.9999875578526571  0.884837770428705    0.9982167582724721  0.9020700224693057   0.8796961457049058   0.9996699776549759  0.9999999985986469  0.847761190544915    0.9298363034575258   0.9582568651524458  0.9999877884769466  0.8767225279891727   0.9999879929567966  0.9516648673736028  0.9662458449575307  0.9995044256574146  0.9982099227850594  0.9823996481505533  0.8736916729437374   0.8818683466087043   0.9999999000266955  0.9999957532333951  0.9943326553812137  0.9999999979475201  0.9999957127508523  0.9999140102929691  0.9944043726295562  0.9999699332368178  0.9999957056384405
25.78125    0.8596757974715844   0.8893712654073219   0.9655050919977711  0.9488973735209125   0.9999999985246749  0.9999719827591843  0.9576147236457524  0.9999999979250235  0.9989902512675226  0.9827537166827038  0.9999163724270297  0.9999999007828391  0.9999957564154891  0.9613021491709672  0.9999877668072233  0.8891618232393663   0.9982419268245742  0.9032585769769491   0.8810382144084781   0.9996875400110281  0.9999999985774001  0.84902922397066     0.9315808507742711   0.9590796246286141  0.9999880694196609  0.8789445463857781   0.9999879288660899  0.9549755198897195  0.9676130550808578  0.9994864331270539  0.998226525074357   0.9825452083940617  0.8757306472568077   0.8825302186069282   0.9999999012156641  0.999995822515446   0.9944314562397985  0.9999999979687485  0.9999957842116328  0.9999165630570375  0.9944401315963766  0.9999701880391195  0.9999957604593851
25.8984375  0.8610836668755463   0.8905092083477701   0.9649540527423301  0.9483196577301353   0.9999999985113625  0.9999731504882823  0.9546842117891072  0.9999999979442813  0.9990024910031048  0.9828784489787791  0.9999197511998484  0.9999999020512003  0.9999958442552429  0.961868004339699   0.9999880885163227  0.892025236198935    0.9982619270390443  0.9044217738337184   0.8826629260909771   0.9996609740888996  0.9999999985659612  0.8503791304071178   0.9331102172926476   0.9601207737323546  0.9999884039036628  0.8812080857130429   0.9999877195439555  0.9583398206901479  0.9687642409231358  0.9995132156626433  0.9982442798304151  0.9827183349885411  0.8779897114938479   0.8829764316344282   0.9999999024505125  0.9999958765250476  0.9945292643991233  0.9999999979901499  0.9999958352339967  0.9999188036508176  0.9944722052081164  0.9999709218597821  0.9999958464654578
26.015625   0.8625561664413057   0.8916174890839206   0.9644353505239336  0.9476114861379874   0.999999998510759   0.9999730066188675  0.9542176389203967  0.9999999979615021  0.9990138200936283  0.9829942723534696  0.9999220939817378  0.9999999033437446  0.9999959432593762  0.9621117469077562  0.9999884880391843  0.8933400574913356   0.9982765334638016  0.9055577604349705   0.8845470167641946   0.9996039418474694  0.999999998566457   0.8518278675898752   0.9343984228065824   0.9613331962073554  0.9999887295622114  0.883459086692952    0.9999876383984744  0.9608625504832717  0.9695282434297735  0.9995723336919711  0.9982636490745143  0.982915021097905   0.8804047658131314   0.8832714182218918   0.9999999037132946  0.9999959154304339  0.9946157684190124  0.9999999980099014  0.9999958668899088  0.9999207877450398  0.9945109279414612  0.9999720457593136  0.9999959455547908
26.1328125  0.8640974686726087   0.8926991418267336   0.9641152329743692  0.9468594129616262   0.9999999985242503  0.9999721369527509  0.9565146305781214  0.9999999979767311  0.999025299052905   0.9831136352606702  0.9999231620910913  0.99999990466005    0.9999960310379731  0.962108232079634   0.9999889042683519  0.8931616073581276   0.9982870391898703  0.9066649148867707   0.8866540643735613   0.9995446566956702  0.9999999985802566  0.8533858580053195   0.9354300231793252   0.9626215818204322  0.9999889990850528  0.885644456723356    0.9999878827803584  0.9621166757436117  0.9697790635724522  0.9996364779906276  0.9982838783729222  0.9831270883802102  0.8828998137729552   0.883490200535528    0.999999905002419   0.999995940673336   0.9946837076979645  0.9999999980277933  0.9999958912629384  0.9999224204661816  0.9945638378710908  0.9999729184593693  0.9999960343522296
26.25       0.8657085197439      0.8937571772952      0.9640999328597     0.9461493087018      0.9999999985519     0.9999713629591     0.9611720652483     0.9999999979924     0.9990373449726     0.9832491167408     0.9999231759632     0.9999999060056     0.9999960906166     0.961971235649      0.9999892610438     0.8916853820753      0.9982958185929     0.907743534088       0.8889369404637      0.9995119871604     0.9999999986072     0.8550565819358      0.9362007502004      0.9638597504741     0.9999891938336     0.8877155789335      0.9999883179008     0.962238858004      0.969464703973      0.9996772707769     0.998303415942      0.9833446702389     0.8853926234201      0.8837128447173      0.9999999063231     0.9999959595641     0.9947310773598     0.9999999980459     0.9999959210455     0.9999234490923     0.9946334408522     0.999972859089      0.999996095222
26.3671875  0.8673869968756905   0.8947945505514304   0.9644179066380675  0.945560323252671    0.999999998591887   0.999971131520013   0.9671841972446042  0.9999999980058122  0.9990492539998804  0.9834107023973198  0.9999226415031431  0.9999999073598189  0.9999961178570612  0.9618297337430166  0.9999894931996816  0.8892282041423156   0.9983055179010745  0.9087960827393936   0.8913406451280249   0.9995215296031833  0.9999999986457437  0.8568365043947666   0.9367176375337986   0.9649146634106452  0.9999893168864683  0.8896315143363718   0.9999886560483235  0.9618531111841947  0.968621671097576   0.9996791728872363  0.9983206631279896  0.9835588754464448  0.8878008384716016   0.8840184898343341   0.9999999076551684  0.9999959856735023  0.9947617079529691  0.9999999980620495  0.999995961138818   0.9999236207826463  0.994716591189007   0.9999717280177408  0.999996124017755
26.484375   0.8691274076776002   0.8958141343073829   0.9650190026709686  0.9451607280526754   0.9999999986410408  0.9999712633887815  0.9732061616857985  0.9999999980188375  0.9990602173868374  0.9836036013063694  0.9999220604930364  0.9999999086881194  0.9999961235598505  0.9618019771341922  0.9999895734238863  0.8861945716756815   0.9983181851638325  0.9098257243999813   0.8938053334858804   0.999568493212998   0.9999999986925611  0.8587153404661154   0.9369986242571237   0.9656723068050247  0.9999893775263854  0.8913616733672167   0.9999888145428663  0.9618440331584154  0.9673722236141973  0.9996466392095518  0.9983347514335973  0.9837639838586933  0.8900480824938649   0.884479396226249    0.9999999089640159  0.999996030314184   0.9947840719961913  0.9999999980774249  0.9999960102316826  0.9999228989097386  0.9948056380060505  0.9999701155278252  0.9999961310880163
26.6015625  0.8709213288740443   0.8968186971829063   0.9657900448040211  0.9450050624068058   0.9999999986953201  0.9999712282187421  0.97791495908745    0.9999999980328623  0.9990703770496195  0.9838271093691605  0.9999217001695003  0.9999999099741844  0.9999961285032106  0.9619725658684973  0.9999895223956909  0.8830314053004856   0.9983346631921766  0.9108350334921778   0.896269373217409    0.9996307222027287  0.9999999987439876  0.8606766458982049   0.9370716533018      0.9660602715744372  0.9999893842500873  0.8928877658177518   0.9999889973097582  0.9630429695726784  0.9659050737094196  0.9996011724598945  0.9983460340878254  0.9839586635760637  0.8920695964291302   0.8851554652683115   0.9999999102328657  0.9999960932319344  0.9948087134863453  0.9999999980934935  0.9999960669052788  0.9999215862031755  0.9948909418927736  0.9999689828376782  0.9999961359673991
26.71875    0.8727577750729789   0.8978108860275891   0.9665829421119094  0.945131800346975    0.999999998751075   0.9999706971045813  0.9803697955664352  0.9999999980454266  0.999080546859918   0.9840747750489578  0.9999215484032945  0.9999999112130804  0.9999961533466282  0.962377022040182   0.9999893964736805  0.8801774116832789   0.9983544577029047  0.9118263481494101   0.8986723056443422   0.999680102591343   0.9999999987964289  0.8626987028006344   0.9369733073299703   0.9660625501398016  0.999989351942164   0.8942048858639258   0.9999894154417117  0.965919900578539   0.964443594062132   0.9995700036203188  0.9983560943214351  0.9841459491811023  0.8938169828699539   0.8860896547914954   0.9999999114571343  0.9999961633719563  0.9948451079194062  0.999999998108025   0.9999961303877336  0.9999202581121234  0.9949639554056242  0.9999690213938593  0.9999961586442118
26.8359375  0.8746236816359325   0.8987932112410756   0.9672490296005702  0.9455609667673684   0.9999999988053672  0.9999699130303462  0.9802660872965142  0.9999999980590907  0.999091108696239   0.9843358145788934  0.999921457754411   0.9999999123856481  0.9999962086818144  0.9629967455769788  0.9999892633825725  0.8780155520415558   0.9983761046651586  0.9128029282162496   0.9009576105674637   0.9996965104810047  0.9999999988470141  0.8647556553351781   0.9367470476013099   0.9657236697869042  0.9999893125771319  0.8953216485019246   0.9999899950729833  0.9703789229974686  0.9632079648801294  0.999572227705417   0.9983672621045031  0.984332030403908   0.8952616964584379   0.887304652710227    0.9999999126189777  0.9999962288583358  0.9948989160169472  0.9999999981232125  0.99999619585553    0.999919529333829   0.9950199062713677  0.9999702081214408  0.9999962108372593
26.953125   0.8765044796067745   0.8997680340979296   0.9676723442204648  0.9462915590526202   0.9999999988562115  0.9999695732567606  0.97800303383973    0.9999999980739804  0.9991015860051312  0.9845974308509747  0.9999213631538846  0.9999999134666171  0.9999962904857748  0.9637650741563166  0.9999891822879912  0.8768368623125042   0.9983978693038477  0.9137690563423481   0.9030751954032832   0.9996773539046451  0.99999999889405    0.8668188381288342   0.9364411411810186   0.9651414393482145  0.9999893136294205  0.8962593615745201   0.9999904617997536  0.9757293825821038  0.9623786753527074  0.9996093418258802  0.9983818210089958  0.9845241970058353  0.8963970137533078   0.8888010795923912   0.9999999136941912  0.9999962846997368  0.9949704524131359  0.9999999981391375  0.9999962534456043  0.9999197832158865  0.9950592650145537  0.9999718780752088  0.9999962905702937
27.0703125  0.8783847356564508   0.9007375554098789   0.9677950631323846  0.9472990948399693   0.9999999989026115  0.9999703084424624  0.9745400184006154  0.9999999980876437  0.9991113955213858  0.9848474868647314  0.9999214176631147  0.9999999144506029  0.9999963827124231  0.964582915380445   0.9999891960622734  0.8768207138226162   0.9984184877451777  0.914728731438109    0.9049835427062436   0.9996381173888995  0.9999999989366063  0.8688582314047553   0.9361063769984094   0.964449847311418   0.9999893992812932  0.8970502885273207   0.9999906733399573  0.9808519945468693  0.9620682537071764  0.999664785539857   0.9984012037463962  0.9847284900891532  0.8972383305503558   0.8905573759381238   0.9999999146781164  0.9999963311336726  0.9950548013042394  0.9999999981539437  0.9999962954374569  0.9999210195126651  0.9950875689434194  0.9999732392429843  0.9999963827078288
27.1875     0.8802488265359563   0.9017038053348563   0.9676301742930374  0.948534041612725    0.9999999989440438  0.9999721635000938  0.9710858469902562  0.999999998103      0.999120675328025   0.985076918274875   0.999921951041075   0.9999999153488125  0.9999964648564     0.9653405408654875  0.9999893301632188  0.8780318976663      0.9984376563653625  0.91568449607645     0.9066514583028126   0.9996038894992938  0.9999999989746     0.8708439716753688   0.9357936809637063   0.9637955536004937  0.9999895861133687  0.8977351257302625   0.9999907688146688  0.9845147457475437  0.962306243560275   0.9997125136417501  0.9984254761993313  0.9849476707475563  0.8978217617390563   0.8925314002190124   0.9999999155829625  0.9999963682473313  0.9951434992358063  0.9999999981698     0.9999963240260937  0.9999229047691313  0.9951136871148062  0.9999739261039937  0.999996465900675
27.3046875  0.8820816157902304   0.9026686345677724   0.9672591675298282  0.9499222243985048   0.9999999989807179  0.9999744876373621  0.9687219103445991  0.9999999981183563  0.9991300581051263  0.9852813490186718  0.9999232753256582  0.9999999161644414  0.9999965207204994  0.965940998129077   0.9999895880801214  0.8804308110925165   0.998456083157585   0.9166376920864043   0.9080593669916804   0.9995963244005143  0.999999999008226   0.8727478453681207   0.9355517441986139   0.9633125888816351  0.9999898515755533  0.8983598789665483   0.9999909739568542  0.9857400134982295  0.9630397165426613  0.9997301519595962  0.9984532977205057  0.985180027697325   0.8982011493318094   0.8946636289083907   0.9999999164133915  0.9999963960167683  0.9952272299728326  0.9999999981856562  0.9999963498601522  0.9999249869773589  0.9951470865312815  0.9999741777048363  0.9999965228138126
27.421875   0.8837504499405875   0.9036337073586184   0.9668155541432316  0.9513683627677543   0.999999999013151   0.9999763282944184  0.9680819743292031  0.9999999981321616  0.9991397973713918  0.985461573791393   0.9999254629843958  0.9999999168915671  0.9999965450879659  0.9663204087852405  0.9999899417876879  0.8838910608464157   0.9984750940995176  0.9175895059287897   0.9092001068637412   0.9996226473309238  0.9999999990381134  0.8745446948080312   0.9354247765969579   0.9631001013994525  0.9999901440949469  0.898972371230353    0.9999913000395718  0.9841045993284231  0.9641485816067206  0.9997101127862763  0.9984823646550531  0.9854193347846067  0.8984437088753878   0.8968817204429629   0.999999917164417   0.9999964191850127  0.9952986820323587  0.9999999982000259  0.9999963826136344  0.9999269394630538  0.9951949537249968  0.99997453149531    0.9999965481635918
27.5390625  0.88504162179722     0.9046004967626209   0.9664577694962001  0.9527634179109553   0.9999999990419532  0.9999770410596516  0.9691883829183121  0.9999999981469063  0.9991494758593538  0.9856228433533362  0.9999282426057203  0.9999999175396577  0.9999965459481497  0.9664611798902413  0.9999903275881662  0.8882191756777316   0.9984959643300588  0.9185411977069518   0.910079180341829    0.9996721449063244  0.999999999065075   0.8762136712631294   0.9354504907615284   0.9632072599589344  0.9999904070438886  0.8996186442046286   0.9999915276482602  0.9798774054234086  0.9654721908881169  0.9996632018160027  0.9985101643168153  0.9856559842293202  0.8986246540527311   0.8991060918554477   0.9999999178453748  0.9999964479641307  0.9953546916166719  0.9999999982149016  0.9999964243820318  0.9999286880241893  0.9952600522333701  0.9999753022529392  0.9999965503900329
27.65625    0.8863224508270546   0.9055702823298      0.966336998629425   0.95399541542455     0.9999999990681     0.9999766694855406  0.9714881756322405  0.99999999816015    0.9991586552818618  0.9857731709532516  0.9999310841350828  0.9999999181358828  0.9999965410664016  0.9663956676707055  0.9999906596289828  0.8931729745174437   0.9985192509105657  0.919493151363257    0.9107144322064484   0.999721739963332   0.9999999990899766  0.8777392776975312   0.9356584093952149   0.9636279156942914  0.999990599911132   0.9003395258821774   0.9999914907208032  0.9739549667718486  0.9668417373931969  0.9996130847724468  0.9985347568905476  0.9858790267405101  0.8988212244169211   0.9012560704651312   0.9999999184830539  0.9999964906689477  0.9953970346710344  0.9999999982285734  0.9999964720841954  0.9999303551138211  0.9953399531892531  0.9999762729524024  0.9999965468418055
27.7734375  0.8875936047790987   0.9065599088128734   0.9665663584086504  0.954962223035439    0.9999999990922515  0.9999758311603661  0.9740662229378443  0.9999999981711737  0.9991675225051928  0.9859211037195236  0.9999334282917743  0.999999918696656   0.9999965498934339  0.9661997719337346  0.9999908613801992  0.8984773701439831   0.9985444013886703  0.9204440692474493   0.9111351430821231   0.9997477013092738  0.9999999991134576  0.8791121540199779   0.9360685730622139   0.9643056703132364  0.9999907057974545  0.9011676256526909   0.9999912830743437  0.9676204456548576  0.9681120230570607  0.999584538835912   0.9985553032657448  0.9860786221730983  0.8991065939362867   0.9032561317155126   0.9999999190930386  0.9999965459540341  0.99543166174379    0.9999999982401437  0.9999965231870797  0.9999320733357733  0.995427843680994   0.9999768455586006  0.9999965557366582
27.890625   0.888855672523351    0.90788593202731     0.9671975321676349  0.9555840030907644   0.9999999991147014  0.999975222290693   0.9759588039281819  0.9999999981823486  0.9991766057083267  0.9860735167454054  0.9999349277243474  0.9999999192219159  0.9999965845707134  0.9659779146109326  0.9999908994525222  0.9038382975526612   0.9985698028110722  0.921391466621343    0.9113805499987356   0.9997375519099293  0.9999999991357125  0.8803295693997812   0.9366907059993347   0.9651479896089749  0.9999907277409363  0.9021249862797999   0.9999911481026682  0.9622042444795046  0.9691864865706701  0.9995914188621919  0.9985721638159966  0.9862483072059511  0.8995441520295627   0.9050417236559318   0.9999999196741396  0.9999966040492996  0.9954666524033643  0.9999999982517521  0.9999965762370308  0.9999338024395189  0.9955146323094599  0.9999765309732195  0.9999965887333448
28.0078125  0.8901091618815756   0.9090006807088719   0.9682094600831742  0.9558131144304761   0.9999999991348878  0.9999751310399483  0.9764656192599833  0.9999999981909817  0.9991860009267828  0.986233964241182   0.9999355642666654  0.9999999197251943  0.9999966447798795  0.9658424935358852  0.9999907990237631  0.9089567080760423   0.9985932629182345  0.9223328861399129   0.9114978330740916   0.9996962591985062  0.9999999991563997  0.8813955992974379   0.9375238762532018   0.9660463196599174  0.9999906840291725  0.903221571734473    0.9999911809971885  0.9587485165460458  0.9700308420315167  0.9996302705792991  0.998586562086492   0.9863865391349042  0.9001826214699664   0.9065642101198089   0.9999999202380563  0.9999966555829024  0.9955095532818703  0.999999998261131   0.9999966269200344  0.9999352785110018  0.9955916830232739  0.9999753886864355  0.9999966466404523
28.125      0.8913544987099      0.9099090440293      0.9695113179542     0.9556399776117      0.9999999991523     0.9999753213051     0.975366408319      0.9999999981973     0.9991952231707     0.9864019708711     0.9999355867576     0.9999999202366     0.9999967189412     0.9658919072614     0.999990631335      0.913544874699       0.9986127470588     0.9232663346256      0.911539639053       0.9996437410681     0.9999999991745     0.8823209787446      0.9385566637786      0.9668981137526     0.9999906081335     0.904454704743       0.9999912213081     0.9577686288088     0.9706730143047     0.9996825921328     0.9985999720503     0.9864971528492     0.9010524101945      0.9077945359488      0.9999999208118     0.9999966977513     0.995564938929      0.9999999982684     0.9999966678311     0.9999361362009     0.9956533335018     0.9999740647255     0.999996719485
28.2421875  0.8925920269145773   0.9106948704715566   0.9709592047640899  0.9550943371653916   0.999999999165862   0.9999753454620763  0.9729837945608519  0.9999999982037869  0.9992038874597553  0.9865734056851478  0.9999353244462897  0.9999999207664182  0.9999967904089289  0.9661914492579081  0.9999904822419348  0.9173463405720323   0.9986270958927193  0.9241895804766014   0.9115612489627908   0.9996050746871856  0.9999999991893193  0.883122637652306    0.9397678269725549   0.9676264374771659  0.9999905484701833  0.9058094934288038   0.9999910828167331  0.9591691728891305  0.9711898284607978  0.9997244017118878  0.9986134854112927  0.9865886247405097  0.9021634921500372   0.9087253222976999   0.9999999214038499  0.9999967322554529  0.9956329484632049  0.9999999982757874  0.9999966949799391  0.99993613140012    0.9956984414470293  0.9999733869554984  0.9999967910738559
28.359375   0.8938220091677667   0.9119519844498943   0.9723832312644822  0.9542420395094328   0.9999999991754388  0.9999750086349021  0.9700821172937462  0.9999999982076605  0.9992122650144898  0.9867418092694749  0.9999350017833248  0.999999921304709   0.9999968449812073  0.9667607496725262  0.9999904213850538  0.9201588564607489   0.9986364564856917  0.9250995595319249   0.9116175371696112   0.9995983675199894  0.9999999992003409  0.8838229381712124   0.9411274351631688   0.9681934121473361  0.9999905589544286  0.9072602084305518   0.9999908072364149  0.9623253021932633  0.9716835931388801  0.9997381951828254  0.9986274103695336  0.9866723175477026  0.9035049844280055   0.9093712293465904   0.9999999220026987  0.999996760491238   0.995709235238501   0.9999999982808625  0.999996713628776   0.9999353247315227  0.99573051820425    0.9999738160798596  0.9999968464518274
28.4765625  0.8950446282094404   0.9130634831902539   0.9736195936164701  0.9531778687761768   0.9999999991812638  0.9999745984627842  0.9676376569228938  0.9999999982107013  0.999220894783812   0.9869003098317349  0.9999346686090703  0.9999999218599441  0.9999968764549235  0.9675701300891821  0.9999904865041132  0.9218574051245609   0.9986422679113063  0.9259929977165113   0.9117599071089801   0.9996262928505408  0.9999999992076237  0.8844486477927608   0.9425984130196965   0.9686050531296295  0.9999906786836404  0.9087724968140481   0.9999906298608091  0.966296367128359   0.9722537798064478  0.9997213230421006  0.9986412500070946  0.9867601200110062  0.9050464457887719   0.9097675663647238   0.999999922614333   0.9999967836857773  0.99578632294834    0.9999999982851612  0.9999967353447644  0.9999341073264956  0.9957564517537155  0.9999751577322623  0.9999968787527708
28.59375    0.896259988679582    0.9140350737574531   0.9745411731167055  0.9520153642719594   0.9999999991840711  0.9999746909284437  0.9665482930715399  0.9999999982150789  0.9992297829112718  0.9870436262911492  0.99993427626665    0.999999922456      0.9999968886267289  0.9685464849286946  0.9999906830783281  0.92241299367755     0.9986468092887032  0.9268676667150305   0.91203341518035     0.9996753488457805  0.9999999992119695  0.8850296945999476   0.9441384234395797   0.9689068042947148  0.9999909112515462  0.9103062527468578   0.9999906969600313  0.9701072371527423  0.9729700075710046  0.9996867411838063  0.9986540505015765  0.9868620323242563  0.9067407771781367   0.9099672738397383   0.9999999232597195  0.9999968067688422  0.9958559284309867  0.9999999982906734  0.9999967686731102  0.9999330455649632  0.9957842518086836  0.9999767383628797  0.9999968918301531
28.7109375  0.8974681194046917   0.9153176901730596   0.9750804809530149  0.9508749765028138   0.9999999991846812  0.9999756844631926  0.9673644601236598  0.9999999982183563  0.9992383391255711  0.9871696488967536  0.9999338310603074  0.9999999230931297  0.9999968933736326  0.9695874475052086  0.9999909894745344  0.9219028972040109   0.9986524793063956  0.9277227708714002   0.9124742979533318   0.9997230003328332  0.9999999992139829  0.8855977620570703   0.9457019991707536   0.969170925008264   0.9999912187290497  0.9118189131844056   0.9999908887384863  0.9730184392729033  0.9738522554348163  0.999655970037056   0.998664960484978   0.9869842476487433  0.9085284708563903   0.9100365366758193   0.9999999239379125  0.9999968381362396  0.9959115217395597  0.9999999982952532  0.9999968135525188  0.9999326306770094  0.995820536810087   0.9999778966793286  0.9999968975479956
28.828125   0.8986689760358556   0.9167207238734739   0.9752403489543469  0.9498722222715681   0.9999999991835804  0.9999774324504843  0.9701164800129979  0.9999999982236251  0.9992461677301115  0.9872802055920442  0.9999335060120386  0.9999999237583186  0.9999969053842962  0.9705809930234618  0.9999913600413541  0.9205091815708021   0.9986610705288638  0.9285581013873111   0.9131080920119834   0.9997489777657826  0.999999999214032   0.8861847894981512   0.9472428206202665   0.9694784955928504  0.9999915366389435  0.9132689141044945   0.9999909910869109  0.9747115689953887  0.9748626670877063  0.9996479642735185  0.9986737607200052  0.9871281489699497  0.9103428420298432   0.9100494003166236   0.9999999246345697  0.9999968828403305  0.9959503337160848  0.9999999983014886  0.9999968643438208  0.9999330745914092  0.995868541705531   0.9999784142312913  0.9999969098209809
28.9453125  0.8998624439667191   0.9181173976176508   0.9750903430917087  0.94910741034371     0.9999999991809929  0.9999792707950489  0.9742881400707675  0.999999998232225   0.999253643658074   0.9873808297058064  0.9999336214644396  0.9999999244647647  0.9999969355427933  0.9714265877445003  0.9999917284642765  0.9185039531064547   0.9986732946582418  0.9293732512063717   0.9139484813438968   0.9997450568337723  0.9999999992124884  0.8868214504646201   0.9487160287899808   0.9698998564472395  0.9999918012082301  0.9146190321866458   0.9999909754187699  0.9753401002005542  0.975910796168592   0.9996695355385539  0.9986811378980862  0.9872904178282301  0.9121157953247181   0.9100818425504457   0.9999999253604961  0.9999969366341177  0.9959742043930698  0.999999998310825   0.9999969156632506  0.999934262110173   0.9959272441488071  0.9999785734069677  0.9999969387585438
29.0625     0.901048341546125    0.9194821168587125   0.974749897252525   0.9486579760157438   0.9999999991773     0.99998043320535    0.9789480350546937  0.9999999982419938  0.9992614335735376  0.98747958988425    0.9999344940570063  0.9999999252376313  0.9999969861965126  0.9720537209139251  0.9999920194150751  0.9162216077960187   0.9986887123005312  0.9301680689555063   0.9149969330062437   0.9997185426524062  0.99999999920935    0.887535683919075    0.9500804587226562   0.970477575942425   0.999991971976225   0.915839348842025    0.99999102993255    0.975432779399325   0.9768712005364251  0.999712067662375   0.9986885803988562  0.9874641877336625  0.9137836350420937   0.9102057954800813   0.9999999261383375  0.9999969886019375  0.9959890310824188  0.99999999832115    0.99999696349085    0.9999358728915312  0.99599184690025    0.999978829835725   0.9999969871858749
29.1796875  0.9022264236741547   0.9207890817582793   0.9743619638804324  0.9485737200274665   0.9999999991724117  0.9999805499987128  0.9830061246097157  0.9999999982564625  0.9992695712680869  0.9875852698707568  0.9999362435974514  0.999999926074298   0.99999705087305    0.9724342133312498  0.9999921712307915  0.9140206230679746   0.9987060650438262  0.9309437705765432   0.9162431021193311   0.9996879215640948  0.9999999992047081  0.8883513526954456   0.9513006808754509   0.9712154923293229  0.9999920384041872  0.9169096066202899   0.999991314045587   0.975679348966238   0.9776095847622258  0.9997560303016428  0.9986979257314921  0.9876409285833295  0.9152924269562628   0.910483610755065    0.9999999269651427  0.9999970299581404  0.9960030072340007  0.9999999983354615  0.999997002624634   0.9999375861732933  0.9960554306289834  0.9999793824645072  0.9999970501901063
29.296875   0.9033963858808784   0.9220140563483744   0.9740629272306732  0.9488746375360612   0.9999999991664831  0.9999798646786056  0.985523613643024   0.9999999982752376  0.9992773797446184  0.9877053409986521  0.9999386801823854  0.9999999269602757  0.9999971182771337  0.9725859749457791  0.9999921608256839  0.9122401597594186   0.9987238593087755  0.9317031263059689   0.9176659174018716   0.9996730630640666  0.9999999991984186  0.8892870989980562   0.9523487455780952   0.9720761030145444  0.9999920128229937  0.9178207786916046   0.9999917445176879  0.9766655979000154  0.9780118238903527  0.9997808887638028  0.9987107288980154  0.9878125814721573  0.9166024604051942   0.910963417187659    0.9999999278268571  0.999997059463483   0.9960242080538588  0.9999999983538009  0.9999970275212763  0.9999392386316466  0.9961112354542389  0.9999799907045933  0.9999971175160822
29.4140625  0.9045578689672692   0.923136049610929    0.9739551908206036  0.9495507592041954   0.9999999991592781  0.9999790194195002  0.9859829800458925  0.9999999982960908  0.999284388223127   0.9878442170905481  0.9999413486551136  0.9999999279035706  0.9999971778001712  0.9725676263607033  0.9999920158664639  0.9111590213373421   0.9987409657578289  0.9324493056770058   0.9192352099437952   0.9996849979149404  0.9999999991904875  0.8903554580822985   0.9532055391825811   0.9729868377865252  0.9999919211616198  0.91857573592513     0.9999921017197857  0.9786440214565819  0.9780095989480324  0.9997753774160262  0.9987276931358793  0.9879734220481939  0.917691438947133    0.9116757377302709   0.9999999287310486  0.9999970809952583  0.9960582659978789  0.9999999983738125  0.9999970382310294  0.999940845220552   0.9961548487359123  0.9999801800845707  0.999997178073191
29.53125    0.9057104642869492   0.9241387924423422   0.9740881039324609  0.9505634961184704   0.9999999991504696  0.9999785848167109  0.984432800346811   0.9999999983219734  0.99929103209045    0.9880022100778648  0.9999437130677711  0.999999928913689   0.9999972235119546  0.9724652478457687  0.9999918053497969  0.9109648991920274   0.9987570052801453  0.9331847545204266   0.9209137258153524   0.9997203864513117  0.9999999991807461  0.8915622787466296   0.9538616791931382   0.9738539096043828  0.9999917998018102  0.9191889675674187   0.9999923001815336  0.9814180902751327  0.9775971568078368  0.9997430272600321  0.9987483821194071  0.9881212211627938  0.9185561385114086   0.9126316208183414   0.999999929688364   0.999997098662804   0.9961068351799477  0.9999999983981445  0.9999970442276781  0.9999424863868938  0.9961856220017719  0.9999796647046617  0.9999972247649789
29.6484375  0.9068537197646949   0.9250119086220779   0.9744509379411015  0.9518482229795046   0.9999999991400875  0.9999786915494417  0.9814601690714281  0.9999999983506938  0.999298151474754   0.9881754344532132  0.999945379902228   0.9999999299698561  0.9999972553745081  0.9723740935170658  0.9999916135595013  0.9117395056897717   0.9987723869955695  0.9339114412728128   0.9226593646028116   0.9997633395174541  0.9999999991690935  0.8929064833355043   0.9543178992196965   0.9745808705656508  0.9999916960223107  0.9196853854910667   0.9999924688090263  0.9843820461718957  0.9768356699096689  0.9997002961951921  0.9987713259866287  0.9882574602827457  0.9192124044999875   0.913822409476467    0.9999999306821202  0.999997116173541   0.9961672872082463  0.999999998425195   0.9999970595568348  0.999944151904114   0.996206892168849   0.9999786426178985  0.9999972571262815
29.765625   0.9079871467369403   0.9257516993869793   0.9749789381349163  0.9533180923080127   0.9999999991282283  0.9999790274823906  0.9780002282292658  0.9999999983803639  0.9993058964351555  0.9883566806901417  0.9999462376048815  0.9999999310439514  0.999997278122051   0.9723790808423891  0.9999915122176841  0.9134608542645216   0.9987880059607197  0.9346319471366318   0.9244275036631407   0.9997935376200192  0.999999999155827   0.8943801815572001   0.9545849017619629   0.97508804085239    0.999991662561778   0.9200983127869393   0.9999927636211181  0.9867054677224614  0.9758443458342495  0.9996683685145288  0.9987944846266864  0.988386594667562   0.9196935020715306   0.9152211325412767   0.999999931688184   0.9999971391484098  0.9962336954102511  0.9999999984528514  0.9999970928363536  0.9999456570469807  0.9962249591770789  0.9999777205349265  0.999997280307047
29.8828125  0.909110227630306    0.9263614885358588   0.9755706759113231  0.9548692592562702   0.9999999991156814  0.9999791736614936  0.9750468359097133  0.9999999984131371  0.9993135235093538  0.9885370450889732  0.9999464434669673  0.999999932130612   0.9999972987091821  0.9725380391188223  0.9999915436533966  0.9160194638875927   0.9988047414691458  0.9353496940625835   0.9261732928594605   0.9997964834305691  0.9999999991418187  0.8959691324784346   0.9546826862056698   0.9753288239062209  0.9999917412921829  0.9204668203102885   0.9999931517754562  0.9876015883654231  0.974780216014743   0.9996627983281069  0.998815897713286   0.9885145875327345  0.9200469751379887   0.916785367396597    0.999999932704025   0.9999971739436875  0.9962987707755122  0.9999999984832317  0.9999971419966323  0.9999466965372683  0.9962471555058808  0.9999775145951475  0.9999973014146621
30.         0.9102224243866995   0.9268515060634996   0.9761126157475999  0.9563877545148997   0.9999999991038     0.9999789946907002  0.9733596239182999  0.9999999984459     0.9993203987962996  0.9887079226355     0.9999462820696002  0.9999999332259999  0.9999973231992997  0.97287104887       0.9999917171350003  0.9192424985034996   0.9988229699717996  0.9360677328799001   0.9278538297924999   0.9997707471505002  0.9999999991282998  0.8976535290067      0.9546393889568998   0.9752995972255996  0.9999919427644997  0.9208326249955001   0.9999934505015997  0.9865898040793004  0.9738108170942998  0.9996861444461002  0.9988342831505002  0.9886471118918001  0.9203302972657998   0.9184613037555999   0.9999999337295996  0.9999972218106006  0.9963561319470001  0.9999999985139998  0.9999971979497992  0.9999470094845001  0.9962796153049999  0.9999782262380998  0.9999973259021999
\end{filecontents}

\usepgfplotslibrary{fillbetween}
\usetikzlibrary{patterns}
\usetikzlibrary{pgfplots.groupplots}

\begin{tikzpicture}
  \begin{groupplot}[
      group style={
          group name=my plots,
          group size=1 by 2,
          xlabels at=edge bottom,
          xticklabels at=edge bottom,
          vertical sep=2pt,
      },
      width=\columnwidth,
      height=0.618034\columnwidth,
      xlabel={Time (\si{\pico\second})},
      %enlarge x limits=0.05,
      %enlarge y limits=0.15
  ]
  \nextgroupplot[ylabel={Population (low density)}]
    \addplot[smooth] table[x = time, y = lbound] {low_density_stats.dat};
    \addplot[smooth] table[x = time, y = ubound] {low_density_stats.dat};

    \addplot+[smooth, no marks] table[x = time, y = p1] {low_density_stats.dat};
    \addplot+[smooth, no marks] table[x = time, y = p2] {low_density_stats.dat};
    \addplot+[smooth, no marks] table[x = time, y = p3] {low_density_stats.dat};
    \addplot+[smooth, no marks] table[x = time, y = p4] {low_density_stats.dat};
    \addplot+[smooth, no marks] table[x = time, y = p5] {low_density_stats.dat};

  \nextgroupplot[ylabel={Population (high density)}] 
    \addplot[smooth] table[x = time, y = lbound] {high_density_stats.dat};
    \addplot[smooth] table[x = time, y = ubound] {high_density_stats.dat};

    \foreach \n in {3, ..., 43} 
    {
      \addplot+[smooth, no marks] table[x = time, y index={\n}] {high_density_stats.dat};
    }

  \end{groupplot}

\end{tikzpicture}

\end{figure}
