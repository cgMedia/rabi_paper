\section{Numerical Results}
Here we detail the results of several investigations into coupled \qd{} behavior with the model presented thusfar.
Our algorithm reliably handles tens of thousands of \qds{} and can simulate ten picoseconds of system dynamics in two days on a single processor.
We perform simulations of systems of \qds{} randomly distributed throughout a cubic simulation volume at various densities.
\Cref{table:parameters} gives approximate values of the system parameters.

\begin{table}
  \begin{ruledtabular}
    \begin{tabular}{lll}
      Quantity                 & Symbol         & Value                        \\ \hline
      Transition frequency     & $\omega_0$     & $\SI{1500}{\milli\eV}/\hbar$ \\
      Transition dipole moment & $\abs{\vb{d}}$ & \SI{10}{\elementarycharge\bohr} \\
      Relaxation times         & $T_{1}, T_{2}$ & \SIlist{10;20}{\pico\second} \\
      Laser frequency          & $\omega_L$     & $\SI{1500}{\milli\eV}/\hbar$ \\ \hline
      Speed of light           & $c$            & \SI{299.792458}{\micro\meter\per\pico\second} \\
      Reduced Planck constant  & $\hbar$        & \SI{0.65821193}{\milli\eV \pico\second} \\
      Vacuum permeability      & $\mu_0$        & \SI{2.0133545e-4}{\milli\eV \pico\second\squared \per \elementarycharge \per \micro\meter}
    \end{tabular}
  \end{ruledtabular}
  \caption{\label{table:parameters}Rough simulation parameters.}
\end{table}

\begin{figure}
  \usetikzlibrary{pgfplots.groupplots}

\begin{filecontents}{ipr_1024.dat}
0.000000000000000000e+00 0.000000000000000000e+00
2.500000000000000139e-02 3.381020394210842994e-10
5.000000000000000278e-02 1.840404700598902155e-09
7.499999999999999722e-02 8.033211570338677499e-09
1.000000000000000056e-01 2.158837546033763599e-08
1.250000000000000000e-01 4.766930931281512795e-08
1.499999999999999944e-01 9.100822592755797302e-08
1.749999999999999889e-01 1.532972601608461466e-07
2.000000000000000111e-01 2.369038891636334604e-07
2.250000000000000056e-01 3.422420040461067459e-07
2.500000000000000000e-01 4.681486278444030313e-07
2.750000000000000222e-01 6.117639690802426783e-07
2.999999999999999889e-01 7.692248697121172041e-07
3.250000000000000111e-01 9.360082609832526962e-07
3.499999999999999778e-01 1.107389336343087352e-06
3.750000000000000000e-01 1.278841105706958123e-06
4.000000000000000222e-01 1.446443866589749994e-06
4.249999999999999889e-01 1.607074230902281672e-06
4.500000000000000111e-01 1.758516512770412835e-06
4.749999999999999778e-01 1.899458037815876868e-06
5.000000000000000000e-01 2.029400021924402178e-06
5.250000000000000222e-01 2.148513335702715091e-06
5.500000000000000444e-01 2.257463509529235995e-06
5.749999999999999556e-01 2.357237091006256224e-06
5.999999999999999778e-01 2.448984700587075736e-06
6.250000000000000000e-01 2.533892668434706286e-06
6.500000000000000222e-01 2.613087782425260047e-06
6.750000000000000444e-01 2.687574294546078366e-06
6.999999999999999556e-01 2.758199257206928739e-06
7.249999999999999778e-01 2.825640380475780145e-06
7.500000000000000000e-01 2.890410412151062786e-06
7.750000000000000222e-01 2.952872560571615651e-06
8.000000000000000444e-01 3.013262491573676788e-06
8.249999999999999556e-01 3.071713496050759717e-06
8.499999999999999778e-01 3.128282391075395616e-06
8.750000000000000000e-01 3.182974453097651285e-06
9.000000000000000222e-01 3.235766212925875879e-06
9.250000000000000444e-01 3.286625301406704766e-06
9.499999999999999556e-01 3.335526797607415994e-06
9.749999999999999778e-01 3.382465748892702490e-06
1.000000000000000000e+00 3.427465741071434452e-06
1.024999999999999911e+00 3.470583607187958941e-06
1.050000000000000044e+00 3.511910568371667471e-06
1.074999999999999956e+00 3.551570282031525291e-06
1.100000000000000089e+00 3.589714412759225162e-06
1.125000000000000000e+00 3.626516425651871156e-06
1.149999999999999911e+00 3.662164324548305636e-06
1.175000000000000044e+00 3.696853022110277096e-06
1.199999999999999956e+00 3.730776944928715019e-06
1.225000000000000089e+00 3.764123359608620087e-06
1.250000000000000000e+00 3.797066771237455826e-06
1.274999999999999911e+00 3.829764608610482582e-06
1.300000000000000044e+00 3.862354283488923443e-06
1.324999999999999956e+00 3.894951602124952991e-06
1.350000000000000089e+00 3.927650421283162413e-06
1.375000000000000000e+00 3.960523379467704302e-06
1.399999999999999911e+00 3.993623496010775820e-06
1.425000000000000044e+00 4.026986413490241806e-06
1.449999999999999956e+00 4.060633059392408650e-06
1.475000000000000089e+00 4.094572516820035868e-06
1.500000000000000000e+00 4.128804918072026195e-06
1.524999999999999911e+00 4.163324205460391253e-06
1.550000000000000044e+00 4.198120637795288435e-06
1.574999999999999956e+00 4.233182956060645846e-06
1.600000000000000089e+00 4.268500155741672755e-06
1.625000000000000000e+00 4.304062844353098716e-06
1.649999999999999911e+00 4.339864189782357065e-06
1.675000000000000044e+00 4.375900487067329453e-06
1.699999999999999956e+00 4.412171387983962120e-06
1.725000000000000089e+00 4.448679848966397704e-06
1.750000000000000000e+00 4.485431858933632346e-06
1.774999999999999911e+00 4.522436009863706544e-06
1.800000000000000044e+00 4.559702970323995311e-06
1.824999999999999956e+00 4.597244916378692009e-06
1.850000000000000089e+00 4.635074966307262817e-06
1.875000000000000000e+00 4.673206656210550369e-06
1.899999999999999911e+00 4.711653483786305829e-06
1.925000000000000044e+00 4.750428537845814009e-06
1.949999999999999956e+00 4.789544222175189450e-06
1.975000000000000089e+00 4.829012074641881799e-06
2.000000000000000000e+00 4.868842675949305171e-06
2.024999999999999911e+00 4.909045637700856432e-06
2.049999999999999822e+00 4.949629656082667889e-06
2.075000000000000178e+00 4.990602615678350548e-06
2.100000000000000089e+00 5.031971727404880787e-06
2.125000000000000000e+00 5.073743685204100407e-06
2.149999999999999911e+00 5.115924827499502073e-06
2.174999999999999822e+00 5.158521291627601521e-06
2.200000000000000178e+00 5.201539151873570827e-06
2.225000000000000089e+00 5.244984534391898198e-06
2.250000000000000000e+00 5.288863704942846702e-06
2.274999999999999911e+00 5.333183127787400474e-06
2.299999999999999822e+00 5.377949496222951542e-06
2.325000000000000178e+00 5.423169736981257524e-06
2.350000000000000089e+00 5.468850992031626043e-06
2.375000000000000000e+00 5.515000582160395348e-06
2.399999999999999911e+00 5.561625957198031503e-06
2.424999999999999822e+00 5.608734637735455358e-06
2.450000000000000178e+00 5.656334152962828876e-06
2.475000000000000089e+00 5.704431978694173705e-06
2.500000000000000000e+00 5.753035478895046975e-06
2.524999999999999911e+00 5.802151853255077887e-06
2.549999999999999822e+00 5.851788092389384466e-06
2.575000000000000178e+00 5.901950941449496044e-06
2.600000000000000089e+00 5.952646872081981955e-06
2.625000000000000000e+00 6.003882062003817327e-06
2.649999999999999911e+00 6.055662380899514904e-06
2.674999999999999822e+00 6.107993380893590035e-06
2.700000000000000178e+00 6.160880289657636941e-06
2.725000000000000089e+00 6.214328003987298205e-06
2.750000000000000000e+00 6.268341081792612804e-06
2.774999999999999911e+00 6.322923730495101299e-06
2.799999999999999822e+00 6.378079790043738803e-06
2.825000000000000178e+00 6.433812709043208386e-06
2.850000000000000089e+00 6.490125512807016703e-06
2.875000000000000000e+00 6.547020762395468554e-06
2.899999999999999911e+00 6.604500504117649662e-06
2.924999999999999822e+00 6.662566209181697979e-06
2.950000000000000178e+00 6.721218703464957450e-06
2.975000000000000089e+00 6.780458087564042446e-06
3.000000000000000000e+00 6.840283647454193511e-06
3.024999999999999911e+00 6.900693756160813930e-06
3.049999999999999822e+00 6.961685766915248763e-06
3.075000000000000178e+00 7.023255898226969315e-06
3.100000000000000089e+00 7.085399111335767899e-06
3.125000000000000000e+00 7.148108980325855455e-06
3.149999999999999911e+00 7.211377555236663656e-06
3.174999999999999822e+00 7.275195218258388719e-06
3.200000000000000178e+00 7.339550533118528730e-06
3.225000000000000089e+00 7.404430087596097082e-06
3.250000000000000000e+00 7.469818329078076885e-06
3.274999999999999911e+00 7.535697392971850428e-06
3.299999999999999822e+00 7.602046923803167100e-06
3.325000000000000178e+00 7.668843888807312312e-06
3.350000000000000089e+00 7.736062383853666014e-06
3.375000000000000000e+00 7.803673431631442947e-06
3.399999999999999911e+00 7.871644772088808376e-06
3.424999999999999822e+00 7.939940645274338668e-06
3.450000000000000178e+00 8.008521566842553147e-06
3.475000000000000089e+00 8.077344096680833668e-06
3.500000000000000000e+00 8.146360601298495430e-06
3.524999999999999911e+00 8.215519010817831994e-06
3.549999999999999822e+00 8.284762571616208562e-06
3.575000000000000178e+00 8.354029595936942817e-06
3.600000000000000089e+00 8.423253209975797504e-06
3.625000000000000000e+00 8.492361102189517619e-06
3.649999999999999911e+00 8.561275273927596019e-06
3.674999999999999822e+00 8.629911794571501135e-06
3.700000000000000178e+00 8.698180563810557070e-06
3.725000000000000089e+00 8.765985083825913668e-06
3.750000000000000000e+00 8.833222244512264551e-06
3.774999999999999911e+00 8.899782125118072332e-06
3.799999999999999822e+00 8.965547815996009404e-06
3.825000000000000178e+00 9.030395264488172771e-06
3.850000000000000089e+00 9.094193149239730364e-06
3.875000000000000000e+00 9.156802787619114370e-06
3.899999999999999911e+00 9.218078081236211543e-06
3.924999999999999822e+00 9.277865504890006262e-06
3.950000000000000178e+00 9.336004144650314474e-06
3.975000000000000089e+00 9.392325791127793463e-06
4.000000000000000000e+00 9.446655094339862294e-06
4.025000000000000355e+00 9.498809786956169052e-06
4.049999999999999822e+00 9.548600983021659702e-06
4.075000000000000178e+00 9.595833559687683145e-06
4.099999999999999645e+00 9.640306629723561019e-06
4.125000000000000000e+00 9.681814113000420177e-06
4.150000000000000355e+00 9.720145415417292059e-06
4.174999999999999822e+00 9.755086224117219193e-06
4.200000000000000178e+00 9.786419428103919696e-06
4.224999999999999645e+00 9.813926173748597701e-06
4.250000000000000000e+00 9.837387065026901786e-06
4.275000000000000355e+00 9.856583518654113071e-06
4.299999999999999822e+00 9.871299284769548266e-06
4.325000000000000178e+00 9.881322144262650112e-06
4.349999999999999645e+00 9.886445794415237368e-06
4.375000000000000000e+00 9.886471935267523992e-06
4.400000000000000355e+00 9.881212569889203864e-06
4.424999999999999822e+00 9.870492532880296987e-06
4.450000000000000178e+00 9.854152262633386963e-06
4.474999999999999645e+00 9.832050834546190998e-06
4.500000000000000000e+00 9.804069274282900537e-06
4.525000000000000355e+00 9.770114172616113993e-06
4.549999999999999822e+00 9.730121626254535638e-06
4.575000000000000178e+00 9.684061532556215157e-06
4.599999999999999645e+00 9.631942270291703190e-06
4.625000000000000000e+00 9.573815803548763395e-06
4.650000000000000355e+00 9.509783251869389012e-06
4.674999999999999822e+00 9.440000976700373830e-06
4.700000000000000178e+00 9.364687242437207473e-06
4.724999999999999645e+00 9.284129519924327889e-06
4.750000000000000000e+00 9.198692511452020992e-06
4.775000000000000355e+00 9.108826989143907829e-06
4.799999999999999822e+00 9.015079553561676159e-06
4.825000000000000178e+00 8.918103436446811125e-06
4.849999999999999645e+00 8.818670491233561448e-06
4.875000000000000000e+00 8.717684537443539175e-06
4.900000000000000355e+00 8.616196250843619758e-06
4.924999999999999822e+00 8.515419820661262670e-06
4.950000000000000178e+00 8.416751628676204938e-06
4.974999999999999645e+00 8.321791243200236249e-06
5.000000000000000000e+00 8.232365064525862253e-06
5.025000000000000355e+00 8.150553007950367412e-06
5.049999999999999822e+00 8.078718666893478298e-06
5.075000000000000178e+00 8.019543462894863570e-06
5.099999999999999645e+00 7.976065362302192652e-06
5.125000000000000000e+00 7.951722822751921757e-06
5.150000000000000355e+00 7.950404727257625107e-06
5.174999999999999822e+00 7.976507171554569897e-06
5.200000000000000178e+00 8.034998093165270430e-06
5.224999999999999645e+00 8.131490870351941458e-06
5.250000000000000000e+00 8.272328178115465752e-06
5.275000000000000355e+00 8.464677569155207651e-06
5.299999999999999822e+00 8.716640453140583542e-06
5.325000000000000178e+00 9.037376380791918573e-06
5.349999999999999645e+00 9.437244803676622662e-06
5.375000000000000000e+00 9.927966779887335141e-06
5.400000000000000355e+00 1.052280943392939344e-05
5.424999999999999822e+00 1.123679636040163937e-05
5.450000000000000178e+00 1.208694758961497556e-05
5.474999999999999645e+00 1.309255321366316858e-05
5.500000000000000000e+00 1.427548530759208123e-05
5.525000000000000355e+00 1.566055337624155420e-05
5.549999999999999822e+00 1.727590921580114507e-05
5.575000000000000178e+00 1.915350780230177900e-05
5.599999999999999645e+00 2.132963160694626749e-05
5.625000000000000000e+00 2.384548658838220314e-05
5.650000000000000355e+00 2.674787901824461980e-05
5.674999999999999822e+00 3.008998324868891912e-05
5.700000000000000178e+00 3.393221151155628897e-05
5.724999999999999645e+00 3.834319782406564979e-05
5.750000000000000000e+00 4.340090902933669247e-05
5.775000000000000355e+00 4.919389687402567733e-05
5.799999999999999822e+00 5.582270575438903412e-05
5.825000000000000178e+00 6.340145126215225059e-05
5.849999999999999645e+00 7.205958482459757196e-05
5.875000000000000000e+00 8.194385942184999501e-05
5.900000000000000355e+00 9.322051041211988284e-05
5.924999999999999822e+00 1.060776636958014364e-04
5.950000000000000178e+00 1.207279805608178162e-04
5.974999999999999645e+00 1.374115442882690129e-04
6.000000000000000000e+00 1.563989876386554043e-04
6.025000000000000355e+00 1.779948523254017752e-04
6.049999999999999822e+00 2.025411611370681909e-04
6.075000000000000178e+00 2.304211701012414369e-04
6.099999999999999645e+00 2.620632516353107750e-04
6.125000000000000000e+00 2.979448396904708744e-04
6.150000000000000355e+00 3.385963442823760813e-04
6.174999999999999822e+00 3.846049154956003146e-04
6.200000000000000178e+00 4.366179063064233937e-04
6.224999999999999645e+00 4.953458499823907950e-04
6.250000000000000000e+00 5.615647324527816599e-04
6.275000000000000355e+00 6.361173045056280236e-04
6.299999999999999822e+00 7.199131452141581660e-04
6.325000000000000178e+00 8.139271595745932193e-04
6.349999999999999645e+00 9.191961735978745190e-04
6.375000000000000000e+00 1.036813283305771565e-03
6.400000000000000355e+00 1.167919624946724697e-03
6.424999999999999822e+00 1.313693267071955805e-03
6.450000000000000178e+00 1.475334985406596836e-03
6.474999999999999645e+00 1.654050772240552762e-03
6.500000000000000000e+00 1.851031055266060552e-03
6.525000000000000355e+00 2.067426755934826700e-03
6.549999999999999822e+00 2.304322500867509935e-03
6.575000000000000178e+00 2.562707504401924748e-03
6.599999999999999645e+00 2.843444854851292711e-03
6.625000000000000000e+00 3.147240146588531177e-03
6.650000000000000355e+00 3.474610586962521494e-03
6.674999999999999822e+00 3.825855851846530348e-03
6.700000000000000178e+00 4.201032047327631774e-03
6.724999999999999645e+00 4.599930141212649551e-03
6.750000000000000000e+00 5.022060145114380647e-03
6.775000000000000355e+00 5.466642151613310303e-03
6.799999999999999822e+00 5.932605065565188344e-03
6.825000000000000178e+00 6.418593527470541418e-03
6.849999999999999645e+00 6.922983131876544148e-03
6.875000000000000000e+00 7.443903623827027340e-03
6.900000000000000355e+00 7.979269343535421236e-03
6.924999999999999822e+00 8.526815816875399009e-03
6.950000000000000178e+00 9.084141085998956658e-03
6.974999999999999645e+00 9.648750163179328765e-03
7.000000000000000000e+00 1.021810088567150047e-02
7.025000000000000355e+00 1.078964945319233620e-02
7.049999999999999822e+00 1.136089403635549139e-02
7.075000000000000178e+00 1.192941503868675049e-02
7.099999999999999645e+00 1.249291085555314859e-02
7.125000000000000000e+00 1.304922827528647543e-02
7.150000000000000355e+00 1.359638698591449611e-02
7.174999999999999822e+00 1.413259796133174359e-02
7.200000000000000178e+00 1.465627578424140778e-02
7.224999999999999645e+00 1.516604520518402902e-02
7.250000000000000000e+00 1.566074242850051465e-02
7.275000000000000355e+00 1.613941175387613447e-02
7.299999999999999822e+00 1.660129828684622430e-02
7.325000000000000178e+00 1.704583746841571365e-02
7.349999999999999645e+00 1.747264216924202321e-02
7.375000000000000000e+00 1.788148805664039470e-02
7.400000000000000355e+00 1.827229788108904879e-02
7.424999999999999822e+00 1.864512525175903573e-02
7.450000000000000178e+00 1.900013838488753132e-02
7.474999999999999645e+00 1.933760422100810200e-02
7.500000000000000000e+00 1.965787322160726658e-02
7.525000000000000355e+00 1.996136507653397188e-02
7.549999999999999822e+00 2.024855548246192252e-02
7.575000000000000178e+00 2.051996409118309397e-02
7.599999999999999645e+00 2.077614367488024236e-02
7.625000000000000000e+00 2.101767051388565843e-02
7.650000000000000355e+00 2.124513597940922219e-02
7.674999999999999822e+00 2.145913925930280700e-02
7.700000000000000178e+00 2.166028115738602086e-02
7.724999999999999645e+00 2.184915888530385847e-02
7.750000000000000000e+00 2.202636175938275584e-02
7.775000000000000355e+00 2.219246771225638279e-02
7.799999999999999822e+00 2.234804052942293562e-02
7.825000000000000178e+00 2.249362772363637519e-02
7.849999999999999645e+00 2.262975896416962590e-02
7.875000000000000000e+00 2.275694498343874481e-02
7.900000000000000355e+00 2.287567688949646363e-02
7.924999999999999822e+00 2.298642581915191427e-02
7.950000000000000178e+00 2.308964287295068446e-02
7.974999999999999645e+00 2.318575927949062532e-02
8.000000000000000000e+00 2.327518674258610698e-02
8.025000000000000355e+00 2.335831793049822011e-02
8.050000000000000711e+00 2.343552707180802594e-02
8.074999999999999289e+00 2.350717062728551257e-02
8.099999999999999645e+00 2.357358801180233637e-02
8.125000000000000000e+00 2.363510234420298714e-02
8.150000000000000355e+00 2.369202120692290997e-02
8.175000000000000711e+00 2.374463740026628092e-02
8.199999999999999289e+00 2.379322967935249347e-02
8.224999999999999645e+00 2.383806346426767600e-02
8.250000000000000000e+00 2.387939151631241971e-02
8.275000000000000355e+00 2.391745457522923349e-02
8.300000000000000711e+00 2.395248195404663799e-02
8.324999999999999289e+00 2.398469208968434171e-02
8.349999999999999645e+00 2.401429304877187795e-02
8.375000000000000000e+00 2.404148298913459658e-02
8.400000000000000355e+00 2.406645057831360152e-02
8.425000000000000711e+00 2.408937537116508446e-02
8.449999999999999289e+00 2.411042814917547725e-02
8.474999999999999645e+00 2.412977122440155067e-02
8.500000000000000000e+00 2.414755871135420076e-02
8.525000000000000355e+00 2.416393677020737055e-02
8.550000000000000711e+00 2.417904382481648670e-02
8.574999999999999289e+00 2.419301075904059517e-02
8.599999999999999645e+00 2.420596109470529028e-02
8.625000000000000000e+00 2.421801115455015155e-02
8.650000000000000355e+00 2.422927021321213958e-02
8.675000000000000711e+00 2.423984063912237863e-02
8.699999999999999289e+00 2.424981803007635398e-02
8.724999999999999645e+00 2.425929134483719232e-02
8.750000000000000000e+00 2.426834303308740212e-02
8.775000000000000355e+00 2.427704916560495876e-02
8.800000000000000711e+00 2.428547956652226530e-02
8.824999999999999289e+00 2.429369794912247457e-02
8.849999999999999645e+00 2.430176205655940058e-02
8.875000000000000000e+00 2.430972380858890253e-02
8.900000000000000355e+00 2.431762945537082124e-02
8.925000000000000711e+00 2.432551973901950562e-02
8.949999999999999289e+00 2.433343006372560083e-02
8.974999999999999645e+00 2.434139067488635261e-02
9.000000000000000000e+00 2.434942684772879382e-02
9.025000000000000355e+00 2.435755908573753944e-02
9.050000000000000711e+00 2.436580332907367988e-02
9.074999999999999289e+00 2.437417117315380383e-02
9.099999999999999645e+00 2.438267009726740805e-02
9.125000000000000000e+00 2.439130370318125240e-02
9.150000000000000355e+00 2.440007196340597193e-02
9.175000000000000711e+00 2.440897147865973490e-02
9.199999999999999289e+00 2.441799574394451500e-02
9.224999999999999645e+00 2.442713542241795391e-02
9.250000000000000000e+00 2.443637862613231432e-02
9.275000000000000355e+00 2.444571120240290538e-02
9.300000000000000711e+00 2.445511702451431263e-02
9.324999999999999289e+00 2.446457828521868619e-02
9.349999999999999645e+00 2.447407579138710654e-02
9.375000000000000000e+00 2.448358925799895211e-02
9.400000000000000355e+00 2.449309759957526630e-02
9.425000000000000711e+00 2.450257921713422729e-02
9.449999999999999289e+00 2.451201227865591400e-02
9.474999999999999645e+00 2.452137499113439881e-02
9.500000000000000000e+00 2.453064586222603732e-02
9.525000000000000355e+00 2.453980394978012847e-02
9.550000000000000711e+00 2.454882909737131358e-02
9.574999999999999289e+00 2.455770215437253629e-02
9.599999999999999645e+00 2.456640517910199314e-02
9.625000000000000000e+00 2.457492162376247180e-02
9.650000000000000355e+00 2.458323650015078674e-02
9.675000000000000711e+00 2.459133652526020869e-02
9.699999999999999289e+00 2.459921024611001067e-02
9.724999999999999645e+00 2.460684814332428361e-02
9.750000000000000000e+00 2.461424271324548149e-02
9.775000000000000355e+00 2.462138852843897760e-02
9.800000000000000711e+00 2.462828227678258430e-02
9.824999999999999289e+00 2.463492277938441818e-02
9.849999999999999645e+00 2.464131098786670471e-02
9.875000000000000000e+00 2.464744996159249149e-02
9.900000000000000355e+00 2.465334482568259586e-02
9.925000000000000711e+00 2.465900271071825092e-02
9.949999999999999289e+00 2.466443267521956431e-02
9.974999999999999645e+00 2.466964561207560247e-02
1.000000000000000000e+01 2.467465414018863573e-02
1.002500000000000036e+01 2.467947248274215966e-02
1.005000000000000071e+01 2.468411633349863396e-02
1.007499999999999929e+01 2.468860271271210854e-02
1.009999999999999964e+01 2.469294981413359263e-02
1.012500000000000000e+01 2.469717684480047387e-02
1.015000000000000036e+01 2.470130385918641036e-02
1.017500000000000071e+01 2.470535158939832915e-02
1.019999999999999929e+01 2.470934127307072500e-02
1.022499999999999964e+01 2.471329448055642475e-02
1.025000000000000000e+01 2.471723294308322480e-02
1.027500000000000036e+01 2.472117838337234072e-02
1.030000000000000071e+01 2.472515235025389210e-02
1.032499999999999929e+01 2.472917605869618535e-02
1.034999999999999964e+01 2.473327023647389766e-02
1.037500000000000000e+01 2.473745497874509439e-02
1.040000000000000036e+01 2.474174961150046176e-02
1.042500000000000071e+01 2.474617256477201765e-02
1.044999999999999929e+01 2.475074125628968363e-02
1.047499999999999964e+01 2.475547198613571934e-02
1.050000000000000000e+01 2.476037984270585204e-02
1.052500000000000036e+01 2.476547862019073606e-02
1.055000000000000071e+01 2.477078074754766879e-02
1.057499999999999929e+01 2.477629722889332364e-02
1.059999999999999964e+01 2.478203759505925169e-02
1.062500000000000000e+01 2.478800986598559547e-02
1.065000000000000036e+01 2.479422052356109601e-02
1.067500000000000071e+01 2.480067449452973485e-02
1.069999999999999929e+01 2.480737514300777533e-02
1.072499999999999964e+01 2.481432427224377485e-02
1.075000000000000000e+01 2.482152213524577850e-02
1.077500000000000036e+01 2.482896745396353730e-02
1.080000000000000071e+01 2.483665744678233839e-02
1.082499999999999929e+01 2.484458786411447226e-02
1.084999999999999964e+01 2.485275303196188279e-02
1.087500000000000000e+01 2.486114590328867766e-02
1.090000000000000036e+01 2.486975811713504000e-02
1.092500000000000071e+01 2.487858006536037767e-02
1.094999999999999929e+01 2.488760096686109050e-02
1.097499999999999964e+01 2.489680894915025436e-02
1.100000000000000000e+01 2.490619113695287032e-02
1.102500000000000036e+01 2.491573374764543755e-02
1.105000000000000071e+01 2.492542219294880540e-02
1.107499999999999929e+01 2.493524118649827898e-02
1.109999999999999964e+01 2.494517485661254530e-02
1.112500000000000000e+01 2.495520686345326089e-02
1.115000000000000036e+01 2.496532051984334810e-02
1.117500000000000071e+01 2.497549891472660918e-02
1.119999999999999929e+01 2.498572503823870417e-02
1.122499999999999964e+01 2.499598190731505687e-02
1.125000000000000000e+01 2.500625269065101172e-02
1.127500000000000036e+01 2.501652083184742470e-02
1.130000000000000071e+01 2.502677016948353164e-02
1.132499999999999929e+01 2.503698505300597926e-02
1.134999999999999964e+01 2.504715045315005287e-02
1.137500000000000000e+01 2.505725206591677065e-02
1.140000000000000036e+01 2.506727640894396147e-02
1.142500000000000071e+01 2.507721090942096470e-02
1.144999999999999929e+01 2.508704398268099198e-02
1.147499999999999964e+01 2.509676510078125491e-02
1.150000000000000000e+01 2.510636485050359717e-02
1.152500000000000036e+01 2.511583498033856734e-02
1.155000000000000071e+01 2.512516843624037696e-02
1.157499999999999929e+01 2.513435938588595361e-02
1.159999999999999964e+01 2.514340323155275086e-02
1.162500000000000000e+01 2.515229661163820296e-02
1.165000000000000036e+01 2.516103739109072993e-02
1.167500000000000071e+01 2.516962464101188748e-02
1.169999999999999929e+01 2.517805860788972783e-02
1.172499999999999964e+01 2.518634067283866823e-02
1.175000000000000000e+01 2.519447330148856870e-02
1.177500000000000036e+01 2.520245998501215168e-02
1.180000000000000071e+01 2.521030517293302090e-02
1.182499999999999929e+01 2.521801419850316917e-02
1.184999999999999964e+01 2.522559319723724838e-02
1.187500000000000000e+01 2.523304901951420667e-02
1.190000000000000036e+01 2.524038913805745632e-02
1.192500000000000071e+01 2.524762155115689924e-02
1.194999999999999929e+01 2.525475468269584856e-02
1.197499999999999964e+01 2.526179727987743970e-02
1.200000000000000000e+01 2.526875830979527504e-02
1.202500000000000036e+01 2.527564685592801133e-02
1.205000000000000071e+01 2.528247201558605009e-02
1.207499999999999929e+01 2.528924279955865537e-02
1.209999999999999964e+01 2.529596803490593085e-02
1.212500000000000000e+01 2.530265627200050868e-02
1.215000000000000036e+01 2.530931569684598606e-02
1.217500000000000071e+01 2.531595404941242369e-02
1.219999999999999929e+01 2.532257854895953939e-02
1.222499999999999964e+01 2.532919582686381579e-02
1.225000000000000000e+01 2.533581186755491654e-02
1.227500000000000036e+01 2.534243195797192955e-02
1.230000000000000071e+01 2.534906064576996493e-02
1.232499999999999929e+01 2.535570170637884910e-02
1.234999999999999964e+01 2.536235811901364767e-02
1.237500000000000000e+01 2.536903205135009748e-02
1.240000000000000036e+01 2.537572485274491169e-02
1.242500000000000071e+01 2.538243705560210006e-02
1.244999999999999929e+01 2.538916838443394317e-02
1.247499999999999964e+01 2.539591777215091661e-02
1.250000000000000000e+01 2.540268338298936820e-02
1.252500000000000036e+01 2.540946264146253658e-02
1.255000000000000071e+01 2.541625226670089777e-02
1.257499999999999929e+01 2.542304831156277922e-02
1.259999999999999964e+01 2.542984620582820765e-02
1.262500000000000000e+01 2.543664080283363632e-02
1.265000000000000036e+01 2.544342642897039555e-02
1.267500000000000071e+01 2.545019693545445133e-02
1.269999999999999929e+01 2.545694575183840569e-02
1.272499999999999964e+01 2.546366594074126952e-02
1.275000000000000000e+01 2.547035025341910439e-02
1.277500000000000036e+01 2.547699118575018698e-02
1.280000000000000071e+01 2.548358103425330326e-02
1.282499999999999929e+01 2.549011195195140592e-02
1.284999999999999964e+01 2.549657600366793744e-02
1.287500000000000000e+01 2.550296522056546070e-02
1.290000000000000036e+01 2.550927165369671484e-02
1.292500000000000071e+01 2.551548742632427785e-02
1.294999999999999929e+01 2.552160478467189325e-02
1.297499999999999964e+01 2.552761614698942014e-02
1.300000000000000000e+01 2.553351415052228582e-02
1.302500000000000036e+01 2.553929169614397102e-02
1.305000000000000071e+01 2.554494199032975726e-02
1.307499999999999929e+01 2.555045858412600610e-02
1.309999999999999964e+01 2.555583540880638355e-02
1.312500000000000000e+01 2.556106680789614616e-02
1.315000000000000036e+01 2.556614756520968201e-02
1.317500000000000071e+01 2.557107292868943810e-02
1.319999999999999929e+01 2.557583862974626907e-02
1.322499999999999964e+01 2.558044089793514581e-02
1.325000000000000000e+01 2.558487647081236424e-02
1.327500000000000036e+01 2.558914259881852477e-02
1.330000000000000071e+01 2.559323704524328266e-02
1.332499999999999929e+01 2.559715808118518537e-02
1.334999999999999964e+01 2.560090447561477744e-02
1.337500000000000000e+01 2.560447548065418813e-02
1.340000000000000036e+01 2.560787081218911243e-02
1.342500000000000071e+01 2.561109062608598469e-02
1.344999999999999929e+01 2.561413549013693419e-02
1.347499999999999964e+01 2.561700635217769792e-02
1.350000000000000000e+01 2.561970450451757711e-02
1.352500000000000036e+01 2.562223154522999882e-02
1.355000000000000071e+01 2.562458933652179971e-02
1.357499999999999929e+01 2.562677996075189008e-02
1.359999999999999964e+01 2.562880567453170583e-02
1.362500000000000000e+01 2.563066886136151565e-02
1.365000000000000036e+01 2.563237198340370671e-02
1.367500000000000071e+01 2.563391753294342809e-02
1.369999999999999929e+01 2.563530798402048269e-02
1.372499999999999964e+01 2.563654574491817245e-02
1.375000000000000000e+01 2.563763311196257805e-02
1.377500000000000036e+01 2.563857222525135413e-02
1.380000000000000071e+01 2.563936502681777099e-02
1.382499999999999929e+01 2.564001322165262650e-02
1.384999999999999964e+01 2.564051824213956290e-02
1.387500000000000000e+01 2.564088121619629429e-02
1.390000000000000036e+01 2.564110293956113965e-02
1.392500000000000071e+01 2.564118385248075754e-02
1.394999999999999929e+01 2.564112402108531183e-02
1.397499999999999964e+01 2.564092312366464746e-02
1.400000000000000000e+01 2.564058044203090456e-02
1.402500000000000036e+01 2.564009485810581079e-02
1.405000000000000071e+01 2.563946485582041523e-02
1.407499999999999929e+01 2.563868852835179610e-02
1.409999999999999964e+01 2.563776359081820777e-02
1.412500000000000000e+01 2.563668739824342957e-02
1.415000000000000036e+01 2.563545696886827124e-02
1.417500000000000071e+01 2.563406901249468128e-02
1.419999999999999929e+01 2.563251996384217565e-02
1.422499999999999964e+01 2.563080602046789030e-02
1.425000000000000000e+01 2.562892318510313905e-02
1.427500000000000036e+01 2.562686731178786745e-02
1.430000000000000071e+01 2.562463415557043184e-02
1.432499999999999929e+01 2.562221942498248486e-02
1.434999999999999964e+01 2.561961883693138903e-02
1.437500000000000000e+01 2.561682817318404470e-02
1.440000000000000036e+01 2.561384333785453674e-02
1.442500000000000071e+01 2.561066041511892225e-02
1.444999999999999929e+01 2.560727572642170977e-02
1.447499999999999964e+01 2.560368588643687632e-02
1.450000000000000000e+01 2.559988785692790883e-02
1.452500000000000036e+01 2.559587899788214396e-02
1.455000000000000071e+01 2.559165711501748541e-02
1.457499999999999929e+01 2.558722050305097667e-02
1.459999999999999964e+01 2.558256798398898801e-02
1.462500000000000000e+01 2.557769893979320100e-02
1.465000000000000036e+01 2.557261333886802143e-02
1.467500000000000071e+01 2.556731175578568632e-02
1.469999999999999929e+01 2.556179538391588346e-02
1.472499999999999964e+01 2.555606604045275454e-02
1.475000000000000000e+01 2.555012616375228890e-02
1.477500000000000036e+01 2.554397880263729881e-02
1.480000000000000071e+01 2.553762759782023453e-02
1.482499999999999929e+01 2.553107675535602750e-02
1.484999999999999964e+01 2.552433101235654020e-02
1.487500000000000000e+01 2.551739559530100468e-02
1.490000000000000036e+01 2.551027617120206828e-02
1.492500000000000071e+01 2.550297879224280861e-02
1.494999999999999929e+01 2.549550983429011222e-02
1.497499999999999964e+01 2.548787593003901814e-02
1.500000000000000000e+01 2.548008389737500476e-02
1.502500000000000036e+01 2.547214066370457816e-02
1.505000000000000071e+01 2.546405318699678302e-02
1.507499999999999929e+01 2.545582837434254286e-02
1.509999999999999964e+01 2.544747299886166575e-02
1.512500000000000000e+01 2.543899361573617388e-02
1.515000000000000036e+01 2.543039647829670180e-02
1.517500000000000071e+01 2.542168745508036412e-02
1.519999999999999929e+01 2.541287194869107716e-02
1.522499999999999964e+01 2.540395481754285922e-02
1.525000000000000000e+01 2.539494030144200457e-02
1.527500000000000036e+01 2.538583195200610002e-02
1.530000000000000071e+01 2.537663256907012055e-02
1.532499999999999929e+01 2.536734414403225488e-02
1.534999999999999964e+01 2.535796781136905043e-02
1.537500000000000000e+01 2.534850380922959445e-02
1.540000000000000036e+01 2.533895145018574477e-02
1.542500000000000071e+01 2.532930910310297185e-02
1.544999999999999929e+01 2.531957418694251796e-02
1.547499999999999964e+01 2.530974317724619255e-02
1.550000000000000000e+01 2.529981162582123008e-02
1.552500000000000036e+01 2.528977419416100422e-02
1.555000000000000071e+01 2.527962470065264755e-02
1.557499999999999929e+01 2.526935618173262177e-02
1.559999999999999964e+01 2.525896096663312468e-02
1.562500000000000000e+01 2.524843076541989639e-02
1.565000000000000036e+01 2.523775676954137515e-02
1.567500000000000071e+01 2.522692976405497070e-02
1.569999999999999929e+01 2.521594025050414439e-02
1.572499999999999964e+01 2.520477857894754328e-02
1.575000000000000000e+01 2.519343508793324762e-02
1.577500000000000036e+01 2.518190025058647183e-02
1.580000000000000071e+01 2.517016482515260084e-02
1.582499999999999929e+01 2.515822000812416304e-02
1.584999999999999964e+01 2.514605758798585042e-02
1.587500000000000000e+01 2.513367009765067953e-02
1.590000000000000036e+01 2.512105096361877399e-02
1.592500000000000071e+01 2.510819464981373442e-02
1.594999999999999929e+01 2.509509679431631834e-02
1.597499999999999964e+01 2.508175433707870300e-02
1.600000000000000000e+01 2.506816563697127770e-02
1.602499999999999858e+01 2.505433057647769857e-02
1.605000000000000071e+01 2.504025065269869024e-02
1.607499999999999929e+01 2.502592905332713669e-02
1.610000000000000142e+01 2.501137071639169401e-02
1.612500000000000000e+01 2.499658237298347727e-02
1.614999999999999858e+01 2.498157257198853690e-02
1.617500000000000071e+01 2.496635168631412133e-02
1.619999999999999929e+01 2.495093190010143347e-02
1.622500000000000142e+01 2.493532717664130594e-02
1.625000000000000000e+01 2.491955320682024549e-02
1.627499999999999858e+01 2.490362733811047474e-02
1.630000000000000071e+01 2.488756848425561358e-02
1.632499999999999929e+01 2.487139701600367056e-02
1.635000000000000142e+01 2.485513463327341036e-02
1.637500000000000000e+01 2.483880421959209356e-02
1.639999999999999858e+01 2.482242967942733039e-02
1.642500000000000071e+01 2.480603575963855056e-02
1.644999999999999929e+01 2.478964785623342590e-02
1.647500000000000142e+01 2.477329180786800103e-02
1.650000000000000000e+01 2.475699367783683325e-02
1.652499999999999858e+01 2.474077952642638953e-02
1.655000000000000071e+01 2.472467517574773732e-02
1.657499999999999929e+01 2.470870596939924932e-02
1.660000000000000142e+01 2.469289652936590071e-02
1.662500000000000000e+01 2.467727051288726853e-02
1.664999999999999858e+01 2.466185037196333418e-02
1.667500000000000071e+01 2.464665711832528250e-02
1.669999999999999929e+01 2.463171009674172685e-02
1.672500000000000142e+01 2.461702676951170801e-02
1.675000000000000000e+01 2.460262251486798266e-02
1.677499999999999858e+01 2.458851044203127362e-02
1.680000000000000071e+01 2.457470122543059521e-02
1.682499999999999929e+01 2.456120296051732960e-02
1.685000000000000142e+01 2.454802104324819989e-02
1.687500000000000000e+01 2.453515807523561423e-02
1.689999999999999858e+01 2.452261379623168688e-02
1.692500000000000071e+01 2.451038504526661943e-02
1.694999999999999929e+01 2.449846575153775963e-02
1.697500000000000142e+01 2.448684695586179211e-02
1.700000000000000000e+01 2.447551686298238965e-02
1.702499999999999858e+01 2.446446092493074348e-02
1.705000000000000071e+01 2.445366195514580426e-02
1.707499999999999929e+01 2.444310027268626825e-02
1.710000000000000142e+01 2.443275387563073608e-02
1.712500000000000000e+01 2.442259864240227049e-02
1.714999999999999858e+01 2.441260855928679627e-02
1.717500000000000071e+01 2.440275597227934204e-02
1.719999999999999929e+01 2.439301186090611639e-02
1.722500000000000142e+01 2.438334613156557806e-02
1.725000000000000000e+01 2.437372792741783478e-02
1.727499999999999858e+01 2.436412595178059914e-02
1.730000000000000071e+01 2.435450880171605853e-02
1.732499999999999929e+01 2.434484530818289971e-02
1.735000000000000142e+01 2.433510487907184286e-02
1.737500000000000000e+01 2.432525784125102314e-02
1.739999999999999858e+01 2.431527577756011776e-02
1.742500000000000071e+01 2.430513185475358454e-02
1.744999999999999929e+01 2.429480113823983059e-02
1.747500000000000142e+01 2.428426088953667544e-02
1.750000000000000000e+01 2.427349084246692781e-02
1.752499999999999858e+01 2.426247345416483042e-02
1.755000000000000071e+01 2.425119412714908662e-02
1.757499999999999929e+01 2.423964139917589300e-02
1.760000000000000142e+01 2.422780709753543857e-02
1.762500000000000000e+01 2.421568645519888568e-02
1.764999999999999858e+01 2.420327818644092688e-02
1.767500000000000071e+01 2.419058452013450600e-02
1.769999999999999929e+01 2.417761118946584048e-02
1.772500000000000142e+01 2.416436737737721885e-02
1.775000000000000000e+01 2.415086561767812445e-02
1.777499999999999858e+01 2.413712165240174123e-02
1.780000000000000071e+01 2.412315424660449806e-02
1.782499999999999929e+01 2.410898496239952116e-02
1.785000000000000142e+01 2.409463789471201489e-02
1.787500000000000000e+01 2.408013937170767893e-02
1.789999999999999858e+01 2.406551762341831158e-02
1.792500000000000071e+01 2.405080242265772786e-02
1.794999999999999929e+01 2.403602470243930422e-02
1.797500000000000142e+01 2.402121615490528453e-02
1.800000000000000000e+01 2.400640881664227802e-02
1.802499999999999858e+01 2.399163464568006277e-02
1.805000000000000071e+01 2.397692509563347646e-02
1.807499999999999929e+01 2.396231069242221040e-02
1.810000000000000142e+01 2.394782061903942896e-02
1.812500000000000000e+01 2.393348231384339816e-02
1.814999999999999858e+01 2.391932108763113879e-02
1.817500000000000071e+01 2.390535976461270751e-02
1.819999999999999929e+01 2.389161835219049540e-02
1.822500000000000142e+01 2.387811374404294820e-02
1.825000000000000000e+01 2.386485946075728518e-02
1.827499999999999858e+01 2.385186543177578861e-02
1.830000000000000071e+01 2.383913782193312556e-02
1.832499999999999929e+01 2.382667890548388276e-02
1.835000000000000142e+01 2.381448698973636161e-02
1.837500000000000000e+01 2.380255639006091314e-02
1.839999999999999858e+01 2.379087745722099564e-02
1.842500000000000071e+01 2.377943665742328697e-02
1.844999999999999929e+01 2.376821670479272244e-02
1.847500000000000142e+01 2.375719674526696745e-02
1.850000000000000000e+01 2.374635259024555564e-02
1.852499999999999858e+01 2.373565699757284111e-02
1.855000000000000071e+01 2.372507999686753258e-02
1.857499999999999929e+01 2.371458925547504162e-02
1.860000000000000142e+01 2.370415048068027705e-02
1.862500000000000000e+01 2.369372785324462549e-02
1.864999999999999858e+01 2.368328448692282734e-02
1.867500000000000071e+01 2.367278290781998223e-02
1.869999999999999929e+01 2.366218554745123590e-02
1.872500000000000142e+01 2.365145524270968400e-02
1.875000000000000000e+01 2.364055573582544631e-02
1.877499999999999858e+01 2.362945216736721255e-02
1.880000000000000071e+01 2.361811155509140059e-02
1.882499999999999929e+01 2.360650325167382851e-02
1.885000000000000142e+01 2.359459937451414274e-02
1.887500000000000000e+01 2.358237520105152876e-02
1.889999999999999858e+01 2.356980952356732897e-02
1.892500000000000071e+01 2.355688495774187555e-02
1.894999999999999929e+01 2.354358820012327630e-02
1.897500000000000142e+01 2.352991023023408984e-02
1.900000000000000000e+01 2.351584645373339805e-02
1.902499999999999858e+01 2.350139678412760738e-02
1.905000000000000071e+01 2.348656566121706118e-02
1.907499999999999929e+01 2.347136200550383606e-02
1.910000000000000142e+01 2.345579910868292317e-02
1.912500000000000000e+01 2.343989446122367662e-02
1.914999999999999858e+01 2.342366951915423462e-02
1.917500000000000071e+01 2.340714941282797876e-02
1.919999999999999929e+01 2.339036260138587647e-02
1.922500000000000142e+01 2.337334047755671232e-02
1.925000000000000000e+01 2.335611692787279217e-02
1.927499999999999858e+01 2.333872785438678016e-02
1.930000000000000071e+01 2.332121066420169211e-02
1.932499999999999929e+01 2.330360373382064920e-02
1.935000000000000142e+01 2.328594585579809792e-02
1.937500000000000000e+01 2.326827567506363145e-02
1.939999999999999858e+01 2.325063112297134221e-02
1.942500000000000071e+01 2.323304885696992131e-02
1.944999999999999929e+01 2.321556371364920032e-02
1.947500000000000142e+01 2.319820818305346963e-02
1.950000000000000000e+01 2.318101191180585094e-02
1.952499999999999858e+01 2.316400124218363082e-02
1.955000000000000071e+01 2.314719879392043364e-02
1.957499999999999929e+01 2.313062309505353048e-02
1.960000000000000142e+01 2.311428826731866740e-02
1.962500000000000000e+01 2.309820377113978862e-02
1.964999999999999858e+01 2.308237421423173924e-02
1.967500000000000071e+01 2.306679922730119070e-02
1.969999999999999929e+01 2.305147340911736439e-02
1.972500000000000142e+01 2.303638634259424708e-02
1.975000000000000000e+01 2.302152268236964255e-02
1.977499999999999858e+01 2.300686231356007666e-02
1.980000000000000071e+01 2.299238058020425246e-02
1.982499999999999929e+01 2.297804858112996104e-02
1.985000000000000142e+01 2.296383352992013155e-02
1.987500000000000000e+01 2.294969917472871462e-02
1.989999999999999858e+01 2.293560627286255621e-02
1.992500000000000071e+01 2.292151311425300736e-02
1.994999999999999929e+01 2.290737608696842217e-02
1.997500000000000142e+01 2.289315027761365529e-02
2.000000000000000000e+01 2.287879009847454750e-02
2.002499999999999858e+01 2.286424993306641201e-02
2.005000000000000071e+01 2.284948479119426265e-02
2.007499999999999929e+01 2.283445096448017547e-02
2.010000000000000142e+01 2.281910667312276111e-02
2.012500000000000000e+01 2.280341269475618174e-02
2.014999999999999858e+01 2.278733296617036289e-02
2.017500000000000071e+01 2.277083514932410832e-02
2.019999999999999929e+01 2.275389115311817270e-02
2.022500000000000142e+01 2.273647760313972366e-02
2.025000000000000000e+01 2.271857625223700286e-02
2.027499999999999858e+01 2.270017432541866775e-02
2.030000000000000071e+01 2.268126479373010901e-02
2.032499999999999929e+01 2.266184657231357785e-02
2.035000000000000142e+01 2.264192463932347341e-02
2.037500000000000000e+01 2.262151007311670112e-02
2.039999999999999858e+01 2.260062000647261593e-02
2.042500000000000071e+01 2.257927749768282380e-02
2.044999999999999929e+01 2.255751131947066404e-02
2.047500000000000142e+01 2.253535566794533790e-02
2.050000000000000000e+01 2.251284979496600766e-02
2.052499999999999858e+01 2.249003756807598994e-02
2.055000000000000071e+01 2.246696696368762070e-02
2.057499999999999929e+01 2.244368949974861333e-02
2.060000000000000142e+01 2.242025961517540508e-02
2.062500000000000000e+01 2.239673400414726387e-02
2.064999999999999858e+01 2.237317091391849422e-02
2.067500000000000071e+01 2.234962941548411519e-02
2.069999999999999929e+01 2.232616865672302706e-02
2.072500000000000142e+01 2.230284710810670826e-02
2.075000000000000000e+01 2.227972181102593624e-02
2.077499999999999858e+01 2.225684763899403798e-02
2.080000000000000071e+01 2.223427658168239049e-02
2.082499999999999929e+01 2.221205706171088673e-02
2.085000000000000142e+01 2.219023329345858747e-02
2.087500000000000000e+01 2.216884469296192400e-02
2.089999999999999858e+01 2.214792534702555993e-02
2.092500000000000071e+01 2.212750354924274826e-02
2.094999999999999929e+01 2.210760140950319627e-02
2.097500000000000142e+01 2.208823454295068267e-02
2.100000000000000000e+01 2.206941184304254955e-02
2.102499999999999858e+01 2.205113534262081315e-02
2.105000000000000071e+01 2.203340016544910740e-02
2.107499999999999929e+01 2.201619456994983667e-02
2.110000000000000142e+01 2.199950008506522045e-02
2.112500000000000000e+01 2.198329173772724243e-02
2.114999999999999858e+01 2.196753836944744659e-02
2.117500000000000071e+01 2.195220303883231236e-02
2.119999999999999929e+01 2.193724350544693988e-02
2.122500000000000142e+01 2.192261278929747564e-02
2.125000000000000000e+01 2.190825979908333707e-02
2.127499999999999858e+01 2.189413002143424103e-02
2.130000000000000071e+01 2.188016626225186209e-02
2.132499999999999929e+01 2.186630943055513712e-02
2.135000000000000142e+01 2.185249935438556348e-02
2.137500000000000000e+01 2.183867561786162481e-02
2.139999999999999858e+01 2.182477840789845761e-02
2.142500000000000071e+01 2.181074935882424889e-02
2.144999999999999929e+01 2.179653238317886751e-02
2.147500000000000142e+01 2.178207447668526961e-02
2.150000000000000000e+01 2.176732648599268363e-02
2.152499999999999858e+01 2.175224382781133781e-02
2.155000000000000071e+01 2.173678714896345315e-02
2.157499999999999929e+01 2.172092291733949271e-02
2.160000000000000142e+01 2.170462393474955134e-02
2.162500000000000000e+01 2.168786976374726880e-02
2.164999999999999858e+01 2.167064706153249212e-02
2.167500000000000071e+01 2.165294981545038203e-02
2.169999999999999929e+01 2.163477947585781130e-02
2.172500000000000142e+01 2.161614498372590662e-02
2.175000000000000000e+01 2.159706269169553439e-02
2.177499999999999858e+01 2.157755617895291272e-02
2.180000000000000071e+01 2.155765596183179325e-02
2.182499999999999929e+01 2.153739910347492875e-02
2.185000000000000142e+01 2.151682872760311491e-02
2.187500000000000000e+01 2.149599344252938263e-02
2.189999999999999858e+01 2.147494668328861836e-02
2.192500000000000071e+01 2.145374598067308416e-02
2.194999999999999929e+01 2.143245216729457864e-02
2.197500000000000142e+01 2.141112853156915824e-02
2.200000000000000000e+01 2.138983993157111616e-02
2.202499999999999858e+01 2.136865188109419306e-02
2.205000000000000071e+01 2.134762962100392181e-02
2.207499999999999929e+01 2.132683718900302630e-02
2.210000000000000142e+01 2.130633650118445022e-02
2.212500000000000000e+01 2.128618645871123080e-02
2.214999999999999858e+01 2.126644209257894103e-02
2.217500000000000071e+01 2.124715375900511002e-02
2.219999999999999929e+01 2.122836639753961085e-02
2.222500000000000142e+01 2.121011886296325644e-02
2.225000000000000000e+01 2.119244334126172552e-02
2.227499999999999858e+01 2.117536485886062358e-02
2.230000000000000071e+01 2.115890089295032958e-02
2.232499999999999929e+01 2.114306108956446120e-02
2.235000000000000142e+01 2.112784709437342048e-02
2.237500000000000000e+01 2.111325249972407711e-02
2.239999999999999858e+01 2.109926290974986976e-02
2.242500000000000071e+01 2.108585612360755826e-02
2.244999999999999929e+01 2.107300243521025018e-02
2.247500000000000142e+01 2.106066504596580333e-02
2.250000000000000000e+01 2.104880058531029274e-02
2.252499999999999858e+01 2.103735973212355823e-02
2.255000000000000071e+01 2.102628792842837005e-02
2.257499999999999929e+01 2.101552617514033591e-02
2.260000000000000142e+01 2.100501189846332509e-02
2.262500000000000000e+01 2.099467987398138252e-02
2.264999999999999858e+01 2.098446319459393999e-02
2.267500000000000071e+01 2.097429426753096710e-02
2.269999999999999929e+01 2.096410582501345710e-02
2.272500000000000142e+01 2.095383193279127046e-02
2.275000000000000000e+01 2.094340898060500708e-02
2.277499999999999858e+01 2.093277663873843730e-02
2.280000000000000071e+01 2.092187876528997861e-02
2.282499999999999929e+01 2.091066424931548387e-02
2.285000000000000142e+01 2.089908777608738555e-02
2.287500000000000000e+01 2.088711050166980332e-02
2.289999999999999858e+01 2.087470062557387693e-02
2.292500000000000071e+01 2.086183385170763821e-02
2.294999999999999929e+01 2.084849372962703506e-02
2.297500000000000142e+01 2.083467187018802139e-02
2.300000000000000000e+01 2.082036803137151532e-02
2.302499999999999858e+01 2.080559007258781304e-02
2.305000000000000071e+01 2.079035377763387149e-02
2.307499999999999929e+01 2.077468254884656459e-02
2.310000000000000142e+01 2.075860697697759841e-02
2.312500000000000000e+01 2.074216429359974218e-02
2.314999999999999858e+01 2.072539771473946163e-02
2.317500000000000071e+01 2.070835568623421649e-02
2.319999999999999929e+01 2.069109104308302152e-02
2.322500000000000142e+01 2.067366009661498333e-02
2.325000000000000000e+01 2.065612166438152500e-02
2.327499999999999858e+01 2.063853605881227349e-02
2.330000000000000071e+01 2.062096405155275716e-02
2.332499999999999929e+01 2.060346583064787823e-02
2.335000000000000142e+01 2.058609996816404206e-02
2.337500000000000000e+01 2.056892241566195267e-02
2.339999999999999858e+01 2.055198554461639196e-02
2.342500000000000071e+01 2.053533724821108436e-02
2.344999999999999929e+01 2.051902011998832723e-02
2.347500000000000142e+01 2.050307072384632487e-02
2.350000000000000000e+01 2.048751896804376291e-02
2.352499999999999858e+01 2.047238759474812389e-02
2.355000000000000071e+01 2.045769179430529175e-02
2.357499999999999929e+01 2.044343895166361377e-02
2.360000000000000142e+01 2.042962853010405963e-02
2.362500000000000000e+01 2.041625209502563251e-02
2.364999999999999858e+01 2.040329347825997966e-02
2.367500000000000071e+01 2.039072908089340982e-02
2.369999999999999929e+01 2.037852831001374376e-02
2.372500000000000142e+01 2.036665414263814575e-02
2.375000000000000000e+01 2.035506380739678448e-02
2.377499999999999858e+01 2.034370957280153339e-02
2.380000000000000071e+01 2.033253962841053558e-02
2.382499999999999929e+01 2.032149904380915750e-02
2.385000000000000142e+01 2.031053078857593397e-02
2.387500000000000000e+01 2.029957679503791024e-02
2.389999999999999858e+01 2.028857904502530743e-02
2.392500000000000071e+01 2.027748066080501191e-02
2.394999999999999929e+01 2.026622698040366280e-02
2.397500000000000142e+01 2.025476659759088000e-02
2.400000000000000000e+01 2.024305234726433630e-02
2.402499999999999858e+01 2.023104221777032030e-02
2.405000000000000071e+01 2.021870017303304540e-02
2.407499999999999929e+01 2.020599686879654550e-02
2.410000000000000142e+01 2.019291024920302074e-02
2.412500000000000000e+01 2.017942601192583216e-02
2.414999999999999858e+01 2.016553793258914901e-02
2.417500000000000071e+01 2.015124804156993854e-02
2.419999999999999929e+01 2.013656664913942088e-02
2.422500000000000142e+01 2.012151221744230140e-02
2.425000000000000000e+01 2.010611108087716203e-02
2.427499999999999858e+01 2.009039701905514230e-02
2.430000000000000071e+01 2.007441068933893402e-02
2.432499999999999929e+01 2.005819892855312259e-02
2.435000000000000142e+01 2.004181393593808605e-02
2.437500000000000000e+01 2.002531235169104085e-02
2.439999999999999858e+01 2.000875424725564317e-02
2.442500000000000071e+01 1.999220204564628248e-02
2.444999999999999929e+01 1.997571939082627701e-02
2.447500000000000142e+01 1.995936998691041908e-02
2.450000000000000000e+01 1.994321642804471745e-02
2.452499999999999858e+01 1.992731904047381417e-02
2.455000000000000071e+01 1.991173475808797472e-02
2.457499999999999929e+01 1.989651605215149097e-02
2.460000000000000142e+01 1.988170993536471365e-02
2.462500000000000000e+01 1.986735705886695513e-02
2.464999999999999858e+01 1.985349091948999598e-02
2.467500000000000071e+01 1.984013719257659084e-02
2.469999999999999929e+01 1.982731320359261765e-02
2.472500000000000142e+01 1.981502754935046684e-02
2.475000000000000000e+01 1.980327987704099887e-02
2.477499999999999858e+01 1.979206082646448039e-02
2.480000000000000071e+01 1.978135213812437510e-02
2.482499999999999929e+01 1.977112692661018439e-02
2.485000000000000142e+01 1.976135011593543878e-02
2.487500000000000000e+01 1.975197903032168203e-02
2.489999999999999858e+01 1.974296413128453706e-02
2.492500000000000071e+01 1.973424988884899672e-02
2.494999999999999929e+01 1.972577577225861911e-02
2.497500000000000142e+01 1.971747734335167063e-02
2.500000000000000000e+01 1.970928743333348609e-02
2.502499999999999858e+01 1.970113738244061788e-02
2.505000000000000071e+01 1.969295832010492520e-02
2.507499999999999929e+01 1.968468246267257235e-02
2.510000000000000142e+01 1.967624440487363360e-02
2.512500000000000000e+01 1.966758238119536636e-02
2.514999999999999858e+01 1.965863947377237408e-02
2.517500000000000071e+01 1.964936474389239116e-02
2.519999999999999929e+01 1.963971426561221187e-02
2.522500000000000142e+01 1.962965204146700088e-02
2.525000000000000000e+01 1.961915078233951101e-02
2.527499999999999858e+01 1.960819253574922968e-02
2.530000000000000071e+01 1.959676914966560976e-02
2.532499999999999929e+01 1.958488256155179949e-02
2.535000000000000142e+01 1.957254490586138318e-02
2.537500000000000000e+01 1.955977843605268690e-02
2.539999999999999858e+01 1.954661526084770090e-02
2.542500000000000071e+01 1.953309689763962562e-02
2.544999999999999929e+01 1.951927364933209522e-02
2.547500000000000142e+01 1.950520381426889757e-02
2.550000000000000000e+01 1.949095274168244352e-02
2.552499999999999858e+01 1.947659174827594611e-02
2.555000000000000071e+01 1.946219691393915804e-02
2.557499999999999929e+01 1.944784777694938752e-02
2.560000000000000142e+01 1.943362595091667824e-02
2.562500000000000000e+01 1.941961368727235585e-02
2.564999999999999858e+01 1.940589240818554884e-02
2.567500000000000071e+01 1.939254123554307388e-02
2.569999999999999929e+01 1.937963554171734729e-02
2.572500000000000142e+01 1.936724554774432380e-02
2.575000000000000000e+01 1.935543499380764851e-02
2.577499999999999858e+01 1.934425990575595888e-02
2.580000000000000071e+01 1.933376747997276462e-02
2.582499999999999929e+01 1.932399510678740537e-02
2.585000000000000142e+01 1.931496955060692244e-02
2.587500000000000000e+01 1.930670630205597732e-02
2.589999999999999858e+01 1.929920911477005230e-02
2.592500000000000071e+01 1.929246973617964461e-02
2.594999999999999929e+01 1.928646783856594743e-02
2.597500000000000142e+01 1.928117115304122178e-02
2.600000000000000000e+01 1.927653580577468745e-02
2.602499999999999858e+01 1.927250685229034421e-02
2.605000000000000071e+01 1.926901900216464181e-02
2.607499999999999929e+01 1.926599752325478324e-02
2.610000000000000142e+01 1.926335931126196485e-02
2.612500000000000000e+01 1.926101410767278727e-02
2.614999999999999858e+01 1.925886584626549172e-02
2.617500000000000071e+01 1.925681410618850301e-02
2.619999999999999929e+01 1.925475564745937965e-02
2.622500000000000142e+01 1.925258600324939243e-02
2.625000000000000000e+01 1.925020110214515495e-02
2.627499999999999858e+01 1.924749889277115392e-02
2.630000000000000071e+01 1.924438094306760508e-02
2.632499999999999929e+01 1.924075398673671597e-02
2.635000000000000142e+01 1.923653139003508178e-02
2.637500000000000000e+01 1.923163451344061040e-02
2.639999999999999858e+01 1.922599394432780717e-02
2.642500000000000071e+01 1.921955057898968103e-02
2.644999999999999929e+01 1.921225653470776465e-02
2.647500000000000142e+01 1.920407587567500227e-02
2.650000000000000000e+01 1.919498513963563760e-02
2.652499999999999858e+01 1.918497365549908656e-02
2.655000000000000071e+01 1.917404364591841076e-02
2.657499999999999929e+01 1.916221011250960823e-02
2.660000000000000142e+01 1.914950050527297609e-02
2.662500000000000000e+01 1.913595418138085841e-02
2.664999999999999858e+01 1.912162166243615985e-02
2.667500000000000071e+01 1.910656370278985411e-02
2.669999999999999929e+01 1.909085018474728987e-02
2.672500000000000142e+01 1.907455885971279935e-02
2.675000000000000000e+01 1.905777395708863065e-02
2.677499999999999858e+01 1.904058468497215120e-02
2.680000000000000071e+01 1.902308364882689812e-02
2.682499999999999929e+01 1.900536521567904730e-02
2.685000000000000142e+01 1.898752385249308922e-02
2.687500000000000000e+01 1.896965246780621048e-02
2.689999999999999858e+01 1.895184078577986583e-02
2.692500000000000071e+01 1.893417378124535172e-02
2.694999999999999929e+01 1.891673020339987993e-02
2.697500000000000142e+01 1.889958121397577670e-02
2.700000000000000000e+01 1.888278916424735235e-02
2.702499999999999858e+01 1.886640653219117877e-02
2.705000000000000071e+01 1.885047503881654568e-02
2.707499999999999929e+01 1.883502495914512326e-02
2.710000000000000142e+01 1.882007464018823836e-02
2.712500000000000000e+01 1.880563023416163826e-02
2.714999999999999858e+01 1.879168565191543031e-02
2.717500000000000071e+01 1.877822273694989413e-02
2.719999999999999929e+01 1.876521165699640906e-02
2.722500000000000142e+01 1.875261150583615311e-02
2.725000000000000000e+01 1.874037110439537301e-02
2.727499999999999858e+01 1.872842998649045312e-02
2.730000000000000071e+01 1.871671955115226244e-02
2.732499999999999929e+01 1.870516436050979892e-02
2.735000000000000142e+01 1.869368355934969764e-02
2.737500000000000000e+01 1.868219239038144963e-02
2.739999999999999858e+01 1.867060377718994177e-02
2.742500000000000071e+01 1.865882994579163429e-02
2.744999999999999929e+01 1.864678405468913128e-02
2.747500000000000142e+01 1.863438180332909269e-02
2.750000000000000000e+01 1.862154298899713900e-02
2.752499999999999858e+01 1.860819298323948545e-02
2.755000000000000071e+01 1.859426410027468016e-02
2.757499999999999929e+01 1.857969683185509147e-02
2.760000000000000142e+01 1.856444092558349676e-02
2.762500000000000000e+01 1.854845628652869796e-02
2.764999999999999858e+01 1.853171368538896607e-02
2.767500000000000071e+01 1.851419526008400013e-02
2.769999999999999929e+01 1.849589480168840547e-02
2.772500000000000142e+01 1.847681781969792020e-02
2.775000000000000000e+01 1.845698138570162766e-02
2.777499999999999858e+01 1.843641375929820289e-02
2.780000000000000071e+01 1.841515380365427168e-02
2.782499999999999929e+01 1.839325020282586362e-02
2.785000000000000142e+01 1.837076049643138892e-02
2.787500000000000000e+01 1.834774995080633297e-02
2.789999999999999858e+01 1.832429028949790137e-02
2.792500000000000071e+01 1.830045830784939737e-02
2.794999999999999929e+01 1.827633439967840320e-02
2.797500000000000142e+01 1.825200102483361120e-02
2.800000000000000000e+01 1.822754114848977244e-02
2.802499999999999858e+01 1.820303668284567120e-02
2.805000000000000071e+01 1.817856696224762897e-02
2.807499999999999929e+01 1.815420728209327442e-02
2.810000000000000142e+01 1.813002753035696316e-02
2.812500000000000000e+01 1.810609093888842364e-02
2.814999999999999858e+01 1.808245297926933115e-02
2.817500000000000071e+01 1.805916042484389389e-02
2.819999999999999929e+01 1.803625059747274509e-02
2.822500000000000142e+01 1.801375081350104282e-02
2.825000000000000000e+01 1.799167803966893187e-02
2.827499999999999858e+01 1.797003876505489164e-02
2.830000000000000071e+01 1.794882909101285992e-02
2.832499999999999929e+01 1.792803503639005952e-02
2.835000000000000142e+01 1.790763305088229063e-02
2.837500000000000000e+01 1.788759072520578267e-02
2.839999999999999858e+01 1.786786768248407528e-02
2.842500000000000071e+01 1.784841663171957485e-02
2.844999999999999929e+01 1.782918456050234038e-02
2.847500000000000142e+01 1.781011404169829601e-02
2.850000000000000000e+01 1.779114462596337667e-02
2.852499999999999858e+01 1.777221429045032150e-02
2.855000000000000071e+01 1.775326091285853863e-02
2.857499999999999929e+01 1.773422373931008572e-02
2.860000000000000142e+01 1.771504481486143104e-02
2.862500000000000000e+01 1.769567034604431488e-02
2.864999999999999858e+01 1.767605196651271526e-02
2.867500000000000071e+01 1.765614787851985112e-02
2.869999999999999929e+01 1.763592384620487827e-02
2.872500000000000142e+01 1.761535401911580243e-02
2.875000000000000000e+01 1.759442156857876985e-02
2.877499999999999858e+01 1.757311912309894239e-02
2.880000000000000071e+01 1.755144899361631150e-02
2.882499999999999929e+01 1.752942318365200564e-02
2.885000000000000142e+01 1.750706318417775617e-02
2.887500000000000000e+01 1.748439955756969644e-02
2.889999999999999858e+01 1.746147131965258142e-02
2.892500000000000071e+01 1.743832513310018817e-02
2.894999999999999929e+01 1.741501432962929211e-02
2.897500000000000142e+01 1.739159778213669391e-02
2.900000000000000000e+01 1.736813865124155198e-02
2.902499999999999858e+01 1.734470303333788405e-02
2.905000000000000071e+01 1.732135853977760653e-02
2.907499999999999929e+01 1.729817283797027705e-02
2.910000000000000142e+01 1.727521218649606671e-02
2.912500000000000000e+01 1.725253999631523941e-02
2.914999999999999858e+01 1.723021544982290473e-02
2.917500000000000071e+01 1.720829220846645594e-02
2.919999999999999929e+01 1.718681723766818034e-02
2.922500000000000142e+01 1.716582977553489028e-02
2.925000000000000000e+01 1.714536046893699178e-02
2.927499999999999858e+01 1.712543069684640748e-02
2.930000000000000071e+01 1.710605209705664750e-02
2.932499999999999929e+01 1.708722630808168913e-02
2.935000000000000142e+01 1.706894493334025487e-02
2.937500000000000000e+01 1.705118973014092820e-02
2.939999999999999858e+01 1.703393302090839417e-02
2.942500000000000071e+01 1.701713831945410316e-02
2.944999999999999929e+01 1.700076116030382542e-02
2.947500000000000142e+01 1.698475011435191320e-02
2.950000000000000000e+01 1.696904797040872312e-02
2.952499999999999858e+01 1.695359305789603810e-02
2.955000000000000071e+01 1.693832068299079216e-02
2.957499999999999929e+01 1.692316464765643291e-02
2.960000000000000142e+01 1.690805881902540317e-02
2.962500000000000000e+01 1.689293871502522623e-02
2.964999999999999858e+01 1.687774307166677729e-02
2.967500000000000071e+01 1.686241535733456684e-02
2.969999999999999929e+01 1.684690520037122538e-02
2.972500000000000142e+01 1.683116969766240353e-02
2.975000000000000000e+01 1.681517457436286847e-02
2.977499999999999858e+01 1.679889516774201244e-02
2.980000000000000071e+01 1.678231721165615586e-02
2.982499999999999929e+01 1.676543740228178536e-02
2.985000000000000142e+01 1.674826373021413503e-02
2.987500000000000000e+01 1.673081556896207495e-02
2.989999999999999858e+01 1.671312351480868166e-02
2.992500000000000071e+01 1.669522897848622900e-02
2.994999999999999929e+01 1.667718353413670876e-02
2.997500000000000142e+01 1.665904803642641319e-02
3.000000000000000000e+01 1.664089152141648431e-02
3.002499999999999858e+01 1.662278991168268660e-02
3.005000000000000071e+01 1.660482455033891053e-02
3.007499999999999929e+01 1.658708059235972782e-02
3.010000000000000142e+01 1.656964528486033067e-02
3.012500000000000000e+01 1.655260617063188805e-02
3.014999999999999858e+01 1.653604925079939919e-02
3.017500000000000071e+01 1.652005714400732461e-02
3.019999999999999929e+01 1.650470727961851625e-02
3.022500000000000142e+01 1.649007016208313067e-02
3.025000000000000000e+01 1.647620774254592718e-02
3.027499999999999858e+01 1.646317193186173575e-02
3.030000000000000071e+01 1.645100328647694510e-02
3.032499999999999929e+01 1.643972989549384936e-02
3.035000000000000142e+01 1.642936649347629882e-02
3.037500000000000000e+01 1.641991381891728594e-02
3.039999999999999858e+01 1.641135823396912621e-02
3.042500000000000071e+01 1.640367161541545285e-02
3.044999999999999929e+01 1.639681152204181327e-02
3.047500000000000142e+01 1.639072163767004223e-02
3.050000000000000000e+01 1.638533248396706679e-02
3.052499999999999858e+01 1.638056239146714124e-02
3.055000000000000071e+01 1.637631871236586503e-02
3.057499999999999929e+01 1.637249925346484156e-02
3.060000000000000142e+01 1.636899390321358561e-02
3.062500000000000000e+01 1.636568642297432305e-02
3.064999999999999858e+01 1.636245636888343513e-02
3.067500000000000071e+01 1.635918110813404727e-02
3.069999999999999929e+01 1.635573789134407707e-02
3.072500000000000142e+01 1.635200594119261275e-02
3.075000000000000000e+01 1.634786851711437020e-02
3.077499999999999858e+01 1.634321491609974294e-02
3.080000000000000071e+01 1.633794237045902917e-02
3.082499999999999929e+01 1.633195780553397397e-02
3.085000000000000142e+01 1.632517942265424016e-02
3.087500000000000000e+01 1.631753807621639379e-02
3.089999999999999858e+01 1.630897841737109777e-02
3.092500000000000071e+01 1.629945978166217821e-02
3.094999999999999929e+01 1.628895680259198966e-02
3.097500000000000142e+01 1.627745973883951147e-02
3.100000000000000000e+01 1.626497450802657369e-02
3.102499999999999858e+01 1.625152242623402812e-02
3.105000000000000071e+01 1.623713965777346876e-02
3.107499999999999929e+01 1.622187638587991101e-02
3.110000000000000142e+01 1.620579572010813832e-02
3.112500000000000000e+01 1.618897236193675773e-02
3.114999999999999858e+01 1.617149105417303376e-02
3.117500000000000071e+01 1.615344484476254061e-02
3.119999999999999929e+01 1.613493319861421249e-02
3.122500000000000142e+01 1.611605999448270654e-02
3.125000000000000000e+01 1.609693144589564515e-02
3.127499999999999858e+01 1.607765398673319593e-02
3.130000000000000071e+01 1.605833216257865348e-02
3.132499999999999929e+01 1.603906656908016259e-02
3.135000000000000142e+01 1.601995187703641327e-02
3.137500000000000000e+01 1.600107498289817312e-02
3.139999999999999858e+01 1.598251332008220754e-02
3.142500000000000071e+01 1.596433336393861604e-02
3.144999999999999929e+01 1.594658935884249104e-02
3.147500000000000142e+01 1.592932229183807030e-02
3.150000000000000000e+01 1.591255913213026132e-02
3.152499999999999858e+01 1.589631235056663747e-02
3.155000000000000071e+01 1.588057972759983338e-02
3.157499999999999929e+01 1.586534445279576960e-02
3.160000000000000142e+01 1.585057551272044118e-02
3.162500000000000000e+01 1.583622835882740262e-02
3.164999999999999858e+01 1.582224584105942911e-02
3.167500000000000071e+01 1.580855938800539151e-02
3.169999999999999929e+01 1.579509040912769474e-02
3.172500000000000142e+01 1.578175189064305595e-02
3.175000000000000000e+01 1.576845015260348523e-02
3.177499999999999858e+01 1.575508673159214251e-02
3.180000000000000071e+01 1.574156035098579418e-02
3.182499999999999929e+01 1.572776893922388924e-02
3.185000000000000142e+01 1.571361165529413378e-02
3.187500000000000000e+01 1.569899088099314943e-02
3.189999999999999858e+01 1.568381413996587648e-02
3.192500000000000071e+01 1.566799590516795501e-02
3.194999999999999929e+01 1.565145925893163606e-02
3.197500000000000142e+01 1.563413737256842337e-02
3.200000000000000000e+01 1.561597477661797601e-02
3.202499999999999858e+01 1.559692839677793985e-02
3.204999999999999716e+01 1.557696833580023837e-02
3.207500000000000284e+01 1.555607838682051755e-02
3.210000000000000142e+01 1.553425626901244749e-02
3.212500000000000000e+01 1.551151358263710284e-02
3.214999999999999858e+01 1.548787548592050686e-02
3.217499999999999716e+01 1.546338010251911352e-02
3.220000000000000284e+01 1.543807767353162427e-02
3.222500000000000142e+01 1.541202947356310957e-02
3.225000000000000000e+01 1.538530651515179065e-02
3.227499999999999858e+01 1.535798807030387680e-02
3.229999999999999716e+01 1.533016004154457573e-02
3.232500000000000284e+01 1.530191321811096711e-02
3.235000000000000142e+01 1.527334145492780307e-02
3.237500000000000000e+01 1.524453981375795825e-02
3.239999999999999858e+01 1.521560270629709194e-02
3.242499999999999716e+01 1.518662207892045729e-02
3.245000000000000284e+01 1.515768567749789866e-02
3.247500000000000142e+01 1.512887542888571857e-02
3.250000000000000000e+01 1.510026597294101755e-02
3.252499999999999858e+01 1.507192337541434293e-02
3.254999999999999716e+01 1.504390404792114999e-02
3.257500000000000284e+01 1.501625389674826963e-02
3.260000000000000142e+01 1.498900771682280461e-02
3.262500000000000000e+01 1.496218884180989389e-02
3.264999999999999858e+01 1.493580905584581189e-02
3.267499999999999716e+01 1.490986876611256895e-02
3.270000000000000284e+01 1.488435743015344842e-02
3.272500000000000142e+01 1.485925422601903217e-02
3.275000000000000000e+01 1.483452894783751641e-02
3.277499999999999858e+01 1.481014310481653176e-02
3.279999999999999716e+01 1.478605119684579805e-02
3.282500000000000284e+01 1.476220213631927736e-02
3.285000000000000142e+01 1.473854078246982988e-02
3.287500000000000000e+01 1.471500955223375290e-02
3.289999999999999858e+01 1.469155007001820071e-02
3.292499999999999716e+01 1.466810481805998212e-02
3.295000000000000284e+01 1.464461874919364276e-02
3.297500000000000142e+01 1.462104082490502631e-02
3.300000000000000000e+01 1.459732544336647675e-02
3.302499999999999858e+01 1.457343372476903667e-02
3.304999999999999716e+01 1.454933462485691331e-02
3.307500000000000284e+01 1.452500585139059573e-02
3.310000000000000142e+01 1.450043456301790094e-02
3.312500000000000000e+01 1.447561783522945296e-02
3.314999999999999858e+01 1.445056288324483330e-02
3.317499999999999716e+01 1.442528703768404183e-02
3.320000000000000284e+01 1.439981747424109136e-02
3.322500000000000142e+01 1.437419070468643097e-02
3.325000000000000000e+01 1.434845184166790889e-02
3.327499999999999858e+01 1.432265365530774673e-02
3.329999999999999716e+01 1.429685544419893525e-02
3.332500000000000284e+01 1.427112174781426483e-02
3.335000000000000142e+01 1.424552093088234973e-02
3.337500000000000000e+01 1.422012367328475725e-02
3.339999999999999858e+01 1.419500140113485152e-02
3.342499999999999716e+01 1.417022469607094257e-02
3.345000000000000284e+01 1.414586172023038838e-02
3.347500000000000142e+01 1.412197669391445157e-02
3.350000000000000000e+01 1.409862846184859950e-02
3.352499999999999858e+01 1.407586918161689293e-02
3.354999999999999716e+01 1.405374316528451052e-02
3.357500000000000284e+01 1.403228590116749330e-02
3.360000000000000142e+01 1.401152327910907755e-02
3.362500000000000000e+01 1.399147103717847396e-02
3.364999999999999858e+01 1.397213444310547309e-02
3.367499999999999716e+01 1.395350821787212300e-02
3.370000000000000284e+01 1.393557670341563984e-02
3.372500000000000142e+01 1.391831427057740357e-02
3.375000000000000000e+01 1.390168595794694269e-02
3.377499999999999858e+01 1.388564832649606673e-02
3.379999999999999716e+01 1.387015051024204429e-02
3.382500000000000284e+01 1.385513543822103691e-02
3.385000000000000142e+01 1.384054119914095507e-02
3.387500000000000000e+01 1.382630251658486902e-02
3.389999999999999858e+01 1.381235230010930196e-02
3.392499999999999716e+01 1.379862323556411453e-02
3.395000000000000284e+01 1.378504937694810181e-02
3.397500000000000142e+01 1.377156770210836816e-02
3.400000000000000000e+01 1.375811959514257506e-02
3.402499999999999858e+01 1.374465221997992100e-02
3.404999999999999716e+01 1.373111975215152158e-02
3.407500000000000284e+01 1.371748443875381827e-02
3.410000000000000142e+01 1.370371746064637415e-02
3.412500000000000000e+01 1.368979957533954066e-02
3.414999999999999858e+01 1.367572152419447781e-02
3.417499999999999716e+01 1.366148419273237671e-02
3.420000000000000284e+01 1.364709851885745166e-02
3.422500000000000142e+01 1.363258514939450113e-02
3.425000000000000000e+01 1.361797385139566254e-02
3.427499999999999858e+01 1.360330269032856282e-02
3.429999999999999716e+01 1.358861699275814433e-02
3.432500000000000284e+01 1.357396811626671203e-02
3.435000000000000142e+01 1.355941205399582850e-02
3.437500000000000000e+01 1.354500790511396942e-02
3.439999999999999858e+01 1.353081624594147485e-02
3.442499999999999716e+01 1.351689743880910878e-02
3.445000000000000284e+01 1.350330991763998290e-02
3.447500000000000142e+01 1.349010848977814751e-02
3.450000000000000000e+01 1.347734269376041523e-02
3.452499999999999858e+01 1.346505525144923825e-02
3.454999999999999716e+01 1.345328065137269338e-02
3.457500000000000284e+01 1.344204389707654747e-02
3.460000000000000142e+01 1.343135945106661670e-02
3.462500000000000000e+01 1.342123040055618818e-02
3.464999999999999858e+01 1.341164786646494023e-02
3.467499999999999716e+01 1.340259067186735584e-02
3.470000000000000284e+01 1.339402528024114664e-02
3.472500000000000142e+01 1.338590600808981478e-02
3.475000000000000000e+01 1.337817551020609695e-02
3.477499999999999858e+01 1.337076553001715143e-02
3.479999999999999716e+01 1.336359790121196059e-02
3.482500000000000284e+01 1.335658578146319171e-02
3.485000000000000142e+01 1.334963509354596643e-02
3.487500000000000000e+01 1.334264614464900425e-02
3.489999999999999858e+01 1.333551539022105256e-02
3.492499999999999716e+01 1.332813730570834437e-02
3.495000000000000284e+01 1.332040632654185076e-02
3.497500000000000142e+01 1.331221881525473591e-02
3.500000000000000000e+01 1.330347501368430753e-02
3.502499999999999858e+01 1.329408093824076091e-02
3.504999999999999716e+01 1.328395017738307100e-02
3.507500000000000284e+01 1.327300555219607971e-02
3.510000000000000142e+01 1.326118060391786799e-02
3.512500000000000000e+01 1.324842087564765195e-02
3.514999999999999858e+01 1.323468496009511455e-02
3.517499999999999716e+01 1.321994528976808310e-02
3.520000000000000284e+01 1.320418865197184669e-02
3.522500000000000142e+01 1.318741641632425673e-02
3.525000000000000000e+01 1.316964446931305661e-02
3.527499999999999858e+01 1.315090285595601517e-02
3.529999999999999716e+01 1.313123513566611528e-02
3.532500000000000284e+01 1.311069746504370942e-02
3.535000000000000142e+01 1.308935742650518992e-02
3.537500000000000000e+01 1.306729262709943086e-02
3.539999999999999858e+01 1.304458909663805052e-02
3.542499999999999716e+01 1.302133951898968439e-02
3.545000000000000284e+01 1.299764133349133773e-02
3.547500000000000142e+01 1.297359474674627741e-02
3.550000000000000000e+01 1.294930069657059500e-02
3.552499999999999858e+01 1.292485881116311672e-02
3.554999999999999716e+01 1.290036540647465993e-02
3.557500000000000284e+01 1.287591156402657748e-02
3.560000000000000142e+01 1.285158132960547740e-02
3.562500000000000000e+01 1.282745007045375113e-02
3.564999999999999858e+01 1.280358302551523782e-02
3.567499999999999716e+01 1.278003407848464154e-02
3.570000000000000284e+01 1.275684477914539386e-02
3.572500000000000142e+01 1.273404363256964468e-02
3.575000000000000000e+01 1.271164567022586497e-02
3.577499999999999858e+01 1.268965231076232998e-02
3.579999999999999716e+01 1.266805151197032166e-02
3.582500000000000284e+01 1.264681820914599630e-02
3.585000000000000142e+01 1.262591502882345375e-02
3.587500000000000000e+01 1.260529326095576053e-02
3.589999999999999858e+01 1.258489406707002413e-02
3.592499999999999716e+01 1.256464989684219838e-02
3.595000000000000284e+01 1.254448608117215806e-02
3.597500000000000142e+01 1.252432256612519626e-02
3.600000000000000000e+01 1.250407574913746898e-02
3.602499999999999858e+01 1.248366037695441098e-02
3.604999999999999716e+01 1.246299146353039428e-02
3.607500000000000284e+01 1.244198618598852009e-02
3.610000000000000142e+01 1.242056571736488771e-02
3.612500000000000000e+01 1.239865695657643707e-02
3.614999999999999858e+01 1.237619411848721492e-02
3.617499999999999716e+01 1.235312015040694079e-02
3.620000000000000284e+01 1.232938794513679243e-02
3.622500000000000142e+01 1.230496132567374806e-02
3.625000000000000000e+01 1.227981578181268005e-02
3.627499999999999858e+01 1.225393894444916885e-02
3.629999999999999716e+01 1.222733078944886718e-02
3.632500000000000284e+01 1.220000356910072928e-02
3.635000000000000142e+01 1.217198147509777895e-02
3.637500000000000000e+01 1.214330004315368645e-02
3.639999999999999858e+01 1.211400531508024028e-02
3.642499999999999716e+01 1.208415277938033536e-02
3.645000000000000284e+01 1.205380611641370531e-02
3.647500000000000142e+01 1.202303577831745005e-02
3.650000000000000000e+01 1.199191743738403995e-02
3.652499999999999858e+01 1.196053033946322709e-02
3.654999999999999716e+01 1.192895560060384247e-02
3.657500000000000284e+01 1.189727448651443573e-02
3.660000000000000142e+01 1.186556671416090156e-02
3.662500000000000000e+01 1.183390881432760423e-02
3.664999999999999858e+01 1.180237259218523682e-02
3.667499999999999716e+01 1.177102372039435553e-02
3.670000000000000284e+01 1.173992049607600584e-02
3.672500000000000142e+01 1.170911278893562585e-02
3.675000000000000000e+01 1.167864120328531569e-02
3.677499999999999858e+01 1.164853647161173331e-02
3.679999999999999716e+01 1.161881909184534108e-02
3.682500000000000284e+01 1.158949921478112308e-02
3.685000000000000142e+01 1.156057678220509186e-02
3.687500000000000000e+01 1.153204191042317764e-02
3.689999999999999858e+01 1.150387550818506022e-02
3.692499999999999716e+01 1.147605011248805512e-02
3.695000000000000284e+01 1.144853092074341223e-02
3.697500000000000142e+01 1.142127699321237623e-02
3.700000000000000000e+01 1.139424259567486768e-02
3.702499999999999858e+01 1.136737864912659192e-02
3.704999999999999716e+01 1.134063425083489327e-02
3.707500000000000284e+01 1.131395822952913598e-02
3.710000000000000142e+01 1.128730069679509569e-02
3.712500000000000000e+01 1.126061455693937174e-02
3.714999999999999858e+01 1.123385693875405232e-02
3.717499999999999716e+01 1.120699051445262634e-02
3.720000000000000284e+01 1.117998467404015747e-02
3.722500000000000142e+01 1.115281652668928883e-02
3.725000000000000000e+01 1.112547170516173071e-02
3.727499999999999858e+01 1.109794495394861996e-02
3.729999999999999716e+01 1.107024048704903507e-02
3.732500000000000284e+01 1.104237210698786123e-02
3.735000000000000142e+01 1.101436308233337026e-02
3.737500000000000000e+01 1.098624578679524913e-02
3.739999999999999858e+01 1.095806110884401698e-02
3.742499999999999716e+01 1.092985764612583975e-02
3.745000000000000284e+01 1.090169070420141280e-02
3.747500000000000142e+01 1.087362112385547100e-02
3.750000000000000000e+01 1.084571396518769354e-02
3.752499999999999858e+01 1.081803708023991277e-02
3.754999999999999716e+01 1.079065960842017467e-02
3.757500000000000284e+01 1.076365043092002323e-02
3.760000000000000142e+01 1.073707662116483302e-02
3.762500000000000000e+01 1.071100192844245090e-02
3.764999999999999858e+01 1.068548533101638942e-02
3.767499999999999716e+01 1.066057969332060047e-02
3.770000000000000284e+01 1.063633055930645645e-02
3.772500000000000142e+01 1.061277511075009103e-02
3.775000000000000000e+01 1.058994131535251695e-02
3.777499999999999858e+01 1.056784728491794167e-02
3.779999999999999716e+01 1.054650085889405862e-02
3.782500000000000284e+01 1.052589942316668921e-02
3.785000000000000142e+01 1.050602996842858565e-02
3.787500000000000000e+01 1.048686938672653036e-02
3.789999999999999858e+01 1.046838499919346480e-02
3.792499999999999716e+01 1.045053530245064824e-02
3.795000000000000284e+01 1.043327091604467904e-02
3.797500000000000142e+01 1.041653570853325224e-02
3.800000000000000000e+01 1.040026807561847573e-02
3.802499999999999858e+01 1.038440234016796242e-02
3.804999999999999716e+01 1.036887024110840286e-02
3.807500000000000284e+01 1.035360247607101242e-02
3.810000000000000142e+01 1.033853026143606406e-02
3.812500000000000000e+01 1.032358687302590942e-02
3.814999999999999858e+01 1.030870913112245067e-02
3.817499999999999716e+01 1.029383879482492770e-02
3.820000000000000284e+01 1.027892383289777910e-02
3.822500000000000142e+01 1.026391954109554146e-02
3.825000000000000000e+01 1.024878947964688418e-02
3.827499999999999858e+01 1.023350620868773188e-02
3.829999999999999716e+01 1.021805180424626043e-02
3.832500000000000284e+01 1.020241814247407550e-02
3.835000000000000142e+01 1.018660694526495394e-02
3.837500000000000000e+01 1.017062958598657918e-02
3.839999999999999858e+01 1.015450665966601690e-02
3.842499999999999716e+01 1.013826732748119089e-02
3.845000000000000284e+01 1.012194845071113375e-02
3.847500000000000142e+01 1.010559353422471086e-02
3.850000000000000000e+01 1.008925150404217132e-02
3.852499999999999858e+01 1.007297534741705310e-02
3.854999999999999716e+01 1.005682064709088885e-02
3.857500000000000284e+01 1.004084404389391470e-02
3.860000000000000142e+01 1.002510166355010182e-02
3.862500000000000000e+01 1.000964754439805961e-02
3.864999999999999858e+01 9.994532102788037178e-03
3.867499999999999716e+01 9.979800672049725374e-03
3.870000000000000284e+01 9.965492149278052866e-03
3.872500000000000142e+01 9.951637781731142648e-03
3.875000000000000000e+01 9.938260121433598396e-03
3.877499999999999858e+01 9.925372172752178895e-03
3.879999999999999716e+01 9.912976753287018258e-03
3.882500000000000284e+01 9.901066083556842035e-03
3.885000000000000142e+01 9.889621615699369442e-03
3.887500000000000000e+01 9.878614105972530202e-03
3.889999999999999858e+01 9.868003930242735464e-03
3.892499999999999716e+01 9.857741636118100628e-03
3.895000000000000284e+01 9.847768719943760657e-03
3.897500000000000142e+01 9.838018611772825564e-03
3.900000000000000000e+01 9.828417846656558113e-03
3.902499999999999858e+01 9.818887396370566678e-03
3.904999999999999716e+01 9.809344131998166960e-03
3.907500000000000284e+01 9.799702384869636573e-03
3.910000000000000142e+01 9.789875571073738611e-03
3.912500000000000000e+01 9.779777843357385578e-03
3.914999999999999858e+01 9.769325733619273547e-03
3.917499999999999716e+01 9.758439749432459495e-03
3.920000000000000284e+01 9.747045889122975953e-03
3.922500000000000142e+01 9.735077041793493979e-03
3.925000000000000000e+01 9.722474241372921697e-03
3.927499999999999858e+01 9.709187747075916086e-03
3.929999999999999716e+01 9.695177926695844328e-03
3.932500000000000284e+01 9.680415923608069739e-03
3.935000000000000142e+01 9.664884093342254387e-03
3.937500000000000000e+01 9.648576200820898827e-03
3.939999999999999858e+01 9.631497374796847352e-03
3.942499999999999716e+01 9.613663821516616609e-03
3.945000000000000284e+01 9.595102305132840320e-03
3.947500000000000142e+01 9.575849407568596053e-03
3.950000000000000000e+01 9.555950585560728933e-03
3.952499999999999858e+01 9.535459047085101045e-03
3.954999999999999716e+01 9.514434473391951072e-03
3.957500000000000284e+01 9.492941616293024826e-03
3.960000000000000142e+01 9.471048802989186569e-03
3.962500000000000000e+01 9.448826382726090850e-03
3.964999999999999858e+01 9.426345150691180924e-03
3.967499999999999716e+01 9.403674784907325654e-03
3.970000000000000284e+01 9.380882331386741074e-03
3.972500000000000142e+01 9.358030771499279213e-03
3.975000000000000000e+01 9.335177703442585490e-03
3.977499999999999858e+01 9.312374166860870606e-03
3.979999999999999716e+01 9.289663636234377769e-03
3.982500000000000284e+01 9.267081204550550039e-03
3.985000000000000142e+01 9.244652974285647723e-03
3.987500000000000000e+01 9.222395667803386793e-03
3.989999999999999858e+01 9.200316464082715015e-03
3.992499999999999716e+01 9.178413063447419043e-03
3.995000000000000284e+01 9.156673976582863778e-03
3.997500000000000142e+01 9.135079028987273461e-03
\end{filecontents}

\begin{tikzpicture}
  \begin{groupplot}[
      group style={
          group name=my plots,
          group size=1 by 2,
          xlabels at=edge bottom,
          xticklabels at=edge bottom,
          vertical sep=2pt,
          every axis yticklabel/.style={/pgf/number format/fixed}
      },
      width=\columnwidth,
      height=0.5\columnwidth,
      xlabel={Time (\si{\pico\second})},
      %xmin=0, xmax=40,
      %ytick align=outside,
      %xtick align=outside
  ]
  \nextgroupplot[ylabel = {Population}, tick label style={/pgf/number format/fixed}]
  \addplot graphics[xmin=0, xmax=40, ymin=0, ymax=1]{figures/rendered_1024.png};
  \nextgroupplot[ylabel = {IPR}, yticklabel style={/pgf/number format/fixed}] 
  \addplot[smooth, thick] table {ipr_1024.dat};
  \end{groupplot}

\end{tikzpicture}

  \caption{\label{fig:population dynamics}Population dynamics of 512 interacting \qds{}.
    The majority of the \qds{} follow a trajectory prescribed by the incident pulse; several pairs of dots, however, become strongly correlated due to the small distance between particles.
  }
\end{figure}
