\section{\label{section:problem statement}Problem Statement}
We wish to consider the evolution of a \qd{} in response to an externally produced electric field.
If we concern ourselves only with electric dipole transitions in a two-level system, we may write the time-dependence of the \qd's state as
\begin{equation}
  i \hbar \dv{\hat{\rho}}{t} = \commutator{\hat{\mathcal{H}}(t)}{\hat{\rho}} - \hat{\mathcal{D}}\qty[\hat{\rho}].
  \label{eq:liouville}
\end{equation}
Here, $\hat{\mathcal{H}}(t)$ represents a local Hamiltonian that governs the internal two-level structure of the \qd{} as well as its interaction with an electromagnetic field while $\hat{\mathcal{D}}$ provides dissipation terms that account for emission effects phenomenologically.
Formally,
\begin{subequations}
  \begin{align}
    \hat{\mathcal{H}}(t) &\equiv \mqty(0 & \hbar \chi(t) \\ \hbar \chi^*(t) & \hbar \omega_0) \label{eq:hamiltonian}\\
    \hat{\mathcal{D}}\qty[\hat{\rho}] &\equiv \mqty( \qty(\rho_{00} - 1)/{T_1} & \rho_{01}/{T_2} \\ \rho_{10}/{T_2} & \rho_{11}/T_1 ) \label{eq:dissipator}
  \end{align}
\end{subequations}
where \todo{Expression tree; bad form?}$\chi(t) \equiv \vb{d} \cdot \hat{\vb{E}}(\vb{r}, t)/\hbar$, $\vb{d} \equiv \matrixel{1}{e \hat{\vb{r}}}{0}$, and $\ket{0}$ \& $\ket{1}$ respectively represent the highest valence and lowest conduction states of the \qd{} under consideration.
Finally, the $T_1$ and $T_2$ constants characterize average emission and relaxation times~\cite{}.
Assuming the radiation in the system contains only wavelengths much longer than the size of any given \qd{} we may neglect the spatial variation in $\vb{E}(\vb{r}, t)$ in calculating $\chi(t)$ and thereby consider each \qd{} as a localized point in space.

As $\expval{\hat{A}} = \Tr[\hat{\rho}\hat{A}]$ yields the expectation value of an operator $\hat{A}$ in a density matrix formulation, the evolution of \cref{eq:liouville} for a \qd{} located at $\vb{r}_j$ gives rise to a classical polarization density $\vb{P} = \expval{\hat{\vb{P}}}$ that produces additional radiation through secondary emission effects.
Writing $\vb{P}$ as
\begin{equation}
  \begin{aligned}
    \vb{P}(\vb{r}, t) &= \Tr[\hat{\rho} \hat{\vb{d}}] \delta(\vb{r} - \vb{r}_j) \\
                      &= 2 \Re[\rho_{01}\vb{d}]\delta(\vb{r} - \vb{r}_j),
  \end{aligned}
\end{equation}
we may compute the total electric field at any point with a Green's function formulation
\begin{equation}
  \vb{E}(\vb{r}, t) = \vb{E}_0(\vb{r}, t) + \int \hat{G}(\vb{r} - \vb{r}'; t - t')\vb{P}(\vb{r}', t') \dd[3]{\vb{r}'} \dd{t'}.
  \label{eq:total field}
\end{equation}
In three dimensions,
\begin{equation}
  \hat{G} = \mathcal{F}_{\omega \to t}^{-1}\qty[ \qty(\tensor{I} + \frac{\nabla \nabla}{\qty(\omega/c)^2})\frac{-e^{i \omega \abs{\vb{r} - \vb{r}'}/c}}{4\pi \epsilon \abs{\vb{r} - \vb{r}'}} ]
  \label{eq:dyadic}
\end{equation}
(where $\mathcal{F}_{\omega \to t}^{-1}$ denotes an inverse Fourier transform with respect to $\omega$) and relates $\vb{E}$ to $\vb{P}$, $\dot{\vb{P}}$, and $\ddot{\vb{P}}$~\cite{Rothwell2009}.
Thus, in a system composed of multiple dots, \cref{eq:total field} couples the the evolution of each \qd{} by way of the off-diagonal matrix elements appearing in \cref{eq:hamiltonian}.
