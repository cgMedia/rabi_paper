\section{Computational approach}
\begin{figure}
  \centering
  \begin{filecontents}{frames.dat}
0.  1.000000000001855 1.000000000001851
0.0078125 1.000000000099064 1.0000000000019755
0.015625  1.0000000001963325  1.000000000001922
0.0234375 1.0000000002930804  1.000000000001333
0.03125 1.0000000003887002  0.9999999999996733
0.0390625 1.0000000004816654  0.99999999999578
0.046875  1.0000000005701766  0.9999999999868765
0.0546875 1.000000000648538 0.999999999967912
0.0625  1.0000000007038823  0.9999999999284281
0.0703125 1.0000000007228276  0.9999999998480593
0.078125  1.0000000006597871  0.9999999996866216
0.0859375 1.0000000004196619  0.9999999993672697
0.09375 0.999999999913817 0.999999998743509
0.1015625 0.9999999988220319  0.9999999975472348
0.109375  0.9999999965210244  0.9999999952767697
0.1171875 0.999999992495834 0.9999999910144274
0.125 0.999999984749681 0.999999983117343
0.1328125 0.9999999696463988  0.999999968709761
0.140625  0.999999944574401 0.9999999426909157
0.1484375 0.9999998984012756  0.9999998962233709
0.15625 0.9999998122242137  0.9999998142275746
0.1640625 0.9999996745759964  0.9999996716013094
0.171875  0.9999994297548589  0.9999994261139672
0.1796875 0.9999989902485159  0.9999990082085595
0.1875  0.9999983132221766  0.9999983051436087
0.1953125 0.9999971480620461  0.9999971381500145
0.203125  0.9999951345961271  0.9999952222108605
0.2109375 0.9999921417268969  0.9999921112188368
0.21875 0.9999871541975468  0.9999871186923618
0.2265625 0.9999788578642649  0.9999792086530892
0.234375  0.9999669552750505  0.999966816490646
0.2421875 0.9999477381984327  0.9999476168903155
0.25  0.9999169738485708  0.9999182151963328
0.2578125 0.9998743637294126  0.999873741974957
0.265625  0.99980767794116  0.9998072351437279
0.2734375 0.9997049637635869  0.999708882963078
0.28125 0.9995675863746575  0.9995651136235735
0.2890625 0.9993590717629919  0.9993574572954752
0.296875  0.9990501873704873  0.9990609519240244
0.3046875 0.9986511506048743  0.9986423197103002
0.3125  0.9980634868593959  0.9980580611387568
0.3203125 0.9972267476443237  0.997252212373583
0.328125  0.9961824638596517  0.9961535437869212
0.3359375 0.9946899240673321  0.9946727073988877
0.34375 0.992649182213192 0.9927000143584426
0.3515625 0.9901887592838116  0.9901031269954944
0.359375  0.986776789247096 0.9867249236762389
0.3671875 0.9823030803488498  0.9823823547545908
0.375 0.9770950478054405  0.9768676559442772
0.3828125 0.9700944843845721  0.9699505060528694
0.390625  0.9613088848309946  0.961382322341978
0.3984375 0.951444634256464 0.9509035658116819
0.40625 0.9386164534254466  0.9382542473883523
0.4140625 0.9232415015740091  0.9231863590706296
0.421875  0.9066244654085627  0.9054790865778088
0.4296875 0.8857731477146324  0.8849573328595965
0.4375  0.8619532539265771  0.861507853220744
0.4453125 0.8372374076458438  0.8350955940124856
0.453125  0.807399607396371 0.8057775375556345
0.4609375 0.7749346211857239  0.7737150194202129
0.46875 0.7426950502672157  0.7391728479160571
0.4765625 0.7053389437418488  0.7025158862517759
0.484375  0.6665569214321613  0.6641971477031197
0.4921875 0.6298124494607032  0.6247395986046391
0.5 0.588990040104439 0.5847063225831133
0.5078125 0.5483009802567272  0.5446705525378415
0.515625  0.5115743243354371  0.5051830367462833
0.5234375 0.4724042158791771  0.46673629561749463
0.53125 0.43439374240395795 0.4297382500700015
0.5390625 0.401543969323369 0.39449067328032006
0.546875  0.3677568924000411  0.36117830297357223
0.5546875 0.33500332812090355 0.3298625024533398
0.5625  0.30733818378974814 0.3004980187241476
0.5703125 0.2797074923749053  0.2729524850232183
0.578125  0.25206040657657114 0.2470361293078564
0.5859375 0.22838311156005775 0.2225376170178301
0.59375 0.2054225126794082  0.19926468284697763
0.6015625 0.18148787188838852 0.1770807254158882
0.609375  0.1602790821756127  0.1559350345878618
0.6171875 0.14078025239518704 0.135892162867752
0.625 0.12054225557983804 0.1171359322294348
0.6328125 0.10259822385626297 0.09996485644779055
0.64  0.08810711896012689 0.08590234347027573
0.640078125 0.08814108013224162 0.08575994015812884
0.64015625  0.08820740455562986 0.08561777604236251
0.640234375 0.08827897706626837 0.08547585158440088
0.6403125 0.088328087682513 0.0853341672456689
0.640390625 0.08832993955631298 0.08519272348759069
0.64046875  0.0882657843872815  0.08505152077159096
0.640546875 0.08812528842769907 0.08491055955909446
0.640625  0.08790782690598993 0.08476984031152532
0.640625  0.08790782690598993 0.08476984031152532
0.640703125 0.08762255053414066 0.08462936349030827
0.64078125  0.08728721438033753 0.08448912955686802
0.640859375 0.08692591578047199 0.08434913897262872
0.6409375 0.08656603615500583 0.0842093921990151
0.641015625 0.08623476900186558 0.08406988969745187
0.64109375  0.08595567704531715 0.08393063192936318
0.641171875 0.0857457157484245  0.08379161935617395
0.64125 0.0856130926867289  0.08365285243930831
0.641328125 0.08555624377941436 0.083514331640191
0.64140625  0.08556404548384543 0.08337605742024673
0.641484375 0.08561723470782384 0.08323803024089965
0.6415625 0.08569086192458497 0.08310025056357448
0.641640625 0.08575745751699647 0.08296271884969593
0.64171875  0.08579051322370179 0.08282543556068817
0.641796875 0.08576783788264283 0.0826884011579759
0.641875  0.08567435938716023 0.08255161610298387
0.641953125 0.08550402384489701 0.08241508085713618
0.64203125  0.08526054974172712 0.0822787958818576
0.642109375 0.08495694122551838 0.08214276163857283
0.6421875 0.08461382976013858 0.082006978588706
0.642265625 0.08425685108265485 0.08187144719368204
0.64234375  0.08391339089112217 0.0817361679149251
0.642421875 0.08360911628330307 0.0816011412138599
0.6425  0.08336472618833213 0.08146636755191115
0.642578125 0.08319333903767537 0.08133184739050302
0.64265625  0.08309884364013581 0.0811975811910602
0.642734375 0.08307541683797595 0.08106356941500743
0.6428125 0.08310827858528745 0.08092981252376885
0.642890625 0.0831755758934856  0.08079631097876919
0.64296875  0.0832511592262399  0.08066306524143316
0.643046875 0.08330790171617586 0.08053007577318491
0.643125  0.08332113421617802 0.08039734303544936
0.643203125 0.08327176632738928 0.08026486748965064
0.64328125  0.08314869575005751 0.0801326495972135
0.643359375 0.08295020088022002 0.08000068981956263
0.6434375 0.0826841467591189  0.0798689886181222
0.643515625 0.08236697521888331 0.07973754645431691
0.64359375  0.08202160898574397 0.0796063637895715
0.643671875 0.08167454095406408 0.0794754410853101
0.64375 0.08135247326063887 0.07934477880295746
0.643828125 0.08107893562021959 0.07921437740393826
0.64390625  0.08087131138755854 0.07908423734967668
0.643984375 0.0807386424408842  0.07895435910159743
0.6440625 0.08068049917743224 0.07882474312112521
0.644140625 0.08068704906777732 0.0786953898696842
0.64421875  0.08074031322574558 0.07856629980869931
0.644296875 0.08081645858623532 0.07843747339959466
0.644375  0.08088882347043677 0.078308911103795
0.644453125 0.08093129844440325 0.07818061338272503
0.64453125  0.0809216326115854  0.0780525806978089
0.644609375 0.08084424424813852 0.07792481351047136
0.6446875 0.0806921859364035  0.07779731228213707
0.644765625 0.08046801112432066 0.07767007747423024
0.64484375  0.08018343544178011 0.07754310954817556
0.644921875 0.07985784117000153 0.07741640896539774
0.645 0.07951581066995092 0.07728997618732096
0.645078125 0.07918400482890496 0.0771638116753701
0.64515625  0.07888778444737768 0.07703791589096935
0.645234375 0.07864799858885671 0.07691228929554339
0.6453125 0.07847835202157986 0.07678693235051694
0.645390625 0.0783836806343218  0.07666184551731418
0.64546875  0.07835934949746455 0.07653703223473818
0.645546875 0.0783918574388659  0.07641250644045161
0.645625  0.07846055964083352 0.07628825212136112
0.645703125 0.07854029201556798 0.07616426972076673
0.64578125  0.07860456727594056 0.07604055968196852
0.645859375 0.07862893003818357 0.07591712244826598
0.6459375 0.07859405267040945 0.07579395846295918
0.646015625 0.07848817500888901 0.07567106816934817
0.64609375  0.07830858128410229 0.07554845201073246
0.646171875 0.07806193269485938 0.0754261104304123
0.64625 0.0777634086540806  0.07530404387168718
0.646328125 0.07743477073187746 0.07518225277785719
0.64640625  0.07710159774820448 0.07506073759222234
0.646484375 0.07679003905267649 0.0749394987580822
0.6465625 0.07652350177906327 0.07481853671873678
0.646640625 0.07631969044488397 0.07469785191748614
0.64671875  0.07618837101421562 0.07457744479762982
0.646796875 0.07613014809759341 0.07445731580246785
0.646875  0.07613640207098007 0.0743374653753003
0.646953125 0.07619039298212005 0.07421789395942667
0.64703125  0.07626939642454834 0.0740986019981472
0.647109375 0.07634758980697254 0.07397958993476142
0.6471875 0.07639932858000073 0.07386085821256938
0.647265625 0.0764023949181377  0.07374240727487111
0.64734375  0.07634080543909538 0.07362423756496615
0.647421875 0.07620682880906347 0.07350634952615456
0.6475  0.07600195112478404 0.07338874360173636
0.647578125 0.07573667305519621 0.07327142023501111
0.64765625  0.07542916797738253 0.07315437986927883
0.647734375 0.07510296685228852 0.07303762294783958
0.6478125 0.07478396919375077 0.0729211499139929
0.647890625 0.0744971590122656  0.07280496121103883
0.64796875  0.07426343979545755 0.0726890572822774
0.648046875 0.07409699373305043 0.07257343857100815
0.648125  0.07400349500568426 0.07245810552053131
0.648203125 0.07397940226677097 0.07234305857414641
0.64828125  0.07401242405527123 0.07222829817515349
0.648359375 0.07408308862853392 0.07211382476685259
0.6484375 0.07416722115440885 0.07199963879254326
0.6484375 0.07416722115440885 0.07199963879254326
0.648515625 0.07423901562321328 0.07188574069552553
0.64859375  0.07427430507721927 0.07177213091909944
0.648671875 0.0742536228526816  0.07165880990656455
0.64875 0.07416466052782565 0.07154577810122088
0.648828125 0.07400381548959395 0.07143303594636849
0.64890625  0.073776636746983 0.07132058388530692
0.648984375 0.07349710592468317 0.07120842236133636
0.6490625 0.07318585298314453 0.07109655181775636
0.649140625 0.07286753349800455 0.07098497269786697
0.64921875  0.07256769725175928 0.07087368544496822
0.649296875 0.07230955026261175 0.07076269050235967
0.649375  0.07211101774993015 0.07065198831334134
0.649453125 0.07198247981638828 0.07054157932121327
0.64953125  0.07192546961230217 0.07043146396927503
0.649609375 0.07193249193680298 0.07032164270082664
0.6496875 0.07198798592675404 0.07021211595916813
0.649765625 0.07207031159474599 0.07010288418759908
0.64984375  0.0721545000378863  0.06999394782941949
0.649921875 0.07221542311829683 0.06988530732792941
0.65  0.07223097815798733 0.06977696312642842
0.650078125 0.07218488250664157 0.06966891566821667
0.65015625  0.07206873081468022 0.06956116539659375
0.650234375 0.07188304566104599 0.06945371275485968
0.6503125 0.07163719756554845 0.0693465581863145
0.650390625 0.07134820625411407 0.06923970213425774
0.65046875  0.07103857013368038 0.06913314504198949
0.650546875 0.07073340744909123 0.06902688735280972
0.650625  0.07045726932496857 0.06892092951001803
0.650703125 0.07023102829958552 0.06881527195691445
0.65078125  0.0700692399937986  0.06870991513679899
0.650859375 0.06997830724006503 0.06860485949297122
0.6509375 0.06995568056897085 0.06850010546873134
0.651015625 0.06999019714120225 0.0683956535073789
0.65109375  0.07006350772833699 0.06829150405221396
0.651171875 0.07015241327962812 0.0681876575465365
0.65125 0.07023181398024571 0.0680841144336461
0.651328125 0.0702778918911127  0.06798087515684283
0.65140625  0.07027112946025262 0.06787794015942666
0.651484375 0.0701987753030866  0.0677753098846972
0.6515625 0.07005644903638406 0.06767298477595447
0.651640625 0.06984868678724282 0.06757096527649847
0.65171875  0.06958834967009737 0.06746925182962882
0.651796875 0.06929498165972357 0.06736789543045793
0.651875  0.06899232343674756 0.06726684908832516
0.651953125 0.0687052952450832  0.0671661100314298
0.65203125  0.06845683745381864 0.06706567865297201
0.652109375 0.06826500497740279 0.06696555534615134
0.6521875 0.06814068582370984 0.0668657405041678
0.652265625 0.06808623346033273 0.06676623452022136
0.65234375  0.06809518034754494 0.06666703778751161
0.652421875 0.06815307139932492 0.06656815069923852
0.6525  0.06823931027264164 0.0664695736486021
0.652578125 0.06832977852438309 0.06637130702880191
0.65265625  0.0683998991611126  0.06627335123303794
0.652734375 0.06842775289029021 0.0661757066545102
0.6528125 0.06839685147692595 0.06607837368641822
0.652890625 0.06829822215735327 0.06598135272196216
0.65296875  0.06813153117495734 0.06588464415434159
0.653046875 0.06790511290327368 0.0657882483767565
0.653125  0.06763490265604469 0.06569216578240686
0.653203125 0.06734240221335941 0.06559639676449228
0.65328125  0.06705194671293012 0.06550094171621271
0.653359375 0.06678761509429794 0.06540580103076815
0.6534375 0.06657017702975157 0.06531097510135819
0.653515625 0.0664144660139083  0.06521646432118282
0.65359375  0.06632750624855682 0.065122269083442
0.653671875 0.06630763439276209 0.06502838978133534
0.65375 0.06634472515877095 0.0649348268080628
0.653828125 0.06642148671243726 0.06484158055682437
0.65390625  0.06651566468462637 0.06474865142081965
0.653984375 0.06660287216060744 0.06465603979324873
0.6540625 0.06665968397842269 0.06456374606731122
0.654140625 0.06666660681081031 0.0644717706362071
0.65421875  0.06661054360169429 0.06438011389313633
0.654296875 0.06648644391635915 0.06428877623129853
0.654375  0.06629793484732202 0.06419775804389366
0.654453125 0.06605684508229143 0.0641070597241217
0.65453125  0.06578169352102156 0.06401668166518225
0.654609375 0.0654953322715458  0.06392662426027529
0.6546875 0.06522204052383124 0.06383688790260078
0.654765625 0.06498444510746813 0.06374747298535835
0.65484375  0.06480065239077341 0.06365837990174808
0.654921875 0.06468195934600758 0.06356960904496957
0.655 0.06463143243530757 0.0634811608082228
0.655078125 0.0646435294051861  0.06339303558470774
0.65515625  0.06470481616783372 0.06330523376762401
0.655234375 0.06479568334493462 0.06321775575017155
0.6553125 0.06489284133869494 0.06313060192555037
0.655390625 0.06497228043544613 0.06304377268696006
0.65546875  0.06501231649996626 0.06295726842760059
0.655546875 0.06499633690440566 0.06287108954067194
0.655625  0.06491490152022052 0.06278523641937371
0.655703125 0.0647669271689718  0.06269970945690587
0.65578125  0.06455981273867285 0.0626145090464684
0.655859375 0.06430849033740865 0.0625296355812609
0.6559375 0.06403351763225325 0.06244508945448348
0.656015625 0.06375846358488597 0.06236087105933573
0.65609375  0.06350691435869253 0.062276980789017625
0.656171875 0.06329948118200093 0.06219341903672914
0.65625 0.06315119179722412 0.06211018619566988
0.65625 0.06315119179722412 0.06211018619566988
0.656328125 0.06306959053352984 0.062027282659039826
0.65640625  0.06305379380492306 0.061944708820038934
0.656484375 0.06309461557870902 0.06186246507186683
0.6565625 0.06317574335306572 0.061780551807723474
0.656640625 0.06327581925301778 0.06169896942080884
0.65671875  0.06337115688327749 0.061617718304322534
0.656796875 0.06343874874064366 0.06153679885146465
0.656875  0.06345918480114096 0.06145621145543481
0.656953125 0.06341910817266055 0.06137595650943297
0.65703125  0.06331290060313956 0.061296034406659095
0.657109375 0.06314338552108156 0.06121644554031282
0.6571875 0.06292145521958731 0.06113719030359409
0.657265625 0.06266467824727003 0.06105826908970288
0.65734375  0.06239506196449207 0.06097968229183882
0.657421875 0.06213625247489198 0.060901430303201866
0.6575  0.06191053400465184 0.06082351351699198
0.657578125 0.06173600201700682 0.06074593232640879
0.65765625  0.06162427447254098 0.06066868712465226
0.657734375 0.06157902817894703 0.06059177830492235
0.6578125 0.06159554158986679 0.06051520626041869
0.657890625 0.06166130799486874 0.06043897138434135
0.65796875  0.06175763390702269 0.06036307406988996
0.658046875 0.061862018787504844  0.06028757584560126
0.658125  0.06195101661513764 0.0602124386425244
0.658203125 0.06200321141210541 0.060137640045443044
0.65828125  0.06200193158574122 0.06006318036934351
0.658359375 0.0619373594250424  0.059989059929212114
0.6584375 0.061807765256678686  0.059915279040034866
0.658515625 0.061619716089415175  0.0598418380167981
0.65859375  0.06138723175212602 0.05976873717448813
0.658671875 0.06112999338072957 0.059695976828090955
0.65875 0.06087083938772382 0.05962355729259302
0.658828125 0.06063286253375529 0.05955147888298032
0.65890625  0.06043647882291566 0.05947974191423917
0.658984375 0.060296841900937835  0.05940834670135591
0.6590625 0.06022192485323164 0.059337293559316544
0.659140625 0.06021152069367314 0.05926658280310739
0.65921875  0.06025728050114216 0.05919621474771476
0.659296875 0.06034378259839797 0.05912618970812468
0.659375  0.060450501633719705  0.059056507999323465
0.659453125 0.060554420179853664  0.05898716993629743
0.65953125  0.06063295308233593 0.058918175834032596
0.659609375 0.06066681397779414 0.05884952600751527
0.6596875 0.06064245711329003 0.058781220771731785
0.659765625 0.060553789182434016  0.05871326044166815
0.65984375  0.06040293305640231 0.05864564533231078
0.659921875 0.0601999452874749  0.0585783757586457
0.66  0.05996153002298003 0.05851145203565921
0.660078125 0.05970891017553081 0.05844487447833765
0.66015625  0.059465126387464075  0.05837864340166703
0.660234375 0.05925211148797754 0.058312759120633655
0.6603125 0.05908790640039823 0.05824722195022385
0.660390625 0.058984376598680895  0.058182032205423645
0.66046875  0.05894571523791116 0.05811719020121934
0.660546875 0.05896791858432998 0.058052696252597256
0.660625  0.05903930816605973 0.05798855067454342
0.660703125 0.05914202326510009 0.05792475378204423
0.66078125  0.05925429551667054 0.05786130589008573
0.660859375 0.05935321894816722 0.05779820731365422
0.6609375 0.05941765820851877 0.057735458367736005
0.661015625 0.05943092970103981 0.057673059367317134
0.66109375  0.059382913803351284  0.057611010627383906
0.661171875 0.0592713293332474  0.05754931246292263
0.66125 0.059102013966473806  0.05748796518891935
0.661328125 0.05888817271483653 0.057426969120360355
0.66140625  0.058648690538352435  0.057366324572231966
0.661484375 0.058405728678881115  0.057306031859520226
0.6615625 0.05818190630589151 0.05724609129721143
0.661640625 0.057997428137140816  0.05718650320029189
0.66171875  0.05786752347206813 0.05712726788374764
0.661796875 0.057800516628577804  0.05706838566256508
0.661875  0.05779678132407209 0.05700985685173025
0.661953125 0.057848703673508584  0.056951681766229466
0.66203125  0.057941657111501975  0.05689386072104901
0.662109375 0.05805587122254006 0.056836394031174936
0.6621875 0.05816894764648636 0.05677928201159356
0.662265625 0.05825870744730225 0.05672252497729116
0.66234375  0.05830600755162276 0.056666123243253805
0.662421875 0.05829716625624721 0.05661007712446779
0.6625  0.05822569479862231 0.05655438693591941
0.662578125 0.058093111982682896  0.05649905299259472
0.66265625  0.05790874006941133 0.05644407560948011
0.662734375 0.05768851270885682 0.05638945510156162
0.6628125 0.05745294315345653 0.05633519178382556
0.662890625 0.05722451310320259 0.05628128597125822
0.66296875  0.05702481639453347 0.05622773797884566
0.663046875 0.05687181651329287 0.05617454812157418
0.663125  0.056777570867026915  0.05612171671443007
0.663203125 0.05674670738435166 0.056069244072399396
0.66328125  0.05677584335738952 0.05601713051046846
0.663359375 0.05685402773932955 0.05596537634362353
0.6634375 0.05696414054351219 0.0559139818868507
0.663515625 0.05708507355954645 0.05586294745513625
0.66359375  0.05719441771208172 0.05581227336346647
0.663671875 0.057271309277280844  0.055761959926827434
0.66375 0.05729907883630142 0.05571200746020551
0.663828125 0.057267363008492035  0.05566241627858675
0.66390625  0.05717341200835664 0.055613186696957466
0.663984375 0.05702243158267338 0.05556431903030394
0.6640625 0.05682691184781345 0.05551581359361225
0.6640625 0.05682691184781345 0.05551581359361225
0.664140625 0.05660503171073579 0.05546767070186867
0.66421875  0.05637834449983053 0.0554198906700595
0.664296875 0.05616903565673643 0.055372529215186435
0.664375  0.055997104284671695  0.055325582982748464
0.664453125 0.05587782587173819 0.05527900040357609
0.66453125  0.055819815605573114  0.05523278169269518
0.664609375 0.055823943576816 0.055186927065131834
0.6646875 0.05588323304545358 0.05514143673591191
0.664765625 0.0559837537058872  0.055096310920061446
0.66484375  0.05610640181495475 0.05505154983260648
0.664921875 0.056229331566222895  0.05500715368857288
0.665 0.056330733549222085  0.05496312270298669
0.665078125 0.05639160562438954 0.054919457090873945
0.66515625  0.05639816236235884 0.0548761570672605
0.665234375 0.05634358231404482 0.054833222847172396
0.6653125 0.05622886582997582 0.054790654645635685
0.665390625 0.05606269876210981 0.05474845267767622
0.66546875  0.05586034236070288 0.05470661715832004
0.665546875 0.05564168647003848 0.054665148302593176
0.665625  0.05542871765718874 0.05462404632552151
0.665703125 0.05524272433218179 0.05458331144213113
0.66578125  0.055101591734310985  0.0545429438674479
0.665859375 0.05501753424366715 0.05450294381649786
0.6659375 0.05499554943110425 0.05446331150430704
0.666015625 0.05503278904905412 0.05442404714590132
0.66609375  0.055118931453641974  0.05438515095630673
0.666171875 0.05523750017016062 0.0543466231505493
0.66625 0.055367962923988184  0.05430846394365491
0.666328125 0.05548834633061735 0.054270673550649594
0.66640625  0.05557802777313756 0.05423325218655938
0.666484375 0.05562035582757459 0.05419620006641015
0.6665625 0.055604761060289644  0.05415951740522798
0.666640625 0.05552809199321641 0.05412320441803875
0.66671875  0.05539501028981515 0.0540872613198685
0.666796875 0.05521738917445242 0.054051688325743245
0.666875  0.055012797595185584  0.05401648565068888
0.666953125 0.05480226335952101 0.053981653509731435
0.66703125  0.05460759674439812 0.053947192117896926
0.667109375 0.05444861880433486 0.05391310169021125
0.6671875 0.05434064444736565 0.05387938244170043
0.667265625 0.05429253945343874 0.05384603458739049
0.66734375  0.05430560155237896 0.053813058342307325
0.667421875 0.05437340210327998 0.053780453921476956
0.6675  0.0544826092889188  0.053748221539925416
0.667578125 0.0546146909073177  0.05371636141267858
0.66765625  0.054748273223192305  0.05368487375476253
0.667734375 0.05486186030931973 0.05365375878120316
0.6678125 0.05493656623998307 0.05362301670702649
0.667890625 0.0549585114742147  0.05359264774725854
0.66796875  0.054920584584085726  0.05356265211692521
0.668046875 0.05482333805367139 0.053533030031052524
0.668125  0.05467491108510307 0.05350378170466649
0.668203125 0.05448999042186758 0.05347490735279303
0.66828125  0.05428793651696975 0.05344640719045815
0.668359375 0.05409031939935218 0.05341828143268787
0.6684375 0.05391817560971195 0.0533905302945081
0.668515625 0.05378933407194843 0.053363153990944885
0.66859375  0.053716153782714396  0.053336152737024134
0.668671875 0.053703956865006335  0.053309526747771875
0.66875 0.053750357236090174  0.05328327623821412
0.668828125 0.05384557208455499 0.05325740142337677
0.66890625  0.05397367052878838 0.05323190251828584
0.668984375 0.054114603369124506  0.053206779737967363
0.6690625 0.054246756077184655  0.05318203329744724
0.669140625 0.05434969610277104 0.053157663411751474
0.66921875  0.05440677066474427 0.0531336702959061
0.669296875 0.054407219299299085  0.05311005416493701
0.669375  0.054347536217319053  0.05308681523387024
0.669453125 0.05423191291655616 0.05306395371773179
0.66953125  0.05407169714392436 0.05304146983154757
0.669609375 0.05388394556059359 0.05301936379034364
0.6696875 0.05368925240903529 0.052997635809145904
0.669765625 0.05350912775481609 0.05297628610298038
0.66984375  0.053363263369406486  0.05295531488687308
0.669921875 0.053267030210645404  0.052934722375849924
0.67  0.053229527082951235  0.05291450878493692
0.670078125 0.0532524297655755  0.05289467432916007
0.67015625  0.053329782201164355  0.052875219223545315
0.670234375 0.0534487588777664  0.05285614368311865
0.6703125 0.05359130158812412 0.05283744792290609
0.670390625 0.053736417708587905  0.05281913215793355
0.67046875  0.0538628515928262  0.05280119660322708
0.670546875 0.05395178699017423 0.05278367868391237
0.670625  0.05398923632928322 0.05276662624627358
0.670703125 0.05396781809747505 0.052749954515420354
0.67078125  0.05388768869572703 0.05273366359252092
0.670859375 0.05375651773814722 0.05271775357874357
0.6709375 0.053588510391647706  0.0527022245752566
0.671015625 0.053402595141347875  0.05268707668322822
0.67109375  0.05322001544129964 0.052672310003826744
0.671171875 0.05306162745952843 0.052657924638220444
0.67125 0.05294524804658106 0.05264392068757756
0.671328125 0.05288339238916874 0.052630298253066386
0.67140625  0.05288168466698189 0.052617057435855194
0.671484375 0.052938147127645524  0.05260419833711224
0.6715625 0.05304345722105608 0.05259172105800581
0.671640625 0.053182135973596174  0.05257962569970415
0.67171875  0.05333451994705697 0.05256791236337553
0.671796875 0.053479264402815946  0.05255658115018827
0.671875  0.053596057140996245  0.05254563216131058
0.671875  0.053596057140996245  0.05254563216131058
0.671953125 0.05366820182865582 0.05253506549791075
0.67203125  0.053684737803655284  0.05252488126115708
0.672109375 0.053641830029818593  0.05251507955221779
0.6721875 0.05354325556404356 0.052505660472261184
0.672265625 0.05339991679810188 0.05249662412245553
0.67234375  0.05322845253376252 0.05248797060396908
0.672421875 0.05304912067663186 0.05247970001797013
0.6725  0.052883219193602816  0.05247181246562692
0.672578125 0.05275037877117718 0.052464308048107734
0.67265625  0.05266606636258341 0.05245718686658086
0.672734375 0.052639620554941494  0.05245044902221453
0.6728125 0.05267306791713632 0.052444094616177044
0.672890625 0.05276086703115364 0.052438123749636675
0.67296875  0.052890617133887666  0.05243253652376167
0.673046875 0.053044638824040174  0.05242733303972031
0.673125  0.05320222341153903 0.05242251339868087
0.673203125 0.0533422682825256  0.052418077701811616
0.67328125  0.053445960153726726  0.05241402605028082
0.673359375 0.05349916578249165 0.05241035854525675
0.6734375 0.05349422894129572 0.05240707528790767
0.673515625 0.05343093956507867 0.05240417637940186
0.67359375  0.05331655780507775 0.05240166192090759
0.673671875 0.05316489017182321 0.05239953201359312
0.67375 0.05299452920052582 0.05239778675862673
0.673828125 0.05282648890715073 0.052396426257176686
0.67390625  0.05268153150992295 0.05239545061041126
0.673984375 0.05257752688042467 0.05239485991949872
0.6740625 0.0525271821795235  0.05239465428560734
0.674140625 0.05253642573149452 0.05239483380990539
0.67421875  0.052603656197067424  0.052395398593561134
0.674296875 0.05271994925324703 0.05239634873774284
0.674375  0.05287019325247082 0.052397684343618786
0.674453125 0.053035013729877296  0.05239940551235724
0.67453125  0.05319323858007412 0.05240151234512648
0.674609375 0.05332459031524763 0.05240400494309476
0.6746875 0.053412265447394365  0.052406883407430355
0.674765625 0.05344506926754515 0.05241014783930154
0.67484375  0.053418837983677595  0.05241379833987658
0.674921875 0.0533369682010393  0.05241783501032376
0.675 0.05320998111603933 0.05242225795181134
0.675078125 0.05305418387401389 0.05242706726550757
0.67515625  0.05288959561572  0.05243226305258075
0.675234375 0.05273739943975437 0.052437845414199144
0.6753125 0.052617249810284185  0.052443814451531
0.675390625 0.052544771811885806  0.05245017026574462
0.67546875  0.0525295755579817  0.052456912958008244
0.675546875 0.05257403584398542 0.05246404262949017
0.675625  0.052672989302112075  0.052471559381358654
0.675703125 0.052814393727253335  0.05247946331478196
0.67578125  0.052980860801578177  0.052487754530928375
0.675859375 0.05315186721944054 0.05249643313096616
0.6759375 0.05330636609242617 0.05250549921606357
0.676015625 0.05342546295147372 0.0525149528873889
0.67609375  0.053494818199528164  0.052524794246110416
0.676171875 0.05350647115671658 0.052535023393396364
0.67625 0.05345985011150632 0.05254564043041505
0.676328125 0.05336184495239859 0.0525566454583347
0.67640625  0.053225931637344716  0.052568038578323636
0.676484375 0.05307045619215959 0.052579819891550104
0.6765625 0.05291630267321628 0.05259198949918235
0.676640625 0.052784236798591976  0.05260454750238868
0.67671875  0.05269226491445671 0.05261749400233736
0.676796875 0.05265334538004773 0.05263084098833951
0.676875  0.05267373838524919 0.05264469304721515
0.676953125 0.05275221174858598 0.05265893377561693
0.67703125  0.05288019798056963 0.05267356315507839
0.677109375 0.05304288223739702 0.05268858116713311
0.6771875 0.053221087966358084  0.052703987793314644
0.677265625 0.05339371521005323 0.05271978301515651
0.67734375  0.053540423563488745  0.0527359668141923
0.677421875 0.05364421959211331 0.05275253917195552
0.6775  0.053693616749063026  0.05276950006997977
0.677578125 0.053684096818793846  0.05278684948979858
0.67765625  0.05361868561254975 0.05280458741294547
0.677734375 0.053507566961917004  0.052822713820954055
0.6778125 0.053366789267172844  0.052841228695357845
0.677890625 0.05321622559399106 0.052860132017690374
0.67796875  0.053077045773452736  0.05287942376948525
0.678046875 0.05296902579864751 0.05289910393227599
0.678125  0.05290803109262367 0.05291917248759612
0.678203125 0.052903999606372216  0.05293962941697924
0.67828125  0.05295967794625774 0.05296047470195886
0.678359375 0.053070268680805874  0.05298170832406857
0.6784375 0.053224041746264404  0.053003330264841914
0.678515625 0.05340382480590419 0.053025340505812396
0.67859375  0.05358918528019979 0.05304773902851364
0.678671875 0.053759029363404334  0.05307052581447916
0.67875 0.053894283327145034  0.053093700845242474
0.678828125 0.05398031973530614 0.0531172641023372
0.67890625  0.054008818594251905  0.053141215567296854
0.678984375 0.05397882508944606 0.05316555522165496
0.6790625 0.053896873504469624  0.05319028304694514
0.679140625 0.053776158288800315  0.0532153990247009
0.67921875  0.053634856474161025  0.053240903136455756
0.679296875 0.05349381870333758 0.05326679536374334
0.679375  0.05337391799373811 0.05329307568809713
0.679453125 0.05329339553819971 0.05331974409105074
0.67953125  0.05326554159270441 0.0533468005541377
0.679609375 0.053297001675734264  0.05337424505889151
0.6796875 0.053386931992530716  0.0534020775868458
0.6796875 0.053386931992530716  0.0534020775868458
0.679765625 0.05352710462636057 0.05343029811953409
0.67984375  0.05370294977037782 0.053458906638489886
0.679921875 0.05389540877338757 0.05348790312524682
0.68  0.05408335530519383 0.05351728756133841
0.680078125 0.05424628103043752 0.053547059928298155
0.68015625  0.05436690394102859 0.05357722020765969
0.680234375 0.05443336551431159 0.053607768380956496
0.6803125 0.054440741335039414  0.05363870442972219
0.680390625 0.05439166942864437 0.0536700283354903
0.68046875  0.05429601623587821 0.053701740079794315
0.680546875 0.05416962601010019 0.05373383964416789
0.680625  0.05403230834126502 0.05376632701014452
0.680703125 0.05390532040622395 0.05379920215925772
0.68078125  0.05380866582820875 0.05383246507304113
0.680859375 0.05375854942955377 0.053866115733028255
0.6809375 0.053765316417695275  0.053900154120752594
0.681015625 0.05383213523383991 0.0539345802177478
0.68109375  0.053954589132603814  0.05396939400554738
0.681171875 0.054121237365424575  0.054004595465684826
0.68125 0.054315066060604766  0.05404018457969379
0.681328125 0.054515648060584045  0.05407616132910773
0.68140625  0.054701739952576565  0.054112525695460295
0.681484375 0.05485398121361293 0.05414927766028498
0.6815625 0.05495735735711809 0.05418641720511529
0.681640625 0.0550031104005989  0.05422394431148489
0.68171875  0.05498985399951346 0.054261858960927264
0.681796875 0.05492375469217327 0.05430016113497591
0.681875  0.05481775114982532 0.0543388508151645
0.681953125 0.05468991198103426 0.05437792798302652
0.68203125  0.0545611425095169  0.05441739262009547
0.682109375 0.054452527924774355  0.054457244707905014
0.6821875 0.05438265319691373 0.054497484227988585
0.682265625 0.05436523969651705 0.05453811116187985
0.68234375  0.054407396387794224  0.05457912549111231
0.682421875 0.05450871443132233 0.05462052719721945
0.6825  0.05466131494589671 0.05466231626173494
0.682578125 0.05485084445135392 0.054704492666192275
0.68265625  0.05505829841923942 0.054747056392124946
0.682734375 0.055262433165399936  0.05479000742106663
0.6828125 0.05544246527426892 0.0548333457345508
0.682890625 0.0555807139748245  0.054877071314110966
0.68296875  0.055664849051136445  0.0549211841412808
0.683046875 0.05568946291991004 0.054965684197593775
0.683125  0.05565676112914973 0.05501068301457238
0.683203125 0.05557628591284415 0.05505608405184906
0.68328125  0.05546370941210232 0.05510187204068137
0.683359375 0.055338844797502874  0.05514804684445322
0.6834375 0.05522313056958169 0.055194608326548315
0.683515625 0.05513690743546578 0.05524155635035036
0.68359375  0.055096831147170754  0.05528889077924324
0.683671875 0.05511375418525358 0.05533661147661066
0.68375 0.05519134251612104 0.05538471830583633
0.683828125 0.05532560302344418 0.05543321113030414
0.68390625  0.05550538813113239 0.055482089813397785
0.683984375 0.05571380699900308 0.05553135421850098
0.6840625 0.055930367423216464  0.055581004208997614
0.684140625 0.05613357928337414 0.05563103964827132
0.68421875  0.05630368343710382 0.055681460399706006
0.684296875 0.05642516545945917 0.05573226632668536
0.684375  0.05648872931864892 0.0557834572925931
0.684453125 0.056492482244531544  0.055835033160813105
0.68453125  0.05644218260200674 0.0558869937947291
0.684609375 0.05635051239962431 0.055939339057724746
0.6846875 0.05623547074503349 0.055992068813183984
0.684765625 0.056118091745411894  0.0560451829244905
0.68484375  0.05601977303875805 0.056098681255027975
0.684921875 0.05595955760857611 0.05615256366818033
0.685 0.05595171214026582 0.056206830027331275
0.685078125 0.056003909176236814  0.05626148019586447
0.68515625  0.056116248608505837  0.05631651403716386
0.685234375 0.05628123841356586 0.05637193141461304
0.6853125 0.056484739363887065  0.05642773219159594
0.685390625 0.05670775834534491 0.05648391623149625
0.68546875  0.05692885650915942 0.056540483397697655
0.685546875 0.057126871762436165  0.05659743355358408
0.685625  0.05728360787847496 0.056654766562539205
0.685703125 0.0573861474456664  0.05671248228794673
0.68578125  0.057428499611186005  0.05677058059319058
0.685859375 0.05741236512226799 0.056829061341654434
0.6859375 0.057346926464397985  0.05688792439672198
0.686015625 0.05724768943463807 0.05694716962177715
0.68609375  0.05713451728064613 0.05700679688020354
0.686171875 0.057029111506164416  0.05706680603538509
0.68625 0.056952256861220005  0.05712719695070548
0.686328125 0.05692117918750957 0.05718796948954839
0.68640625  0.056947354994216035  0.05724912351529776
0.686484375 0.057035047732598314  0.057310658891337275
0.6865625 0.057180758737708594  0.057372575481050604
0.686640625 0.05737366648338269 0.0574348731478217
0.68671875  0.0575969936520282  0.05749755175503423
0.686796875 0.05783013258476451 0.057560611166071876
0.686875  0.058051260695413914  0.05762405124431859
0.686953125 0.05824010985643852 0.05768787185315804
0.68703125  0.05838054399483016 0.05775207285597391
0.687109375 0.05846261110239163 0.05781665411615014
0.6871875 0.05848381228365139 0.05788161549707032
0.687265625 0.05844942863123504 0.0579469568621184
0.68734375  0.058371855606632606  0.05801267807467805
0.687421875 0.05826903599349113 0.05807877899813294
0.6875  0.058162187490099465  0.05814525949586704
0.6875  0.058162187490099465  0.05814525949586704
0.687578125 0.058073110004902714  0.05821211943126401
0.68765625  0.058021418471719 0.058279358667707525
0.687734375 0.05802204900866198 0.05834697706858154
0.6878125 0.058083356790927104  0.05841497449726972
0.687890625 0.05820604974439511 0.05848335081715575
0.68796875  0.058383089916971054  0.05855210589162359
0.688046875 0.05860057908701757 0.05862123958405679
0.688125  0.058839518347862184  0.05869075175783933
0.688203125 0.05907821447713426 0.058760642276354885
0.68828125  0.059295032585520456  0.0588309110029871
0.688359375 0.05947114343357251 0.05890155780111996
0.6884375 0.05959291591533275 0.05897258253413713
0.688515625 0.059653655773104684  0.059043985065422265
0.68859375  0.059654460266503916  0.05911576525835934
0.688671875 0.05960408701851977 0.059187922976332026
0.68875 0.05951785214930684 0.059260458082723975
0.688828125 0.059415690444517257  0.059333370440919166
0.68890625  0.059319630226157403  0.05940665991430127
0.688984375 0.059250998783417726  0.05948032636625393
0.6890625 0.05922771320145153 0.05955436966016114
0.689140625 0.05926200300502098 0.05962878965940644
0.68921875  0.05935885006862658 0.059703586227373824
0.689296875 0.059515348146524205  0.059778759227446944
0.689375  0.05972106456460329 0.05985438813485205
0.689453125 0.05995935524615899 0.059930432614869376
0.68953125  0.060209471143817694  0.060006852931824534
0.689609375 0.06044918853469796 0.06008364883802902
0.6896875 0.06065762748128572 0.060160820085794674
0.689765625 0.06081790525333238 0.06023836642743301
0.68984375  0.060919282552774665  0.060316287615255516
0.689921875 0.0609585327495841  0.06039458340157404
0.69  0.06094036261344605 0.06047325353869997
0.690078125 0.06087682087960525 0.060552297778945156
0.69015625  0.06078577815848654 0.060631715874621094
0.690234375 0.060688666415328545  0.06071150757803928
0.6903125 0.06060776139754951 0.060791672641511564
0.690390625 0.06056335758225517 0.06087221081734945
0.69046875  0.06057118901485225 0.06095312185786444
0.690546875 0.060640427125383735  0.06103440551536837
0.690625  0.06077251016033745 0.061116061542172744
0.690703125 0.060960950025547654  0.06119808969058907
0.69078125  0.061192146898048375  0.06128048971292919
0.690859375 0.0614471077984146  0.0613632613615046
0.6909375 0.06170384951575162 0.0614464043886268
0.691015625 0.0619401856466906  0.061529918546607644
0.69109375  0.06213654152850876 0.06161380358775851
0.691171875 0.06227844049628004 0.06169805926439125
0.69125 0.062358348780390094  0.061782685328817366
0.691328125 0.062376637557048975  0.061867681533348345
0.69140625  0.06234154529455782 0.06195304763029605
0.691484375 0.06226814413735013 0.06203878337197197
0.6915625 0.062176433135171176  0.062124888510687606
0.691640625 0.06208880819951931 0.06221136279875481
0.69171875  0.06202722319334778 0.06229820598848508
0.691796875 0.06201040351212839 0.06238541783218991
0.691875  0.06205146750567381 0.062472998082181166
0.691953125 0.06215625332405646 0.06256094649077021
0.69203125  0.06232257003506483 0.06264926281026889
0.692109375 0.0625404664960253  0.06273794679298873
0.6921875 0.06279348272432612 0.0628269981912412
0.692265625 0.06306073090112459 0.06291641675733818
0.69234375  0.06331953998039586 0.06300620224359114
0.692421875 0.06354832871539585 0.06309635440231158
0.6925  0.06372934579605521 0.06318687298581137
0.692578125 0.06385092509057916 0.063277757746402
0.69265625  0.0639089728258254  0.06336900843639497
0.692734375 0.06390749824465679 0.06346062480810212
0.6928125 0.06385811278893531 0.06355260661383495
0.692890625 0.06377856776235412 0.06364495360590497
0.69296875  0.06369051176211904 0.06373766553662402
0.693046875 0.06361674820356536 0.06383074215830348
0.693125  0.06357834618852502 0.06392418322325522
0.693203125 0.06359196448676761 0.06401798848379071
0.69328125  0.06366773373969098 0.06411215769222146
0.693359375 0.06380796404453672 0.06420669060085935
0.6934375 0.06400683974241762 0.06430158696201582
0.693515625 0.0642511478392468  0.0643968465280024
0.69359375  0.06452194445681705 0.06449246905113096
0.693671875 0.06479694875025269 0.06458845428371297
0.69375 0.06505336511440946 0.06468480197805992
0.693828125 0.06527077261933092 0.06478151188648369
0.69390625  0.06543371733774662 0.06487858376129563
0.693984375 0.06553367956592168 0.06497601735480764
0.6940625 0.06557016121002278 0.06507381241933118
0.694140625 0.06555075987109456 0.06517196870717772
0.69421875  0.06549021759915101 0.06527048597065917
0.694296875 0.06540855899197369 0.06536936396208701
0.694375  0.06532856071495473 0.0654686024337727
0.694453125 0.06527286699312149 0.06556820113802815
0.69453125  0.06526111821467019 0.06566815982716484
0.694609375 0.06530745815669103 0.06576847825349422
0.6946875 0.06541873099814634 0.06586915616932822
0.694765625 0.06559360556264372 0.06597019332697827
0.69484375  0.06582273347458965 0.0660715894787559
0.694921875 0.06608992173736856 0.06617334437697296
0.695 0.06637417794685677 0.06627545777394081
0.695078125 0.06665236473640455 0.06637792942197134
0.69515625  0.0669021294746023  0.06648075907337601
0.695234375 0.06710473960504786 0.06658394648046634
0.6953125 0.0672474606458014  0.06668749139555417
0.6953125 0.0672474606458014  0.06668749139555417
0.695390625 0.06732517873264274 0.06679139357095103
0.69546875  0.06734106003392393 0.06689565275896835
0.695546875 0.0673061585670147  0.06700026871191804
0.695625  0.06723802689015049 0.06710529032193435
0.695703125 0.06715850088900645 0.06721072669447381
0.69578125  0.06709093683353566 0.06731651894263423
0.695859375 0.0670572549483194  0.06742266671830739
0.6959375 0.0670751574670367  0.0675291696733857
0.696015625 0.06715588058531095 0.06763602745976108
0.69609375  0.06730276226907587 0.0677432397293255
0.696171875 0.06751080568411967 0.06785080613397133
0.69625 0.06776730329823567 0.06795872632559052
0.696328125 0.0680534363810208  0.06806699995607501
0.69640625  0.06834665027618637 0.06817562667731722
0.696484375 0.06862350866799677 0.06828460614120904
0.6965625 0.06886266114664567 0.06839393799964245
0.696640625 0.06904755169412995 0.06850362190450984
0.69671875  0.06916852351353425 0.06861365750770314
0.696796875 0.06922405035762888 0.0687240444611143
0.696875  0.06922094203418032 0.0688347824166357
0.696953125 0.06917349370169258 0.06894587102615915
0.69703125  0.06910168347762165 0.06905730994157702
0.697109375 0.06902865062043749 0.0691690988147813
0.6971875 0.06897776791243912 0.06928123729766386
0.697265625 0.06896968077932047 0.06939372504211716
0.69734375  0.06901968934606698 0.06950656170003311
0.697421875 0.0691357997099443  0.06961974692330365
0.6975  0.0693177019738064  0.06973328036382119
0.697578125 0.0695567977712118  0.06984716167347765
0.69765625  0.0698372756968936  0.06996139050416499
0.697734375 0.07013810645660974 0.07007596650777559
0.6978125 0.07043569852434933 0.07019088933620124
0.697890625 0.07070688274162465 0.07030615864133435
0.69796875  0.07093184826635462 0.07042177407506688
0.698046875 0.07109665510745666 0.0705377352892907
0.698125  0.07119500888268747 0.07065404193589828
0.698203125 0.071229068737573 0.07077069366678153
0.69828125  0.07120918411456224 0.07088769013383238
0.698359375 0.07115259681781769 0.07100503098894324
0.6984375 0.0710812670972228  0.07112271588400605
0.698515625 0.07101909835906259 0.07124074447091273
0.69859375  0.07098891404865725 0.07135911640155573
0.698671875 0.07100956422452519 0.07147783132782694
0.69875 0.07109353389085368 0.0715968889016183
0.698828125 0.07124535372587422 0.07171628877482224
0.69890625  0.0714610118480995  0.07183603059933051
0.698984375 0.07172845292356707 0.07195611402703554
0.6990625 0.07202909248809232 0.07207653870982927
0.699140625 0.07234015982929153 0.07219730429960361
0.69921875  0.07263757701781381 0.07231841044825098
0.699296875 0.0728990044273108  0.07243985680766332
0.699375  0.07310667253324118 0.07256164302973254
0.699453125 0.07324963797596666 0.07268376876635109
0.69953125  0.07332517719357923 0.07280623366941087
0.699609375 0.07333914427921141 0.07292903739080382
0.6996875 0.07330524157191155 0.07305217958242237
0.699765625 0.07324329499391606 0.07317565989615825
0.69984375  0.07317675419929794 0.07329947798390393
0.699921875 0.07312972846833483 0.07342363349755131
0.7 0.07312393577065796 0.0735481260889923
0.700078125 0.07317595093138535 0.07367295541011937
0.70015625  0.07329509787124475 0.07379812111282441
0.700234375 0.07348226109144958 0.07392362284899935
0.7003125 0.07372976172729948 0.07404946027053663
0.700390625 0.07402231471528886 0.07417563302932818
0.70046875  0.07433895580643193 0.0743021407772659
0.700546875 0.07465568562250574 0.07442898316624222
0.700625  0.07494850313509337 0.07455615984814909
0.700703125 0.07519644395106898 0.07468367047487841
0.70078125  0.07538423643606544 0.07481151469832262
0.700859375 0.07550424414369043 0.07493969217037347
0.7009375 0.0755574421175822  0.07506820254292339
0.701015625 0.07555330452403944 0.07519704546786432
0.70109375  0.0755086193367531  0.07532622059708816
0.701171875 0.07544537348558283 0.07545572758248735
0.70125 0.07538797757129402 0.07558556607595383
0.701328125 0.07536018126496528 0.07571573572937948
0.70140625  0.07538206576455682 0.07584623619465677
0.701484375 0.07546749858363776 0.07597706712367763
0.7015625 0.07562237010449682 0.07610822816833393
0.701640625 0.07584383449994679 0.07623971898051816
0.70171875  0.0761206600454704  0.07637153921212203
0.701796875 0.07643463764408699 0.07650368851503801
0.701875  0.0767628735193669  0.0766361896098022
0.701953125 0.07708068133143506 0.07676909043142963
0.70203125  0.07736470126268614 0.07690231917045483
0.702109375 0.07759585869192455 0.07703587539069193
0.7021875 0.07776178266760123 0.07716975865595505
0.702265625 0.07785837912555758 0.07730396853005887
0.70234375  0.07789036250216577 0.07743850457681752
0.702421875 0.07787067034392721 0.0775733663600451
0.7025  0.07781883779225476 0.07770855344355633
0.702578125 0.07775853707303573 0.07784406539116531
0.70265625  0.07771458791223783 0.07797990176668616
0.702734375 0.07770982007334755 0.07811606213393357
0.7028125 0.07776218301572005 0.07825254605672148
0.702890625 0.07788246694670724 0.07838935309886458
0.70296875  0.0780729296483746  0.07852648282417697
0.703046875 0.07832699832680179 0.07866393479647278
0.703125  0.07863008666047096 0.07880170857956673
0.703125  0.07863008666047096 0.07880170857956673
0.703203125 0.07896143096480265 0.07893980373727288
0.70328125  0.07929670631195232 0.07907821983340539
0.703359375 0.07961109827296795 0.07921695643177895
0.7034375 0.07988244109077725 0.07935601309620767
0.703515625 0.08009402302352246 0.07949538939050566
0.70359375  0.08023670968139754 0.07963508487848762
0.703671875 0.08031010788863413 0.07977509912396749
0.70375 0.08032262692411908 0.07991543169075994
0.703828125 0.08029042939729862 0.0800560821426791
0.70390625  0.08023539673284359 0.0801970500435391
0.703984375 0.08018237050010764 0.08033833495715462
0.7040625 0.08015601604026974 0.08047993644733976
0.704140625 0.08017770230581521 0.08062185407790867
0.70421875  0.08026279637937435 0.08076408741267603
0.704296875 0.08041871152041948 0.08090663601545596
0.704375  0.08064395721385652 0.08104949945006257
0.704453125 0.08092831732959949 0.08119267728031056
0.70453125  0.08125412860740436 0.08133616907001405
0.704609375 0.08159850270900748 0.08147997438298714
0.7046875 0.08193621452148304 0.08162409278304456
0.704765625 0.08224288633039151 0.08176852383400018
0.70484375  0.08249807269303124 0.08191326709966876
0.704921875 0.08268784931100992 0.08205832214386438
0.705 0.08280658118318061 0.08220368853040115
0.705078125 0.08285764900765444 0.0823493658230938
0.70515625  0.08285303178104467 0.08249535358575641
0.705234375 0.0828118045150265  0.0826416513822031
0.7053125 0.08275773824631069 0.08278825877624858
0.705390625 0.0827163004101374  0.08293517533170698
0.70546875  0.08271143893215636 0.08308240061239239
0.705546875 0.08276255285807055 0.08322993418211952
0.705625  0.08288203314727495 0.08337777560470228
0.705703125 0.08307368838333629 0.08352592444395537
0.70578125  0.08333225047709808 0.08367438026369294
0.705859375 0.08364402709151525 0.08382314262772905
0.7059375 0.08398862263302168 0.08397221109987844
0.706015625 0.0843415054685071  0.08412158524395522
0.70609375  0.08467710307643303 0.08427126462377346
0.706171875 0.08497203293729273 0.08442124880314793
0.70625 0.0852080587807026  0.0845715373458927
0.706328125 0.0853744050365268  0.08472212981582188
0.70640625  0.08546912587300419 0.08487302577675021
0.706484375 0.08549936278898165 0.08502422479249178
0.7065625 0.0854804568745924  0.08517572642686067
0.706640625 0.08543401893588914 0.08532753024367164
0.70671875  0.08538520810886217 0.08547963580673856
0.706796875 0.08535955803950986 0.08563204267987617
0.706875  0.08537975014358505 0.08578475042689858
0.706953125 0.08546274510607277 0.08593775861161987
0.70703125  0.08561763097391971 0.08609106679785478
0.707109375 0.0858444637000029  0.08624467454941741
0.7071875 0.08613424952738269 0.08639858143012184
0.707265625 0.08647006772982387 0.08655278700378283
0.70734375  0.08682919765177728 0.08670729083421447
0.707421875 0.08718598276293898 0.08686209248523084
0.7075  0.08751506672734768 0.0870171915206467
0.707578125 0.08779459866783815 0.08717258750427591
0.70765625  0.08800899718584593 0.08732827999993323
0.707734375 0.0881509257374443  0.08748426857143274
0.7078125 0.08822223366761561 0.08764055278258853
0.707890625 0.0882337312055898  0.08779713219721537
0.70796875  0.08820383649899396 0.08795400637912731
0.708046875 0.0881562607058447  0.08811117489213847
0.708125  0.08811701806642591 0.08826863989961423
0.708203125 0.08811114435644697 0.08842647662663969
0.70828125  0.08815953253377802 0.08858460629836398
0.708359375 0.08827628798119219 0.08874302840318275
0.7084375 0.08846693925744928 0.08890174242949096
0.708515625 0.08872772677741281 0.08906074786568362
0.70859375  0.08904606548277125 0.08922004420015635
0.708671875 0.08940212413345723 0.08937963092130392
0.70875 0.08977131910178758 0.08953950751752199
0.708828125 0.09012741365881562 0.0896996734772055
0.70890625  0.09044582949457051 0.08986012828874948
0.708984375 0.09070675167895147 0.09002087144054956
0.7090625 0.09089763939346213 0.09018190242100071
0.709140625 0.09101481664878777 0.09034322071849792
0.70921875  0.09106394743800333 0.09050482582143682
0.709296875 0.0910593355239927  0.09066671721821243
0.709375  0.0910221259316479  0.09082889439721968
0.709453125 0.09097764521216861 0.09099135684685425
0.70953125  0.09095220905891357 0.09115410405551089
0.709609375 0.0909697998182658  0.09131713551158525
0.7096875 0.0910490367288213  0.09148045070347233
0.709765625 0.09120081624775145 0.09164404911956706
0.70984375  0.09142692671936781 0.09180793024826514
0.709921875 0.0917198120546346  0.09197209357796152
0.71  0.0920635119909429  0.09213653859705118
0.710078125 0.09243566714173629 0.09230126479392978
0.71015625  0.09281033457718402 0.0924662716569923
0.710234375 0.09316125755001371 0.0926315586746337
0.7103125 0.0934651812764063  0.09279712533524966
0.710390625 0.093704791449256 0.09296297112723514
0.71046875  0.09387090682095103 0.09312909553898513
0.710546875 0.09396365086456386 0.09329549805889527
0.710625  0.09399244476170508 0.0934621781753603
0.710703125 0.09397483025051745 0.09362913537677592
0.71078125  0.0939342673342321  0.09379636915153708
0.710859375 0.09389717780972667 0.09396387898803873
0.7109375 0.09388961562995757 0.09413166437467657
0.7109375 0.09388961562995757 0.09413166437467657
0.711015625 0.09393397651038396 0.09429972479984555
0.71109375  0.09404616670348545 0.09446805975194066
0.711171875 0.09423358835659844 0.09463666871935754
0.71125 0.09449419225474029 0.0948055511904912
0.711328125 0.09481672585347346 0.09497470665373656
0.71140625  0.09518214290469422 0.09514413459748931
0.711484375 0.09556599629623026 0.0953138345101442
0.7115625 0.09594151804089142 0.0954838058800969
0.711640625 0.09628299462985361 0.09565404819574237
0.71171875  0.09656901293323342 0.09582456094547558
0.711796875 0.0967851693985376  0.09599534361769219
0.711875  0.09692589417486687 0.09616639570078718
0.711953125 0.09699516472390521 0.09633771668315552
0.71203125  0.09700601751876382 0.09650930605319288
0.712109375 0.09697891087857316 0.09668116329929423
0.7121875 0.09693915364869723 0.09685328790985452
0.712265625 0.0969137176231005  0.09702567937326945
0.71234375  0.09692783503044806 0.09719833717793398
0.712421875 0.09700181417967764 0.09737126081224307
0.7125  0.09714846850640008 0.09754444976459238
0.712578125 0.09737149208776541 0.09771790352337667
0.71265625  0.09766498329306705 0.0978916215769916
0.712734375 0.09801417566946286 0.09806560341383216
0.7128125 0.09839729213108761 0.09823984852229327
0.712890625 0.09878828446618651 0.09841435639077065
0.71296875  0.09916011399916765 0.09858912650765925
0.713046875 0.09948816305250238 0.09876415836135401
0.713125  0.09975334253804906 0.09893945144025065
0.713203125 0.09994450674794149 0.09911500523274411
0.71328125  0.1000598702215807  0.09929081922722936
0.713359375 0.10010724127422711 0.09946689291210209
0.7134375 0.10010304886128626 0.099643225775757
0.713515625 0.10007028127965475 0.0998198173065898
0.71359375  0.10003559158693212 0.09999666699299545
0.713671875 0.10002594145591935 0.10017377432336888
0.71375 0.10006519886365349 0.10035113878610583
0.713828125 0.10017112433136764 0.1005287598696012
0.71390625  0.100353124581734 0.10070663706224998
0.713984375 0.10061105327139792 0.10088476985244788
0.7140625 0.10093522066240546 0.10106315772858983
0.714140625 0.10130760503558267 0.10124180017907078
0.71421875  0.10170411471678573 0.10142069669228646
0.714296875 0.10209762159348008 0.1015998467566318
0.714375  0.10246137877977401 0.10177924986050176
0.714453125 0.1027723945450986  0.10195896097146197
0.71453125  0.10301433735170666 0.10213893622340728
0.714609375 0.10317960120390438 0.10231916286522331
0.7146875 0.10327027528386652 0.10249964032220746
0.714765625 0.10329789232529807 0.10268036801965709
0.71484375  0.10328198220709352 0.10286134538287031
0.714921875 0.10324761921622846 0.10304257183714453
0.715 0.10322226605910272 0.1032240468077771
0.715078125 0.10323231226392797 0.10340576972006615
0.71515625  0.10329974685895565 0.10358773999930908
0.715234375 0.10343937751178346 0.10376995707080323
0.7153125 0.10365695747493915 0.10395242035984675
0.715390625 0.1039484503948123  0.10413512929173675
0.71546875  0.10430052570398454 0.10431808329177139
0.715546875 0.1046922320093553  0.10450128178524802
0.715625  0.10509763018905638 0.104684724197464
0.715703125 0.10548905814613208 0.10486840995371749
0.71578125  0.10584061818561594 0.10505233847930585
0.715859375 0.10613144306887974 0.10523650919952646
0.7159375 0.10634833319037869 0.10542092153967746
0.716015625 0.10648742952942249 0.10560557492505619
0.71609375  0.10655470797973843 0.10579046878096005
0.716171875 0.10656523750361278 0.10597560253268717
0.71625 0.10654129030877013 0.10616097560553493
0.716328125 0.10650953977572276 0.10634658742480067
0.71640625  0.10649770386391837 0.10653243741578257
0.716484375 0.10653105040076277 0.10671852500377771
0.7165625 0.10662920950169387 0.10690484961408425
0.716640625 0.10680369415969357 0.10709141067199957
0.71671875  0.1070564365237697  0.10727820760282099
0.716796875 0.10737953756957411 0.1074652398318467
0.716875  0.10775625209842234 0.10765250678437405
0.716953125 0.10816308856942522 0.1078400078857004
0.71703125  0.10857276579544485 0.10802774256112392
0.717109375 0.10895764696119968 0.10821571023594195
0.7171875 0.10929322348976944 0.10840391033545188
0.717265625 0.10956120809556694 0.10859234228495183
0.71734375  0.10975184450398007 0.10878100550973892
0.717421875 0.10986514712547872 0.1089698994351113
0.7175  0.10991090904431307 0.10915902348636634
0.717578125 0.10990747562664704 0.10934837708880138
0.71765625  0.10987944226961407 0.10953795966771461
0.717734375 0.10985456027182054 0.10972777064840336
0.7178125 0.1098602416786628  0.109917809456165
0.717890625 0.10992010536910034 0.1101080755162977
0.71796875  0.11005099399144691 0.1102985682540988
0.718046875 0.1102608463818311  0.11048928709486568
0.718125  0.11054768864072745 0.11068023146389648
0.718203125 0.11089986953386288 0.11087140078648858
0.71828125  0.11129752297247346 0.11106279448793932
0.718359375 0.111715062394299 0.11125441199354688
0.7184375 0.11212439916126876 0.11144625272860834
0.718515625 0.11249848135724135 0.11163831611842186
0.71859375  0.11281470236262084 0.1118306015882848
0.718671875 0.11305775534056138 0.11202310856349451
0.71875 0.11322156870026892 0.11221583646934918
0.71875 0.11322156870026892 0.11221583646934918
0.718828125 0.11331007827531372 0.11240878473114614
0.71890625  0.11333674250649879 0.11260195277418276
0.718984375 0.11332285503714608 0.11279534002375721
0.7190625 0.1132948678585075  0.11298894590516684
0.719140625 0.1132810644452736  0.11318276984370901
0.71921875  0.1133079964569978  0.11337681126468187
0.719296875 0.11339713715955041 0.11357106959338252
0.719375  0.11356217175491096 0.11376554425510914
0.719453125 0.11380726272889674 0.11396023467515906
0.71953125  0.11412651937841042 0.11415514027882964
0.719609375 0.1145047302485513  0.11435026049141907
0.7196875 0.11491926862106991 0.11454559473822469
0.719765625 0.11534293957767565 0.11474114244454385
0.71984375  0.11574740043354687 0.11493690303567473
0.719921875 0.11610673140198402 0.11513287593691468
0.72  0.11640070342972643 0.11532906057356104
0.7265625 0.13316826541759097 0.13250611193655962
0.734375  0.15504207832938785 0.15438661324569714
0.7421875 0.17953574241826162 0.1771911086805357
0.75  0.20254400028214517 0.20022627963080875
0.7578125 0.22529769555010176 0.22279611019627096
0.765625  0.2485016655560014  0.24422889686058655
0.7734375 0.26837985463118536 0.2639096121818559
0.78125 0.2861909980123255  0.28130843692850754
0.7890625 0.3023743179573015  0.2960038030990309
0.796875  0.3144004821301852  0.307697103834578
0.8046875 0.32351410395942637 0.3162231121607242
0.8125  0.3297924975324139  0.3215437367830273
0.8203125 0.33235566981005277 0.32373458002840905
0.828125  0.3322591220306885  0.32296672578613306
0.8359375 0.3291285584781786  0.3194864445208208
0.84375 0.3235318226284676  0.3135897210449937
0.8515625 0.3162150458074871  0.30559874085831445
0.859375  0.30625879267344747 0.29584170670971255
0.8671875 0.2951624850090331  0.2846348373884085
0.875 0.28342674237170074 0.2722685013583014
0.8828125 0.2695499379785337  0.2589992228384363
0.890625  0.25542682436951486 0.2450464009178136
0.8984375 0.24155405204662345 0.23059324384770563
0.90625 0.22591073555059013 0.21579385393865907
0.9140625 0.21039406022549187 0.2007851573698807
0.921875  0.19584243492611644 0.18570116398907144
0.9296875 0.17991874094931584 0.1706901941843369
0.9375  0.16428926936034544 0.15593340400708972
0.9453125 0.15045256226669942 0.1416599207334627
0.953125  0.13611752009709474 0.12815838280667266
0.9609375 0.12252124509002649 0.11578337077687606
0.96875 0.11190238324301337 0.10495309676430382
0.9765625 0.10247452087422433 0.0961348919890935
0.984375  0.09469191696435886 0.08982411304367804
0.9921875 0.0911979705204661  0.08651321336906115
\end{filecontents}

\begin{filecontents}{no_inter.dat}
0.                    1.000000000001851
0.006266570697492783  1.0000000000019666
0.012533141394985566  1.0000000000019487
0.018799712092478348  1.0000000000017852
0.025066282789971132  1.0000000000010234
0.03133285348746392   0.9999999999994704
0.037599424184956695  0.9999999999963395
0.04386599488244948   0.9999999999901179
0.050132565579942265  0.9999999999784472
0.05639913627743505   0.9999999999572073
0.06266570697492783   0.9999999999189899
0.06893227767242062   0.9999999998509544
0.07519884836991339   0.9999999997314043
0.08146541906740618   0.9999999995238524
0.08773198976489896   0.9999999991668261
0.09399856046239174   0.9999999985575857
0.10026513115988453   0.9999999975271914
0.10653170185737731   0.99999999580022
0.1127982725548701    0.9999999929279755
0.11906484325236288   0.9999999881876291
0.12533141394985567   0.9999999804315655
0.13159798464734845   0.9999999678347511
0.13786455534484124   0.999999947536424
0.14413112604233402   0.9999999150837613
0.15039769673982678   0.9999998635949076
0.15666426743731957   0.9999997825322439
0.16293083813481235   0.9999996558820822
0.16919740883230514   0.9999994595216283
0.17546397952979792   0.999999157396639
0.1817305502272907    0.9999986960709965
0.1879971209247835    0.9999979970040563
0.19426369162227627   0.9999969457055331
0.20053026231976906   0.9999953766624672
0.20679683301726184   0.9999930525935473
0.21306340371475463   0.9999896361665115
0.2193299744122474    0.9999846518169385
0.2255965451097402    0.9999774347265344
0.23186311580723298   0.9999670633385539
0.23812968650472577   0.9999522710103254
0.24439625720221855   0.9999313315752785
0.25066282789971134   0.9999019127017936
0.2569293985972041    0.9998608900746458
0.2631959692946969    0.999804114607643
0.26946253999218966   0.9997261241616495
0.2757291106896825    0.999619791034765
0.28199568138717523   0.9994758962510575
0.28826225208466805   0.9992826223735384
0.2945288227821608    0.9990249578849218
0.30079539347965356   0.9986840083105065
0.3070619641771464    0.9982362127875217
0.31332853487463913   0.9976524694266141
0.31959510557213194   0.9968971792399138
0.3258616762696247    0.9959272263027891
0.3321282469671175    0.9946909215453547
0.33839481766461027   0.9931269489660489
0.3446613883621031    0.9911633658561135
0.35092795905959584   0.9887167224091168
0.35719452975708865   0.9856913801823349
0.3634611004545814    0.9819791222650555
0.3697276711520742    0.9774591592992742
0.375994241849567     0.9719986430141854
0.3822608125470598    0.9654538007870799
0.38852738324455255   0.9576717986194048
0.39479395394204536   0.94849342373534
0.4010605246395381    0.9377566498489481
0.4073270953370309    0.9253011066480541
0.4135936660345237    0.910973420147693
0.41986023673201645   0.8946333231017332
0.42612680742950926   0.8761603581287858
0.432393378127002     0.8554609153732876
0.4386599488244948    0.832475268535357
0.4449265195219876    0.8071842068657494
0.4511930902194804    0.7796148156608546
0.45745966091697315   0.7498449434202389
0.46372623161446597   0.7180059176581706
0.4699928023119587    0.6842831379542581
0.47625937300945154   0.648914284082833
0.4825259437069443    0.6121850235295748
0.4887925144044371    0.5744222754906498
0.49505908510192986   0.5359852719750268
0.5013256557994227    0.4972548323110243
0.5075922264969154    0.45862141598323425
0.5138587971944082    0.420472623216338
0.520125367891901     0.383180860705548
0.5263919385893938    0.34709187550280524
0.5326585092868865    0.312514785087355
0.5389250799843793    0.27971410506492983
0.5451916506818721    0.24890411255950173
0.551458221379365     0.22024570162198742
0.5577247920768577    0.1938457060834472
0.5639913627743505    0.16975850278130067
0.5702579334718433    0.14798957756968428
0.5765245041693361    0.12850064646118453
0.5827910748668288    0.11121587757590579
0.5890576455643216    0.09602875412693612
0.5953242162618144    0.0828091483473904
0.6015907869593071    0.07141023241474359
0.6078573576567999    0.06167492539945901
0.6141239283542927    0.053441655687703815
0.6203904990517856    0.046549298114431925
0.6266570697492783    0.04084121797486567
0.6329236404467711    0.03616841601834192
0.6391902111442639    0.032391817310818305
0.6454567818417567    0.029383782004598125
0.6517233525392494    0.027028938373703237
0.6579899232367422    0.025224449603537424
0.664256493934235     0.023879827845284696
0.6705230646317278    0.02291640418031332
0.6767896353292205    0.022266553500501607
0.6830562060267134    0.021872760758676113
0.6893227767242062    0.02168660112146875
0.695589347421699     0.021667692454141185
0.7018559181191917    0.021782665149735425
0.7081224888166845    0.02200418215376987
0.7143890595141773    0.022310031462226987
0.72065563021167      0.022682304526258468
0.7269222009091628    0.02310666687961621
0.7331887716066556    0.023571721814337643
0.7394553423041484    0.024068463901742054
0.7457219130016411    0.024589816389761987
0.751988483699134     0.025130244790815437
0.7582550543966268    0.02568543809665158
0.7645216250941196    0.02625204882202986
0.7707881957916123    0.02682748331321405
0.7770547664891051    0.027409734310803624
0.7833213371865979    0.027997248506595573
0.7895879078840907    0.0285888226839921
0.7958544785815834    0.02918352290791737
0.8021210492790762    0.029780622080761987
0.808387619976569     0.030379551971091934
0.8146541906740618    0.03097986653124979
0.8209207613715546    0.03158121393898988
0.8271873320690474    0.032183315325658546
0.8334539027665402    0.03278594859313472
0.8397204734640329    0.033388936081496234
0.8459870441615257    0.03399213513855481
0.8522536148590185    0.03459543087114131
0.8585201855565113    0.03519873053630379
0.864786756254004     0.03580195916758799
0.8710533269514968    0.036405056135538905
0.8773198976489897    0.03700797241954257
0.8835864683464825    0.03761066842601035
0.8898530390439752    0.03821311223050122
0.896119609741468     0.038815278152494526
0.9023861804389608    0.03941714559416465
0.9086527511364535    0.04001869809094549
0.9149193218339463    0.040619922533633335
0.9211858925314391    0.041220808530529784
0.9274524632289319    0.04182134788460068
0.9337190339264246    0.04242153416548561
0.9399856046239174    0.04302136235990589
0.9462521753214103    0.04362082858691285
0.9525187460189031    0.04421992986672563
0.9587853167163958    0.044818663933774594
0.9650518874138886    0.04541702908611195
0.9713184581113814    0.0460150240646337
0.9775850288088742    0.046612647956644715
0.9838515995063669    0.04720990011921328
0.9901181702038597    0.04780678011853767
0.9963847409013525    0.04840328768220714
\end{filecontents}

\definecolor{plot1}{RGB}{228,26,28}
\definecolor{plot2}{RGB}{55,126,184}

\begin{tikzpicture}[
    spy using outlines={circle, magnification=12, connect spies},
  ]
  \begin{axis}[
      xlabel = {Time (\si{\pico\second})},
      ylabel = {Population ($\rho_{00}$)},
    ]
    \addplot+[no marks, smooth, ultra thin, plot2] table [x index = {0}, y index = {1}]{frames.dat};
    \addlegendentry{Fixed frame};

    \addplot+[no marks, smooth, ultra thin, plot1] table [x index = {0}, y index = {2}]{frames.dat};
    \addlegendentry{Rotating frame};

    \coordinate (spypoint) at (axis cs:0.69, 0.065);
    \coordinate (magnifyglass) at (axis cs:0.2,0.3);
  \end{axis}
  \spy [gray, size=2.5cm] on (spypoint) in node[fill=white] at (magnifyglass);
\end{tikzpicture}

  \caption{\label{fig:frame comparison}
    Validity of the rotating-frame approximation within an integral equation framework.
    The fixed- and rotating-frame populations for a collection of quantum dots display nearly identical trajectories.
  }
\end{figure}

As the \qds{} in our system contain a discrete set of internal energy levels, they interact with only a narrow band of frequencies in the radiation field and we may treat $\vb{E}(t)$ as a slowly-varying envelope modulated by a sinusoid of frequency $\omega_L$:
\begin{equation}
  \vb{E}(t) = \tilde{\vb{E}}(t)\cos(\omega_L t).
  \label{eq:modulated field}
\end{equation}

Introducing a unitary transformation $\tilde{\rho} = U \hat{\rho} U^\dagger$ where $U = \mathrm{diag}(1, e^{i \omega_L t})$ produces a rotating-frame version of \cref{eq:liouville}
\begin{equation}
  i \hbar \pdv{\tilde{\rho}}{t} = \commutator{U \hat{\mathcal{H}} U^\dagger - i \hbar V}{\tilde{\rho}}; \quad V \equiv U \dot{U}^\dagger.
  \label{eq:rotating liouville}
\end{equation}
