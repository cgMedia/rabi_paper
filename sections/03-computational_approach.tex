\section{Computational approach}
\begin{figure}
  \centering
  \input{figures/frame_comparison.tex}
  \caption{\label{fig:frame comparison}
    Validity of the rotating-frame approximation within an integral equation framework.
    The fixed- and rotating-frame populations for a collection of quantum dots display nearly identical trajectories.
  }
\end{figure}

As the \qds{} in our system contain a discrete set of internal energy levels, they interact with only a narrow band of frequencies in the radiation field and we may treat $\vb{E}(t)$ as a slowly-varying envelope modulated by a sinusoid of frequency $\omega_L$:
\begin{equation}
  \vb{E}(t) = \tilde{\vb{E}}(t)\cos(\omega_L t).
  \label{eq:modulated field}
\end{equation}

Introducing a unitary transformation $\tilde{\rho} = U \hat{\rho} U^\dagger$ where $U = \mathrm{diag}(1, e^{i \omega_L t})$ produces a rotating-frame version of \cref{eq:liouville}
\begin{equation}
  i \hbar \pdv{\tilde{\rho}}{t} = \commutator{U \hat{\mathcal{H}} U^\dagger - i \hbar V}{\tilde{\rho}}; \quad V \equiv U \dot{U}^\dagger.
  \label{eq:rotating liouville}
\end{equation}
