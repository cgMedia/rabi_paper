\section{Computational approach}

\subsection{The rotating-wave approximation}

Models of lone \qds{} or \qd{} ensembles in which the separation between dots far exceeds a typical interaction length (perhaps through a finite-difference time-domain grid) often use a rotating-wave approximation to great effect in eliminating high-frequency terms in the solution to \cref{eq:liouville} with only a small change in the resulting dynamics.
Assuming the incident radiation takes the form
\begin{equation}
  \vb{E}(t) = \tilde{\vb{E}}(t) \cos(\omega_L t)
  \label{eq:field envelope}
\end{equation}
where $\tilde{\vb{E}}(t)$ represents a slowly-varying envelope function, we introduce a unitary transformation $\tilde{\rho} = \hat{U} \hat{\rho} \hat{U}^\dagger$ where $\hat{U} = \mathrm{diag}(1, e^{i \omega_L t})$ to produce a rotating-frame version of \cref{eq:liouville},
\begin{equation}
  i \hbar \pdv{\tilde{\rho}}{t} = \commutator{\hat{U} \hat{\mathcal{H}} \hat{U}^\dagger - i \hbar \hat{V}}{\tilde{\rho}} - \hat{\mathcal{D}}\qty[\tilde{\rho}]; \quad V \equiv U \dot{U}^\dagger,
  \label{eq:rotating liouville}
\end{equation}
and treating treating $\hat{U} \hat{\mathcal{H}} \hat{U}^\dagger - i \hbar \hat{V} \equiv \tilde{\mathcal{H}}$ as a ``rotating-frame Hamiltonian.''
If the electric field appearing inside of $\tilde{\mathcal{H}}$ takes the form of \cref{eq:field envelope}, the resulting equation contains terms proportional to $e^{i \qty(\omega_0 \pm \omega_L) t}$.
By neglecting the $e^{i \qty(\omega_0 + \omega_L) t}$ terms under the assumption that they oscillate quickly and will therefore integrate out over any appreciable timescales, we produce a governing equation much more amenible to numerical integration as it contains only smooth (low-frequency) quantities.


\subsection{Solution of the Liouville equation}

Solving \cref{eq:liouville} for each of $n_s$ \qds{} amounts to solving $n_s$ coupled first-order differential equations in time.
The coupling term in \cref{eq:hamiltonian} and the retardation factor in \cref{eq:total field} add additional complexity, however, making ``straightforward'' solution methods (such as RK4 or other midstep methods) ill-suited for the problem at hand.
Due to its extreme accuracy and inherent bandlimitedness, we make use of the highly-tuned predictor/corrector scheme detailed in~\cite{Glaser2009} to numerically solve \cref{eq:liouville} for each \qd{} in the system.
\begin{equation}
  \rho_i(t) \approx \sum_{j = 1}^{n_\lambda} w_j e^{\lambda_j t}
\end{equation}
where each of the $\lambda_j$ take on appropriately chosen complex values, 

\begin{equation}
  \rho_i(t_{k + 1}) = \sum_{\ell = 1}^k \qty[\mathcal{P} \ell^{\qty(0)} \rho_i(t_i) + \mathcal{P} \ell^{\qty(1)} \dot{\rho}_i(t_i)]
\end{equation}


\subsection{Evaluation of radiated fields}

As the \qds{} in our system contain a discrete set of internal energy levels, they interact with only a narrow band of frequencies in the radiation field and we may treat $\vb{E}(t)$ as a slowly-varying envelope modulated by a sinusoid of frequency $\omega_L$:
\begin{equation}
  \vb{E}(t) = \tilde{\vb{E}}(t)\cos(\omega_L t).
  \label{eq:modulated field}
\end{equation}

