\documentclass{tufte-handout}

\usepackage{newpxtext,newpxmath}

\usepackage[normalem]{ulem}
\usepackage{amsmath}
\usepackage{cleveref}
\usepackage{graphicx}
\usepackage{lipsum}
\usepackage{microtype}
\usepackage{siunitx}
\usepackage{textcomp}
\usepackage{wasysym}

% Set up the drop-caps
\usepackage{Rothdn, lettrine}
\renewcommand\LettrineFontHook{\Rothdnfamily}

\begin{document}

\hfill\today\\
\vspace{0.2cm}

\noindent Dear editor, \\
\vspace{0.2cm}

We wish to submit the attached manuscript for consideration as a regular article in Physical Review A.
In it, we study computationally the behavior of a large number of electromagnetically-coupled quantum dots under intense laser excitation within a semiclassical formulation.
As a departure from previous methods, our novel algorithm uses an integral kernel to describe radiation accurately at all distances.
This allows for fully three-dimensional investigations into short and long-range interaction effects simultaneously.
We detail the results of simulations using our approach for systems of up to \num{10000} particles.
We observe significant coupling effects between extremely close quantum dots whereby they both exhibit a diminished response to an incident light pulse; we attribute this effect to a dynamical detuning effect that shifts the transition frequency of the pair beyond the spectrum of the incident laser.
In extended geometries spanning several wavelengths, we also observe regions of enhanced polarization that do not occur in simulations without particle interactions.
We believe that Physical Review A is the appropriate journal for this manuscript because our results could help understand nonlinear pulse propagation effects in quantum dot (or similar) media with applications in nanolasers or other similar devices.
Moreover, our coded algorithms (which we have made available alongside this publication) could easily extend to model other atomic or molecular systems.
Thank you for receiving our manuscript and considering it for review.
We appreciate your time and look forward to your response.

\vspace{0.6in}
\noindent Regards,\\
\noindent Connor Glosser, Balasubramaniam Shanker, and Carlo Piermarocchi

\end{document}
